\documentclass[a4paper, 12pt]{report}

\usepackage{polyglossia}
\usepackage{longtable}
\usepackage{xcolor}  % can't use color with polyglossia
\usepackage{arydshln}

\interfootnotelinepenalty=10000 % prevents the footnote from breaking across pages
% http://tex.stackexchange.com/questions/32208/footnote-runs-onto-second-page

% Thanks to Manas Tungare (http://manas.tungare.name/software/latex) for these settings.
\setlength{\paperwidth}{210mm}
\setlength{\paperheight}{297mm}

\setlength{\textwidth}{160mm}
\setlength{\textheight}{247mm}

\setlength{\evensidemargin}{1in}
\setlength{\oddsidemargin}{0in}
\setlength{\topmargin}{-0.5in}

\setmainfont[Mapping=tex-text]{Liberation Serif}  % Set the default font for the document (footnotes will also use this font, but see also: http://tex.stackexchange.com/questions/4779/how-to-change-font-family-in-footnote)

\setmainlanguage{english}
\setotherlanguage{arabic}
\setotherlanguage{farsi}  % used to allow typesetting the tile in a different font

\newfontfamily\arabicfont[Script=Arabic, Scale=2]{Scheherazade} % Arabic transcription layer
\newfontfamily\farsifont[Script=Arabic, Scale=2]{GranadaKD} % Arabic transcription layer for titles - uses a version of Granada which has been extended to include glyphs for Swahili.
\newcommand{\AS}[1]{\fontspec[Script=Arabic, Scale=1.5]{Scheherazade} #1\normalfont} % Arabic when used in footnotes - using \newfontfamily resets following English text as well.  IMPORTANT: this can ONLY be used for single words - multiple words will be written LTR, and not RTL as required.
\newfontfamily{\T}[Scale=0.9, Color=00BB33]{Linux Biolinum O} %  transliteration layer - Biolinum handles diacritics well
\newfontfamily{\I}[Scale=0.9, Color=blue]{Linux Biolinum O} % interpolated letters in the transliteration layer
\newfontfamily{\E}[Scale=0.9, Color=666666]{Liberation Serif Italic} % English translation layer

\renewcommand\thefootnote{\textcolor{red}{\arabic{footnote}}}  % Alter the colour of the footnote markers - thanks to Gonzalo Medina (http://tex.stackexchange.com/questions/26693/change-the-color-of-footnote-marker-in-latex#26696):

\begin{document}

\begin{longtable}{r}
\textfarsi{مَئِيشَ} \\*
\T{maisha} \\
\cdashline{1-1}[1pt/3pt] \\
[6mm]
\end{longtable}


\begin{longtable}{rrl} 

\textarabic{نَأَنْدَ نُذُومَ هِينِ} & \textarabic{كْوَ إِينَ لَ رَهَمَانِ} & \textarabic{١} \\* 
\T{naanda nudhuma hini} & \T{kwa ina la rahamani} & \T{1a/b} \\ 
\textarabic{نَمْوَنْدِكِيَ يَهَايَ} & \textarabic{إِينَ يَ پِيلِ رَمَانِ} &  \\* 
\T{namwandikiya yahaya} & \T{ina ya pili ramani} & \T{1c/d} \\ 
\\[8mm] 

\textarabic{بَنَاتِ نَلِوَفُونْدَ} & \textarabic{كْوٖينْيٖ رَمَانِ يَ كْوَانْدَ} & \textarabic{٢} \\* 
\T{banati naliwafunda} & \T{kwenye ramani ya kwanda} & \T{2a/b} \\ 
\textarabic{وَڤُلَانَ كُوَمْبِيَ} & \textarabic{نَ هِينِ نِمٖئُِونْدَ} &  \\* 
\T{wavulana kuwambiya} & \T{na hini nimeiunda} & \T{2c/d} \\ 
\\[8mm] 

\textarabic{نَهِيسِ وَانَ وَتَاكَ} & \textarabic{نَ أَسِيلِ يَ كْوَنْدِيكَ} & \textarabic{٣} \\* 
\T{nahisi wana wataka} & \T{na asili ya kwandika} & \T{3a/b} \\ 
\textarabic{مِيمِ أَلِنَنْدِكِيَ} & \textarabic{بَابَ پِيَ كَذَلِيكَ} &  \\* 
\T{mimi alinandikiya} & \T{baba piya kadhalika} & \T{3c/d} \\ 
\\[8mm] 

\textarabic{أَتَمْجَازِ وَدُودِ} & \textarabic{بَابَ بوَانَ أَهمَادِ} & \textarabic{٤} \\* 
\T{atamjazi wadudi} & \T{baba bwana ahmadi} & \T{4a/b} \\ 
\textarabic{كُنَنْدِكِيَ وَسِيَ} & \textarabic{كْوَانِ أَلِجِتَهِيدِ} &  \\* 
\T{kunandikiya wasiya} & \T{kwani alijitahidi} & \T{4c/d} \\ 
\\[8mm] 

\textarabic{نْدِيٗ لَ وَسِيَ يِينَ} & \textarabic{هَاپٗ زَمَانِ زَ يَانَ} & \textarabic{٥} \\* 
\T{ndiyo la wasiya yina} & \T{hapo zamani za yana} & \T{5a/b} \\ 
\textarabic{بَبَانْڠُ كَنِوَتِيَ} & \textarabic{أَلٗنَنْدِكِئَ بوَانَ} &  \\* 
\T{babangu kaniwatiya} & \T{alonandikia bwana} & \T{5c/d} \\ 
\\[8mm] 

\textarabic{تَيِپِينْدَ كُبَئِينِ} & \textarabic{نَامِ كَتِيكَ رَمَانِ} & \textarabic{٦} \\* 
\T{tayipinda kubaini} & \T{nami katika ramani} & \T{6a/b} \\ 
\textarabic{يَوٖيزَ كُسَئِدِيَ} & \textarabic{يَالٖ نِنَيٗؤَمِينِ} &  \\* 
\T{yaweza kusaidiya} & \T{yale ninayoamini} & \T{6c/d} \\ 
\\[8mm] 

\textarabic{نَ يَالٖ نِلٗيَسٗومَ} & \textarabic{تَنٖينَ نَلٗيَتُومَ} & \textarabic{٧} \\* 
\T{na yale niloyasoma} & \T{tanena naloyatuma} & \T{7a/b} \\ 
\textarabic{كْوَ وَاتُ نِلٗپٗكٖيَ} & \textarabic{تَزِتَايَ نَ هٖكِيمَ} &  \\* 
\T{kwa watu nilopokeya} & \T{tazitaya na hekima} & \T{7c/d} \\ 
\\[8mm] 

\textarabic{مْبَالِ مْبَالِ كُتَنْڠَانْيَ} & \textarabic{يَالٖ نِتَيَكُسَانْيَ} & \textarabic{٨} \\* 
\T{mbali mbali kutanganya} & \T{yale nitayakusanya} & \T{8a/b} \\ 
\textarabic{كَامَ لَ مَئِيشَ بٗويَ} & \textarabic{لٖينْڠٗ لَانْڠُ نِ كُفَانْيَ} &  \\* 
\T{kama la maisha boya} & \T{lengo langu ni kufanya} & \T{8c/d} \\ 
\\[8mm] 

\end{longtable}

% Remember to end each line (except vertical space) with a double backslash

\begin{longtable}{r}

 \\  % Any material after the end of the poem

\end{longtable}

\end{document}

