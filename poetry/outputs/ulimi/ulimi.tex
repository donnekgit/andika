\documentclass[a4paper, 12pt]{report}

\usepackage{polyglossia}
\usepackage{longtable}
\usepackage{xcolor}  % can't use color with polyglossia
\usepackage{arydshln}

\interfootnotelinepenalty=10000 % prevents the footnote from breaking across pages
% http://tex.stackexchange.com/questions/32208/footnote-runs-onto-second-page

% Thanks to Manas Tungare (http://manas.tungare.name/software/latex) for these settings.
\setlength{\paperwidth}{210mm}
\setlength{\paperheight}{297mm}

\setlength{\textwidth}{160mm}
\setlength{\textheight}{247mm}

\setlength{\evensidemargin}{1in}
\setlength{\oddsidemargin}{0in}
\setlength{\topmargin}{-0.5in}

\setmainfont[Mapping=tex-text]{Liberation Serif}  % Set the default font for the document (footnotes will also use this font, but see also: http://tex.stackexchange.com/questions/4779/how-to-change-font-family-in-footnote)

\setmainlanguage{english}
\setotherlanguage{arabic}
\setotherlanguage{farsi}  % used to allow typesetting the tile in a different font

\newfontfamily\arabicfont[Script=Arabic, Scale=2]{Scheherazade} % Arabic transcription layer
\newfontfamily\farsifont[Script=Arabic, Scale=2]{GranadaKD} % Arabic transcription layer for titles - uses a version of Granada which has been extended to include glyphs for Swahili.
\newcommand{\AS}[1]{\fontspec[Script=Arabic, Scale=1.5]{Scheherazade} #1\normalfont} % Arabic when used in footnotes - using \newfontfamily resets following English text as well.  IMPORTANT: this can ONLY be used for single words - multiple words will be written LTR, and not RTL as required.
\newfontfamily{\T}[Scale=0.9, Color=00BB33]{Linux Biolinum O} %  transliteration layer - Biolinum handles diacritics well
\newfontfamily{\I}[Scale=0.9, Color=blue]{Linux Biolinum O} % interpolated letters in the transliteration layer
\newfontfamily{\E}[Scale=0.9, Color=666666]{Liberation Serif Italic} % English translation layer

\renewcommand\thefootnote{\textcolor{red}{\arabic{footnote}}}  % Alter the colour of the footnote markers - thanks to Gonzalo Medina (http://tex.stackexchange.com/questions/26693/change-the-color-of-footnote-marker-in-latex#26696):

\begin{document}

\begin{longtable}{r}
\textfarsi{أُلِيمِ} \\*
\T{ulimi} \\
\cdashline{1-1}[1pt/3pt] \\
[6mm]
\end{longtable}


\begin{longtable}{rrl} 

\textarabic{أُكِؤُتٗنْڠٗوَ} & \textarabic{أُلِيمِ نِ تَامُ} & \textarabic{١} \\* 
\T{ukiuṯongowa} & \T{ulı̄mi ni ṯāmu} & \T{1a/b} \\ 
\textarabic{يَ كُسٗنْڠٗنْيٗوَ} & \textarabic{وَ أَامَ نِ سُومُ} &  \\* 
\T{ya kusongonyowa} & \T{wa āma ni sūmu} & \T{1c/d} \\ 
\textarabic{نَ مْوِيسٗ هُؤُوَ} & \textarabic{أُيُوٖ فَهَامُ} &  \\* 
\T{na mwı̄so huuwa} & \T{uyuwe fahāmu} & \T{1e/f} \\ 
\\[8mm] 

\textarabic{هَئِينَ مْفُوپَ} & \textarabic{نِ نْيَامَ لَئِينِ} & \textarabic{٢} \\* 
\T{haı̄na mfūpa} & \T{ni nyāma laı̄ni} & \T{2a/b} \\ 
\textarabic{مٗولَ أَمٖتُوپَ} & \textarabic{نْجٖيمَ كْوَ لَهَانِ} &  \\* 
\T{mōla ameṯūpa} & \T{njēma kwa lahāni} & \T{2c/d} \\ 
\textarabic{أُسِؤٗ مْشِيپَ} & \textarabic{تهأُونُ يَ ثَمَانِ} &  \\* 
\T{usio mshı̄pa} & \T{ṯʿūnu ya thamāni} & \T{2e/f} \\ 
\\[8mm] 

\textarabic{كْوَ وَاكٗ وٖنْدَانِ} & \textarabic{أُتَتُومَ وٖيمَ} & \textarabic{٣} \\* 
\T{kwa wāko wenḏāni} & \T{uṯaṯūma wēma} & \T{3a/b} \\ 
\textarabic{نَ كُويَ نْيُمْبَانِ} & \textarabic{نْدُوڠُ كُكُڠٖيمَ} &  \\* 
\T{na kūya nyumbāni} & \T{nḏūgu kukugēma} & \T{3c/d} \\ 
\textarabic{هَلِپَتِكَانِ} & \textarabic{لَكُكُسُكُومَ} &  \\* 
\T{halipaṯikāni} & \T{lakukusukūma} & \T{3e/f} \\ 
\\[8mm] 

\textarabic{كُوٖيكْوَ پَأَنِ} & \textarabic{كْوَ بَابَ نَ مَامَ} & \textarabic{٤} \\* 
\T{kuwēkwa paani} & \T{kwa bāba na māma} & \T{4a/b} \\ 
\textarabic{نَ كْوَ مَجِرَانِ} & \textarabic{أُتَكُوَ مْوٖيمَ} &  \\* 
\T{na kwa majirāni} & \T{uṯakuwa mwēma} & \T{4c/d} \\ 
\textarabic{نِ أُسُلُتوَانِ} & \textarabic{كُيُوَ كُسٖيمَ} &  \\* 
\T{ni usuluṯwāni} & \T{kuyuwa kusēma} & \T{4e/f} \\ 
\\[8mm] 

\textarabic{وَؤُوزٖ نِ نْيَانِ؟} & \textarabic{وَسٗكُفَهَامُ} & \textarabic{٥} \\* 
\T{waūze ni nyāni?} & \T{wasokufahāmu} & \T{5a/b} \\ 
\textarabic{وَاجٖ مَسِكَانِ} & \textarabic{وَفَانْيٖ هَمُومُ} &  \\* 
\T{wāje masikāni} & \T{wafānye hamūmu} & \T{5c/d} \\ 
\textarabic{وَاوٖ وَ نْيَؤٗونِ} & \textarabic{نَ وَكُهِشِيمُ} &  \\* 
\T{wāwe wa nyaōni} & \T{na wakuhishı̄mu} & \T{5e/f} \\ 
\\[8mm] 

\textarabic{أُوٖكٖيوٖ مْبٖيكٗ} & \textarabic{سوَرِيفُ هِسَانِ} & \textarabic{٦} \\* 
\T{uwekēwe mbēko} & \T{swarı̄fu hisāni} & \T{6a/b} \\ 
\textarabic{نَ كْوَ وَاتُ وَاكٗ} & \textarabic{كْوَ نْدٖ نَ نْدَانِ} &  \\* 
\T{na kwa wāṯu wāko} & \T{kwa nḏe na nḏāni} & \T{6c/d} \\ 
\textarabic{تٗوَ شَاكَ نْدَاكٗ} & \textarabic{دَرَاجَ يَ شَانِ} &  \\* 
\T{ṯowa shāka nḏāko} & \T{ḏarāja ya shāni} & \T{6e/f} \\ 
\\[8mm] 

\end{longtable}

% Remember to end each line (except vertical space) with a double backslash

\begin{longtable}{r}

 \\  % Any material after the end of the poem

\end{longtable}

\end{document}

