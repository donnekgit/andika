\chapter{Editing fonts}
\renewcommand{\thesection}{B/\arabic{section}}  % redefine the section numbering
\setcounter{section}{0}  % reset counter
\label{appB}

\begin{quotation}
\noindent I am grateful to Khaled Hosny\footnote{\url{khaledhosny.org}} for his advice on using FontForge to edit Arabic glyphs, which has been incorporated in these instructions.
\end{quotation}

\section{Introduction}
\label{appb:intro}

Most Arabic fonts are missing some glyphs that are essential to allow them to be used for writing Swahili.  This appendix deals with how to edit these fonts to add the missing glyphs.  This will entail editing the font with FontForge\footnote{\url{fontforge.github.io/en-US}} (originally developed by George Williams).

\section{Install FontForge}

There are two options here -- the easiest is to use a pre-compiled package.

\subsection{Use a pre-compiled package}

The FontForge package included in Ubuntu 14.04 by default dates from 2012, so it is preferable to install the more up-to-date package from the FontForge Personal Package Archive (PPA).\footnote{\url{https://launchpad.net/~fontforge/+archive/ubuntu/fontforge}}

Check that the helper script add-apt-repository is installed:

\verb|sudo apt-get install software-properties-common|

Add the FontForge PPA (which will also add the authentication key):

\verb|sudo add-apt-repository ppa:fontforge/fontforge|

Update the package list:

\verb|sudo apt-get update|

Install FontForge:

\verb|sudo apt-get install fontforge|


\subsection{Compile from the source code}

In some cases, perhaps because you want access to a feature not yet available in the pre-compiled package, you may wish to compile your own version from the code available on GitHub.\footnote{\url{github.com/fontforge/fontforge}}

\subsubsection{Install preliminary software}

Install packages to allow the building of software:

\verb|sudo apt-get install build-essential automake flex bison|

Install the \textit{unifont} package to get a full display of the reference glyphs.  Unifont\footnote{\url{savannah.gnu.org/projects/unifont}} includes glyphs for all Unicode codepoints, and FontForge will use it if it is installed.

\verb|sudo apt-get install unifont|

Install other required packages: 

\verb|sudo apt-get install packaging-dev pkg-config python-dev libpango1.0-dev|\\
$\hookrightarrow$ \verb|libglib2.0-dev libxml2-dev giflib-dbg libjpeg-dev libtiff-dev uthash-dev|

\subsubsection{Build \textit{libspiro}}

FontForge uses \textit{libspiro}\footnote{\url{github.com/fontforge/libspiro}} (by Raph Levien) to simplify the drawing of curves.

Download the code:

\verb|git clone https://github.com/fontforge/libspiro.git|

Run the following commands in sequence (that is, wait for each one to complete before running the next):
\begin{verbatim}
cd libspiro
autoreconf -i
automake --foreign -Wall
./configure
make
sudo make install
cd ..
\end{verbatim}

\subsubsection{Build \textit{libuninameslist}}

FontForge uses \textit{libuinameslist}\footnote{\url{github.com/fontforge/libuninameslist}} to access attribute data about each Unicode code point.

Download the code:

\verb|git clone https://github.com/fontforge/libuninameslist.git|

Run the following commands in sequence (that is, wait for each one to complete before running the next):
\begin{verbatim}
cd libuninameslist
autoreconf -i
automake --foreign
./configure
make
sudo make install
cd ..
\end{verbatim}

\subsubsection{Build FontForge}

Download the code:

\verb|git clone https://github.com/fontforge/fontforge.git|

Run the following commands in sequence (that is, wait for each one to complete before running the next):
\begin{verbatim}
cd fontforge
./bootstrap
./configure
make
sudo make install
cd ..
\end{verbatim}

Make the system aware of the new libraries:

\verb|sudo ldconfig|


\section{Make a working copy of the font}

The font we will add glyphs to is Graph\footnote{\url{openfontlibrary.org/en/font/graph}} (regenerated by Nadim Shaikli).  A version of the following howto which includes images is available on the \textit{Design with FontForge} website.\footnote{\url{http://designwithfontforge.com/en-US/Adding_Glyphs_to_an_Arabic_Font.html}}

Download the font from the webpage into the \textit{andika} directory.  Unzip it, and delete the zip file:

\verb|unzip -q graph.zip -d fonts && rm graph.zip|

Launch FontForge (in KDE, go to \textbf{K \textrightarrow\ Graphics \textrightarrow\ FontForge}).  Note that FontForge is built using the programming language Tcl,\footnote{\url{tcl.tk/}} and it therefore behaves slightly differently from other software you may be used to.  For instance, every action requires at least one click (so the submenus for menus don't appear as you move across the menu bar -- you have to click each one).

The first time you open FontForge, it will ask to you load a font.  Navigate to \textit{andika/fonts}, select \textit{ae\_Graph.ttf}, and click \textbf{OK}.  FontForge will display a chart of every glyph in the font, each in its own cell.  The smaller cell above it is a reference glyph -- not all reference glyphs will have a font glyph, since few fonts contain glyphs for every single Unicode code point.  Where the font glyph is missing, the cell will contain a grey X.

Save it as an sfd file which will become your working copy: select \textbf{File \textrightarrow\ Save}, edit the suggested name to read \textbf{GraphSwa.sfd} and click \textbf{Save}.

\section{Rename the font}

If you do not rename the font, your adapted font will not install separately from the original -- you will have to uninstall the original font first.  It is also sensible to rename the font if you are going to distribute your adaptations -- if the original author of the font has reserved the font name under the Reserved Font Name (RFN) mechanism, that original name can only be used with the original author's version of the font.

If you adapt a font that was originally under an open license (eg GPL\footnote{\url{gnu.org/copyleft/gpl.html}} or OFL\footnote{\url{scripts.sil.org/OFL-FAQ_web}}) and then distribute it, you must retain the original author's copyright notices and licensing information, although you can append a note at the end of the copyright notice covering your contribution.

Note that adapting a font that was originally under a closed license (eg most fonts by Microsoft, Adobe, Bitstream, Linotype, etc), may be a breach of copyright, depending on the terms of the license.

Select \textbf{Element \textrightarrow\ Font Info}, and in the \textit{PS Names} panel, change \textit{Fontname}, \textit{Family Name}, and \textit{Name For Humans} to \textbf{GraphSwa}.

In the \textit{TTF Names} panel, the names for \textit{Family} and \textit{Fullname} are taken from the \textit{PS Names} entries, and should already be showing \textit{GraphSwa} (you can't edit them directly).  Change the entries for \textit{Preferred Family} and \textit{Compatible Full} to \textbf{GraphSwa}.  These name changes will now allow you to install this font alongside the original one if you wish.

If desired, in the \textit{TTF Names} panel you can also place a "Swahili glyphs added by" message after the text already in the entry for \textit{Designer}.

Click \textbf{OK} to save these changes.  You will get a message about generating a new UniqueID (XUID) for the font -- click \textbf{Change}.

\section{Add the glyph for the isolated form of \textit{peh}}
\label{s:pehisol}

We will add the missing glyph \textit{peh} (U+067E) to the Graph font.

Go to the Arabic section of the font chart: select \textbf{View \textrightarrow\ Go to}, click the dropdown box and select \textbf{Arabic}, then click \textbf{OK}.

Clicking on a cell in the font chart will show its Unicode number and name in blue at the top of the panel.  Go to position \textit{1662 (0x67e) U+067E ``uni067E'' ARABIC LETTER PEH}.  The cell below the reference glyph contains a grey X, showing that the font does not include this glyph.

We will make \textit{peh} by copying \textit{beh} (U+0628) and swapping its single dot for three dots.

Click on the \textit{beh} cell (position 1576), then right-click and select \textbf{Copy}.  Then right-click on the \textit{peh} cell and select \textbf{Paste}.  Now that \textit{beh} is now copied into the \textit{peh} cell, the next thing is to change the dot.

Find a glyph with three dots -- \textit{sheen} (position 1588, U+0634) will do.  Double-click on the cell -- this will open a glyph design panel.  Press \textbf{V} to ensure the pointer tool (arrowhead) in the toolbox is selected, and press \textbf{Z} and enlarge the panel to give you a good view of the glyph.

Click and drag so that the nodes of the three dots above sheen change colour from pink to beige.  If you accidentally include or omit a node, deselect or select it by pressing \textbf{Shift} and clicking.  Press \textbf{Alt+C} to copy.

Go back to the font chart and double-click on the \textit{peh} cell -- this will load \textit{peh} into another tab in the glyph design panel, alongside the \textit{sheen} tab.

Click and drag to highlight the dot below \textit{peh}, then press \textbf{Delete}.  Press \textbf{Alt+V} to paste in the three dots, which will likely appear above the body of \textit{peh}.  Leave the dot nodes highlighted so that you can invert and move them more easily.

Invert the dots: select the flip tool (two triangles with a red dashed line between them) from the toolbox.  (Alternatively, right-click in the middle of the dots, and select \textbf{Flip the selection} from the popup.)  Click on one of the dot nodes and drag the mouse slightly left or right.

Move the inverted dots: press \textbf{V} to select the pointer tool again, click on one of the dot nodes, and drag them down below the body of the glyph.  Position them centrally, above the \textit{ArabicBelow} mark.

Close the glyph design panel. There should now be a new glyph for \textit{peh} in the font chart.  Save the adapted font (\textbf{File \textrightarrow\ Save}).


\section{Add the glyphs for the connected forms of \textit{peh}}

However, this is only the isolated (standalone) form of the glyph.  If you try to use your adapted font, you will find that initial, medial and final forms are not available.  These have to be created separately.  "The[se] forms are built as unencoded glyphs (glyphs whose encoding is -1 in FontForge conventions).  Th[ey] have no predefined slots." (Khaled Hosny)

Select \textbf{Encoding \textrightarrow\ Add Encoding Slots} and enter the number of the glyphs you want -- in this case \textbf{3}.  FontForge will add the same number of slots at the very end of the font, and you will be moved there in the font chart.  The last three cells (positions 65537, 65538, 65539) have a question mark as a reference glyph, and it is in those cells that you will add the unencoded glyphs by repeating the process in \Cref{s:pehisol} above.

Note that if by mistake you start typing when the font chart still has focus, you get moved to the European section at the top.  To get back to the bottom, select \textbf{View \textrightarrow\ Go to}, click the dropdown box and select \textbf{Not a Unicode Character},  and then click \textbf{OK}.

\subsection{Create the final form}

Roll the font chart up a bit until you come to a set of Arabic glyphs at position 65152 (U+FE80) onwards.  At U+FE90 (position 65168) you will see a \textit{behfinal} glyph -- click on it and press \textbf{Ctrl+C} to copy it.  Roll down to the third last cell in the chart (position 65537), click on it, and press \textbf{Ctrl-V} to paste in the \textit{behfinal} glyph.

Right-click on the cell and select \textbf{Glyph Info}.  The naming convention is to use the number of the isolated glyph + a suffix for the form, so change \textit{Glyph Name} to \textbf{uni067E.fina},  and click \textbf{OK}.  The question mark in the reference cell will change to \textit{peh}.

Get the three dots: double-click on \textit{sheen} (U+FEB5) to load it into the glyph design panel, select the three dots and press \textbf{Ctrl+C}.

Double-click on the new \textit{pehfinal} to load it into the glyph design panel, click and drag to highlight the nodes of the dot and press \textbf{Delete}.

Ctrl+V to insert the three dots from \textit{sheen}, flip them, and move them into position below the glyph body.  Press \textbf{Ctrl+S} to save the revised font chart.

\subsection{Create the initial and medial forms}

Copy the initial form U+FE91 (position 65169) to the penultimate cell (position 65538), delete the single dot and paste in the three dots.

Right-click the cell, select \textbf{Glyph Info}, change \textit{Glyph Name} to \textbf{uni067E.init}, and click \textbf{OK}.

Copy the medial form U+FE92 (position 65170) to the last cell (position 65539), delete the single dot and paste in the three dots.

Right-click the cell, select \textbf{Glyph Info}, change \textit{Glyph Name} to \textbf{uni067E.medi}, and click \textbf{OK}.

Select \textbf{File \textrightarrow\ Save} to save the revised font chart.

\subsection{Add the lookups}

The isolated form has to be mapped (linked) to its initial, medial and final forms.

Select \textbf{Element \textrightarrow\ Font Info \textrightarrow\ Lookups}.

Click on the \textbf{+} beside the entry \textit{'init' Initial Forms in Arabic lookup 2}.  This will open a submenu of the same name.  Click on this submenu.

The \textit{Edit Data} button on the right will now become available -- click it.

In the \textit{Lookup Subtable} panel that pops up, ensure that the \textit{Unicode} button is checked.  Roll the list of characters down until you come to the end.

In the box beside \textit{Default Using Suffix}, enter the relevant suffix (in this case, \textbf{init}), and then click \textbf{Default Using Suffix}.

A new mapping will be added to the list of characters, from uni067E (the isolated form of \textit{peh}) to uni067E.init (the initial form).
Click \textbf{OK}.

Do the same for the submenus under the entries \textit{'medi' Medial Forms in Arabic lookup 2} and \textit{'fina' Terminal Forms in Arabic lookup 2}, choosing \textit{medi} and \textit{fina} as the relevant suffix.

Click \textbf{OK} again to close the panel, and save the font chart (\textbf{Ctrl+S}).

Note that \textit{Default Using Suffix} only seems to work on glyphs in the Unicode 06 (\textit{Arabic}) block -- glyphs in Unicode 07 (\textit{Arabic Supplement}), eg \textit{ain} with two dots, may have to be added manually by clicking the line marked \textit{New} and typing in the names.

\section{Generate the adapted font}

Select \textbf{File \textrightarrow\ Generate Fonts}.

In the dropdown showing \textit{PS Type 1 (Binary)}, select \textbf{TrueType}, and check that the filename reads \textit{GraphSwa.ttf}.

Navigate to where you want to save the font, and then click \textbf{Generate}.  Click \textbf{Yes} and \textbf{Generate} to the two information messages that come up.  You can then use your normal font installation procedure (in KDE, \textbf{K \textrightarrow\ System \textrightarrow\ System Settings \textrightarrow\ Font Management}) to install the adapted font.

\section{Next steps}

You will need to carry out the above process to add all the missing glyphs listed in \Cref{tab:missglyphs}.

Note that if you make changes to a font, you need to restart LibreOffice in order to use the changed font, because it will see only the previous version of the font, and not the new changes.
