\chapter{Typesetting poetry}
\label{ch:poetry}

As noted in \Cref{ch:intro}, a key aim of \textbf{Andika!} is to facilitate the production of digital versions of classical Swahili manuscripts.  This chapter deals with the tools provided to do that -- they are based on the concept of importing each word of the text into a database table, and then adding material such as notes on individual words or sections, variant readings and emendations, translations, etc.  The enriched text can then be output in a number of formats allowing for both print and online publication -- see \Cref{s:output}.  This approach also facilitates the automatic production of word frequency lists, glossaries, concordances, n-grams, and so on, which open the way for detailed linguistic analysis of the text.

These chapter focusses on the possibilities for typesetting traditional poetry in Arabic script.  However, a similar approach can be used for prose, and a further chapter dealing with this may be added at a future date.


\section{Input}

The first step is to manually transcribe the manuscript letter-for-letter into a LibreOffice \textit{odt} document -- virtually all the Arabic glyphs likely to be used in a manuscript are already on the layout described in \Cref{ch:keyboard}, and it is actually faster to type the Arabic text into the computer than it is to type in a close transliteration. 

Each \textit{kipande} of a poem in traditional metre should be placed on its own line, and each stanza should be separated by a blank line. The LibreOffice document should end with one blank line.

Particles such as  \AS{لَ زَ يَ نَ} (\textbf{na, ya, za, la}) are best written according to the manuscript rendering. In many instances the author or copyist may have attached them to the following word, or (in the case of non-connecting letters like \AS{زَ}) placed them very close to it. But where there is a larger space in the manuscript between the particle and the following word, it may be appropriate to write it separately from that word in the transcription. These decisions are subjective, and may produce some inconsistency in the transcription, but they will reflect the manuscript more faithfully.  Connection or disconnection of the particles (in line with standard Roman orthography) will be handled by annotating the entries in the database.

An input file for stanzas 138-147 of \AS{أُتٖينْزِ وَ ڤِيتَ ڤِكُؤُ} (\textbf{Utenzi wa Vita Vikuu}, \textit{The Ballad of the Great Battle}), a manuscript in the SOAS collection, is at \url{andika/convert/inputs/vita_vikuu/vita_vikuu.odt}.  An input file for \AS{كِسْوَاحِلِ} (\textbf{Kiswahili}), which forms the basis of the output in \Cref{appD}, is at \url{andika/convert/inputs/kiswahili/kiswahili.odt}.


\section{Import}
\label{s:import}

Once the manuscript is transcribed in the input file, it needs to be imported into the database.  To do this, open a terminal and either run:

\verb|convert/convert.sh|

choose your poem, and then select \textit{Insert into database} on the \textit{Output} screen.  Alternatively, run the following command (replacing \textit{vita_vikuu} with the name of the poem) :

\verb|php convert/convert.php convert/inputs/vita_vikuu/vita_vikuu.odt+Arabic+Poetry+db+kip-line|

A database table of the same name as the poem (in the above case, \textit{vita_vikuu}) will be created in the \textit{andika} database, and each \textit{kipande} of the poem will be imported into that table.  During this import, the Arabic text is transliterated into standard Roman orthography, and also into a close transliteration which more closely reflects the Arabic glyphs.

To view the database table, you can access it using phpPgAdmin: in a browser, enter \url{localhost/phppgadmin} and log in to the PostgreSQL server.  In the left panel, click the + beside \textit{andika}, and then \textbf{Tables}, and then click on \textbf{Browse} alongside the name of the poem (in this case, \textit{vita_vikuu}).  You should see the first part of the contents of the table: for each \textit{kipande} of the poem, there will be an index number (\textit{poemline_id}), the stanza number (\textit{stanza}), the location of the line in the stanza (\textit{loc}),\footnote{The default import numbers the stanzas from 1 onwards, and for simplicity, that is retained here, but it is possible to change the numbering to reflect the actual stanza numbers (138-147) by opening \url{convert/convert.php} and changing the line \texttt{\$stanza_no=0;} to \texttt{\$stanza_no=137;}.} the Arabic text (\textit{arabic}),\footnote{Some cells in the Arabic column may appear empty, but this seems to be a display issue -- if you press \textbf{Edit} you can see the text is there, and a \textit{select} query will also show it.} the close transcription (\textit{close}) and the standard transcription (\textit{standard}).

Note that each time the conversion is done, it will delete and then recreate this poem table.  If you want to preserve a specific conversion (perhaps for archival purposes), you need to save (dump) the database table before re-running the conversion.  You can do this in a terminal:

\verb|pg_dump -U dbuser --table=name_of_poem > name_of_poem.sql andika|

in this case, for me:

\verb|pg_dump -U kevin --table=vita_vikuu > vita_vikuu.sql andika|

This will create a file which can be loaded into a PostgreSQL database.  If you want to create a file which can be opened in a spreadsheet, run the following commands:

\verb|psql -U dbuser -d andika|\\
(replacing \textit{dbuser} with your database user)

\verb|\copy (select * from vita_vikuu) to 'vita_vikuu.csv' with delimiter ',' csv header|\\
(replacing \textit{vita_vikuu} with the name of the poem)

\verb|\q|

You may want to add dates to the backup filenames, and perhaps create a directory in which to store them.

You can do the same thing with phpPgAdmin.  On the \textit{Tables} screen of the \textit{andika} database, click on the name of the poem (in this case, \textit{vita_vikuu}).  Then click on the \textbf{Export} button at the top.  For a backup which can be loaded into PostgreSQL, tick \textbf{Structure and data}, select \textbf{SQL} from the \textit{Format} drop-down list, tick \textbf{Download}, and then click \textbf{Export}.  For a backup which can be opened in a spreadsheet, tick \textbf{Data only}, select \textbf{CSV} from the \textit{Format} drop-down list, tick \textbf{Download}, and then click \textbf{Export}.  In either case, select a location in which to save the file, and rename it as appropriate.


\section{Split lines into words}

The next step is to split each \textit{kipande} of the poem into words, which will allow each word to have individual annotations added to it.  To do this, open a terminal and run:

\verb|php db/import_words.php name_of_the_poem|

replacing \textit{name_of_the_poem} with whatever your poem name is.  In the example, you would run:

\verb|php db/import_words.php vita_vikuu|

Note also that the code relating to annotation and other editorial work in the database are located in the directory \textit{db}, and not in \textit{convert}.

The import command will create a new table, \textit{name_of_the_poem_words} (in this case, \textit{vita_vikuu_words}), which you can again inspect using phpPgAdmin.  Each word has an entry of its own, consisting of an index number (\textit{word_id}), the stanza (\textit{stanza}) and \textit{kipande} (\textit{loc}) it occurs in, the position it occupies in the \textit{kipande}, eg first word, second word or whatever (\textit{position}), the Arabic text, and the close and standard transliterations.  A copy of the close transliteration is also created (\textit{edclose}) --  this is the entry that will be edited by you to bring the close transcription more into line with standard Swahili.  For instance, going back to the \textbf{Mwana Kupona} example (\Cref{fig:mwanakupona}), the entry \textbf{nawulimiḡu} in \textit{close} would be edited to \textbf{na ulimwengu} in \textit{edclose}.  Other fields are also created to hold annotations for variants, general notes, the root of the word, and an English translation, as explained in \Cref{s:annotation}.

Note that each time \textit{import_words.php} is run (eg as you complete another batch of stanzas), it will delete and then recreate this words table.  Since you may well have devoted considerable time to editing the table and adding annotations, it is important not to lose these!  So \textit{import_words.php} does the following things:
\begin{enumerate}
\item Creates a backup of the \textit{words} table, which will include your annotations and edits.
\item Does the new import, as explained above.
\item Adds your annotations and edits from the backup into the new import.
\end{enumerate}

Note that \textit{import_words.php} should not be run twice in succession if you notice problems with the new \textit{words} table (eg missing annotations compared to your old \textit{words} table.).  Instead, you should make a manual copy of the \textit{backup} table using an SQL query like:  

\verb|create table mymanual_backup as select * from my_backup;|

Then copy fields manually from there to your \textit{words} table by using an SQL query like:

\verb|update my_words w set (variant, note) = (b.variant, b.note) from mymanual_backup b|
$\hookrightarrow$ \verb|where w.stanza=b.stanza and w.loc=b.loc and w.position=b.position;|

The above query, for instance, will copy the variant and note fields from your manual backup table to your \textit{words} table, and it is then safe to fix the other problems in your \textit{words} table, and run \textit{import_words.php} again.

If you do not correct data losses manually after a first problematic re-import, running \textit{import_words.php} again will lead to data loss -- the first run creates a \textit{backup} table with all your data, but a second run will delete that good \textit{backup} table and create a new one based on the now-faulty \textit{words} table, possibly replacing good entries with faulty ones. 


\section{Annotations}
\label{s:annotation}

It is possible to create as many fields in the \textit{words} table as are required to handle various annotation types, and the content of each of these fields can be selected and edited via code, which means that a versatile framework exists for any editorial apparatus.  The current selection of annotation types is described below, with examples.


\subsection{Specifying fonts in the annotations}

Changes from the default font can be marked in any of the annotation types, and it is recommended to to make such changes directly in the annotation, rather than try to edit the output file afterwards.  The alternate fonts use the font definitions as set up in the file \url{andika/convert/tex/fontdefs.tex}, and you can add more to that list using the patterns seen there (see also \Cref{s:changefont}).  To apply the font changes, the LaTeX command format is used: a backslash, then the abbreviation for the desired font, and then the text inside braces.  Likely font changes that might be applied are:
\begin{itemize}
\item Arabic script: \verb|\AS{text here}| -- Scheherazade.  To use Amiri, use \verb|\Am{text here}|
\item transliteration: \verb|\Tr{text here}| -- Linux Biolinum O in grey.
\item inserted letters: \verb|\In{text here}| -- Linux Biolinum O in blue.
\item standard Swahili: \verb|\Swa{text here}| -- Linux Biolinum O in blue.
\item English: \verb|\E{text here}| -- Liberation Serif in grey italics, smaller than the default.
\item English italics: \verb|\Eit{text here}| -- Liberation Serif in black italics.	
\item standout type in footnote: \verb|\FN{text here}| -- Liberation Serif in green italics. 
\end{itemize}


\subsection{Adding and editing the annotations}

SQL Workbench (\Cref{ss:workbench}) allows the annotations to be added directly to the table.  Open SQL Workbench and connect to the \textit{andika} database.  In the top panel, enter an SQL query to show the whole of the \textit{vita_vikuu_words} table in order of stanza and \textit{kipande}:

\verb|select * from vita_vikuu_words order by stanza, loc;|\\
(note that the semi-colon at the end is essential)

Move the cursor somewhere inside the query and press \textbf{Ctrl+Return}.  You can now edit individual cells -- remember to press the \textbf{Save} icon or select \textbf{Data \textrightarrow\ Save Changes to Database} to save any material you add.

Note that if SQL Workbench is showing data from an \textit{andika} database table, it places a lock on the table which prevents any other program accessing it for some operations (re-creating a table, renaming a field in the table, etc).  If you sometimes find that phpPgAdmin or \textbf{Andika!} commands seem to ``hang'', close SQL Workbench.  You can then reopen it after the other command has completed.


\subsection{\textit{edclose} field}

This field holds an edited version of the close transliteration.  The automatically-generated close transliteration tries to recreate the Arabic glyphs as faithfully as possible using Roman glyphs.  With some writers (eg Sheikh Yahya), this will result in text that is very close to standard Swahili in Roman script, because their Arabic script marks distinctions that reflect the phonology of Swahili.  Other writers (usually from earlier periods) have used the Arabic script as an approximation of the  phonology of Swahili (see \Cref{ch:fonts}) so the close transliteration will reflect standard Swahili in Roman script less well, and it is therefore necessary to adjust the transliteration to do this.

Apart from obvious spelling corrections (eg changing \textit{ḡ} to \textit{ng}, or \textit{i} to \textit{e}), the \textit{edclose} field currently handles two other types of edits: segmentation of the transliteration to reflect standard Swahili word boundaries, and insertion of additional letters.
\begin{itemize}
\item To separate particles like \AS{لَ زَ يَ نَ} (\textbf{na, ya, za, la}), etc that have been written connected in the Arabic text, put a space after them.  For example, in 1c of \textit{vita_vikuu}, edit the \textit{edclose} field to read \textbf{na khubuzi} (\textit{and bread}) instead of \AS{نَخُبُوزِ} (\textbf{nakhubuzi}).
\item To connect elements that have been written separately in the Arabic text, enter $\sim$ against the element to be joined, and move the element to the proper cell.  For instance, in 1g of \textit{kiswahili}, standard Swahili requires \textbf{nimewatendani} (\textit{what have I done to you?}) instead of \AS{نِ مٖوَتٖنْدَانِ} (\textbf{ni mewatendani}).  So against \textbf{ni} we put a $\sim$ in the \textit{edclose} field, and edit the \textit{edclose} field of \textbf{mewatendani} to read \textbf{nimewatendani}.
\item Epenthetic vowels to support proper scansion can be added using \verb|\In{text here}|.  For instance, in 1c of \textit{vita_vikuu} an epenthetic vowel can be added to the transliteration of \AS{مِلْحِ} (\textbf{milḥi}, \textit{and salt}) to give \textbf{mil\In{i}ḥi} by editing the \textit{edclose} field to read \verb|na mil\In{i}ḥi|.
\end{itemize}


\subsection{\textit{standard} field}

This field holds an automatically-generated version of the text in standard Swahili orthography.  With most texts, you would give the edited version of the close transliteration from the \textit{edclose} field, but with some texts there may already be a standard Swahili transliteration that you wish to re-use.  In this case, you would edit the standard entries to reflect that existing transliteration, and they will be saved and reapplied in the same way as the other annotation fields every time you re-import the document.  In \textit{kiswahili}, the entries in \textit{standard} are based on the transliteration in \citet{Abdulkadir2013}.


\subsection{\textit{variant} field}

This field allows the recording of variant readings (\textit{variae lectiones}) of the word in different manuscript versions of the same poem.  For instance, 6a of \textit{vita_vikuu} reads: \AS{نَسٖيْفُ نْجٖيْمَ أَسِيْسِ} (\textbf{nasēfu njēma ası̄si}, \textit{and a good, stout sword}).  If you have another manuscript B where this \textit{kipande} reads: \AS{نَسٖيْفُ نْزُوْرِ أَسِيْسِ} (\textbf{nasēfu nzūri ası̄si}), in the variant column against \textbf{njēma} we put (for instance):

\verb|B: \AS{|\texttt{\AS{نْزُوْرِ}}\verb|}, nzūri|

which will be converted to a footnote reading: B: \AS{نْزُوْرِ }, nzūri.


\subsection{\textit{note} field}

This field holds notes on the meaning or reference of the word, or any other material which may help to elucidate its meaning or usage.  There is no practical limit on the length of these notes.\footnote{The longest possible content that can be stored in any one field is about 1 GB.}  \Cref{appD} demonstrates how \textbf{Andika!} caters easily for a significant\footnote{This is the first note.} \footnote{And this is the second note.}  number of notes -- for more information the relevant database table, \textit{kiswahili_words}, can be inspected.


\subsection{\textit{root} field}

Recording the root of the word is useful for building concordances, frequency lists, etc.  Instead of having to search for different forms of the word, it is possible to search on the root and have all forms presented.

For Bantu words, use the stem minus any elements such as class prefixes, verbal \textit{-a} or verbal extensions.  Thus \textbf{fik} would find \textbf{akafika} (\textit{he arrived}), \textit{watafikia} (\textit{they will arrive at}), \textbf{mfiko} (\textit{arrival}), and so on.

For Arabic words, use the triconsonantal root (\AS{جذر}).\footnote{\url{en.wikipedia.org/wiki/Semitic_root}}  Thus \textbf{klm} would find \textbf{katakalama} (\textit{he spoke}), \textbf{kalima} (\textit{word}), and so on.

Ideally, the roots would be filled in automatically during import by looking up the word against a digital Swahili dictionary, such as the one used for my Swahili verb segmenter.\footnote{\url{kevindonnelly.org.uk/swahili/segmenter}.  The dictionary here is a heavily-customised version of Beata Wójtowicz's English-Swahili dictionary at \url{freedict.org}.}  That dictionary is not yet extensive enough for that purpose.


\subsection{\textit{english} field}

This field holds an English translation of the \textit{kipande}, placed against the first word of each \textit{kipande}.  See the \textit{kiswahili_words} table for examples.


\subsection{Inserting citations in the annotations}

All the annotation fields can include citations, but they are perhaps most likely in the \textit{note} field.  The citations use a LaTeX package called \textit{biblatex}, which has already been installed (\Cref{s:latex}).\footnote{The older \textit{bibtex} package does not support citations in footnotes.}  This draws on a list of the citations you wish to use, which need to be in BibTeX format -- see \url{andika/docs/andika.bib} for an example of a short bibliography file.  It is possible to write this file using just a text editor, but it is easier to use a frontend such as KBibTex, which has already been installed.\footnote{An alternative is JabRef -- \url{jabref.org}.}  With a frontend, you simply type the bibliographic details of the citation into a series of dialogue boxes.  Each citation is referred to by a ``key'' -- I use the name of the first author and the year, so the key for Sacleux' \textit{Dictionnaire} would be \textit{Sacleux1939}, but you can use anything you like.

Using this as an example, we can then refer to a citation in the note by using \verb|\textcite{Sacleux1939}| where the work is a subject:
\begin{quotation}
Sacleux (1939) was a major achievement in Swahili lexicography.
\end{quotation}
and \verb|\parencite{Sacleux1939}| where the work is referred to parenthetically:
\begin{quotation}
This word is found mainly in northern dialects (Sacleux 1939).
\end{quotation}

The \textit{author-year} citation style is default, but it can be changed.  However, since citation management and style is a wide-ranging topic not directly relevant to \textbf{Andika!}, this aspect is not dealt with here.

The bibliography file has to be called \textit{andika.bib}, and has to be in the \url{andika/docs} directory.  You can either add your citation details to that file, or delete it and start your own \textit{andika.bib}.  If you need to use a bibliography file of a different name in a different location, you can delete the existing \textit{andika.bib} and set up a symbolic link to your own file.  For instance, if your bibliography file were called \textit{thesis.bib}, and it was located in \url{/home/USER/thesis}, you would run the following commands in a terminal open at \textit{andika}:

\verb|cd docs|\\
(change into the \url{andika/docs} directory)

\verb|ln -s ~/thesis/thesis.bib andika.bib|\\
(create an \textit{andika.bib} link (alias or shortcut) to your bibliography file)

\verb| cd ..|\\
(change back up to the \url{andika} directory)

It is worth noting that in cases where citations are not printed properly, it is virtually always the case that there is an error in the bibliography file, or that the key being used is incorrect.\footnote{I recently spent almost an hour trying to work out why the entry for \citet{Abdulkadir2013}, key \texttt{Abdulkadir2013}, was not appearing correctly - there were definitely no mistakes in the bibliography entry.  Yet LaTeX reported it could not find that entry.  I tried various increasingly complicated fixes, until eventually I noticed that I had typed the key as \texttt{Adbulkadir2013} when citing the work.  When a problem like this arises, we tend to think that the most likely cause is a fault in the software, but it is far more likely to be a fault in the wetware connecting the chair to the computer.}


\section{Output}
\label{s:output}

Once the database table holding the words of the poem has been edited to include all the annotations required, it can be output in a variety of formats.  The most important is likely to be \textit{pdf} format, since the LaTeX typesetting system produces a very attractive, beautifully laid-out text equally suitable for traditional printing or digital distribution.  The \textit{pdf} output format is two \textit{vipande} to the line, with a space between them (compare the conversion options in \Cref{ss:pacint}).

To print your annotated poem to a pdf file, run:

\verb|php db/output_pdf.php vita_vikuu|

replacing \textit{vita_vikuu} with the name of your document.


\subsection{Changing the transliteration source}

By default, the \textit{pdf} output uses the entries in \textit{edclose}, but you can override this to use the entries in \textit{standard} instead by including the word \textbf{standard} at the end of the command.  For instance, to print out \textit{kiswahili} with the \textit{standard} entries as the main transliteration instead of the \textit{edclose} entries, use:

\verb|php db/output_pdf.php kiswahili standard|


\subsection{Changing the font size}
\label{s:outfont}

The default font size for the output is 12pt, which is good for computer use.  However, if you need a slightly smaller font for print use, open \textit{andika/db/tex/tex_header.tex} and change \textit{12pt} in the first line to \textit{10pt}.  Save the file, and then run \textit{output_pdf.php} again.

\subsection{Location of annotations}
\label{s:outloc}

By default, all annotations are output as footnotes -- since they refer to individual words in the text, having them on the same page is easier to read.  But if you need endnotes instead, this can be achieved by minimal editing of two files:
\begin{enumerate}
\item Find the \textit{Endnotes} section near the bottom of \textit{andika/db/tex/tex_header.tex}.  Uncomment the two lines there by removing the initial \textit{\%}, and save the file.
\item Find the \textit{Endnotes} section towards the bottom of \textit{andika/db/output_pdf.php}.  Uncomment the three lines there by removing the initial \textit{//}, and save the file.
\end{enumerate}

After this, running:

\verb|php output_pdf.php name_of_poem|

will produce a \textit{pdf} with endnotes.  If you want to go back to footnotes, simply comment the five lines again.

To get an impression of the difference in output, compare \Cref{appD} and \Cref{appF}.

% \subsection{Output to html}
% 
% Output to html (the most likely format for web publication) is shown in this html file. In this version, the transcription colour has been changed to blue to fit in with the site livery, and lemma adjustments are in purple. Notes are displayed in a pop-up when the mouse hovers over the note number (to dismiss the note, click on it and move the mouse away). An example of how variant readings might be handled is given in stanza 142 - hovering over the word with a grey background pops up the variant. (Note that this variant reading is an invention for the purposes of this example.)
% 
% A paging mechanism could be added to allow easy navigation through a long text.
% 
% Again, it would be relatively simple to adjust the layout and contents of the html output to meet other requirements, and indeed, a variant is shown below.
% 
% 
% \subsection{Output to odt}
% 
% Output to odt (LibreOffice format) is only indirectly supported - it is handled by generating a standalone html file. In this, notes to the text are printed at the end of the entire text, with the endnote number marked in red in the text. The endnote number in the text is a link to the endnote, and the number beside the endnote itself is a link back to the relevant word in the text. To create an odt document from this, open the file in the default KDE web-browser Rekonq (see below), and copy the text using Ctrl+A and Crtl+C. Then open a new LibreOffice document and use Ctrl+V to paste the text into that. When saving the file in LibreOffice, select odt as the save format.
% 
% Note that, of the browsers available on Linux, Rekonq provides the best results. It retains all the formatting (fonts, colours, etc), and converts the endnote links in the resulting odt file so that they can be accessed by pressing Ctrl while clicking. Chromium retains the formatting, but does not convert the links, so that they still point to the original html file, and clicking them opens that file in a browser. Chromium also converts some spaces to non-breaking spaces, which are marked with a grey background - to get rid of this, select Tools → Options → LibreOffice Writer → Formatting Aids, untick Non-breaking spaces, and press OK. Firefox is worst of all, losing all the formatting, and leaving the links unconverted. Note that even with Rekonq and Chromium, though, you will need to select all the endnotes in LibreOffice and press the LTR button in order to align them to the left.
% 
% You can also open the html file in LibreOffice itself, but in this case you will not be able to save it as an odt file. 
