\chapter{Writing contemporary Swahili in Arabic script}
\label{ch:spelling}

\section{Introduction}

The spelling conventions suggested here for writing contemporary Swahili in Arabic script are based on those developed by Sheikh Yahya Ali Omar, as evidenced in his own manuscripts and in \citet{Omar1997}.  However, I am wholly responsible for the conventions set out here, and for any unwitting misinterpretation!  In particular, the issue of vowel sequences\footnote{I have tried to build on the discussion in \citet{Omar1997}: \textit{Appendix B: The Hamza in Swahili Arabic script}.} (\Cref{s:vseq} below) is a complex one, and may need revision based on input from first-language speakers who are literate in Swahili in Arabic script. I would be happy to hear from anyone who has any comments on the conventions.

\section{General principles}

Word segmentation is as for standard Swahili in Roman script. This means that items such as \AS{لَ زَ يَ نَ} \textbf{na, ya, za, la} are written separately from the following word, even though in older manuscripts they may be written attached to that word.

All short vowels are marked. Although short vowels are usually omitted in Arabic, this is inadvisable in Swahili because of the different structure of the language, and also because Swahili has five vowels instead of three.

The penultimate syllable of a word has its stress marked by writing it with a long vowel. \AS{ا} is used for \textbf{a}, \AS{ي} for \textbf{e} and \textbf{i}, and \AS{و} for \textbf{o} and \textbf{u}.\footnote{The short vowels \textbf{a, i, u} may be omitted when they occur before a long vowel, eg \AS{ساسَ} instead of \AS{سَاسَ} (\textbf{sasa}, \textit{now}), but this is not recommended.} This also helps to delimit individual words in the Arabic script.

Initial vowels use the vowel-carriers \AS{أ} (\textbf{AltGr+A}, for \textbf{a, o, u}) or \AS{إ} (\textbf{AltGr+\textbackslash}, for \textbf{e, i}), eg \AS{أَنَسٖيمَ} (\textbf{anasema}, \textit{he is speaking}), \AS{أُڠَالِ} (\textbf{ugali}, \textit{porridge}), \AS{إِذِينِ} (\textbf{idhini}, \textit{permission}).\footnote{\citet[p69]{Omar1997} recommends omission of the \textit{hamza}, presumably in order to limit the number of diacritics in the text, but the current convention in \textbf{Andika!} is to write it.} The order of typing is: vowel carrier, then short vowel, then long vowel (if applicable).

Arabic sounds in loanwords should ideally use the original Arabic glyph, but they can also be written as an Arabic transliteration of the Roman letter, eg \AS{ذ} instead of \AS{ض} or \AS{ظ}.\footnote{Note that the Roman to Arabic converter will always do this, since standard Swahili in Roman script does not preserve these distinctions.}


\section{Representation of consonants}

The representation of Swahili vowels in Arabic script is set out in \Cref{tab:consonants}.

\begin{longtable}[c]{p{4cm}rp{3cm}rp{5cm}}  % [c] means the table will be centered.
\textbf{Roman} & \textbf{Arabic} & \textbf{Keystrokes} & & \textbf{Example} \\
\noalign{\bigskip}\hline\noalign{\bigskip}

b & \AS{ب} & b & \AS{كِبُورِ} & ki\textbf{b}uri (\textit{arrogance}) \\
\noalign{\medskip}

ch & \AS{چ} & c & \AS{چُونڠوَ} & \textbf{ch}ungwa (\textit{large orange}) \\
\noalign{\medskip}
ch (aspirated, Mombasa) & \AS{چه} & c, h & \AS{چهُونڠوَ} & \textbf{ch'}ungwa (\textit{medium-sized orange}) \\
\noalign{\medskip}

d & \AS{د} & d & \AS{كُدَنڠَانيَ} & ku\textbf{d}anganya (\textit{to deceive}) \\
\noalign{\medskip}
d - alveolar d (Mombasa) & \AS{ڈ} & AltGr+Shift+d & \AS{ٹُونڈُ} & tun\textbf{d}u (\textit{chicken coop}) \\
\noalign{\medskip}
dh & \AS{ذ} & Shift+d & \AS{ذَهَابُ} & \textbf{dh}ahabu (\textit{gold}) \\
\noalign{\medskip}
dh (pharyngeal) & \AS{ض} & AltGr+d & \AS{ضِيكِ} & \textbf{dh}iki (\textit{distress}) \\
\noalign{\medskip}
dh (pharyngeal) & \AS{ظ} & AltGr+z & \AS{أَظُهُورِ} & a\textbf{dh}uhuri (\textit{noon}) \\
\noalign{\medskip}

f & \AS{ف} & f & \AS{فِيڠٗ} & \textbf{f}igo (\textit{kidneys}) \\
\noalign{\medskip}

g & \AS{ڠ} & g & \AS{ڠُنِئَ} & \textbf{g}unia (\textit{sack}) \\
\noalign{\medskip}
gh & \AS{غ} & h & \AS{غَضَابُ} & \textbf{gh}adhabu (\textit{anger}) \\
\noalign{\medskip}

h & \AS{ه} & h & \AS{هَاكٗ} & \textbf{h}ako (\textit{he is not here}) \\
\noalign{\medskip}
h (pharyngeal) & \AS{ح} & Shift+h & \AS{حَسَن} & \textbf{H}asan (\textit{Hasan} [name]) \\
\noalign{\medskip}
[k]h & \AS{خ} & x & \AS{خَبَارِ} & \textbf{[k]h}abari (\textit{news}) \\
\noalign{\medskip}

j & \AS{ج} & j & \AS{جَانَ} & \textbf{j}ana (\textit{yesterday}) \\
\noalign{\medskip}

k & \AS{ك} & k & \AS{كُوكُ} & \textbf{k}u\textbf{k}u (\textit{large hen}) \\
\noalign{\medskip}
k (aspirated, Mombasa) & \AS{كه} & k, h & \AS{كهُوكُ} & \textbf{k'}uku (\textit{medium-sized hen}) \\
\noalign{\medskip}

l & \AS{ل} & l & \AS{كُلِيمَ} & ku\textbf{l}ima (\textit{to dig}) \\
\noalign{\medskip}

m & \AS{م} & m & \AS{مِيمِ} & \textbf{m}imi (\textit{I}) \\
\noalign{\medskip}
% m , Shift+ . (full stop) & \AS{مْ} & m - syllabic (eg Classes 1 and 3) & \AS{مْپٖينزِ} & \textbf{m}penzi (\textit{beloved}) \\

n & \AS{ن} & n & \AS{نَانِ} & \textbf{n}a\textbf{n}i (\textit{who?}) \\
\noalign{\medskip}
ng' & \AS{نݝ} & n, Shift+n & \AS{نݝٗومبٖ} & \textbf{ng'}ombe (\textit{cattle}) \\
\noalign{\medskip}

p & \AS{پ} & p & \AS{كُپَاكَ} & ku\textbf{p}aka (\textit{to paint}) \\
\noalign{\medskip}

q & \AS{ق} & q & \AS{وَقْفُ} & wa\textbf{q}fu (\textit{consecrated}) \\
\noalign{\medskip}

r & \AS{ر} & r & \AS{كُرُودِ} & ku\textbf{r}udi (\textit{to come back}) \\
\noalign{\medskip}

s & \AS{س} & s & \AS{كُسِمَامَ} & ku\textbf{s}imama (\textit{to stand}) \\
\noalign{\medskip}
s (pharyngeal) & \AS{ص} & AltGr+s & \AS{صَحِيبُ} & \textbf{s}ahibu (\textit{friend}) \\
\noalign{\medskip}
sh & \AS{ش} & Shift+s & \AS{كُشِيكَ} & ku\textbf{sh}ika (\textit{to hold}) \\
\noalign{\medskip}

t & \AS{ت} & t & \AS{فِتِينَ} & fi\textbf{t}ina (\textit{intrigue}) \\
\noalign{\medskip}
t (aspirated dental, Mombasa) & \AS{ته} & t, h & \AS{تهُوپَ} & \textbf{t'}upa (\textit{bottle}) \\
\noalign{\medskip}
t (alveolar, Mombasa) & \AS{ٹ} & AltGr+Shift+t & \AS{ٹُونڈُ} & \textbf{t}undu (\textit{chicken coop}) \\
\noalign{\medskip}
t (pharyngeal) & \AS{ط} & t & \AS{كُطَهِرِيشَ} & ku\textbf{t}ahirisha (\textit{to purify}) \\
\noalign{\medskip}
th & \AS{ث} & Shift+t & \AS{ثَمَنِينِ} & \textbf{th}amanini (\textit{eighty}) \\
\noalign{\medskip}

v & \AS{ڤ} & v & \AS{كُڤِيمبَ} & ku\textbf{v}imba (\textit{to swell}) \\
\noalign{\medskip}

z & \AS{ز} & z & \AS{كُزِيمَ} & ku\textbf{z}ima (\textit{to extinguish}) \\
\noalign{\medskip}
zh (Northern) & \AS{ژ} & Shift+z & \AS{ژِينَ} & \textbf{zh}ina (\textit{name}) \\
\noalign{\medskip}

w & \AS{و} & w & \AS{كُوَ} & ku\textbf{w}a (\textit{to be}) \\
\noalign{\medskip}
w (labio-dental) & \AS{ۏ} & AltGr+Shift+w & \AS{ۏِينٗ} & \textbf{w}ino (\textit{ink}) \\
\noalign{\medskip}

y & \AS{ي} & y & \AS{يَاكٗ} & \textbf{y}ako (\textit{your}) \\
\noalign{\medskip}

ʕ (pharyngeal) & \AS{ع} & ` (single quote) & \AS{مَعَانَ} & ma\textbf{'}ana (\textit{meaning}) \\
\noalign{\medskip}

\textit{hamza} (vowel-carrier) & \AS{ء} & AltGr+Shift+h & \AS{تَاءٗ} & ta\textbf{o} (\textit{arch}) \\
\noalign{\medskip}

\textit{hamza} (marks long vowels used as vowel-carriers) & \AS{ٔ} & Shift+ , (comma) & \AS{كُپِكِئَ} & kupik\textbf{i}a (\textit{to cook for}) \\
\noalign{\medskip}

\textit{sakani} (marks a consonant without a following vowel) & \AS{ْ} & Shift+ . (full stop) & \AS{أَسْلَارِ} & a\textbf{s}kari (\textit{soldier}) \\
\noalign{\medskip}

\textit{shada} (marks a doubled consonant in Arabic words) & \AS{ّ} & Shift+ ` (single quote) & \AS{وَالنَّهَارِ} & wa-\textbf{nn}ahari (\textit{and day}) \\
\noalign{\bigskip}\hline\noalign{\bigskip}

\multicolumn{5}{p{12cm}}{\textbf{NOTE}: In the \textbf{Keystrokes} column, the comma stands for \textit{followed by}.} \\
% can't use quotes in the multicolumn field.
\noalign{\bigskip}

\caption{Representation of consonants}\\
\label{tab:consonants}
\end{longtable}



\section{Representation of vowels}

The representation of Swahili vowels in Arabic script is set out in \Cref{tab:vowels}.

\begin{longtable}[c]{p{1.5cm}rp{2cm}rp{5cm}}  % [c] means the table will be centered.
\textbf{Roman} & \textbf{Arabic} & \textbf{Keystrokes} & & \textbf{Example} \\
\noalign{\bigskip}\hline\noalign{\bigskip}

a-...  & \AS{أَ} & AltGr+a, a & \AS{أَسٗومَ} & \textbf{a}soma (\textit{he reads}) \\
\noalign{\medskip}
...-a-... & \AS{َ} &a & \AS{بَهَرِينِ} & b\textbf{a}h\textbf{a}rini (\textit{in the sea}) \\
\noalign{\medskip}
...-a\CV{CV} & \AS{َ  ا} & a, Shift+a & \AS{سَاسَ} & s\textbf{a}sa (\textit{now}) \\
\noalign{\medskip}
...-a\CV{V} & \AS{َ   ا ء} & a, Shift+a,  & \AS{مَفَاءَ} & maf\textbf{a}a (\textit{usefulness}) \\
&& AltGr+Shift+h & \AS{تَاءِ} & t\textbf{a}i \textit{(vulture)} \\
&&& \AS{بَاءٗ} & b\textbf{a}o \textit{(plank)} \\
\noalign{\bigskip}\hline\noalign{\bigskip}

e-...  & \AS{إٖ} & AltGr+\textbackslash, e & \AS{إٖندٖلٖئَ} & \textbf{e}ndelea (\textit{go on!}) \\
\noalign{\medskip}
...-e-... & \AS{ٖ} & e & \AS{كٖلٖيلٖ} & k\textbf{e}lel\textbf{e} (\textit{shout}) \\
\noalign{\medskip}
...-e\CV{CV} & \AS{ٖ ي} & e, Shift+e & \AS{نجٖيمَ} & nj\textbf{e}ma (\textit{good}) \\
\noalign{\medskip}
...-e\CV{V} & \AS{ٖ ئ} & e, AltGr+e & \AS{كُپٖئَ} & kup\textbf{e}a (\textit{to sweep}) \\
&&& \AS{كُپٗكٖئَ} & kupok\textbf{e}a \textit{(plank)} \\
\noalign{\bigskip}\hline\noalign{\bigskip}

i-...  & \AS{إِ} & AltGr+\textbackslash, i & \AS{إِسِپٗكُوَ} & \textbf{i}sipokuwa (\textit{unless}) \\
\noalign{\medskip}
...-i-... & \AS{ِ} & i & \AS{كِتَابُ} & k\textbf{i}tabu (\textit{book}) \\
\noalign{\medskip}
...-i\CV{CV} & \AS{ِ  ي} & i, Shift+i & \AS{مَشِيزِ} & mash\textbf{i}zi (\textit{soot}) \\
\noalign{\medskip}
...-i\CV{V} & \AS{ِ  ئ} & i, AltGr+i & \AS{كُتِئَ} & kut\textbf{i}a (\textit{to place}) \\
\noalign{\bigskip}\hline\noalign{\bigskip}

o-...  & \AS{أٗ} & AltGr+a, o & \AS{أٗكتٗوبَ} & \textbf{O}ktoba (\textit{Oktober}) \\
\noalign{\medskip}
...-o-... & \AS{ٗ} & o & \AS{كِلِيمٗ} & kilim\textbf{o} (\textit{cultivation}) \\
\noalign{\medskip}
...-o\CV{CV} & \AS{ٗ  و} & o, Shift+o & \AS{مْكٗونڠَ} & mk\textbf{o}nga (\textit{elephant's trunk}) \\
\noalign{\medskip}
...-o\CV{V} & \AS{ٗ  ؤ} & o, AltGr+o & \AS{كُپٗؤَ} & kup\textbf{o}a (\textit{to cool}) \\
\noalign{\bigskip}\hline\noalign{\bigskip}

u-...  & \AS{أُ} & AltGr+a, u & \AS{أُلِيمِ} & \textbf{u}limi (\textit{tongue}) \\
\noalign{\medskip}
...-u-... & \AS{ُ} & u & \AS{كُشُكُورُ} & k\textbf{u}sh\textbf{u}kur\textbf{u} (\textit{ to give thanks}) \\
\noalign{\medskip}
...-u\CV{CV} & \AS{ُ  و} & u, Shift+u & \AS{كُومِ} & k\textbf{u}mi (\textit{ten}) \\
\noalign{\medskip}
...-u\CV{V} & \AS{ُ  ؤ} & u, AltGr+u & \AS{كُسُڠُؤَ} & kusug\textbf{u}a (\textit{to rub}) \\
\noalign{\bigskip}\hline\noalign{\bigskip}

\multicolumn{5}{p{12cm}}{\textbf{NOTE}: In the \textbf{Roman} column, \CV{C} stands for \textit{consonant} or \textit{consonant cluster} and \CV{V} for \textit{vowel}, and the entries refer respectively to (1) non-initial, (2) non-initial and non-penultimate, (3) penultimate followed by a consonant, (4) penultimate followed by a vowel.  For a discussion of vowel-sequences, see \Cref{s:vseq}.  In the \textbf{Keystrokes} column, the comma stands for \textit{followed by}.} \\
\noalign{\bigskip}

\caption{Representation of single vowels}
\label{tab:vowels}
\end{longtable}

\section{Vowel sequences}
\label{s:vseq}

Vowel sequences have matching vowel-carriers inserted between them, as set out in \Cref{tab:carriers}.

\begin{longtable}[c]{p{2cm}p{2cm}rp{4.5cm}}  % [c] means the table will be centered.
\textbf{Vowel} & \textbf{Carrier} & \textbf{Arabic} & \textbf{Keystrokes} \\
\noalign{\smallskip}\hline\noalign{\smallskip}
e, i & \textit{yeh+hamza} & \AS{ئ} & AltGr+Shift+I or E or Y \\
\noalign{\medskip}
o, u & \textit{waw+hamza} & \AS{ؤ} & AltGr+Shift+O or U or W \\
\noalign{\medskip}
a & \textit{alef+hamza} & \AS{ا ء} or \AS{أ}& Shift+A, AltGr+Shift+H\\
\noalign{\bigskip}
\caption{Vowel-carriers}
\label{tab:carriers}
\end{longtable}

\subsection{Stressed+unstressed vowel sequences}

When the vowel is first in the vowel sequence and is also stressed (which will only happen when it is in penultimate position in the word), a vowel-carrier is inserted after it.  Where \textbf{a} is concerned, the \textit{hamza} on the carrier is written as a full letter rather than a diacritic.

\hangindent=3cm  % controls the amount of indentation from left (positive value) or right (negative value).
\hangafter=0  % controls the number of full-width lines before/after changing the indent (\hangindent) - a positive number produces full-width lines at the beginning, whereas a negative number produces them at the end. 0 means we want them all indented, including the first line.
\textbf{kupea} (\textit{to sweep}) \textrightarrow\ kupe\SPSB{\AS{ء}}{y}a \textrightarrow\ \AS{كُپٖئَ} \\
\textbf{kupokea} (\textit{to receive}) \textrightarrow\ kupoke\SPSB{\AS{ء}}{y}a \textrightarrow\ \AS{كُپٗكٖئَ} \\
\textbf{kutia} (\textit{to place}) \textrightarrow\ kuti\SPSB{\AS{ء}}{y}a \textrightarrow\ \AS{كُتِئَ} \\
\textbf{kupoa} (\textit{to cool}) \textrightarrow\ kupo\SPSB{\AS{ء}}{w}a \textrightarrow\ \AS{كُپٗؤَ} \\
\textbf{kusugua} (\textit{to rub}) \textrightarrow\ kusugu\SPSB{\AS{ء}}{w}a \textrightarrow\ \AS{كُسُڠُؤَ} \\
\textbf{kutoa} (\textit{to produce}) \textrightarrow\ kuto\SPSB{\AS{ء}}{w}a \textrightarrow\ \AS{كُتٗؤَ} \\
\textbf{mafaa} (\textit{usefulness}) \textrightarrow\ mafa\SB{a\AS{ء}}a \textrightarrow\ \AS{مَفَاءَ} \\
\textbf{tai} (\textit{vulture}) \textrightarrow\ ta\SB{a\AS{ء}}i \textrightarrow\ \AS{تَاءِ} \\
\textbf{bao} (\textit{plank}) \textrightarrow\ ba\SB{a\AS{ء}}o \textrightarrow\ \AS{بَاءٗ}

Since the vowel-carrier accompanies the stress, there is no need to add another long vowel to mark the stress. Thus \AS{كُپٖئَ} (\textbf{kupea}, \textit{to sweep}), and not \AS{كُپٖيئَ}, and \AS{كُتٗؤَ} (\textbf{kutoa}, \textit{to produce}), and not \AS{كُتٗوؤَ}.

\subsection{Unstressed+stressed vowel sequences}

When the vowel \textbf{e, i, o, u} is second in the vowel sequence and is also stressed (i.e. again appearing in penultimate position), it has a matching vowel-carrier as in \Cref{tab:carriers} inserted before it.  Since the vowel-carrier comes before the stress, the stressed vowel is marked as normal with a long vowel.

\hangindent=3cm
\hangafter=0
\textbf{shairi} (\textit{poetry}) \textrightarrow\ sha\SPSB{\AS{ء}}{y}iri \textrightarrow\ \AS{شَئِيرِ} \\
\textbf{kiini} (\textit{pith}) \textrightarrow\ ki\SPSB{\AS{ء}}{y}ini \textrightarrow\ \AS{كِئِينِ} \\
\textbf{kuita} (\textit{to call}) \textrightarrow\ ku\SPSB{\AS{ء}}{y}ita \textrightarrow\ \AS{كُئِيتَ} \\
\textbf{shauri} (\textit{advice}) \textrightarrow\ sha\SPSB{\AS{ء}}{w}uri \textrightarrow\ \AS{شَؤُورِ} \\
\textbf{meupe} (\textit{white} [class 6]) \textrightarrow\ me\SPSB{\AS{ء}}{w}upe \textrightarrow\ \AS{مٖؤُوپٖ} \\
\textbf{kuona} (\textit{to see}) \textrightarrow\ ku\SPSB{\AS{ء}}{w}ona \textrightarrow\ \AS{كُؤٗونَ}

However, where the second (stressed) vowel of the sequence is \textbf{a}, the vowel-carrier matches the preceding vowel unless that preceding vowel is itself \textbf{a}, in which case the \textit{hamza} on the carrier is written as a diacritic rather than as a full letter.

\hangindent=3cm
\hangafter=0
\textbf{viazi} (\textit{potatoes}) \textrightarrow\ vi\SPSB{\AS{ء}}{y}azi \textrightarrow\ \AS{ڤِئَازِ} \\
\textbf{akaacha} (\textit{then he left behind}) \textrightarrow\ aka\SPSB{\AS{ء}}{a}acha \textrightarrow\ \AS{أَكَأَاچَ}

\subsection{Unstressed vowel sequences}

In vowel sequences where there is no stress (i.e. none of the vowels in the sequence appear in penultimate position), the vowel-carrier matches the first vowel.  Again, in the case of \textbf{a}, the \textit{hamza} on the carrier is written as a diacritic rather than as a full letter.

\hangindent=3cm
\hangafter=0
\textbf{tuondoke} (\textit{let us leave}) \textrightarrow\ tu\SPSB{\AS{ء}}{w}ondoke \textrightarrow\ \AS{تُؤٗندٗوكٖ} \\
\textbf{kuandika} (\textit{to write}) \textrightarrow\ ku\SPSB{\AS{ء}}{w}andika \textrightarrow\ \AS{كُؤَندِيكَ} \\
\textbf{maandishi} (\textit{manuscripts}) \textrightarrow\ ma\SPSB{\AS{ء}}{a}andishi \textrightarrow\ \AS{مَأَندِيشِ} \\

\subsection{Longer vowel sequences}

Longer sequences are handled in line with the principles above.

\hangindent=3cm
\hangafter=0
\textbf{kuua} \textrightarrow\ ku\SPSB{\AS{ء}}{w}u\SPSB{\AS{ء}}{w}a \textrightarrow\ \AS{كُؤُؤَ}



\section{Comparing conventions}
\label{s:comparison}

\Cref{tab:comp} summarises the differences between the writing systems used in Sheikh Yahya's manuscripts, \citet{Omar1997}, and \textbf{Andika!}

\begin{longtable}[c]{lccc}
\textbf{Feature} & \textbf{Manuscripts} & \textbf{Article} & \textbf{\textit{Andika!}} \\
\hline\noalign{\medskip}
\textit{Sakani} is marked on long vowels & ✓ & $\times$ & $\times$ \\
All short vowels are marked & ✓ & $\times$ & ✓ \\
\textit{Sakani} on consonants denotes syllabicity only & $\times$ & ✓ & $\times$ \\
Distinction between syllabicity and prenasalisation & ✓ & ✓ & $\times$ \\
\label{tab:comp}
\end{longtable}

\subsection{Sakani on long vowels}

In Sheikh Yahya's manuscripts, \AS{ي و} carry a \textit{sakani} when used to mark length/stress in the penultimate syllable, eg \AS{مَزِيْوَ} (\textbf{maziwa}, \textit{milk}). However, in \citet{Omar1997}, \textit{sakani} is not used here (eg \AS{مَزيوَ}). The suggested spelling in \textbf{Andika!} reflects this (though users can of course mark \textit{sakani} if they wish).

\subsection{Marking short vowels}

In Sheikh Yahya's manuscripts, all short vowels are marked, and \textbf{Andika!} follows this.  However, \citet{Omar1997} proposed that marking these is unnecessary in certain situations:
\begin{itemize}
\item If the short (unstressed, non-penultimate) vowel they represent is identical to a preceding short vowel. For example, in \AS{ثَمنين}  (\textbf{thamanini}, \textit{eighty}) the second \textbf{a} is omitted because it is preceded by an \textbf{a} (\textit{fataha}).

\item If the short vowel they represent is identical to a preceding or following stressed (penultimate) vowel represented by \AS{ي و ا}. For example, in \AS{ثَمنين} (\textbf{thamanini}, \textit{eighty}) the last \textbf{i} (\textit{kasiri}) is omitted because it is preceded by \AS{ي}, and in \AS{ذهابُ} (\textbf{dhahabu}, \textit{gold}) the first \textbf{a} (\textit{fataha}) is omitted because it is followed by \AS{ا}.

\item Where all the vowels in a word are identical, except for stress. For example: \AS{تپكاز} (\textbf{tapakaza}, \textit{scatter}), \AS{فكير} (\textbf{fikiri}, \textit{think}), \AS{شكور} (\textbf{shukuru}, \textit{give thanks}).
\end{itemize}

However, the suggested spelling convention in \textbf{Andika!}, as in Sheikh Yahya's own manuscripts, is that all short vowels are marked, thus: \AS{ثَمَنِينِ}, \AS{ذَهَابُ}, \AS{تَپَكَازَ}, \AS{فِكِيرِ}, \AS{شُكُورُ}.  There are a few practical reasons for this:
\begin{itemize}
\item Short \textbf{e, o} need to be marked anyway, since Arabic script has no way otherwise of distinguishing \AS{ي}  meaning \textbf{i} from \AS{ي} meaning \textbf{e}, or \AS{و} meaning \textbf{o} from \AS{و} meaning \textbf{u}.

\item Omitting short vowels may conceivably save time when writing, once the rules above are mastered, but this is unlikely to apply when typing -- it is probably faster simply to type more or less what would be typed when using Roman script, including short vowels.

\item The omission of short vowels means that transliteration into Roman script would require post-editing to add vowels. It might be possible to automate the application of the above rules to avoid this, but the resulting system would likely be cumbersome, and simply typing the short vowels is a more practical solution.
\end{itemize}

\subsection{\textit{Sakani} on consonants}

Arabic \textit{sukun} marks the absence of a vowel after a consonant. In Sheikh Yahya's manuscripts, \textit{sakani} is used consistently for this purpose (alongside its use on long vowels). Thus: \AS{أُنَڤْيٗوٖيزَ} (\textbf{unavyoweza}, \textit{how you can}), \AS{كْوَ} (\textbf{kwa}, \textit{to, by, for}). Its most common occurrence is on a nasal before another consonant: \AS{أِنْڠَوَ} (\textbf{ingawa}, \textit{although}), \AS{نْجٖيمَ} (\textbf{njema}, \textit{good}).

Its use on nasals means that \textit{sakani} can also denote syllabicity, and in \citet{Omar1997} its function appears to be limited solely to that. The aim, as with the omission of short vowels, was most likely to limit the number of diacritics in the text.

The suggested convention in \textbf{Andika!} currently is to follow the manuscript practice, and use \textit{sakani} on the first consonant of multi-consonant clusters. However, since \textit{sakani} is not strictly necessary if all vowels are being marked, this convention is open to change.  (If users feel that marking \textit{sakani} leads to clutter, they can of course omit it).

\subsection{Distinction between syllabicity and prenasalisation}

Although the Roman orthography does not distinguish these two sounds, both Sheikh Yahya's manuscripts and \citet{Omar1997} make a distinction between a syllabic nasal followed by a voiced plosive (eg \textbf{m̩b}) and a prenasalised voiced plosive (eg \textbf{nɓ}). The former is written with a preceding \AS{ْم}, and the latter with a preceding \AS{ن}, as in \AS{مْبَيَ} (\textbf{mbaya}, \textit{bad} [Class1]) compared to \AS{نبَايَ} (\textbf{mbaya}, \textit{bad} [Class 9]).

\textbf{Andika!} will of course allow this distinction to be made in the Arabic script should a writer wish to do so. However, the Roman to Arabic converter cannot do this (since the distinction is not reflected in the standard orthography), and will always convert mb to \AS{مْب}, so automatically-converted text will need post-editing to reflect this distinction if the user wishes to make it.
