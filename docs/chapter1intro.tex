\chapter{Introduction}
\label{ch:intro}

For centuries, Swahili was written in Arabic script, and hundreds of manuscripts in collections around the world testify to its long tradition of written literature. Over the last century, however, Swahili in Roman script has become the norm.

\textbf{Andika!} (meaning \textbf{Write!} in Swahili) has two aims.  The first is to make Swahili in Arabic script as easy to use as Swahili in Roman script -- it is equally easy to read and write the the language in either script. The tools, based on the work of Marehemu Mu'allim Sheikh Yahya Ali Omar \citep{Omar1997} provide a consistent, standardised transliteration of Swahili in Arabic script, and a one-to-one mapping of this to Swahili in Roman script.  Documents can be typed in either script, and automatically transliterated to the other.

\begin{itemize}
\item New writing in Swahili can be composed in Arabic script and published easily via word-processors, webpages, or pdfs created by typesetting systems such as LaTeX.

\item The ability to convert Arabic script at any time into Roman script means that there is very little overhead involved in choosing to write Swahili in Arabic script. Material can be produced simultaneously in both scripts with the minimum of effort (although the converted text will need minor editing to cover such things as capital letters, which do not exist in Arabic script).

\item Existing Swahili content in Roman script can be converted to Arabic script, making it possible to reuse content already published in Roman script. This means that large amounts of material in Arabic script can be be made available very quickly.

\item The Roman-to-Arabic conversion can be adjusted to convert numerals, to add or remove markers such as \textit{sakani} (\textit{sukun}), and so on.
\end{itemize}

The second aim of \textbf{Andika!} is to allow the creation of digital versions of existing Swahili manuscripts written in Arabic script.

\begin{itemize}
\item Perishable Swahili manuscripts in Arabic script can be directly transcribed and made available in digital format, which is more versatile than a photocopy or scan of the manuscript. At present, most Swahili literature from earlier periods has only been published in Roman transliteration, even though the manuscripts were written in Arabic script.

\item A direct transcription can be augmented with a fully-vocalised Arabic transcription, a close phonetic transliteration (a variety of different ones can be easily created), a transliteration in the standard Roman orthography, and so on. The tools allow much of these to be generated automatically, reducing the effort this would otherwise involve.

\item A critical apparatus (English translation, notes on words, variant readings, emendations, etc) can easily be added to the digital version, with high-quality typeset output in a variety of formats.

\item Apart from allowing easier typesetting and dissemination, having manuscripts in digital form will make it possible for the first time to use computers to look at word frequency, stylistic variation, etc, within the texts, to build corpora for classical Swahili, and so on.
\end{itemize}

\textbf{Andika!} is licensed under version 3 of the Free Software Foundation's General Public License.\footnote{http://www.gnu.org/licenses/gpl.html}  This means that, apart from costing nothing to use, it can be adapted and extended as required by the user, subject to the same license being used for any new version thus created.
