\chapter{Changing the \textbf{Andika!} keyboard layout}
\renewcommand{\thesection}{C/\arabic{section}}  % redefine the section numbering
\setcounter{section}{0}  % reset counter 
\label{appC}

\section{Introduction}

The layout of the \textbf{Andika!} keyboard is specified in the file \textit{layout/tz}.  The file (reproduced in \Cref{appE}) is a simple text file, and can be easily adapted to add new glyphs or change the position of existing glyphs.

Each line follows the pattern below:

\verb|key <AC03> { [Arabic_dal, Arabic_thal, Arabic_dad, Arabic_ddal] };|

The key number (in this case AC03, for the \textbf{D} key) is followed by a sequence of 4 glyph names (in this case representing \AS{ڈ ض ذ د}).  The sequence specifies the glyph that will be output when (respectively) the user presses \textbf{D}, \textbf{Shift+D}, \textbf{AltGr+D}, and \textbf{AltGr+Shift+D}.

Some lines have less than four entries.  For instance, the \textbf{P} key only has one entry (\AS{پ}):

\verb|key <AD10> { [Arabic_peh] };|

because that is the only glyph output by that key, and the \textbf{S} key only has three entries ( \AS{ص ش س}):

\verb|key <AC02> { [Arabic_seen, Arabic_sheen, Arabic_sad] }|

giving the glyphs that will be output by pressing \textbf{S}, \textbf{Shift+S} and \textbf{AltGr+S}.

If it is desired to block one of the slots, to enforce a particular keypress for a glyph, the entry \verb|NoSymbol| can be used.  Thus in the line for the \textbf{5} key:

\verb|key <AE05> { [Arabic_5, NoSymbol, KP_5, percent] };|

the output will be \AS{٥} for \textbf{5}, nothing for \textbf{Shift+5}, Western 5 for \textbf{AltGr+5} and a percent sign for \textbf{AltGr+Shift+5}.  Without the \verb|NoSymbol|, the output would be \AS{٥} for \textbf{5}, Western 5 for \textbf{Shift+5}, a percent sign for \textbf{AltGr+5} and nothing for \textbf{AltGr+Shift+5}.

Glyph names are available for some, but by no means all, of the possible glyphs.\footnote{\url{http://wiki.linuxquestions.org/wiki/List_of_Keysyms_Recognised_by_Xmodmap}}  Where no name is available, the Unicode codepoint can be used instead.  Thus, in the line for the \textbf{n} key:

\verb|key <AB06> { [Arabic_noon, U075D] };|

\AS{ن} will be output when the \textbf{N} key is pressed, and the glyph represented by Unicode 075D (\AS{ݝ}, \textit{ain} with two dots above) will be output when \textbf{Shift+N} is pressed.  It would be possible to use nothing but Unicode codepoints in the file, but using the glyph names makes it a bit easier to read.

From the above, it will be obvious that adjusting the location of a particular glyph merely consists of moving it to the desired slot on the desired key.  For example, if the user wanted \AS{ض} to appear when Shift+D is pressed, and \AS{ذ} when AltGr+D is pressed, all that needs to be done is to open the file:

\verb|sudo nano layout/tz|

and change the line:

\verb|key <AC03> { [Arabic_dal, Arabic_thal, Arabic_dad, Arabic_ddal] };|

to:

\verb|key <AC03> { [Arabic_dal, Arabic_dad, Arabic_thal, Arabic_ddal] };|

Then save the file by pressing \textbf{Ctrl+X}, \textbf{Y}, and \textbf{Return}.

Likewise, adding a new glyph to the keyboard is as simple as deciding which slot on which key it should occupy, and then inserting the Unicode codepoint (or the glyph name where one exists) at that slot.  For instance, if the user needs to access the glyph \textit{rreh} (\textit{ra} with \textit{tah} as a diacritic, Unicode 0691), and decides to put it on the \textbf{R} key so that it will be output when \textbf{AltGr+R} is pressed, all that needs to be done is to change the line:

\verb|key <AD04> { [Arabic_ra] };|

to:

\verb|key <AD04> { [Arabic_ra, NoSymbol, U0691] };|

(Remember that if \verb|NoSymbol| is omitted here, \textit{rreh} will appear when \textbf{Shift+R} is pressed.)

In either case, the new layout has to be activated.  So, after saving the file, copy it to the correct location:

\verb|sudo cp layout/tz /usr/share/X11/xkb/symbols/|

Delete the cache files relating to the old layout (new ones will be created when the new layout is activated):

\verb|sudo rm /var/lib/xkb/server-*|

Then remove the \textit{tz} keyboard layout using your desktop's language setup utility and re-add it.  For KDE, this simply means going to \textbf{K} \textrightarrow\ \textbf{Settings} \textrightarrow\ \textbf{System Settings} \textrightarrow\ \textbf{Input Devices} \textrightarrow\ \textbf{Keyboard} \textrightarrow\ \textit{Layouts} tab, unticking \textbf{Configure layouts}, clicking \textbf{Apply}, and then reticking \textbf{Configure layouts} and clicking \textbf{Apply} again.

The new layout should then be ready for use.



