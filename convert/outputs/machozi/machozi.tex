\documentclass[a4paper, 12pt]{report}

\usepackage{polyglossia}
\usepackage{longtable}
\usepackage{xcolor}  % can't use color with polyglossia

\interfootnotelinepenalty=10000 % prevents the footnote from breaking across pages
% http://tex.stackexchange.com/questions/32208/footnote-runs-onto-second-page

% Thanks to Manas Tungare (http://manas.tungare.name/software/latex) for these settings.
\setlength{\paperwidth}{210mm}
\setlength{\textwidth}{160mm}
\setlength{\textheight}{247mm}

\setlength{\evensidemargin}{1in}
\setlength{\oddsidemargin}{0in}
\setlength{\topmargin}{-0.5in}

\setmainfont[Mapping=tex-text]{Liberation Serif}  % Set the default font for the document (footnotes will also use this font, but see also: http://tex.stackexchange.com/questions/4779/how-to-change-font-family-in-footnote)

\setmainlanguage{english}
\setotherlanguage{arabic}
\setotherlanguage{farsi}  % used to allow typesetting the tile in a different font

\newfontfamily\arabicfont[Script=Arabic, Scale=2]{Scheherazade} % Arabic transcription layer
\newfontfamily\farsifont[Script=Arabic, Scale=2]{GranadaKD} % Arabic transcription layer for titles - uses a version of Granada which has been extended to include glyphs for Swahili.
\newcommand{\AS}[1]{\fontspec[Script=Arabic, Scale=1.5]{Scheherazade} #1\normalfont} % Arabic when used in footnotes - using \newfontfamily resets following English text as well.  IMPORTANT: this can ONLY be used for single words - multiple words will be written LTR, and not RTL as required.
\newfontfamily{\Tr}[Scale=0.9, Color=00BB33]{Linux Biolinum O} %  transliteration layer - Biolinum handles diacritics well
\newfontfamily{\I}[Scale=0.9, Color=blue]{Linux Biolinum O} % interpolated letters in the transliteration layer
\newfontfamily{\S}[Color=00BB33]{Linux Biolinum O} % standard spelling layer
\newfontfamily{\E}[Scale=0.9, Color=666666]{Liberation Serif Italic} % English translation layer
\newfontfamily{\FN}[Color=00BB33]{Liberation Serif Italic} % Alternate type in footnotes.

\renewcommand\thefootnote{\textcolor{red}{\arabic{footnote}}}  % Alter the colour of the footnote markers - thanks to Gonzalo Medina (http://tex.stackexchange.com/questions/26693/change-the-color-of-footnote-marker-in-latex#26696).

\usepackage[obeyspaces]{url}  % Use urls in text and captions with sensible linewrap.
\urlstyle{rm}  % Set urls in roman.

\begin{document}

\begin{longtable}{r}
\textfarsi{مَچٗوزِ} \\*
\Tr{machozi} \\
[6mm]
\end{longtable}


\begin{longtable}{rl} 

\makebox[8cm][r]{} & & \makebox[8cm][r]{} \\ 

\textarabic{مَچٗوزِ يَ هُوبَ} & \textarabic{١} \\* 
\Tr{machozi ya huba} & \Tr{1a} \\ 
\textarabic{يَمٖنِدٗنْدٗوكَ} &  \\* 
\Tr{yamenidondoka} & \Tr{1b} \\ 
\textarabic{سِيلِ نِكَشِيبَ} &  \\* 
\Tr{sili nikashiba} & \Tr{1c} \\ 
\textarabic{نِكَفُرَهِيكَ} &  \\* 
\Tr{nikafurahika} & \Tr{1d} \\ 
\textarabic{وَانْڠُ مَهَبُوبَ} &  \\* 
\Tr{wangu mahabuba} & \Tr{1e} \\ 
\textarabic{أَمٖشَنِتٗوكَ} &  \\* 
\Tr{ameshanitoka} & \Tr{1f} \\ 
\\[8mm] 

\textarabic{أَمٖشَنِتٗوكَ} & \textarabic{٢} \\* 
\Tr{ameshanitoka} & \Tr{2a} \\ 
\textarabic{مْپٖنْزِ جَمَانِ} &  \\* 
\Tr{mpenzi jamani} & \Tr{2b} \\ 
\textarabic{مْسِيتُ وَ نْيِيكَ} &  \\* 
\Tr{msitu wa nyika} & \Tr{2c} \\ 
\textarabic{چٗوزِ كِفُؤَانِ} &  \\* 
\Tr{chozi kifuani} & \Tr{2d} \\ 
\textarabic{نِنَهَنْڠَئِيكَ} &  \\* 
\Tr{ninahangaika} & \Tr{2e} \\ 
\textarabic{وَالَ سِمُؤٗونِ} &  \\* 
\Tr{wala simuoni} & \Tr{2f} \\ 
\\[8mm] 

\textarabic{وَالَ سِمُؤٗونِ} & \textarabic{٣} \\* 
\Tr{wala simuoni} & \Tr{3a} \\ 
\textarabic{سِجُئِ أَلِيكٗ} &  \\* 
\Tr{sijui aliko} & \Tr{3b} \\ 
\textarabic{أَنِيپَ مَشَاكَ} &  \\* 
\Tr{anipa mashaka} & \Tr{3c} \\ 
\textarabic{نَ مَسِكِتِيكٗ} &  \\* 
\Tr{na masikitiko} & \Tr{3d} \\ 
\textarabic{سَؤُوتِ نَغَانِ} &  \\* 
\Tr{sauti naghani} & \Tr{3e} \\ 
\textarabic{أَجُئٖ نِلِيكٗ} &  \\* 
\Tr{ajue niliko} & \Tr{3f} \\ 
\\[8mm] 

\end{longtable}

\begin{longtable}{r}
% Remember to end each line (except vertical space) with a double backslash
 \\  % Any material after the end of the poem

\end{longtable}

\end{document}

