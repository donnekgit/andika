\documentclass[a4paper, 12pt]{report}

\usepackage{polyglossia}
\usepackage{longtable}
\usepackage{xcolor}  % can't use color with polyglossia
\usepackage{arydshln}

\interfootnotelinepenalty=10000 % prevents the footnote from breaking across pages
% http://tex.stackexchange.com/questions/32208/footnote-runs-onto-second-page

% Thanks to Manas Tungare (http://manas.tungare.name/software/latex) for these settings.
\setlength{\paperwidth}{210mm}
\setlength{\paperheight}{297mm}

\setlength{\textwidth}{160mm}
\setlength{\textheight}{247mm}

\setlength{\evensidemargin}{1in}
\setlength{\oddsidemargin}{0in}
\setlength{\topmargin}{-0.5in}

\setmainfont[Mapping=tex-text]{Liberation Serif}  % Set the default font for the document (footnotes will also use this font, but see also: http://tex.stackexchange.com/questions/4779/how-to-change-font-family-in-footnote)

\setmainlanguage{english}
\setotherlanguage{arabic}
\setotherlanguage{farsi}  % used to allow typesetting the tile in a different font

\newfontfamily\arabicfont[Script=Arabic, Scale=2]{Scheherazade} % Arabic transcription layer
\newfontfamily\farsifont[Script=Arabic, Scale=2]{GranadaKD} % Arabic transcription layer for titles - uses a version of Granada which has been extended to include glyphs for Swahili.
\newcommand{\AS}[1]{\fontspec[Script=Arabic, Scale=1.5]{Scheherazade} #1\normalfont} % Arabic when used in footnotes - using \newfontfamily resets following English text as well.  IMPORTANT: this can ONLY be used for single words - multiple words will be written LTR, and not RTL as required.
\newfontfamily{\Tr}[Scale=0.9, Color=00BB33]{Linux Biolinum O} %  transliteration layer - Biolinum handles diacritics well
\newfontfamily{\I}[Scale=0.9, Color=blue]{Linux Biolinum O} % interpolated letters in the transliteration layer
\newfontfamily{\S}[Color=00BB33]{Linux Biolinum O} % standard spelling layer
\newfontfamily{\E}[Scale=0.9, Color=666666]{Liberation Serif Italic} % English translation layer
\newfontfamily{\FN}[Color=00BB33]{Liberation Serif Italic} % Alternate type in footnotes.

\renewcommand\thefootnote{\textcolor{red}{\arabic{footnote}}}  % Alter the colour of the footnote markers - thanks to Gonzalo Medina (http://tex.stackexchange.com/questions/26693/change-the-color-of-footnote-marker-in-latex#26696).

\usepackage[obeyspaces]{url}  % Use urls in text and captions with sensible linewrap.
\urlstyle{rm}  % Set urls in roman.

\begin{document}

\begin{longtable}{{rl}} 

\textarabic{kwa ina la rahamani * naanda nudhuma hini * ina ya pili ramani * namwandikiya yahaya
} & \textarabic{١} \\* 
\Tr{kwa ina la rahamani * naanḏa nudhuma hini * ina ya pili ramani * namwanḏikiya yahaya} & \Tr{1}\\* 
\textarabic{
} & \textarabic{٢} \\* 
\Tr{} & \Tr{2}\\* 
\textarabic{kwenye ramani ya kwanda * banati naliwafunda * na hini nimeiunda * wavulana kuwambiya
} & \textarabic{٣} \\* 
\Tr{kwenye ramani ya kwanḏa * banaṯi naliwafunḏa * na hini nimeiunḏa * wavulana kuwambiya} & \Tr{3}\\* 
\textarabic{
} & \textarabic{٤} \\* 
\Tr{} & \Tr{4}\\* 
\textarabic{na asili ya kwandika * nahisi wana wataka * baba piya kadhalika * mimi alinandikiya
} & \textarabic{٥} \\* 
\Tr{na asili ya kwanḏika * nahisi wana waṯaka * baba piya kadhalika * mimi alinanḏikiya} & \Tr{5}\\* 
\textarabic{
} & \textarabic{٦} \\* 
\Tr{} & \Tr{6}\\* 
\textarabic{baba bwana ahmadi * atamjazi wadudi * kwani alijitahidi * kunandikiya wasiya 
} & \textarabic{٧} \\* 
\Tr{baba bwana ahmaḏi * aṯamjazi waḏuḏi * kwani alijiṯahiḏi * kunanḏikiya wasiya} & \Tr{7}\\* 
\textarabic{
} & \textarabic{٨} \\* 
\Tr{} & \Tr{8}\\* 
\textarabic{hapo zamani za yana * ndiyo la wasiya yina * alonandikia bwana * babangu kaniwatiya
} & \textarabic{٩} \\* 
\Tr{hapo zamani za yana * nḏiyo la wasiya yina * alonanḏikia bwana * babangu kaniwaṯiya} & \Tr{9}\\* 
\textarabic{
} & \textarabic{١٠} \\* 
\Tr{} & \Tr{10}\\* 
\textarabic{nami katika ramani * tayipinda kubaini * yale ninayoamini * yaweza kusaidiya
} & \textarabic{١١} \\* 
\Tr{nami kaṯika ramani * ṯayipinḏa kubaini * yale ninayoamini * yaweza kusaiḏiya} & \Tr{11}\\* 
\textarabic{
} & \textarabic{١٢} \\* 
\Tr{} & \Tr{12}\\* 
\textarabic{tanena naloyatuma * na yale niloyasoma * tazitaya na hekima * kwa watu nilopokeya
} & \textarabic{١٣} \\* 
\Tr{ṯanena naloyaṯuma * na yale niloyasoma * ṯaziṯaya na hekima * kwa waṯu nilopokeya} & \Tr{13}\\* 
\textarabic{
} & \textarabic{١٤} \\* 
\Tr{} & \Tr{14}\\* 
\textarabic{yale nitayakusanya * mbali mbali kutanganya * lengo langu ni kufanya * kama la maisha boya
} & \textarabic{١٥} \\* 
\Tr{yale niṯayakusanya * mbali mbali kuṯanganya * lengo langu ni kufanya * kama la maisha boya} & \Tr{15}\\* 
\end{longtable}

\begin{longtable}{r}
% Remember to end each line (except vertical space) with a double backslash
 \\  % Any material after the end of the poem

\end{longtable}

\end{document}

