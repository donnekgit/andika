\documentclass[a4paper, 12pt]{report}

\usepackage{polyglossia}
\usepackage{longtable}
\usepackage{xcolor}  % can't use color with polyglossia
\usepackage{arydshln}

\interfootnotelinepenalty=10000 % prevents the footnote from breaking across pages
% http://tex.stackexchange.com/questions/32208/footnote-runs-onto-second-page

% Thanks to Manas Tungare (http://manas.tungare.name/software/latex) for these settings.
\setlength{\paperwidth}{210mm}
\setlength{\paperheight}{297mm}

\setlength{\textwidth}{160mm}
\setlength{\textheight}{247mm}

\setlength{\evensidemargin}{1in}
\setlength{\oddsidemargin}{0in}
\setlength{\topmargin}{-0.5in}

\setmainfont[Mapping=tex-text]{Liberation Serif}  % Set the default font for the document (footnotes will also use this font, but see also: http://tex.stackexchange.com/questions/4779/how-to-change-font-family-in-footnote)

\setmainlanguage{english}
\setotherlanguage{arabic}
\setotherlanguage{farsi}  % used to allow typesetting the tile in a different font

\newfontfamily\arabicfont[Script=Arabic, Scale=2]{Scheherazade} % Arabic transcription layer
\newfontfamily\farsifont[Script=Arabic, Scale=2]{GranadaKD} % Arabic transcription layer for titles - uses a version of Granada which has been extended to include glyphs for Swahili.
\newcommand{\AS}[1]{\fontspec[Script=Arabic, Scale=1.5]{Scheherazade} #1\normalfont} % Arabic when used in footnotes - using \newfontfamily resets following English text as well.  IMPORTANT: this can ONLY be used for single words - multiple words will be written LTR, and not RTL as required.
\newfontfamily{\T}[Scale=0.9, Color=00BB33]{Linux Biolinum O} %  transliteration layer - Biolinum handles diacritics well
\newfontfamily{\I}[Scale=0.9, Color=blue]{Linux Biolinum O} % interpolated letters in the transliteration layer
\newfontfamily{\E}[Scale=0.9, Color=666666]{Liberation Serif Italic} % English translation layer

\renewcommand\thefootnote{\textcolor{red}{\arabic{footnote}}}  % Alter the colour of the footnote markers - thanks to Gonzalo Medina (http://tex.stackexchange.com/questions/26693/change-the-color-of-footnote-marker-in-latex#26696):

\begin{document}

\begin{longtable}{r}
\textfarsi{هَمزِيَ} \\*
\T{hamziya} \\
\cdashline{1-1}[1pt/3pt] \\
[6mm]
\end{longtable}


\begin{longtable}{rrl} 

\textarabic{نِ وٖيمَ أُلٗويٖ وَلَنْڠَافُ كُفِيكَ مٖيمَ} & \textarabic{كْوَ سَبَابُ سِيزٗ كَمُپٗوسَ أَمُلُوٗولٖ} & \textarabic{١} \\* 
\T{ni wema uloye walangafu kufika mema} & \T{kwa sababu sizo kamuposa amuluwole} & \T{1a/b} \\ 
[8mm] 

\textarabic{نَ مْوٖينْيٖ أَكِيلِ كَتِزَامَ مَامْبٗ كِپِيمَ} & \textarabic{كَجِلِيَ تُومْوَ جِبُرِيلِ مْوَاكٖ هَدِيجَ} & \textarabic{٢} \\* 
\T{na mwenye akili katizama mambo kipima} & \T{kajiliya tumwa jiburili mwake hadija} & \T{2a/b} \\ 
[8mm] 

\textarabic{أَمُئِلٖيٖيٗ سٗويٗ نِ وَهَايِ أَامُ إِغِيمَ} & \textarabic{أَكَؤُوتَ مْبَالِ مُتُوَازِ كْوَامْبَ أَجُوٖ} & \textarabic{٣} \\* 
\T{amuileyeyo soyo ni wahayi amu ighima} & \T{akauta mbali mutuwazi kwamba ajuwe} & \T{3a/b} \\ 
[8mm] 

\end{longtable}

% Remember to end each line (except vertical space) with a double backslash

\begin{longtable}{r}

 \\  % Any material after the end of the poem

\end{longtable}

\end{document}

