\documentclass[a4paper, 12pt]{report}

\usepackage{polyglossia}
\usepackage{longtable}
\usepackage{xcolor}  % can't use color with polyglossia

\usepackage{marginnote}
\renewcommand*{\marginfont}{\color{red}\sffamily}

\interfootnotelinepenalty=10000 % prevents the footnote from breaking across pages
% http://tex.stackexchange.com/questions/32208/footnote-runs-onto-second-page

% Thanks to Manas Tungare (http://manas.tungare.name/software/latex) for these settings.
\setlength{\paperwidth}{210mm}
\setlength{\textwidth}{160mm}
\setlength{\textheight}{247mm}

\setlength{\evensidemargin}{1in}
\setlength{\oddsidemargin}{0in}
\setlength{\topmargin}{-0.5in}

\setmainfont[Mapping=tex-text]{Liberation Serif}  % Set the default font for the document (footnotes will also use this font, but see also: http://tex.stackexchange.com/questions/4779/how-to-change-font-family-in-footnote)

\setmainlanguage{english}
\setotherlanguage{arabic}
\setotherlanguage{farsi}  % used to allow typesetting the tile in a different font

\newfontfamily\arabicfont[Script=Arabic, Scale=2]{Scheherazade} % Arabic transcription layer
\newfontfamily\farsifont[Script=Arabic, Scale=2]{GranadaKD} % Arabic transcription layer for titles - uses a version of Granada which has been extended to include glyphs for Swahili.
\newcommand{\AS}[1]{\fontspec[Script=Arabic, Scale=1.5]{Scheherazade} #1\normalfont} % Arabic when used in footnotes - using \newfontfamily resets following English text as well.  IMPORTANT: this can ONLY be used for single words - multiple words will be written LTR, and not RTL as required.
\newfontfamily{\Tr}[Scale=0.9, Color=00BB33]{Linux Biolinum O} %  transliteration layer - Biolinum handles diacritics well
\newfontfamily{\I}[Scale=0.9, Color=blue]{Linux Biolinum O} % interpolated letters in the transliteration layer
\newfontfamily{\S}[Color=00BB33]{Linux Biolinum O} % standard spelling layer
\newfontfamily{\E}[Scale=0.9, Color=666666]{Liberation Serif Italic} % English translation layer
\newfontfamily{\FN}[Color=00BB33]{Liberation Serif Italic} % Alternate type in footnotes.

\renewcommand\thefootnote{\textcolor{red}{\arabic{footnote}}}  % Alter the colour of the footnote markers - thanks to Gonzalo Medina (http://tex.stackexchange.com/questions/26693/change-the-color-of-footnote-marker-in-latex#26696).

\usepackage[obeyspaces]{url}  % Use urls in text and captions with sensible linewrap.
\urlstyle{rm}  % Set urls in roman.

\begin{document}

\begin{flushright}

{\scriptsize\marginnote{1}[2mm]}\textarabic{هِسْتٗرِئَ يَ سِيتِ إِنَلٖيتَ مْڤُوتٗ كْوَ كُؤٗنْيٖيشَ أُئِمَارَ وَاكٖ كَتِيكَ مَتٖينْدٗ يَاكٖ۔ أَلِوٖيزَ} \\ 

\Tr{Historia ya Siti inaleta mvuto kwa kuonyesha uimara wake katika matendo yake. Aliweza} & \\ 
{\scriptsize\marginnote{2}[2mm]}\textarabic{كُفُنْزْوَ كُئِيمْبَ وَكَاتِ هَكُجُؤَ كُسٗومَ وَالَ كُؤَنْدِيكَ كِئَرَابُ وَالَ كِسوَهِيلِ، لُوغَ كُؤُ} \\ 

\Tr{kufunzwa kuimba wakati hakujua kusoma wala kuandika Kiarabu wala Kiswahili, lugha kuu} & \\ 
{\scriptsize\marginnote{3}[2mm]}\textarabic{أَمْبَازٗ زِلِفُنْدِيشْوَ كَتِيكَ ڤِسِيوَ ڤْيَ زَنْزِبَر وَكَاتِ هُؤٗ۔ كْوَنْزَ أَلِفُنْدِيشْوَ كُسٗومَ} \\ 

\Tr{ambazo zilifundishwa katika visiwa vya Zanzibar wakati huo. Kwanza alifundishwa kusoma} & \\ 
{\scriptsize\marginnote{4}[2mm]}\textarabic{كُرَانِ نَ أَلِفَؤُولُ كُپَاتَ لَفُوذِ نْزُورِ يَ كِئَرَابُ۔ كُتٗكَانَ نَ كِپَاجِ چَاكٖ چَ كُوَ نَ} \\ 

\Tr{Kurani na alifaulu kupata lafudhi nzuri ya Kiarabu. Kutokana na kipaji chake cha kuwa na} & \\ 
{\scriptsize\marginnote{5}[2mm]}\textarabic{مٗويٗ مْوٖپٖيسِ وَ كُهِفَاذِ نَ كُپَاتَ لَفُوذِ نْزُورِ، سِيتِ أَلِنْيَنْيُكِئَ كُوَ مْوِمْبَاجِ بٗورَ} \\ 

\Tr{moyo mwepesi wa kuhifadhi na kupata lafudhi nzuri, Siti alinyanyukia kuwa mwimbaji bora} & \\ 
{\scriptsize\marginnote{6}[2mm]}\textarabic{كَتِيكَ كِسوَهِيلِ، كِئَرَابُ هَاتَ نَ نْيِيمْبٗ زَ كِهِينْدِ۔} \\ 

\Tr{katika Kiswahili, Kiarabu hata na nyimbo za Kihindi.} & \\ 
{\scriptsize\marginnote{7}[2mm]}\textarabic{سِيتِ أَلِبَاكِ كَتِيكَ كُفِكِئَ مَفَنِكِؤٗ يَاكٖ لِيچَ يَ كُئٖنْدٖلٖئَ كُسٖيمْوَ مَنٖينٗ ذِيدِ يَاكٖ} \\ 

\Tr{Siti alibaki katika kufikia mafanikio yake licha ya kuendelea kusemwa maneno dhidi yake} & \\ 
{\scriptsize\marginnote{8}[2mm]}\textarabic{هَاسَ يَ كُكِؤُوكَ أُتَمَدُونِ وَ مْوَنَامْكٖ وَ وَكَاتِ وَاكٖ نَ كُجِوٖيكَ مْجِينِ كْوَ} \\ 

\Tr{hasa ya kukiuka utamaduni wa mwanamke wa wakati wake na kujiweka mjini kwa} & \\ 
{\scriptsize\marginnote{9}[2mm]}\textarabic{كُشِرِكِئَانَ نَ وَنَؤُومٖ كَتِيكَ كُئِيمْبَ، پِئَ أَلِپَاتَ كٖجٖيلِ زَ وَاتُ وَ مْجِينِ كُمُؤٗونَ يٖيٖ} \\ 

\Tr{kushirikiana na wanaume katika kuimba, pia alipata kejeli za watu wa mjini kumuona yeye} & \\ 
{\scriptsize\marginnote{10}[2mm]}\textarabic{أَنَتٗوكَ شَامْبَ ؍ أَتَيَوٖزَاجٖ يَ مْجِينِ۔ سِيتِ أَلِجِذَتِئِ كُئٖنْدٖلٖئَ نَ كَازِ يَاكٖ نَ چِينِ يَ} \\ 

\Tr{anatoka shamba - atayawezaje ya mjini. Siti alijidhatii kuendelea na kazi yake na chini ya} & \\ 
{\scriptsize\marginnote{11}[2mm]}\textarabic{أُؤٗنْڠٗوزِ وَ مْوَلِيمُ وَاكٖ أَلِپَاتَ مٗويٗ وَ كُئٖنْدٖلٖئَ مْپَاكَ كُفِكِئَ سِيكُ يَ كُتٗكٖيزَ} \\ 

\Tr{uongozi wa mwalimu wake alipata moyo wa kuendelea mpaka kufikia siku ya kutokeza} & \\ 
{\scriptsize\marginnote{12}[2mm]}\textarabic{هَذَرَانِ۔} \\ 

\Tr{hadharani.} & \\ 
\end{flushright}

\end{document}
