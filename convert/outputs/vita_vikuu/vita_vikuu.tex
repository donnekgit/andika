\documentclass[a4paper, 12pt]{report}

\usepackage{polyglossia}
\usepackage{longtable}
\usepackage{xcolor}  % can't use color with polyglossia

\usepackage{marginnote}
\renewcommand*{\marginfont}{\color{red}\sffamily}

\interfootnotelinepenalty=10000 % prevents the footnote from breaking across pages
% http://tex.stackexchange.com/questions/32208/footnote-runs-onto-second-page

% Thanks to Manas Tungare (http://manas.tungare.name/software/latex) for these settings.
\setlength{\paperwidth}{210mm}
\setlength{\textwidth}{160mm}
\setlength{\textheight}{247mm}

\setlength{\evensidemargin}{1in}
\setlength{\oddsidemargin}{0in}
\setlength{\topmargin}{-0.5in}

%--------------------------------
%%% Font definitions %%%
%--------------------------------
% Note that these definitions malfunction if used in \chapter.
\defaultfontfeatures{Mapping=tex-text}
\setmainfont{Charis SIL}  % Set the default font for the document. = \setdefaultfont
% Footnotes will by default also use this font -- http://tex.stackexchange.com/questions/4779/how-to-change-font-family-in-footnote).
\defaultfontfeatures{Scale=MatchLowercase}  % needs to be below main font declaration

\setsansfont{Liberation Sans}
\setmonofont{Lato Bold}

\setmainlanguage{english}
\setotherlanguage{arabic}

\newfontfamily\arabicfont[Script=Arabic, Scale=2]{Scheherazade} % Arabic transcription -- coloured black, double size.
% One font needs to be called \arabicfont in order for XeTeX to load Arabic-related hyphenation and other stuff.
%  The default \textarabic will use this \arabicfont.  Use the \begin{Arabic} ..... \end{Arabic} environment for longer stretches (eg paras).
% Use \textarabic{\aemph{با}} to give overline emphasis.
% Omitting Script=Arabic for Amiri or Granada will mean that letters are written in their standalone forms, not connected.  (Omitting Script=Arabic for Scheherazade seems to cause no problem, though.)

\newfontfamily\citationfont[Script=Arabic, Scale=1.5]{Scheherazade}  % Citations, or stand-alone Arabic script in the middle of Roman script -- coloured black, one-and-a-half size.
\newcommand\AS[1]{{\citationfont\RLE{#1}}}
% \RLE (from the bidi package, which polyglossia loads automatically) is to allow multiple words of Arabic to be written right-to-left -- if omitted, each word in the sequence will be written RTL, but the sequence as a whole will be written LTR.

%You can either, as above, define a new \fontfamily, and then use it in a \newcommand, or you can, as below, include the font in the \newcommand by calling \fontspec directly.

\newcommand\Atitle[1]{{\fontspec[Script=Arabic, Scale=2]{GranadaKD}\RLE{#1}}}  % Arabic transcription for titles - uses a version of Granada which has been extended to include glyphs for Swahili.

\newcommand\Am[1]{{\fontspec[Script=Arabic]{Amiri}\RLE{#1}}} % Examples using Amiri --  if using Scheherazade's default scale, set Scale=0.8 here.

%\newfontfamily\translitfont[Scale=1, Color=666666]{Linux Biolinum O}
%\newcommand\Tr[1]{{\translitfont\RLE{#1}}}
\newcommand\Tr[1]{{\fontspec[Scale=1, Color=666666]{Linux Biolinum O}#1}}   %  Transliteration -- Biolinum handles diacritics well.  Coloured grey, slightly less than normal size.
% Scale=1 is required because of Scale=MatchLowercase - otherwise the size is too large.

\newcommand\In[1]{{\fontspec[Scale=1, Color=blue]{Linux Biolinum O}#1}}  % Epenthetic letters in the transliteration -- coloured blue, normal size.

\newcommand\Swa[1]{{\fontspec[Color=00BB33, Scale=1]{Linux Biolinum O}#1}}  % Standard spelling -- coloured green, normal size.

\newcommand\E[1]{{\fontspec[Scale=0.9, Color=333333]{Liberation Serif Italic}#1}}  % English translation layer -- coloured grey, slightly less than normal size.

\newcommand\Eit[1]{{\fontspec{Liberation Serif Italic}#1}}  % English italics.

\newcommand\FN[1]{{\fontspec[Color=00BB33]{Liberation Serif Italic}#1}} % Standout type in footnotes -- coloured green, normal size.

% Older versions:
% \newfontfamily{\Tr}[Scale=0.9, Color=00BB33]{Linux Biolinum O}
% This can be used as \Tr{text}.  But this will change the font outside the argument until the end of that stretch.
% This doesn't show up in the poemlines, because they are self-contained, but it does show up in connected text.
% To avoid this, and have the font only changed within the argument, use \newcommand as above.
% Though you can also enclose \Tr in braces to limit it: {\Tr{}}

%----------------------------------------
%%% End of font definitions %%%
%----------------------------------------

\renewcommand\thefootnote{\textcolor{red}{\arabic{footnote}}}  % Alter the colour of the footnote markers - thanks to Gonzalo Medina (http://tex.stackexchange.com/questions/26693/change-the-color-of-footnote-marker-in-latex#26696).

\usepackage[obeyspaces]{url}  % Use urls in text and captions with sensible linewrap.
\urlstyle{rm}  % Set urls in roman.

\begin{document}

\begin{longtable}{rrl} 

\makebox[8cm][r]{} & & \makebox[8cm][r]{} \\ 

\textarabic{نَخُبُوزِ يَشَعِيْرِ} & \textarabic{أَكَتٗؤَ تَمَارِ} & \textarabic{١} \\* 
\Tr{nakhubūzi yasha'ı̄ri} & \Tr{akaṯoa ṯamāri} & \Tr{1b/a} \\ 
\textarabic{كَكهٖيْتِ كَٹٗئٖلٖئَ} & \textarabic{نَمِلْحِ أَصْفَرِ} &  \\* 
\Tr{kakʿēṯi kaţoelea} & \Tr{na mil\In{i}ḥi aṣ\In{u}fari} & \Tr{1d/c} \\ 
\\[8mm] 

\textarabic{ۏَاكٖ إِلَاهِ وَدُوْدِ} & \textarabic{كِشَكُوْلَ كَحِمِيْدِ} & \textarabic{٢} \\* 
\Tr{w̱āke ilāhi waḏūḏi} & \Tr{kishakūla kaḥimı̄ḏi} & \Tr{2b/a} \\ 
\textarabic{مَعَدُوِ نَمَوَلِيْ} & \textarabic{مُؤُوْنْبَ زٗوْتهٖ جَسَادِ} &  \\* 
\Tr{ma'aḏuwi namawalii} & \Tr{muūm̱ba zōṯʿe jasāḏi} & \Tr{2d/c} \\ 
\\[8mm] 

\textarabic{عَمُوْرِ أَكَتٗوْكَ} & \textarabic{هَاتَ كُكِپَنْبَؤُوْكَ} & \textarabic{٣} \\* 
\Tr{'amūri akaṯōka} & \Tr{hāṯa kukipam̱baūka} & \Tr{3b/a} \\ 
\textarabic{سَوْتِ أَكَئِتٗؤَ} & \textarabic{كْوَ عَلِىْ أَكَفِيْكَ} &  \\* 
\Tr{sawṯi akaiṯoa} & \Tr{kwa 'alii akafı̄ka} & \Tr{3d/c} \\ 
\\[8mm] 

\textarabic{أَهْلاً يَا مُكَرَّمَ} & \textarabic{عَلِىْ كَتَكَلَامَ} & \textarabic{٤} \\* 
\Tr{ahlāⁿ yā mukarrama} & \Tr{'alii kaṯakalāma} & \Tr{4b/a} \\ 
\textarabic{نْدِئَ إِنْڠَاۏَ طَوِلِيَ} & \textarabic{ٹُتَوَصِيْل سَلَامَ} &  \\* 
\Tr{nḏia ingāw̱a ṭawiliya} & \Tr{ţuṯawaṣı̄l salāma} & \Tr{4d/c} \\ 
\\[8mm] 

\textarabic{كَئِلَبِيْسِ يُوَانِ} & \textarabic{عَمُوْرِ كَرُوْدِ نْڈَانِ} & \textarabic{٥} \\* 
\Tr{kailabı̄si yuwāni} & \Tr{'amūri karūḏi nḑāni} & \Tr{5b/a} \\ 
\textarabic{أَكَئِفُوْنْڠَ زِكَمْكَاءَ} & \textarabic{دِرِيْعِ زَ أُۏَنْدَانِ} &  \\* 
\Tr{akaifūnga zikamkaa} & \Tr{ḏirī'i za uw̱anḏāni} & \Tr{5d/c} \\ 
\\[8mm] 

\textarabic{نَرُمْحِ نَتُرُوْسِ} & \textarabic{نَسٖيْفُ نْجٖيْمَ أَسِيْسِ} & \textarabic{٦} \\* 
\Tr{narumḥi naṯurūsi} & \Tr{nasēfu njēma\footnote{B: \AS{ننْزُوْرِ}, nzūri} ası̄si} & \Tr{6b/a} \\ 
\textarabic{جُوْ أَكَمْكَلِئَ} & \textarabic{أَكَمْپَانْڈَ فَرَاسِ} &  \\* 
\Tr{juu akamkalia} & \Tr{akampānḑa farāsi} & \Tr{6d/c} \\ 
\\[8mm] 

\textarabic{بِاللَّيْلِ وَالنَّهَارِ} & \textarabic{ۏَكٖيْنْڈَ تهَخُبِيْرِ} & \textarabic{٧} \\* 
\Tr{billayli wannahāri} & \Tr{w̱akēnḑa ṯʿakhubı̄ri} & \Tr{7b/a} \\ 
\textarabic{نَمِيْٹِ نَمَطَرِيَ} & \textarabic{ۏَكَپَنْبَانَ نَبَحَارِ} &  \\* 
\Tr{namı̄ţi namaṭariya} & \Tr{w̱akapam̱bāna nabaḥāri} & \Tr{7d/c} \\ 
\\[8mm] 

\textarabic{عَلِىْ نَمَلِعُوْنِ} & \textarabic{ۏَكٖنٖينْڈَ يُۏَانِ} & \textarabic{٨} \\* 
\Tr{'alii namali'ūni} & \Tr{w̱akenēnḑa yuw̱āni} & \Tr{8b/a} \\ 
\textarabic{عَمُوْرِ كَزِٹَنْبُؤَ} & \textarabic{نْدِئَ زِيْلِ زَمَدِيْنِ} &  \\* 
\Tr{'amūri kaziţam̱bua} & \Tr{nḏia zı̄li zamaḏı̄ni} & \Tr{8d/c} \\ 
\\[8mm] 

\textarabic{كَمْوَنبِئَ حَيْدَرِ} & \textarabic{عَمُوْرِ كَذُكُوْرِ} & \textarabic{٩} \\* 
\Tr{kamwam̱bia ḥayḏari} & \Tr{'amūri kadhukūri} & \Tr{9b/a} \\ 
\textarabic{نْبٗوْنَ ٹْوَئِفُوَٹِئَ} & \textarabic{دُرُوْبُ نْدَ أَنْصَارِ} &  \\* 
\Tr{m̱bōna ţwaifuwaţia} & \Tr{ḏurūbu nḏa anṣāri} & \Tr{9d/c} \\ 
\\[8mm] 

\textarabic{يَكْوٖنٖينْڈَ سَفَارِ} & \textarabic{نَاسِ ٹُمٖفَانْيَ مَشَؤُوْرِ} & \textarabic{١٠} \\* 
\Tr{yakwenēnḑa safāri} & \Tr{nāsi ţumefānya mashaūri} & \Tr{10b/a} \\ 
\textarabic{جُنُوْدِ كُئِكُسَنْيِئَ} & \textarabic{كْوَنْڠَلِئَ أَمْصَارِ} &  \\* 
\Tr{junūḏi kuikusanyia} & \Tr{kwangalia amṣāri} & \Tr{10d/c} \\ 
\\[8mm] 

\end{longtable} 

\end{document}
