\documentclass[a4paper, 12pt]{report}

\usepackage{titlesec}  % Allow the chapter/section heading settings to be fine-tuned.  Needs to come before bidi, in polyglossia.
\usepackage{polyglossia}  % multilingual support
\usepackage{longtable}  % tables that carry across multiple pages
\usepackage{xcolor}  % can't use color with polyglossia

%--------------------------------
%%% Font definitions %%%
%--------------------------------

% Note that these definitions malfunction if used in \chapter{}.
\defaultfontfeatures{Mapping=tex-text}
\setmainfont{Charis SIL}  % Set the default font for the document. = \setdefaultfont
% Footnotes will by default also use this font -- http://tex.stackexchange.com/questions/4779/how-to-change-font-family-in-footnote).
\defaultfontfeatures{Scale=MatchLowercase}  % needs to be below main font declaration

\setsansfont{Liberation Sans}
\setmonofont{DejaVu Sans Mono}

\setmainlanguage{english}
\setotherlanguage{arabic}

\newfontfamily\arabicfont[Script=Arabic, Scale=2]{Scheherazade} % Arabic transcription -- coloured black, double size.
% One font needs to be called \arabicfont in order for XeTeX to load Arabic-related hyphenation and other stuff.
%  The default \textarabic will use this \arabicfont.  Use the \begin{Arabic} ..... \end{Arabic} environment for longer stretches (eg paras).
% Use \textarabic{\aemph{با}} to give overline emphasis.
% Omitting Script=Arabic for Amiri or Granada will mean that letters are written in their standalone forms, not connected.  (Omitting Script=Arabic for Scheherazade seems to cause no problem, though.)

\newfontfamily\citationfont[Script=Arabic, Scale=1.5]{Scheherazade}  % Citations, or stand-alone Arabic script in the middle of Roman script -- coloured black, one-and-a-half size.
\newcommand\AS[1]{{\citationfont\RLE{#1}}}
% \RLE (from the bidi package, which polyglossia loads automatically) is to allow multiple words of Arabic to be written right-to-left -- if omitted, each word in the sequence will be written RTL, but the sequence as a whole will be written LTR.

%You can either, as above, define a new \fontfamily, and then use it in a \newcommand, or you can, as below, include the font in the \newcommand by calling \fontspec directly.

\newcommand\Atitle[1]{{\fontspec[Script=Arabic, Scale=2]{GranadaKD}\RLE{#1}}}  % Arabic transcription for titles - uses a version of Granada which has been extended to include glyphs for Swahili.

\newcommand\Am[1]{{\fontspec[Script=Arabic]{Amiri}\RLE{#1}}} % Examples using Amiri --  if using Scheherazade's default scale, set Scale=0.8 here.

%\newfontfamily\translitfont[Scale=1, Color=666666]{Linux Biolinum O}
%\newcommand\Tr[1]{{\translitfont\RLE{#1}}}
\newcommand\Tr[1]{{\fontspec[Scale=1, Color=666666]{Linux Biolinum O}#1}}   %  Transliteration -- Biolinum handles diacritics well.  Coloured grey, slightly less than normal size.
% Scale=1 is required because of Scale=MatchLowercase - otherwise the size is too large.
\newcommand\Trb[1]{{\fontspec[Scale=1, Color=0000BB]{Linux Biolinum O}#1}} 

\newcommand\In[1]{{\fontspec[Scale=1, Color=blue]{Linux Biolinum O}#1}}  % Epenthetic letters in the transliteration -- coloured blue, normal size.

\newcommand\Swa[1]{{\fontspec[Color=00BB33, Scale=1]{Linux Biolinum O}#1}}  % Standard spelling -- coloured green, normal size.

\newcommand\E[1]{{\fontspec[Scale=0.9, Color=333333]{Liberation Serif Italic}#1}}  % English translation layer -- coloured grey, slightly less than normal size.

\newcommand\Eit[1]{{\fontspec{Liberation Serif Italic}#1}}  % English italics.

\newcommand\FN[1]{{\fontspec[Color=00BB33]{Liberation Serif Italic}#1}} % Standout type in footnotes -- coloured green, normal size.

% Older versions:
% \newfontfamily{\Tr}[Scale=0.9, Color=00BB33]{Linux Biolinum O}
% This can be used as \Tr{text}.  But this will change the font outside the argument until the end of that stretch.
% This doesn't show up in the poemlines, because they are self-contained, but it does show up in connected text.
% To avoid this, and have the font only changed within the argument, use \newcommand as above.
% Though you can also enclose \Tr in braces to limit it: {\Tr{}}

%----------------------------------------
%%% End of font definitions %%%
%----------------------------------------

%--------------------------------
%%% Colour definitions %%%
%--------------------------------

\definecolor{mygreen}{RGB}{0, 187, 50}

%----------------------------------------
%%% End of font definitions %%%
%----------------------------------------

  % Bring in the font definitions.

\usepackage{marginnote}
\renewcommand*{\marginfont}{\color{red}\sffamily}

\interfootnotelinepenalty=10000 % prevents the footnote from breaking across pages
% http://tex.stackexchange.com/questions/32208/footnote-runs-onto-second-page

% Thanks to Manas Tungare (http://manas.tungare.name/software/latex) for these settings.
\setlength{\paperwidth}{210mm}
\setlength{\textwidth}{160mm}
\setlength{\textheight}{247mm}

\setlength{\evensidemargin}{1in}
\setlength{\oddsidemargin}{0in}
\setlength{\topmargin}{-0.5in}

\renewcommand\thefootnote{\textcolor{red}{\arabic{footnote}}}  % Alter the colour of the footnote markers - thanks to Gonzalo Medina (http://tex.stackexchange.com/questions/26693/change-the-color-of-footnote-marker-in-latex#26696).

\usepackage{url}  % Use urls in text and captions with sensible linewrap.  Can't use [obeyspaces] - this option clashes with biblatex.
\urlstyle{rm}  % Set urls in roman.

\begin{document}

\begin{longtable}{rrl} 

\makebox[8cm][r]{} & & \makebox[8cm][r]{} \\ 

\textarabic{تَانْڠَ كْوَ لِينْڠِ شَؤُورِ} & \textarabic{لِئٖمْبٖيتٖ نَ مٗنْڠٗوتِ} & \textarabic{١} \\* 
\Tr{tanga kwa lingi shauri} & \Tr{Liembete na mongoti} & \Tr{1b/a} \\ 
\textarabic{إِمٖتُوَامَ بَهَارِ} & \textarabic{هَؤُتُكُوتِ أُكُوتِ} &  \\* 
\Tr{imetuwama bahari} & \Tr{Hautukuti ukuti} & \Tr{1d/c} \\ 
\textarabic{تُتَپَتَاءٖ بَنْدَارِ؟} & \textarabic{هَاتَ مَاءِ هَيَڤُوتِ} &  \\* 
\Tr{tutapatae bandari?} & \Tr{Hata mai hayavuti} & \Tr{1f/e} \\ 
\\[8mm] 

\textarabic{مَاءِ نْڠَمَانِ هُجِيرِ} & \textarabic{نْڠُرُودِ إِمٖشٗپٗوكَ} & \textarabic{٢} \\* 
\Tr{mai ngamani hujiri} & \Tr{Ngurudi imeshopoka} & \Tr{2b/a} \\ 
\textarabic{نَ هَاتَ كْوَ مِسُمَارِ} & \textarabic{هَئِتَاكِ كُزِبِيكَ} &  \\* 
\Tr{na hata kwa misumari} & \Tr{Haitaki kuzibika} & \Tr{2d/c} \\ 
\textarabic{مِكٗونٗ هُتُهَئِيرِ} & \textarabic{كُيَفُؤَ تُمٖچٗوكَ} &  \\* 
\Tr{mikono hutuhairi} & \Tr{Kuyafua tumechoka} & \Tr{2f/e} \\ 
\\[8mm] 

\textarabic{هَتُونَ تٖينَ شَؤُورِ} & \textarabic{تُمٖكٗوسَ تَرَتِيبُ} & \textarabic{٣} \\* 
\Tr{hatuna tena shauri} & \Tr{Tumekosa taratibu} & \Tr{3b/a} \\ 
\textarabic{هَكُپِيتِ مَنُوَارِ} & \textarabic{نَ بَنْدَارِ سِ كَرِيبُ} &  \\* 
\Tr{hakupiti manuwari} & \Tr{Na bandari si karibu} & \Tr{3d/c} \\ 
\textarabic{مٖتُزُنْڠُوكَ خَتَارِ} & \textarabic{زِمٖتُتَانْدَ ذَرُوبُ} &  \\* 
\Tr{metuzunguka khatari} & \Tr{Zimetutanda dharubu} & \Tr{3f/e} \\ 
\\[8mm] 

\textarabic{تُمٖشِنْدوَ كُفِكِيرِ} & \textarabic{هَتُئِيسِ لَ كُتٖينْدَ} & \textarabic{٤} \\* 
\Tr{tumeshindwa kufikiri} & \Tr{Hatuisi la kutenda} & \Tr{4b/a} \\ 
\textarabic{كٗوتٖ كِمٖپِيجَ دٗورِ} & \textarabic{كِيزَ كِينْڠِ كِمٖتَانْدَ} &  \\* 
\Tr{kote kimepija dori} & \Tr{Kiza kingi kimetanda} & \Tr{4d/c} \\ 
\textarabic{كَامَ هُونُ أُتِرِيرِ} & \textarabic{نِ هٖيرِ مْوَامْبَ كُپَانْدَ} &  \\* 
\Tr{kama hunu utiriri} & \Tr{Ni heri mwamba kupanda} & \Tr{4f/e} \\ 
\\[8mm] 

\textarabic{وَ كُمُؤٗومْبَ جَبَارِ} & \textarabic{هَتُونَ إِيلَ مَنَانِ} & \textarabic{٥} \\* 
\Tr{wa kumuomba Jabari} & \Tr{Hatuna ila Manani} & \Tr{5b/a} \\ 
\textarabic{يَپٗكُوَ كْوَ كِهٗورِ} & \textarabic{أَتوٖڠٖيشٖ نَاسِ پوَانِ} &  \\* 
\Tr{yapokuwa kwa kihori} & \Tr{Atwegeshe nasi pwani} & \Tr{5d/c} \\ 
\textarabic{وَاجَ وَاكٗ تُسِتِيرِ} & \textarabic{تُؤٗكٗوٖ رَهَمَانِ} &  \\* 
\Tr{waja wako tusitiri} & \Tr{Tuokowe Rahamani} & \Tr{5f/e} \\ 
\\[8mm] 

\textarabic{هِيلِ كَتِيكَ شَئِيرِ} & \textarabic{سُوَالِ سِ لَ جَهَازِ} & \textarabic{٦} \\* 
\Tr{hili katika shairi} & \Tr{Suwali si la jahazi} & \Tr{6b/a} \\ 
\textarabic{مَفُونْدِ مُولٗ هٗدَارِ} & \textarabic{لِتَمْبُوٖينِ وَيُوزِ} &  \\* 
\Tr{mafundi mulo hodari} & \Tr{Litambuweni wayuzi} & \Tr{6d/c} \\ 
\textarabic{مُزِزَمِئَايٗ دُورِ} & \textarabic{وَ مْتٗونِ وٖينْدٖ مْبِيزِ} &  \\* 
\Tr{muzizamiayo duri} & \Tr{Wa mtoni wende mbizi} & \Tr{6f/e} \\ 
\\[8mm] 

\end{longtable} 

\end{document}
