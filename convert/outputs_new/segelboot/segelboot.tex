\documentclass[a4paper, 12pt]{report}

\usepackage{polyglossia}
\usepackage{longtable}
\usepackage{xcolor}  % can't use color with polyglossia

\usepackage{marginnote}
\renewcommand*{\marginfont}{\color{red}\sffamily}

\interfootnotelinepenalty=10000 % prevents the footnote from breaking across pages
% http://tex.stackexchange.com/questions/32208/footnote-runs-onto-second-page

% Thanks to Manas Tungare (http://manas.tungare.name/software/latex) for these settings.
\setlength{\paperwidth}{210mm}
\setlength{\textwidth}{160mm}
\setlength{\textheight}{247mm}

\setlength{\evensidemargin}{1in}
\setlength{\oddsidemargin}{0in}
\setlength{\topmargin}{-0.5in}

\setmainfont[Mapping=tex-text]{Liberation Serif}  % Set the default font for the document (footnotes will also use this font, but see also: http://tex.stackexchange.com/questions/4779/how-to-change-font-family-in-footnote)

\setmainlanguage{english}
\setotherlanguage{arabic}
\setotherlanguage{farsi}  % used to allow typesetting the tile in a different font

\newfontfamily\arabicfont[Script=Arabic, Scale=2]{Scheherazade} % Arabic transcription layer
\newfontfamily\farsifont[Script=Arabic, Scale=2]{GranadaKD} % Arabic transcription layer for titles - uses a version of Granada which has been extended to include glyphs for Swahili.
\newcommand{\AS}[1]{\fontspec[Script=Arabic, Scale=1.5]{Scheherazade} #1\normalfont} % Arabic when used in footnotes - using \newfontfamily resets following English text as well.  IMPORTANT: this can ONLY be used for single words - multiple words will be written LTR, and not RTL as required.
\newfontfamily{\Tr}[Scale=0.9, Color=00BB33]{Linux Biolinum O} %  transliteration layer - Biolinum handles diacritics well
\newfontfamily{\I}[Scale=0.9, Color=blue]{Linux Biolinum O} % interpolated letters in the transliteration layer
\newfontfamily{\S}[Color=00BB33]{Linux Biolinum O} % standard spelling layer
\newfontfamily{\E}[Scale=0.9, Color=666666]{Liberation Serif Italic} % English translation layer
\newfontfamily{\FN}[Color=00BB33]{Liberation Serif Italic} % Alternate type in footnotes.

\renewcommand\thefootnote{\textcolor{red}{\arabic{footnote}}}  % Alter the colour of the footnote markers - thanks to Gonzalo Medina (http://tex.stackexchange.com/questions/26693/change-the-color-of-footnote-marker-in-latex#26696).

\usepackage[obeyspaces]{url}  % Use urls in text and captions with sensible linewrap.
\urlstyle{rm}  % Set urls in roman.

\begin{document}

\begin{flushright}

{\scriptsize\marginnote{1}[2mm]}\textarabic{أَنْڠَلِيَ وَڤُڤِ هَوَا} \\ 

\Tr{angaliya wavuvi hawā} \\ 

{\scriptsize\marginnote{2}[2mm]}\textarabic{أُپٖپٗ أُمٖرُفَعِ، نَ مَشُوَ} \\ 

\Tr{upepo umerufa'i, na mashuwa} \\ 

{\scriptsize\marginnote{3}[2mm]}\textarabic{يَاءٗ إِمٖسَلِيَ پَلٖ پَلٖ بِلَ} \\ 

\Tr{yao imesaliya pale pale bila} \\ 

{\scriptsize\marginnote{4}[2mm]}\textarabic{يَ كْوٖنْدَ، كْوَنِ هَونَ مَشِنِ} \\ 

\Tr{ya kwenḏa, kwani hawna mashini} \\ 

{\scriptsize\marginnote{5}[2mm]}\textarabic{نَ لَئِتِ وَنْڠٖلِكُوَ نَايٗ بَسِ} \\ 

\Tr{na laiṯi wangelikuwa nāyo basi} \\ 

{\scriptsize\marginnote{6}[2mm]}\textarabic{وَسِنْڠٖلِپٗتِزَ وَكَتِ وَاءٗ وٗتٖ} \\ 

\Tr{wasingelipoṯiza wakaṯi wao woṯe} \\ 

{\scriptsize\marginnote{7}[2mm]}\textarabic{وَاپٗ هَپَ هَپَ. كْوَ هِڤْيٗ} \\ 

\Tr{wāpo hapa hapa. kwa hivyo} \\ 

{\scriptsize\marginnote{8}[2mm]}\textarabic{مَشِنِ نْدَانِ يَ مَشُوَ ِنَ} \\ 

\Tr{mashini nḏāni ya mashuwa ina} \\ 

{\scriptsize\marginnote{9}[2mm]}\textarabic{كُئٗنْڠٖزَ پَاتٗ نَ أُچُمِ.} \\ 

\Tr{kuongeza pāṯo na uchumi.} \\ 

{\scriptsize\marginnote{10}[2mm]}\textarabic{يَپَسَ وَڤُڤِ وٗتٖ وَفُوَتٖ} \\ 

\Tr{yapasa wavuvi woṯe wafuwaṯe} \\ 

{\scriptsize\marginnote{11}[2mm]}\textarabic{مَئٖنْدٖلٖيٗ يَ وٖنْزَاءٗ نَ كْوَ} \\ 

\Tr{maenḏeleyo ya wenzao na kwa} \\ 

{\scriptsize\marginnote{12}[2mm]}\textarabic{كُفَنْيَ هِڤْيٗ هَپَنَ شَكَ وَتَفَئُلُ.} \\ 

\Tr{kufanya hivyo hapana shaka waṯafaulu.} \\ 

\end{flushright}

\end{document}
