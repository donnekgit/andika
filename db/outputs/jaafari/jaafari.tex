\documentclass[a4paper, 10pt]{report}

\usepackage{titlesec}  % Allow the chapter/section heading settings to be fine-tuned.  Needs to come before bidi, in polyglossia.
\usepackage{polyglossia}  % multilingual support
\usepackage{longtable}  % tables that carry across multiple pages
\usepackage[x11names]{xcolor}  % can't use color with polyglossia
% The following lines set up commands for superscript (SP) and subscript (SB), which can also be stacked (SPSB).
% These can be used to give the stanza number as appearing in the MS (which, because of human error, can be misnumbered).
% http://tex.stackexchange.com/questions/8255/superscript-and-subscript-together and http://tex.stackexchange.com/questions/146098/llap-or-rlap-at-the-begining-of-an-indented-paragraph
\def\SP#1{\textsuperscript{\E{#1}}}
\def\SB#1{\textsubscript{\E{#1}}}
\def\SPSB#1#2{\indent\rlap{\textsuperscript{\E{#1}}}\SB{#2}}

\usepackage{ulem}  % Allow dotted underlines.

\usepackage{pdfpages}

%--------------------------------
%%% Font definitions %%%
%--------------------------------
% Note that these definitions malfunction if used in \chapter.
\defaultfontfeatures{Mapping=tex-text}
\setmainfont{Charis SIL}  % Set the default font for the document. = \setdefaultfont
% Footnotes will by default also use this font -- http://tex.stackexchange.com/questions/4779/how-to-change-font-family-in-footnote).
\defaultfontfeatures{Scale=MatchLowercase}  % needs to be below main font declaration

\setsansfont{Liberation Sans}
\setmonofont{Liberation Mono}

\setmainlanguage{english}
\setotherlanguage{arabic}

\newfontfamily\arabicfont[Script=Arabic, Scale=2]{Scheherazade} % Arabic transcription -- coloured black, double size.
% One font needs to be called \arabicfont in order for XeTeX to load Arabic-related hyphenation and other stuff.
%  The default \textarabic will use this \arabicfont.  Use the \begin{Arabic} ..... \end{Arabic} environment for longer stretches (eg paras).
% Use \textarabic{\aemph{با}} to give overline emphasis.
% Omitting Script=Arabic for Amiri or Granada will mean that letters are written in their standalone forms, not connected.  (Omitting Script=Arabic for Scheherazade seems to cause no problem, though.)

\newfontfamily\citationfont[Script=Arabic, Scale=1.5]{Scheherazade}  % Citations, or stand-alone Arabic script in the middle of Roman script -- coloured black, one-and-a-half size.
\newcommand\AS[1]{{\citationfont\RLE{#1}}}
% \RLE (from the bidi package, which polyglossia loads automatically) is to allow multiple words of Arabic to be written right-to-left -- if omitted, each word in the sequence will be written RTL, but the sequence as a whole will be written LTR.

%You can either, as above, define a new \fontfamily, and then use it in a \newcommand, or you can, as below, include the font in the \newcommand by calling \fontspec directly.

\newcommand\Atitle[1]{{\fontspec[Script=Arabic, Scale=2]{GranadaKD}\RLE{#1}}}  % Arabic transcription for titles - uses a version of Granada which has been extended to include glyphs for Swahili.

\newcommand\Am[1]{{\fontspec[Script=Arabic]{Amiri}\RLE{#1}}} % Examples using Amiri --  if using Scheherazade's default scale, set Scale=0.8 here.

%\newfontfamily\translitfont[Scale=1, Color=666666]{Linux Biolinum O}
%\newcommand\Tr[1]{{\translitfont\RLE{#1}}}
\newcommand\Tr[1]{{\fontspec[Scale=1, Color=666666]{Linux Biolinum O}#1}}   %  Transliteration -- Biolinum handles diacritics well.  Coloured grey, slightly less than normal size.
% Scale=1 is required because of Scale=MatchLowercase - otherwise the size is too large.

\newcommand\In[1]{{\fontspec[Scale=1, Color=blue]{Linux Biolinum O}#1}}  % Epenthetic letters in the transliteration -- coloured blue, normal size.

\newcommand\Swa[1]{{\fontspec[Color=00BB33, Scale=1]{Linux Biolinum O}#1}}  % Standard spelling -- coloured green, normal size.

\newcommand\E[1]{{\fontspec[Scale=0.9, Color=333333]{Liberation Serif Italic}#1}}  % English translation layer -- coloured grey, slightly less than normal size.

\newcommand\Eit[1]{{\fontspec{Liberation Serif Italic}#1}}  % English italics.

\newcommand\FN[1]{{\fontspec[Color=00BB33]{Liberation Serif Italic}#1}} % Standout type in footnotes -- coloured green, normal size.

% Older versions:
% \newfontfamily{\Tr}[Scale=0.9, Color=00BB33]{Linux Biolinum O}
% This can be used as \Tr{text}.  But this will change the font outside the argument until the end of that stretch.
% This doesn't show up in the poemlines, because they are self-contained, but it does show up in connected text.
% To avoid this, and have the font only changed within the argument, use \newcommand as above.
% Though you can also enclose \Tr in braces to limit it: {\Tr{}}

%----------------------------------------
%%% End of font definitions %%%
%----------------------------------------
  % Bring in the font definitions.

\usepackage{marginnote}
\renewcommand*{\marginfont}{\color{red}\sffamily}

\usepackage{csquotes}
\newcommand{\q}[1]{\enquote{#1}}  % Mark quotes by using \q{text to be quoted}.
% \newcommand{\q}[1]{``#1''}  % Alternative when not using csquotes.

\usepackage[multiple]{footmisc}  % Adds commas between multiple footnotemarkers.

% \usepackage{natbib}
\usepackage[backend=biber, style=authoryear]{biblatex}
\addbibresource{bib/andika.bib}

\interfootnotelinepenalty=10000 % prevents the footnote from breaking across pages
% http://tex.stackexchange.com/questions/32208/footnote-runs-onto-second-page

% Thanks to Manas Tungare (http://manas.tungare.name/software/latex) for these settings.
\setlength{\paperwidth}{210mm}
\setlength{\paperheight}{297mm}

\setlength{\textwidth}{160mm}
\setlength{\textheight}{247mm}

\setlength{\evensidemargin}{1in}
\setlength{\oddsidemargin}{0in}
\setlength{\topmargin}{-0.5in}

\renewcommand\thefootnote{\textcolor{red}{\arabic{footnote}}}  % Alter the colour of the footnote markers - thanks to Gonzalo Medina (http://tex.stackexchange.com/questions/26693/change-the-color-of-footnote-marker-in-latex#26696).

\usepackage{url}  % Use urls in text and captions with sensible linewrap.  Can't use [obeyspaces] - this option clashes with biblatex.
% \urlstyle{rm}  % Set urls in roman.

\titleformat{\chapter}[display]{\normalfont\large}{\bfseries\chaptertitlename\ \thechapter}{10pt}{\large\itshape}[\vspace{2ex}\titlerule\vspace{2ex}]  % 10pt is the space between chapter and chapter name.  [display] sets the chapter and chapter name on separate lines.  The square brackets at the end draw a line under each chapter name, with 2ex gap from the name.
\titlespacing{\chapter}{0pt}{0pt}{10pt}  % First is indent from the side, second is length down from the top, third is gap between heading and text.

% \pagenumbering{gobble}  % Suppress page numbering.

% ===== Endnotes =====
% Uncomment the following two lines to get endnotes instead of footnotes.
% Remember to uncomment the three lines in andika/db/output_pdf.php as well.
% \usepackage{endnotes}
% \let\footnote=\endnote
% ==================

\begin{document}

\begin{center}
\Atitle{أُتٖنْزِ وَ جَعْفَر} \\
\Tr{uṯēnzi wa ja'far} \\
\E{The Ballad of Ja'far} \\
[5mm]
\textcolor{red}{\AS{بِسْمِ اللّٰهِ الرَحمَنِ الرَّحِيْمِ}} \\
\Tr{bismillähi ar-rahmani ar-rahīmi} \\
\E{In the name of God, the Compassionate, the Merciful}
\end{center}




\begin{center} 

\textarabic{(١) \textcolor{mygreen}{بِسْمِ اللّٰهِ إِخْوَانِ  * پَمْوٖ نَ رَحْمٰنِ  * نَ الرَّحِيْمُ يُوَنِ  * نْدِيٗ يَلٗأَنْدَمِيَ }} \\* 
 \OLTcl{ yaloandamiya ndiyo *  yuwani rraḥı̄mu na *  raḥmäni na pamwe *  ikhwāni llähi bismi} \\* 
\SB{1} (\textbf{1}) \OLTst{bismillahi  ihwani  * pamwe na rahmani  * na rahimu yuwani  * ndiyo yaloandamia } \\ 
\E{"In the name of God", my friends   along with "the Compassionate"   and "the Merciful" -- know   that that is what goes first.  } \\ 
\\[8mm] 

\textarabic{(٢) \textcolor{mygreen}{پُلِكَنِ نْدُزَنْڠُ  * كهٖنْدَ مَتٖمْبٖزِ يَنْڠُ  * نِوَپٖ خَبَرِ زَنْڠُ  * قِصَ چَلٗنِجِرِيَ }} \\* 
 \OLTcl{ chalonijiriya qiṣa *  zangu khabari niwape *  yangu matembezi kʰenda *  nduzangu pulikani} \\* 
\SB{2} (\textbf{2}) \OLTst{pulikani nduzangu  * kenda matembezi yangu  * niwape habari zangu  * qisa chalonijiria } \\ 
\E{Listen, my brothers,  I went on a journey.   Let me give you my story,    an account of what happened to me.  } \\ 
\\[8mm] 

\textarabic{(٣) \textcolor{mygreen}{نِلِ نٖنْدَءٗ نْدِيَنِ  * كْوَلِ نَ مْٹُ چُمْبَنِ  * مٗيٗ أُكَمْتَمَنِ  * كْوَ حَلَالِ كُرِضِيَ }} \\* 
 \OLTcl{ kuriḍiya ḥalāli kwa *  ukamtamani moyo *  chumbani mţu na kwali *  ndiyani nendao nili} \\* 
\SB{3} (\textbf{3}) \OLTst{nili nendao ndiani  * kwali na mtu\footnote{The woman's name is Atika, but we are not told this until 274d.} chumbani\footnote{Atika probably went indoors to hide from Ali.  The custom is for  women to hide from men of their own status.  Therefore, if they hide when they see a man coming, the man will be pleased, because it is a compliment to him to be considered of noble status.  On the other hand, if the woman does not hide, the man may be angry, because he will think she is looking down on him. Thus, women will not hide from lascivious people, but only from those who aren't, because they are worthy of more respect.}  * moyo ukamtamani  * kwa halali \dotuline{karidhia}\footnote{Ali did not want to sin by committing adultery with her, so he decides \Swa{kuoa kwa siri}, \E{to marry in secret}.  Among the Swahili it is possible to have an \Swa{mke wa siri}, \E{secret wife}, if it is thought that the first wife or other people would object.  An \Swa{mke wa siri} has all the rights of an ordinary wife, except that the marriage is not publicised.} } \\ 
\E{I was going along the road,   and there was a person in a room,    and my heart desired her,  and I was gratified lawfully.   } \\ 
\\[8mm] 

\textarabic{(٤) \textcolor{mygreen}{نَ مَهَرِيٖ يُوَنِ  * نَلٗمْپَ زَيْدَنِ  * نَلِنَ پٖٹٖ چَنْدَنِ  * كَوَهِ كُمْڤَلِيَ }} \\* 
 \OLTcl{ kumvaliya kawahi *  chandani peţe nalina *  zaydani nalompa *  yuwani mahariye na} \\* 
\SB{4} (\textbf{4}) \OLTst{na mahariye yuani  * nalompa zaidani  * nalina pete\footnote{This ring is an important factor in the rest of the story, because it had been given to Ali by his wife Fatima.} chandani  * kawahi kumvalia\footnote{Perhaps emend to \Swa{kumwatiya}, i.e. I succeeded in leaving it with her.} } \\ 
\E{ And know that as for her dowry,  which I gave her as well,   I had a ring on my finger,   and I persuaded her to wear it.  } \\ 
\\[8mm] 

\textarabic{(٥) \textcolor{mygreen}{عَلِى كُتٗكَ كْوَكٖ  * أُنَ رُبَ مٗيٗ وَكٖ  * أَكٖنْدَ كْوَ مْكٖ وَكٖ  * مْكٗنٗ هُمْفُمْبِيَ }} \\* 
 \OLTcl{ humfumbiya mkono *  wake mke kwa akenda *  wake moyo ruba una *  kwake kutoka ʿalii} \\* 
\SB{5} (\textbf{5}) \OLTst{Ali kutoka kwake  * una ruba moyo wake  * akenda kwa mke wake  * mkono humfumbia\footnote{i.e. to hide the fact that he was not wearing the ring.} } \\ 
\E{When Ali left [Atiya]   his heart was troubled --    when he went to his wife [Fatima],    he hid his hand from her.  } \\ 
\\[8mm] 

\textarabic{(٦) \textcolor{mygreen}{مْوَنَ بِنْتِ رَسُوْلِ  * يَلِ هَيَتَأَمَلِ  * كَنٖنْدَ مْوٖنْيٖ عَقِلِ  * خَبَرِ أَكَمْوَمْبِيَ }} \\* 
 \OLTcl{ akamwambiya khabari *  ʿaqili mwenye kanenda *  hayataamali yali *  rasūli binti mwana} \\* 
\SB{6} (\textbf{6}) \OLTst{Mwana binti Rasuli\footnote{i.e. Fatima.}  * yali hayataamali\footnote{\Swa{-taamali}, \E{observe}.}  * kanenda mwenye 'aqili  * habari akamwambia } \\ 
\E{The Lady daughter of the Prophet   was unaware of these things  [until] someone in the know went   and told her the news.  } \\ 
\\[8mm] 

\textarabic{(٧) \textcolor{mygreen}{أَكٖنٖنْدَءٖ كِزٖيْ  * فَاطِمَه أَكَمْوَمْبِيٖ  * شٖيْخٖ عَلِيْ أُوٗزٖيْ  * خَبَرِ هُكِسِكِيَ }} \\* 
 \OLTcl{ hukisikiya khabari *  uwozee ʿalii shēkhe *  akamwambiye fāṭimah *  kizee akenendae} \\* 
\SB{7} (\textbf{7}) \OLTst{akenendae kizee  * Fatima akamwambie  * Shehe Ali uozee  * habari hukisikia } \\ 
\E{The person who went was an old woman,  and she said to Fatima:  Sheikh Ali has got married --   have you heard the news?  } \\ 
\\[8mm] 

\textarabic{(٨) \textcolor{mygreen}{أَكَفَنْيَ مْشَوَشَ  * فَاطِمَه كَمْكَنُشَ  * عَلَامَ كَمُؤٗنٖشَ  * أُتُنْڠُ أَكَمْٹِيَ }} \\* 
 \OLTcl{ akamţiya utungu *  kamuonesha ʿalāma *  kamkanusha fāṭimah *  mshawasha akafanya} \\* 
\SB{8} (\textbf{8}) \OLTst{akafanya mshawasha  * Fatima kamkanusha  * 'alama kamuonyesha  * utungu akamtia } \\ 
\E{She tried to convince Fatima,  but Fatima refused to believe her,  but [the old woman] gave her proof,  and made her worried.  } \\ 
\\[8mm] 

\textarabic{(٩) \textcolor{mygreen}{مْوَنَ بِنْتِ أَمِيْنِ  * هِيٗ نْدِيٗ تَمْكِنِ  * أَلِ نَ پٖٹٖ چَنْدَنِ  * كَوَهِ كُمْڤُلِيَ }} \\* 
 \OLTcl{ kumvuliya kawahi *  chandani peţe na ali *  tamkini ndiyo hiyo *  amı̄ni binti mwana} \\* 
\SB{9} (\textbf{9}) \OLTst{Mwana binti Amini\footnote{The Prophet was called \Eit{al-Amin} from his teenage years onward, because he was reliable and even-handed to all.}  * hiyo ndiyo tamkini\footnote{\Swa{tamkini = hakika, kweli}}  * ali na pete chandani  * kawahi kumvulia } \\ 
\E{Lady, daughter of the Trustworthy One,    [said the old woman], this is true.   He had a ring on his finger,     and he took it off [to leave it] with her."  } \\ 
\\[8mm] 

\textarabic{(١٠) \textcolor{mygreen}{سِجُوِ كهَنٖنَ نِ كوٖلِ  * مِمِ سِمتَأَمَلِ  * أَرُدِپٗ كْوَ رَسُوْلِ  * يٖئٗ تهَمْزِنْڠَتِيَ }} \\* 
 \OLTcl{ tʰamzingatiya yeo *  rasūli kwa arudipo *  simtaamali mimi *  kweli ni kʰanena sijuwi} \\* 
\SB{10} (\textbf{10}) \OLTst{sijui kanena ni kweli  * mimi simtaamali  * arudipo kwa Rasuli  * yeo tamzingatia } \\ 
\E{I don't know, said [Fatima], if that is true.    I didn't pay him any heed.  When he comes back from the Prophet's   today, I will ask him.  } \\ 
\\[8mm] 

\textarabic{(١١) \textcolor{mygreen}{أَكَفَنْيَ هِمَ هِمَ  * أَسِئِوٖزٖ فَاطِمَه  * حَسَنِ أَكَمْٹُمَ  * بَبَكٗ نَمْكُلِيَ }} \\* 
 \OLTcl{ namkuliya babako *  akamţuma ḥasani *  fāṭimah asiiweze *  hima hima akafanya} \\* 
\SB{11} (\textbf{11}) \OLTst{akafanya hima hima  * asiiweze\footnote{\Swa{hakuweza kustahimili}.} Fatima  * Hasani akamtuma  * babako namkulia\footnote{N. \Swa{-amkulia} = S. \Swa{-itia}} } \\ 
\E{But Fatima then acted immediately --   she could not restrain herself.  She sent Hasan, [saying:]  "I am summoning your father."  } \\ 
\\[8mm] 

\textarabic{(١٢) \textcolor{mygreen}{نَمْكُلِيَ كْوَ سِرِ  * أَسِسِكِيٖ بَشِيْرِ  * كٖنٖنْدَ أَكَفَسِرِ  * مْبٖلٖ زَ ٹُمْوَ نَبِيَ }} \\* 
 \OLTcl{ nabiya ţumwa za mbele *  akafasiri kenenda *  bashı̄ri asisikiye *  siri kwa namkuliya} \\* 
\SB{12} (\textbf{12}) \OLTst{namkulia kwa siri  * asisikie Bashiri  * kenenda akafasiri\footnote{\Swa{-fasiri} lit. means \q{explain}.}  * mbele za Tumwa Nabia\footnote{In other words, the child forgot to give the message privately (\Swa{hakusema kwa siri}).  In any case, for a man to be called away from the \Swa{baraza} by a message from home is very worrying, since it implies some emergency for which his presence is required.} } \\ 
\E{I am summoning him secretly,    so that the Bearer of Good Tidings does not hear.  [But Hasan] went and spoke [openly],  in front of the Messenger, the Prophet.    } \\ 
\\[8mm] 

\textarabic{(١٣) \textcolor{mygreen}{كَفَسِرِ مُعَيَنِ  * هَمْكُلِوَ نْيُمْبَنِ  * أَكِسِكِيَ أَمِيْنِ  * عَجَبُ إِكَمْنْڠِيَ }} \\* 
 \OLTcl{ ikamngiya ʿajabu *  amı̄ni akisikiya *  nyumbani hamkuliwa *  muʿayani kafasiri} \\* 
\SB{13} (\textbf{13}) \OLTst{kafasiri mu'ayani  * hamkuliwa nyumbani  * akisikia Amini  * 'ajabu ikamngiya } \\ 
\E{He addressed [Ali] openly [saying:]  You are wanted at home.  When the Trustworthy One heard this  he was filled with wonder.  } \\ 
\\[8mm] 

\textarabic{(١٤) \textcolor{mygreen}{أَكَمُؤُزَ هَشِمَ  * يٖئٗ أُنَنِ فَاطِمَه  * هُكْوَمْكُوَ كْوَ هِمَ  * نَايٗ سِيٗ مَزٗوٖيَ }} \\* 
 \OLTcl{ mazoweya siyo nāyo *  hima kwa hukwamkuwa *  fāṭimah unani yeo *  hashima akamuuza} \\* 
\SB{14} (\textbf{14}) \OLTst{akamuuza Hashima\footnote{The Prophet belonged to the clan of Hashim in the tribe of Quraysh of the Hollow.}  * yeo una-ni Fatima  * hukwamkuwa kwa hima  * nayo siyo mazoea } \\ 
\E{The Hashimite asked [Ali]:  What is the matter with Fatima today?   She wants you in a hurry,    and that is not like her.   } \\ 
\\[8mm] 

\textarabic{(١٥) \textcolor{mygreen}{عَلِى أَكَئِنُكَ  * أَكَنٖنْدَ كْوَ هَرَكَ  * هَتَ نْيُمْبَنِ كِفِكَ  * مْوَنَ فَاطِمَ هُلِيَ }} \\* 
 \OLTcl{ huliya fāṭima mwana *  kifika nyumbani hata *  haraka kwa akanenda *  akainuka ʿalii} \\* 
\SB{15} (\textbf{15}) \OLTst{'alii akainuka  * akanenda kwa haraka  * hata nyumbani kifika  * mwana fatima huliya } \\ 
\E{Ali got up  and went in haste   until he arrived home.   Lady Fatima was crying.   } \\ 
\\[8mm] 

\textarabic{(١٦) \textcolor{mygreen}{هُرُمَ زِكَمْشِكَ  * عَلِيْ كَشَوِشِكَ  * كَمْبَ فَتُمَ پُلِكَ  * أُلِلٗنَلٗ نَمْبِيَ }} \\* 
 \OLTcl{ nambiya ulilonalo *  pulika fatuma kamba *  kashawishika ʿalii *  zikamshika huruma} \\* 
\SB{16} (\textbf{16}) \OLTst{huruma zikamshika  * 'alii kashawishika\footnote{\Swa{-shawishika} = \Swa{-fanya wasiwasi, -fanya tashwish}}  * kamba fatuma pulika  * ulilonalo nambiya } \\ 
\E{Ali was seized with pity,  and became perplexed.  He said: Fatima, listen --    tell me what's wrong with you.  } \\ 
\\[8mm] 

\textarabic{(١٧) \textcolor{mygreen}{أُنَ كِتْوَ هُكُؤُمَ  * أَمَ أُمٖشِكْوَ نِ حُمَ  * أَكَمْجِبُ فَتُمَ  * كْوَ مَرَضِ سِكُلِيَ }} \\* 
 \OLTcl{ sikuliya maraḍi kwa *  fatuma akamjibu *  ḥuma ni umeshikwa ama *  hukuuma kitwa una} \\* 
\SB{17} (\textbf{17}) \OLTst{una kitwa hukuuma  * ama umeshikwa \dotuline{na} huma  * akamjibu fatuma  * kwa maradhi sikuliya } \\ 
\E{Do you have a headache,   or have you a temperature?    And Fatima replied:  I am not crying because I am ill.   } \\ 
\\[8mm] 

\textarabic{(١٨) \textcolor{mygreen}{مِمِ هَپَ نَلِتٗكَ  * وٖوٖ هُنٖنَ هُتٖكَ  * نَ كَمَ كهِٹُ وَتَكَ  * أُسِخٗفُ كُنَمْبِيَ }} \\* 
 \OLTcl{ kunambiya usikhofu *  wataka kʰiţu kama na *  huteka hunena wewe *  nalitoka hapa mimi} \\* 
\SB{18} (\textbf{18}) \OLTst{mimi hapa nalitoka  * wewe hunena huteka\footnote{This is a common expression meaning \q{you were in a good mood}.}  * na kama kitu wataka  * usihofu kunambiya } \\ 
\E{[Ali said:] When I left here   you were talking and laughing.   And if there's anything you want,    don't be afraid to ask me [for it].  } \\ 
\\[8mm] 

\textarabic{(١٩) \textcolor{mygreen}{وَتَكَ نِنِ نْدُيَنْڠُ  * نِئِفِدِ رٗحٗ يَنْڠُ  * مٗيٗ أُوَتٖ مَتُنْڠُ  * نَوٖ أُكٗمٖ كُلِيَ }} \\* 
 \OLTcl{ kuliya ukome nawe *  matungu uwate moyo *  yangu roḥo niifidi *  nduyangu nini wataka} \\* 
\SB{19} (\textbf{19}) \OLTst{wataka nini nduyangu  * niifidi\footnote{\Swa{niifidi} =  \Swa{niokowe}.  Therefore, lit., \q{so that I may save my soul, so that I will not be in distress}.  This expresses humility, and shows that the speaker cares very much about the other person.} roho yangu  * moyo uwate matungu  * nawe ukome kuliya } \\ 
\E{What do you want, my dear,    so that I may console you,   so that your heart will leave aside pain,    so that you will stop crying?   } \\ 
\\[8mm] 

\textarabic{(٢٠) \textcolor{mygreen}{فَتُمَ أَكَتَمْكَ  * پٖٹٖ يَكٗ نَئِتَكَ  * عَلِى أَكَشُٹُكَ  * هَؤٗنِ لَ كُمْوَمْبِيَ }} \\* 
 \OLTcl{ kumwambiya la haoni *  akashuţuka ʿalii *  naitaka yako peţe *  akatamka fatuma} \\* 
\SB{20} (\textbf{20}) \OLTst{fatuma akatamka  * pete yako naitaka  * 'alii akashutuka  * haoni la kumwambiya } \\ 
\E{Fatima replied:  I want your ring.   Ali was startled --  he could not see what he could tell her.   } \\ 
\\[8mm] 

\textarabic{(٢١) \textcolor{mygreen}{كِشَ عَلِيْ حَيْدَرِ  * نٖنٗ أَلِلٗفَسِرِ  * أَنَيٗ أَبُوْ بَكَرِ  * تهَكْوٖنْدَ كُٹْوَلِيَ }} \\* 
 \OLTcl{ kuţwaliya tʰakwenda *  bakari abuu anayo *  alilofasiri neno *  ḥaydari ʿalii kisha} \\* 
\SB{21} (\textbf{21}) \OLTst{kisha 'alii haydari\footnote{\Swa{haidari}, \E{lion}.  The epithet, \q{lion-like}, is so closely associated with Ali that it is now more of a name than a descriptive metaphor.}  * neno alilofasiri  * anayo abuu bakari  * takwenda kutwaliya\footnote{Ali tells this lie to gain some time, but it doesn't work.} } \\ 
\E{Then Ali the Lion-like,   the utterance that he spoke [was this:]  Abu Bakr has it --   I'll go and fetch it.  } \\ 
\\[8mm] 

\textarabic{(٢٢) \textcolor{mygreen}{پٖٹٖ يَكٗ يَ چَنْدَنِ  * أَبُوْ بَكَرِ سِ كِفَنِ  * نِمٖٹْوَاءَ تَمْكِنِ  * كْوَ أُلِيٗنَمْبِيَ }} \\* 
 \OLTcl{ uliyonambiya kwa *  tamkini nimeţwaa *  kifani si bakari abuu *  chandani ya yako peţe} \\* 
\SB{22} (\textbf{22}) \OLTst{pete yako ya chandani  * abuu bakari si kifani  * nimetwaa tamkini  * kwa uliyonambiya } \\ 
\E{[But Fatima said:] "Your ring is for [your] finger --    it will not fit Abu Bakr.    "I have discovered the real reason  for what you told me.  } \\ 
\\[8mm] 

\textarabic{(٢٣) \textcolor{mygreen}{هُنُ مْوٖزِ نِ وَ پِلِ  * مِمِ سِكُتَأَمَلِ  * أُنَ نَ مْكٖ وَ پِلِ  * هَبَرِ نِمٖسِكِيَ }} \\* 
 \OLTcl{ nimesikiya habari *  pili wa mke na una *  sikutaamali mimi *  pili wa ni mwezi hunu} \\* 
\SB{21} (\textbf{23}) \OLTst{hunu mwezi ni wa pili  * mimi sikutaamali  * una na mke wa pili  * habari nimesikiya } \\ 
\E{"This is the second month [that you have not worn it] --     I did not realise this before.  You have a second wife --     I have heard the news.  } \\ 
\\[8mm] 

\textarabic{(٢٤) \textcolor{mygreen}{نَأَپَ وَاللّٰهِ نْدُيَنْڠُ  * نِ وٖيْ پْوٖكٖ مْكٖ وَنْڠُ  * نِ نَنِ عَدُوِ يَنْڠُ  * هُيٗ أَلٗنِزُلِيَ }} \\* 
 \OLTcl{ alonizuliya huyo *  yangu ʿaduwi nani ni *  wangu mke pweke wee ni *  nduyangu wallähi naapa} \\* 
\SB{22} (\textbf{24}) \OLTst{naapa wallahi nduyangu  * ni wee pweke mke wangu\footnote{This is, at best, only half-true.  Note that \Swa{kusema urongo asitete ni vizuri, si vibaya} -- that is, it is justifiable to tell white lies to console your spouse, or in the interests of conciliation and marital harmony.  For instance, once a husband arrived home late accompanied by a friend.  The wife asked the husband why he was late and became suspicious when he did not reply.  The friend saw trouble brewing and stepped in with the lie that the husband had been seeing about getting some money for her as a present, which of course placated her.}  * ni nani 'aduwi yangu  * huyo alonizuliya } \\ 
\E{[Ali said:] I swear to God, dearest --   you are my only wife.     Who is this enemy of mine    who has told you this lie about me?"  } \\ 
\\[8mm] 

\textarabic{(٢٥) \textcolor{mygreen}{كَمْجِبُ كْوَ لِسَنِ  * مْٹُيٖ سِمْبَئِنِ  * پٖٹٖ أُمْپٖيْ نَنِ  * أُنِپَپٗ تهَرِضِيَ }} \\* 
 \OLTcl{ tʰariḍiya unipapo *  nani umpee peţe *  simbaini mţuye *  lisani kwa kamjibu} \\* 
\SB{23} (\textbf{25}) \OLTst{kamjibu kwa lisani\footnote{\Swa{lisani} perhaps < \AS{لسن}, \E{speak eloquently}, note also \AS{لسان}, \E{tongue} and \AS{لسانى}, \E{orally}.  We could also emend to \Swa{hisani}, \E{kindness, goodness}, i.e. politely.  See 250a.}  * mtuye simbaini  * pete umpee nani  * unipapo taridhiya } \\ 
\E{She replied eloquently:   I will not disclose that person.  Who have you given the ring to?   [Only] when you give [it to me] will I be satisfied.  } \\ 
\\[8mm] 

\textarabic{(٢٦) \textcolor{mygreen}{عَلِى أَكَبَئِنِ  * تهَكْوَمْبِيَ تَمْكِنِ  * يَلِنْڠِيَ كِسِمَنِ  * تهَكْوٖنْدَ كُكُتٗلٖيَ }} \\* 
 \OLTcl{ kukutoleya tʰakwenda *  kisimani yalingiya *  tamkini tʰakwambiya *  akabaini ʿalii} \\* 
\SB{24} (\textbf{26}) \OLTst{'alii akabaini  * takwambiya tamkini  * yalingiya kisimani\footnote{Another lie, again to gain some time, \Swa{kamuulize Nabiya}, \E{so that he can go and ask the Prophet}.  But Fatima does not fall for this one either.}  * takwenda kukutoleya } \\ 
\E{Ali declared:  I will tell you what really happened.  It fell into a well --  I'l go and get it out for you.  } \\ 
\\[8mm] 

\textarabic{(٢٧) \textcolor{mygreen}{هَيَ نٖنَ پٖٹٖ يَكٗ  * مَهَلٖ أُئِوٖسٖپٗ  * كَمَ هَيَ مَتَمْكٗ  * بَبَنْڠُ سِتٗمْوَمْبِيَ }} \\* 
 \OLTcl{ sitomwambiya babangu *  matamko haya kama *  uiwesepo mahale *  yako peţe nena haya} \\* 
\SB{25} (\textbf{27}) \OLTst{haya nena\footnote{Or we could emend to \Swa{huyanena}, \E{you still haven't said}.} pete yako  * mahale uiwesepo  * kama haya matamko  * babangu sitomwambiya\footnote{\Swa{anamwogopa sana} -- she is greatly in awe of him.} } \\ 
\E{[Fatima said:] Well, your ring -- say    where you have put it,  for these words --   I do not [want to] tell my father [about them]."  } \\ 
\\[8mm] 

\textarabic{(٢٨) \textcolor{mygreen}{عَلِيْ أَكَؤُذِكَ  * فَتُمَ كُكَسِرِكَ  * أَلِئِنُكَ كَتٗكَ  * أَكَنٖنْدَ كْوَ نَبِيَ }} \\* 
 \OLTcl{ nabiya kwa akanenda *  katoka aliinuka *  kukasirika fatuma *  akaudhika ʿalii} \\* 
\SB{26} (\textbf{28}) \OLTst{'alii akaudhika  * fatuma kukasirika  * aliinuka katoka  * akanenda kwa nabiya } \\ 
\E{Ali was worried  that Fatima was angry.  He got up and left,  and went to the Prophet.   } \\ 
\\[8mm] 

\textarabic{(٢٩) \textcolor{mygreen}{أَكَمُؤُزَ هَشِمَ  * أَلِ نَ نِنِ فَتُمَ  * أَلِكهَاٹَ كَلِمَ  * مْٹُمٖ أَكَمْوَمْبِيَ }} \\* 
 \OLTcl{ akamwambiya mţume *  kalima alikʰāţa *  fatuma nini na ali *  hashima akamuuza} \\* 
\SB{27} (\textbf{29}) \OLTst{akamuuza hashima  * ali na nini\footnote{The northern form of \Swa{alikuwa na nini}, based respectively on two verbs meaning \q{be}: \Swa{-li} and \Swa{-wa}.} fatuma  * alikata kalima\footnote{lit. \q{cut the words}.}  * mtume akamwambiya } \\ 
\E{The Hashimite asked him:  What was wrong with Fatima?    [Ali] interrupted him,  and told the Prophet:  } \\ 
\\[8mm] 

\textarabic{(٣٠) \textcolor{mygreen}{أَكَمْوَمْبِيَ شَرِيْفُ  * فَتُمَ مٖنِكَلِفُ  * زِيَپٗ زِسِزٗ خٗوْفُ  * زِنْڠِ نِمٖمْٹِلِيَ }} \\* 
 \OLTcl{ nimemţiliya zingi *  khōfu zisizo ziyapo *  menikalifu fatuma *  sharı̄fu akamwambiya} \\* 
\SB{28} (\textbf{30}) \OLTst{akamwambiya sharifu  * fatuma menikalifu  * ziyapo\footnote{\Swa{-apa}, \E{curse}, but \Swa{-tia kiapo}, \E{swear an oath}, such as \Swa{nife nili kaffir}, \E{may I die an unbeliever}.  Ali has sworn a few oaths to Fatima out of necessity, but he is not bound by them because he swore them in order to preserve marital harmony.} zisizo\footnote{i.e. oaths that have no frightening consequences.} hofu  * zingi nimemtiliya } \\ 
\E{He told the Noble One:  Fatima is annoyed with me --  white lies,   I have told her many of them.  } \\ 
\\[8mm] 

\textarabic{(٣١) \textcolor{mygreen}{أَلِكِلِيَ يَا رَسُوْلِ  * نَمِ نِمٖٹٖنْدَ كْوٖلِ  * كُمْوَمْبِيَ سِهِمِلِ  * خٗوْفُ زِمٖنِنْڠِيَ }} \\* 
 \OLTcl{ zimeningiya khōfu *  sihimili kumwambiya *  kweli nimeţenda nami *  rasūli yā alikiliya} \\* 
\SB{29} (\textbf{31}) \OLTst{alikiliya\footnote{\Swa{-liki-} is a past continuous tense.  See also 48d, 49a.} ya rasuli  * nami nimetenda kweli  * kumwambiya sihimili  * hofu zimeningiya } \\ 
\E{[Ali] was weeping: Oh Prophet!   I've really done it now.   I haven't the nerve to tell her.  I'm filled with fear.  } \\ 
\\[8mm] 

\textarabic{(٣٢) \textcolor{mygreen}{تٖنَ إٖنٖنْدَ سَيِدِ  * أُمُؤٗنْيٖ أَبُرُدِ  * كَئِنُكَ مُحَمَدِ  * هَپٗ كَأَنْدَمَ نْدِيَ }} \\* 
 \OLTcl{ ndiya kaandama hapo *  muḥamadi kainuka *  aburudi umuonye *  sayidi enenda tena} \\* 
\SB{30} (\textbf{32}) \OLTst{tena enenda sayidi  * umuonye aburudi\footnote{He is speaking as if \Swa{iko moto nyumbani}, \E{there is a fire at home}, and he wants Fatima to \Swa{apowe}, \E{cool down}.
 }  * kainuka muhamadi  * hapo kaandama ndiya } \\ 
\E{So you go, my Lord,   and tell her to calm down.  Muhammad got up,  and then set out on the way [to Ali's house].   } \\ 
\\[8mm] 

\textarabic{(٣٣) \textcolor{mygreen}{أَكٖنْدَ أَكَجِلِسِ  * مٗيٗ أُنَ وَسِوَسِ  * كَئِفَنْيَ كَمَ هَئِسِ  * إٖنٖنْدٖيْ كُمْوَنْڠَلِيَ }} \\* 
 \OLTcl{ kumwangaliya enendee *  haisi kama kaifanya *  wasiwasi una moyo *  akajilisi akenda} \\* 
\SB{31} (\textbf{33}) \OLTst{akenda akajilisi\footnote{\Swa{-jilisi} = \Swa{-keti}.  The Prophet is being subtle, and pretends he does not know what has happened, so that he can get to the bottom of things.}  * moyo una wasiwasi  * kaifanya kama haisi  * \dotuline{endee} kumwangaliya } \\ 
\E{He went [to the house] and sat down.  His heart was confused,   He pretended he knew nothing,   [that] he had just come to look in on her.  } \\ 
\\[8mm] 

\textarabic{(٣٤) \textcolor{mygreen}{كَمُؤُلِزَ هَشِمَ  * وَلِ نَ نِنِ فَتُمَ  * حَسَنِ مٖكُيَ هِمَ  * عَلِيْ كُمْوَنْدَمِيَ }} \\* 
 \OLTcl{ kumwandamiya ʿalii *  hima mekuya ḥasani *  fatuma nini na wali *  hashima kamuuliza} \\* 
\SB{32} (\textbf{34}) \OLTst{kamuuliza hashima  * wali na nini fatuma  * hasani mekuya hima  * 'alii kumwandamiya\footnote{\Swa{-andama}, \E{follow someone directly, the two of you together}, but \Swa{-andamia}, \E{follow someone who has already gone ahead, so that each person is travelling alone}, = \Swa{-fuatia}.} } \\ 
\E{The Hashimite asked her:  Was there anything wrong, Fatima?    Hasan came [to us] in a hurry   to fetch Ali.  } \\ 
\\[8mm] 

\textarabic{(٣٥) \textcolor{mygreen}{فَتُمَ هَكُكَسِرِ  * أَكَمْوَمْبِيَ بَشِيْرِ  * عَلِيْ نْدَكٖ هَبَرِ  * يٖؤٗ نِمٖزِسِكِيَ }} \\* 
 \OLTcl{ nimezisikiya yeo *  habari ndake ʿalii *  bashı̄ri akamwambiya *  hakukasiri fatuma} \\* 
\SB{33} (\textbf{35}) \OLTst{fatuma hakukasiri\footnote{i.e. \Swa{pale pale kampa habari yote}, \E{there and then she gave him the whole story}.}  * akamwambiya bashiri  * 'alii \dotuline{zake} habari  * yeo nimezisikiya } \\ 
\E{Fatima did not hesitate,  she told the Bearer of Glad Tidings:  The news about Ali --   I heard it today.  } \\ 
\\[8mm] 

\textarabic{(٣٦) \textcolor{mygreen}{أَكَمُؤُزَ أَمِيْنِ  * نِ كَمَ هَبَرِ ڠَنِ  * أُسِمْنْڠٗجٖ نْيُمْبَنِ  * أُكَجَ أُكَمُأَمْكُوَ }} \\* 
 \OLTcl{ ukamuamkuwa ukaja *  nyumbani usimngoje *  gani habari kama ni *  amı̄ni akamuuza} \\* 
\SB{34} (\textbf{36}) \OLTst{akamuuza amini  * ni kama habari gani  * usimngoje nyumbani  * ukaja ukamuamkuwa } \\ 
\E{The Trustworthy One asked her:  What sort of news is it    that you couldn't wait for him at home,  and ended up sending for him?  } \\ 
\\[8mm] 

\textarabic{(٣٧) \textcolor{mygreen}{أَكَئِنَمَ فَتُمَ  * كُمْسِتَحِ هَشِمَ  * يَلٗمْتٗكَ كَلِمَ  * بَبَكٖ أَكَمْوَمْبِيَ }} \\* 
 \OLTcl{ akamwambiya babake *  kalima yalomtoka *  hashima kumsitaḥi *  fatuma akainama} \\* 
\SB{35} (\textbf{37}) \OLTst{akainama\footnote{Good children are shy in front of their parents, and show them respect.} fatuma  * kumsitahi hashima  * yalomtoka kalima\footnote{\Swa{ametokwa na maneno} -- this occurs when one feels particularly when one feels strongly about something.  If you want to emphasise a speaker's volubility, you can say \Swa{ametokwa na maneno yake na ya kuwazimu}, lit. \E{he was come out of by his own words and those of his ancestors}.}  * babake akamwambiya } \\ 
\E{Fatima bowed down  to show honour to the Hashimite,  and words came tumbling out,   and she explained [everything] to her father.  } \\ 
\\[8mm] 

\textarabic{(٣٨) \textcolor{mygreen}{يَوَتٖ مَنٖنٗ هَيٗ  * أُسِٹٖٹٖ نَ مُمٖئٗ  * أَكَرُدِشَ كِلِيٗ  * فَاطِمَة الزَّهْرِيَّة }} \\* 
 \OLTcl{ zzahrı̄yaẗ fāṭimaẗ *  kiliyo akarudisha *  mumeo na usiţeţe *  hayo maneno yawate} \\* 
\SB{36} (\textbf{38}) \OLTst{yawate maneno hayo  * usitete na mumeo  * akarudisha kiliyo  * fatimat zzahriyat } \\ 
\E{Leave off these words," [he said].   Don't quarrel with your husband.   And he comforted her,  Fatima the Radiant.  } \\ 
\\[8mm] 

\textarabic{(٣٩) \textcolor{mygreen}{أَكَمْوَمْبِيَ مْوَنَنْڠُ  * أُتَكَپٗ رَضِ يَنْڠُ  * نِ هَيَ مَنٖنٗ يَنْڠُ  * يُوَ أُكِيَسِكِيَ }} \\* 
 \OLTcl{ ukiyasikiya yuwa *  yangu maneno haya ni *  yangu raḍi utakapo *  mwanangu akamwambiya} \\* 
\SB{37} (\textbf{39}) \OLTst{akamwambiya mwanangu  * utakapo radhi\footnote{Move note from 83d to here.} yangu  * ni haya maneno yangu  * yuwa ukiyasikiya } \\ 
\E{He told her: My child,   if you want my blessing,   this is my advice --    you know, if you'll listen to it.  } \\ 
\\[8mm] 

\textarabic{(٤٠) \textcolor{mygreen}{كُتٗكَ كْوَكٖ فَتُمَ  * يَلٖ أَكَيَسُكُمَ  * كِشَ كَڤُٹَ كَلِمَ  * بَبَكٖ أَكَمْوَمْبِيَ }} \\* 
 \OLTcl{ akamwambiya babake *  kalima kavuţa kisha *  akayasukuma yale *  fatuma kwake kutoka} \\* 
\SB{38} (\textbf{40}) \OLTst{kutoka kwake fatuma  * yale akayasukuma  * kisha kavuta kalima  * babake akamwambiya } \\ 
\E{For her part, Fatima   weighed those [words] carefully,  then she spoke [these] words,   and told her father:  } \\ 
\\[8mm] 

\textarabic{(٤١) \textcolor{mygreen}{أَكَمْوَمْبِيَ بَبَنْڠُ  * ٹُنَنِ نَ مُمٖ وَنْڠُ  * كَبِسَ مَؤٗڤُ يَنْڠُ  * سِ مْوٖنْيٖ كُپٖنْدٖلٖيَ }} \\* 
 \OLTcl{ kupendeleya mwenye si *  yangu maovu kabisa *  wangu mume na ţunani *  babangu akamwambiya} \\* 
\SB{39} (\textbf{41}) \OLTst{akamwambiya babangu  * tunani na mume wangu  * kabisa maovu yangu  * si mwenye kupendeleya } \\ 
\E{She said: Father,  what quarrel do I have with my husband?   [It was] my fault entirely,    and I am not pleased [to have done it].   } \\ 
\\[8mm] 

\textarabic{(٤٢) \textcolor{mygreen}{نَ مْٹُ أَلٗنِغُرِ  * سَسَ نِمٖفِكِرِ  * هَٹُپٖنْدٖلٖئِ خٖيْرِ  * هَوٖزِ كُٹْوَنْڠَلِيَ }} \\* 
 \OLTcl{ kuţwangaliya hawezi *  khēri haţupendelei *  nimefikiri sasa *  alonighuri mţu na} \\* 
\SB{40} (\textbf{42}) \OLTst{na mtu alonighuri  * sasa nimefikiri  * hatupendelei heri  * hawezi kutwangaliya } \\ 
\E{And the person who deceived me --   now I have realised  she did not want good fortune for us --  she couldn't look at us [without envy].  } \\ 
\\[8mm] 

\textarabic{(٤٣) \textcolor{mygreen}{تٖنَ هُضُمُ مٗيٗنِ  * مْٹُيٖ نِ شَيْطَانِ  * مْلَنِوَ مَلْعُوْنِ  * نِيَه مْبٗڤُ كُنِٹِيَ }} \\* 
 \OLTcl{ kuniţiya mbovu niyah *  malʿūni mlaniwa *  shayṭāni ni mţuye *  moyoni huḍumu tena} \\* 
\SB{41} (\textbf{43}) \OLTst{tena hudhumu moyoni  * mtuye ni shaytani  * mlaniwa mal'uni  * niyah mbovu kunitiya } \\ 
\E{And the conclusion in my heart   [is that] that person was the Devil,   the Cursed One, the Damned One,  planting evil intentions in me.   } \\ 
\\[8mm] 

\textarabic{(٤٤) \textcolor{mygreen}{هَيَ نِمٖيَخِتِمُ  * نَ مٖنْڠِنٖ تهَنُظُمُ  * جَمِيْعِ مُيَفَهَمُ  * نِمٖپٖنْدَ كُوَمْبِيَ }} \\* 
 \OLTcl{ kuwambiya nimependa *  muyafahamu jamı̄ʿi *  tʰanuẓumu mengine na *  nimeyakhitimu haya} \\* 
\SB{42} (\textbf{44}) \OLTst{haya nimeyahitimu\footnote{The first part of the ballad, describing the \Swa{mke wa siri}, and the resulting friction between Ali and Fatima, and its resolution, is now complete.  The next portion of the tale (\q{\Swa{mengine}}), describing Ja'far's meeting with his father Ali, and its results, now begins.}  * na mengine tanudhumu\footnote{\Swa{nuzumu}, \E{compose}.}  * jami'i muyafahamu\footnote{\q{that you may all understand it}, or \q{that you may understand it all}.}  * nimependa kuwambiya } \\ 
\E{I have completed these [things],  and I will compose other [things],   so that all of you may understand them --   I have been pleased to tell you [them].  } \\ 
\\[8mm] 

\textarabic{(٤٥) \textcolor{mygreen}{فَهَمُنِ وَؤُنْڠْوَنَ  * يٖؤٗ نِوَپٖ مَعَنَ  * پٖٹٖ يَنْڠُ يَ عَيْنَ  * حُجَ نَلٗئِوَتِيَ }} \\* 
 \OLTcl{ naloiwatiya ḥuja *  ʿayna ya yangu peţe *  maʿana niwape yeo *  waungwana fahamuni} \\* 
\SB{43} (\textbf{45}) \OLTst{fahamuni\footnote{It seems that here we should envisage Ali explaining, after the events of the rest of the ballad, about the ring, and why he left it with Atika.} waungwana  * yeo niwape ma'ana  * pete yangu ya 'ayna\footnote{\Swa{ya aina}, \E{one of a kind}, i.e. \Swa{nzuri}.}  * huja naloiwatiya } \\ 
\E{[Ali said:] Pay attention, noble [listeners],  so that today I may give you an explanation:   my distinctive ring --    the reason I left it behind.  } \\ 
\\[8mm] 

\textarabic{(٤٦) \textcolor{mygreen}{نَلِئِوَتَ قَصِدِ  * سِ مَهَبَ كُنِزِدِ  * مَرَ هُزَءَ وَلِدِ  * أَصِلِ إِكَپٗتٖيَ }} \\* 
 \OLTcl{ ikapoteya aṣili *  walidi huzaa mara *  kunizidi mahaba si *  qaṣidi naliiwata} \\* 
\SB{44} (\textbf{46}) \OLTst{naliiwata qasidi\footnote{Amu \Swa{qasidi} = Mvita \Swa{maqusudi}.}  * si mahaba kunizidi  * mara huzaa walidi  * asili ikapoteya\footnote{That is, the child would not know who his father was -- this would be very unfortunate, and Ali is anxious for this not to happen} } \\ 
\E{I left it for the purpose,  not of increasing [her] love for me,   [but lest] once the child was born,   its heritage should be lost.  } \\ 
\\[8mm] 

\textarabic{(٤٧) \textcolor{mygreen}{بَسِ نَلٗيَفِكِرِ  * نْدِيٗ يٗتٖ يَلٗجِرِ  * أَكَزَوَ جَعْفَرِ  * وَ مَوْلَانَا عَلِيَ }} \\* 
 \OLTcl{ ʿaliya mawlānā wa *  jaʿfari akazawa *  yalojiri yote ndiyo *  naloyafikiri basi} \\* 
\SB{45} (\textbf{47}) \OLTst{basi naloyafikiri\footnote{Amu \Swa{n[i]-al[i]-o} = Mvita \Swa{ni-l[i]-o}, subject prefix + past marker + relative marker.}  * ndiyo yote yalojiri  * akazawa ja'fari\footnote{We might surmise that he was named Ja'far after Ali's brother Ja'far.}  * wa maulana 'aliya } \\ 
\E{Indeed, what I had foreseen  was exactly what happened.   Ja'far was born,  [son of] Lord Ali.   } \\ 
\\[8mm] 

\textarabic{(٤٨) \textcolor{mygreen}{أَكَئِسِنْڠَ كِجَنَ  * نَ بَبَكٖ وَكِفَنَ  * كُلَ أَلٗكِمُؤٗنَ  * صُوْرَ زَلِكِمْوَمْبِيَ }} \\* 
 \OLTcl{ zalikimwambiya ṣūra *  alokimuona kula *  wakifana babake na *  kijana akaisinga} \\* 
\SB{46} (\textbf{48}) \OLTst{akaisinga\footnote{lit. \q{he moulded himself} to the appearance of his father.} kijana  * na babake wakifana  * kula alokimuona  * sura zalikimwambiya } \\ 
\E{The boy grew up  resembling his father.   [To] everyone who saw him,  his features said who he was.  } \\ 
\\[8mm] 

\textarabic{(٤٩) \textcolor{mygreen}{وَٹُ وَلِكِنُكُرِ  * عَلِى هَنَ هَبَرِ  * هَتَ مْمٗيَ كْوَ سِرِ  * أَكَفِكَ كُمْوَمْبِيَ }} \\* 
 \OLTcl{ kumwambiya akafika *  siri kwa mmoya hata *  habari hana ʿalii *  walikinukuri waţu} \\* 
\SB{47} (\textbf{49}) \OLTst{watu \dotuline{walikidhukuri}  * 'alii hana habari  * hata mmoya kwa siri  * akafika kumwambiya } \\ 
\E{People were talking about it,  [but] Ali knew nothing of it   not a single [person] secretly    arrived to tell him.  } \\ 
\\[8mm] 

\textarabic{(٥٠) \textcolor{mygreen}{أَلِپٗپٖنْدَ مَنَانِ  * كَمُؤٗنَ مُعَيَنِ  * كُنَ كِسِمَ مْوِٹُنِ  * أَكٖنْدَ كُچَنْڠَلِيَ }} \\* 
 \OLTcl{ kuchangaliya akenda *  mwiţuni kisima kuna *  muʿayani kamuona *  manāni alipopenda} \\* 
\SB{48} (\textbf{50}) \OLTst{alipopenda\footnote{This translation (\Swa{hata mmoya}, \E{not a single [person]}) deals with the Y text, but for the R text we should translate \Swa{hata mmoya}, \E{until a single [person]} to deal with the fact that it adds stanzas here describing someone (actually the Devil) coming along to trick Ali.} manani  * kamuona mu'ayani\footnote{lit. \q{clearly}.}  * kuna kisima mwituni  * akenda kuchangaliya } \\ 
\E{When it pleased Providence  [Ali] saw [Ja'far] in the flesh.  There was a well in the forest,   and [Ali] went to have a look at it.  } \\ 
\\[8mm] 

\textarabic{(٥١) \textcolor{mygreen}{نَاءٖ أَكٖنْدَ كْوَ شَكَ  * مَاءِ أَسِپٗيَتَكَ  * نَاءٖ أَلِكِفُنِكَ  * كِوَزِ أَكِچٖنْدٖيَ }} \\* 
 \OLTcl{ akichendeya kiwazi *  alikifunika nae *  asipoyataka mai *  shaka kwa akenda nae} \\* 
\SB{49} (\textbf{51}) \OLTst{nae akenda kwa shaka  * mai asipoyataka  * nae alikifunika\footnote{Because in such a climate water is very valuable.}  * kiwazi akichendeya } \\ 
\E{And he went from suspicion,    not wanting water.   [Although] he had [earlier] covered it,  it was open when he got there.  } \\ 
\\[8mm] 

\textarabic{(٥٢) \textcolor{mygreen}{هَپٗ عَلِى حَيْدَرِ  * كِوَزَ نَ كُفِكِرِ  * وَمٖكُيَ مَكَفِرِ  * يٖؤٗ كُنِفُنُلِيَ }} \\* 
 \OLTcl{ kunifunuliya yeo *  makafiri wamekuya *  kufikiri na kiwaza *  ḥaydari ʿalii hapo} \\* 
\SB{50} (\textbf{52}) \OLTst{hapo 'alii\footnote{̣} haydari  * kiwaza na kufikiri  * wamekuya makafiri  * yeo kunifunuliya\footnote{To annoy and frustrate him.} } \\ 
\E{Then Ali the Lion-like   pondered and considered:   Unbelievers have come here  to uncover it today in spite of me.  } \\ 
\\[8mm] 

\textarabic{(٥٣) \textcolor{mygreen}{كِشَ أَكَتَمْكَ  * نَ يٖؤٗ تهَكِفُنِكَ  * سِنَ بُدِ تهَمْشِكَ  * مْٹُيٖ أَمٖزٗوٖيَ }} \\* 
 \OLTcl{ amezoweya mţuye *  tʰamshika budi sina *  tʰakifunika yeo na *  akatamka kisha} \\* 
\SB{51} (\textbf{53}) \OLTst{kisha akatamka  * na yeo takifunika  * sina budi tamshika  * mtuye amezoweya } \\ 
\E{Then he said:  I will cover it again today,   and doubtless I will catch   that person who is behaving like that.  } \\ 
\\[8mm] 

\textarabic{(٥٤) \textcolor{mygreen}{أَچٗنْدٗكَ هُكُ نْيُمَ  * جَعْفَرِ كَئٖڠٖمَ  * لِلٖ بَاءٗ كَسُكُمَ  * مْبَلِ أَكَلَتِلِيَ }} \\* 
 \OLTcl{ akalatiliya mbali *  kasukuma bao lile *  kaegema jaʿfari *  nyuma huku achondoka} \\* 
\SB{52} (\textbf{54}) \OLTst{achondoka huku nyuma  * ja'fari kaegema\footnote{The Swahili belief would be that the boy has been led to that particular place \q{by the blood}, i.e. because he is a son of his father's, the two have a bodily affinity, and tend to be attracted to each other, like magnets.}  * lile bao kasukuma  * mbali akalatiliya\footnote{\Swa{-atilia} means \q{drop} in Mvita and \q{throw} in Amu.} } \\ 
\E{When he had gone off [to hide], in the meantime   Ja'far approached,  pushed off the plank [covering the well],   and threw it far away.  } \\ 
\\[8mm] 

\textarabic{(٥٥) \textcolor{mygreen}{مْبُزِ وَكَنْوَ كْوَ هِمَ  * جَعْفَرِ أُكَلِكٗ نْيُمَ  * كِشَ نَاءٖ كَئٖڠٖمَ  * عَلِيْ هُمْوَنْڠَلِيَ }} \\* 
 \OLTcl{ humwangaliya ʿalii *  kaegema nae kisha *  nyuma ukaliko jaʿfari *  hima kwa wakanwa mbuzi} \\* 
\SB{53} (\textbf{55}) \OLTst{mbuzi wakanwa kwa hima  * ja'fari ukaliko nyuma  * kisha nae kaegema  * 'alii humwangaliya } \\ 
\E{His goats drank greedily    and Jaafar was there behind them.   Then he too came forward,    and Ali watched him.  } \\ 
\\[8mm] 

\textarabic{(٥٦) \textcolor{mygreen}{أَچٖڠٖمَ كَرَدِدِ  * ٹُتَشِنْدَنَ قَصِدِ  * كْوَندَ يٖؤٗ أَكِرُدِ  * هَلِدِرِكِ أَكِيَ }} \\* 
 \OLTcl{ akiya halidiriki *  akirudi yeo kwanda *  qaṣidi ţutashindana *  karadidi achegema} \\* 
\SB{54} (\textbf{56}) \OLTst{achegema karadidi\footnote{Move note to 183c to here.}  * tutashindana qasidi\footnote{lit. \E{we will compete in aim}.  It seems we should understand a sequence of events prior to the present sequence (beginning in stanza 50), in which Ali's covering of the well and Ja'far's uncovering of it have gone on for some time.  They have now both resolved to get to the heart of the matter and teach the other person a lesson.}  * kwanda yeo akirudi  * halidiriki akiya } \\ 
\E{As Ja'far approached, he was saying:  We will compete tit-for-tat --  if he comes back today,   he will not find [the plank] when he gets here.  } \\ 
\\[8mm] 

\textarabic{(٥٧) \textcolor{mygreen}{چَمْبَ هُفَنْيَ نِ بِرِ  * أَيَپٗ أَتَنِكِرِ  * هِلِ لِپٖٹٖ بَنْدَرِ  * نْڠٗٹَ تهَمْفِنِكِيَ }} \\* 
 \OLTcl{ tʰamfinikiya ngoţa *  bandari lipeţe hili *  atanikiri ayapo *  biri ni hufanya chamba} \\* 
\SB{55} (\textbf{57}) \OLTst{chamba\footnote{\Swa{chamba}, \E{if}.} hufanya ni\footnote{This line is unclear.  Possibly we should read \Swa{nibiri}, \E{challenge} < \AS{نبر}, \E{raise one's voice, shout}.} biri\footnote{i.e. if Ali is daring Ja'far to do something.}  * ayapo atanikiri  * hili lipete bandari\footnote{lit. \q{this [boat, \Swa{jahazi}] has gained the harbour}, i.e. this state of affairs must come to an end.}  * ngota tamfinikiya } \\ 
\E{If he is challenging me,     when he gets here he will submit to me.  I'm at the end of my tether --   I'll teach him a lesson."  } \\ 
\\[8mm] 

\textarabic{(٥٨) \textcolor{mygreen}{مَاءِ كَٹِكَ كُٹٖكَ  * نَ عَلِى أَكَتٗكَ  * مْكٗنٗ أَكَمْشِكَ  * جَعْفَرِ كَمْوَمْبِيَ }} \\* 
 \OLTcl{ kamwambiya jaʿfari *  akamshika mkono *  akatoka ʿalii na *  kuţeka kaţika mai} \\* 
\SB{56} (\textbf{58}) \OLTst{mai katika kuteka  * na 'alii akatoka  * mkono akamshika  * ja'fari kamwambiya } \\ 
\E{[But] while he was drawing water,    Ali came out [of hiding]  and grabbed him by the arm.  Jaafari said to him:  } \\ 
\\[8mm] 

\textarabic{(٥٩) \textcolor{mygreen}{وٖوٖ هُنِشِكِيَنِ  * مِمِ سِكُچِ سِنَنِ  * هَتَ أُكِوَ نِ جِنِ  * نَيُوَ كُكُسٗمٖيَ }} \\* 
 \OLTcl{ kukusomeya nayuwa *  jini ni ukiwa hata *  sinani sikuchi mimi *  hunishikiyani wewe} \\* 
\SB{57} (\textbf{59}) \OLTst{wewe hunishikiyani  * mimi sikuchi sinani\footnote{\Swa{sina-ni} emphasises the negative -- a person accused of theft may say \Swa{sikuiba sina-ni}, \E{I didn't do any stealing at all}. It may be shortened to \Swa{sini}.  Thus the poem: \Swa{kidege na uliwani? / silicha mtu sina-ni}.  The story is told of a witty tailor from Takaungu.  A group of people had dropped into his shop for a chat, and after a while the tailor got up and went out to answer nature's call to urinate.  He came back very quickly, which made some of the men there ask him jokingly why he had been so fast.  He replied: \Swa{sina kisonono sini!}, \E{I don't have gonorrhea at all!}, which was greeted with laughter.  The men then said that even if this were the case, he should still have taken longer, since it takes some time to wash (\Swa{kutama}) after going to the toilet.  The tailor replied, \Swa{siṣali sini!}, \E{I'm not doing any praying!}.  This made everyone collapse with laughter -- people may not pray, but they certainly would not tell other people that.}  * hata ukiwa ni jini  * nayuwa kukusomeya } \\ 
\E{Why are you grabbing hold of me?  I'm not in the least afraid of you.   Even if you were a jinn    I would know how to read [the Qur'an] against you.  } \\ 
\\[8mm] 

\textarabic{(٦٠) \textcolor{mygreen}{هَيَ نِمٖزٗفَسِرِ  * سِكُيَنٖنَ كْوَ سِرِ  * أُكِتَكَ أَظْهَرِ  * نَ زَيْدِ تهَكْوَمْبِيَ }} \\* 
 \OLTcl{ tʰakwambiya zaydi na *  aẓhari ukitaka *  siri kwa sikuyanena *  nimezofasiri haya} \\* 
\SB{58} (\textbf{60}) \OLTst{haya nimezofasiri  * sikuyanena kwa siri  * ukitaka adhhari\footnote{\Swa{azhari} = \Swa{wazi-wazi, zaidi}.}  * na zaydi\footnote{This is fighting talk -- \Swa{jeuri}!} takwambiya } \\ 
\E{These [things] I have said,  I have not spoken secretly --    if you want it in plain terms  I will say even more to you."   } \\ 
\\[8mm] 

\textarabic{(٦١) \textcolor{mygreen}{كَمْڤُٹِيَ كْوَ مْبَلِ  * وَكَوَنَ سَاءَ مْبِلِ  * كِشَ كَڤُٹَ قَوْلِ  * جَعْفَرِ كَمْوَمْبِيَ }} \\* 
 \OLTcl{ kamwambiya jaʿfari *  qawli kavuţa kisha *  mbili saa wakawana *  mbali kwa kamvuţiya} \\* 
\SB{59} (\textbf{61}) \OLTst{kamvutiya kwa mbali  * wakawana saa mbili\footnote{For a small boy to be able to hold his own against Ali, the champion warrior, is no mean feat.}  * kisha kavuta qauli  * ja'fari kamwambiya } \\ 
\E{He pulled away from [Ali]   and they fought for two hours.   Eventually he spoke,   Ja'far, and addressed him.  } \\ 
\\[8mm] 

\textarabic{(٦٢) \textcolor{mygreen}{كَمْوَمْبِيَ مْبَئِنِ  * مْوَنَ آدَمُ نْ نَنِ  * بَبَ هَكٗ دُنِيَنِ  * نْدِپٗ أُكَنِؤٗنٖيَ }} \\* 
 \OLTcl{ ukanioneya ndipo *  duniyani hako baba *  nani n ãdamu mwana *  mbaini kamwambiya} \\* 
\SB{60} (\textbf{62}) \OLTst{kamwambiya mbaini\footnote{\Swa{-m-} here = \Swa{-ni-}.  See 237b.}  * mwana adamu n nani  * baba hako duniyani  * ndipo ukanioneya\footnote{i.e. why are you picking on an orphan?} } \\ 
\E{He said to him: Explain to me  what sort of person you are.    [My] father is no longer in this world,   and that is why you are bullying me.  } \\ 
\\[8mm] 

\textarabic{(٦٣) \textcolor{mygreen}{كَمُؤُزَ هُنٖنَنِ  * وٖوٖ بَبَكٗ نِ نَنِ  * كَمْبَ نِ پٖٹٖ چَنْدَنِ  * عَلِى كَيَنْڠَلِيَ }} \\* 
 \OLTcl{ kayangaliya ʿalii *  chandani peţe ni kamba *  nani ni babako wewe *  hunenani kamuuza} \\* 
\SB{61} (\textbf{63}) \OLTst{kamuuza hunenani  * wewe babako ni nani  * kamba ni pete chandani  * 'alii kayangaliya } \\ 
\E{[Ali] asked him: What are you saying?  Who is your father?    [Jaafar] said: He is the ring on my finger.    Ali looked at it.  } \\ 
\\[8mm] 

\textarabic{(٦٤) \textcolor{mygreen}{پٖٹٖ كُئِيٗنَ كْوَكٖ  * كِسٗمَ نَ جِنَ لَكٖ  * أَرُدِ أَسِكِتِكٖ  * نَ مَيُتٗ كُمْنْڠِيَ }} \\* 
 \OLTcl{ kumngiya mayuto na *  asikitike arudi *  lake jina na kisoma *  kwake kuiyona peţe} \\* 
\SB{62} (\textbf{64}) \OLTst{pete kuiyona kwake  * kisoma na jina lake  * arudi asikitike  * na mayuto kumngiya\footnote{The motif of a father and son unknowingly fighting each other is a recurrent one in literature -- the most famous example is that of Sohrab and Rustum.  Fortunately, in this case the father recognises his son before any damage has been done.} } \\ 
\E{Once he saw the ring   and read his name [on it],    he stepped back greatly saddened  and was filled with remorse.   } \\ 
\\[8mm] 

\textarabic{(٦٥) \textcolor{mygreen}{كِشَ هَپٗ أَمْوَمْبِئٖ  * سِنْڠَلِكُپِجِئٖ  * إِنَ لَكٗ هُئِٹْوَيٖ  * بَبَكٗ نْدِيٖ عَلِيَ }} \\* 
 \OLTcl{ ʿaliya ndiye babako *  huiţwaye lako ina *  singalikupijie *  amwambie hapo kisha} \\* 
\SB{63} (\textbf{65}) \OLTst{kisha hapo amwambie\footnote{The following stanzas are a bit unclear.  The gist seems to be that Ali says he is Ja'far's father, Ja'far reproaches him for his earlier bullying behaviour, still suspicious and unsure whether or not to believe him, whereupon Ali describes Ja'far's mother to him, which convinces Ja'far.}  * singalikupijie  * ina lako huitwaye  * babako ndiye 'aliya } \\ 
\E{Then he told [Ja'far]:   I should not have attacked you. As for the name you are to be called,   your father is [me], Ali.   } \\ 
\\[8mm] 

\textarabic{(٦٦) \textcolor{mygreen}{نِنْڠَلِپٗتٖزَ دَمُ  * كَمَ سِكُكُفَهَمُ  * أَكِشَ كَتَكَلَمُ  * جَعْفَرِ كَمْوَمْبِيَ }} \\* 
 \OLTcl{ kamwambiya jaʿfari *  katakalamu akisha *  sikukufahamu kama *  damu ningalipoteza} \\* 
\SB{64} (\textbf{66}) \OLTst{ningalipoteza\footnote{\Swa{-poteza} = \Swa{-tupa}.} damu  * kama sikukufahamu  * akisha katakalamu  * ja'fari kamwambiya } \\ 
\E{I would have spilt your blood  if I had not recognised you.  When he finished speaking,  Jaafar spoke to him:  } \\ 
\\[8mm] 

\textarabic{(٦٧) \textcolor{mygreen}{وٖوٖ مْٹُ هُمُؤٗنَ  * هُمُؤُزِ لَكٖ إِنَ  * هُجِؤٗنَ أُجَڠِنَ  * أُلِؤٗنَ تهَكِمْبِيَ }} \\* 
 \OLTcl{ tʰakimbiya uliona *  ujagina hujiona *  ina lake humuuzi *  humuona mţu wewe} \\* 
\SB{65} (\textbf{67}) \OLTst{wewe mtu humuona  * humuuzi lake ina  * hujiona ujagina\footnote{According to Sacleux, \Swa{ujagina} comes from a Galla word meaning \q{brave, courageous}.  It is said that \Swa{Ali sifa yake ni shujaa}, \E{Ali is famed as a warrior}, and Ja'far inherits this martial attribute, as his spirited fighting shows.  Ali was a short man, but very strong.  It is said that once he plunged his sword into the ground and challenged others to pull it out, but it was buried so deep that no-one could.  Again, it is said that once when Ali was praying in the mosque his friends jokingly took his sandals (which in accordance with ritual he had of course removed before entering the mosque) and placed them on top of the lintel, where Ali, being short, could not reach them.  As a retort, Ali took their sandals, grabbed hold of the mosque wall, lifted it up, put the sandals under the wall, and set it down again.  Other important Muslims have their own attributes -- Uthman, for instance, was known for his shyness.}  * uliona takimbiya } \\ 
\E{When you see someone,   you do not even ask his name.   You see yourself as a warrior,  and you thought I would run away.  } \\ 
\\[8mm] 

\textarabic{(٦٨) \textcolor{mygreen}{كْوَنْزَ نِپَ پٖٹٖ يَنْڠُ  * إِنُكَ إٖوٖ بَبَنْڠُ  * هُنْڠَلِوٖزَ مَتُنْڠُ  * مْوِلِنِ كُنِٹِيَ }} \\* 
 \OLTcl{ kuniţiya mwilini *  matungu hungaliweza *  babangu ewe inuka *  yangu peţe nipa kwanza} \\* 
\SB{66} (\textbf{68}) \OLTst{kwanza nipa pete yangu  * inuka ewe\footnote{Ja'far is suspicious.} babangu  * hungaliweza matungu  * mwilini kunitiya } \\ 
\E{First, give me back my ring,    and get up, father --   you would not have been able   to inflict injuries on my body.  } \\ 
\\[8mm] 

\textarabic{(٦٩) \textcolor{mygreen}{أَكَمْوَمْبِيَ مَمَكٖ  * نَمُيُوَ سُرَ زَكٖ  * نَاءٖ هَپٗ أَتَمْكٖ  * إِنَ لَكٖ أَمْوَمْبِيٖ }} \\* 
 \OLTcl{ amwambiye lake ina *  atamke hapo nae *  zake sura namuyuwa *  mamake akamwambiya} \\* 
\SB{67} (\textbf{69}) \OLTst{akamwambiya mamake  * namuyuwa sura zake  * nae hapo atamke  * ina lake amwambiye } \\ 
\E{[Ali] told him [who] his mother [was]:  I recognise her features [in you].   And then [Ja'far] spoke   in order to tell [Ali] his name.   } \\ 
\\[8mm] 

\textarabic{(٧٠) \textcolor{mygreen}{كِمْوَمْبِيَ كَفَسِرِ  * مِمِ هُئِٹْوَ جَعْفَرِ  * وَ عَلِيْ حَيْدَرِ  * نَ مْوَكَ نِ وَ تِسِيَ }} \\* 
 \OLTcl{ tisiya wa ni mwaka na *  ḥaydari ʿalii wa *  jaʿfari huiţwa mimi *  kafasiri kimwambiya} \\* 
\SB{68} (\textbf{70}) \OLTst{kimwambiya kafasiri  * mimi huitwa ja'fari  * wa 'alii haydari  * na mwaka ni wa tisiya } \\ 
\E{He spoke, saying:  I am called Ja'far,   [son] of Ali the Lion-like   and I am nine years old.     } \\ 
\\[8mm] 

\textarabic{(٧١) \textcolor{mygreen}{وَتٗوٖ مْبُزِ مْوِٹُنِ  * نَ أُوَپٖكٖ نْدِيَنِ  * ٹْوٖنٖنْدٖ زٖٹُ مُئِنِ  * جَعْفَرِ كَمْوَمْبِيَ }} \\* 
 \OLTcl{ kamwambiya jaʿfari *  muini zeţu ţwenende *  ndiyani uwapeke na *  mwiţuni mbuzi watowe} \\* 
\SB{69} (\textbf{71}) \OLTst{watowe mbuzi mwituni\footnote{Ali suggests going back with him to Mecca, but Ja'far wishes to take leave of his family first.  He must also ask permission of his teacher (122b), since he cannot leave the \Swa{chuo}, \E{school}, without being allowed.  See 79b.}  * na uwapeke ndiyani  * twenende zetu muini  * ja'fari kamwambiya } \\ 
\E{Bring your goats out of the forest, [said Ali],    and herd them along the road  so that we may go on towards the town.   But Ja'far spoke to him  } \\ 
\\[8mm] 

\textarabic{(٧٢) \textcolor{mygreen}{كَمْوَمْبِيَ نٖنْدَ زَنْڠُ  * نِنَ نَ مْوَلِمُ وَنْڠُ  * نِمُوَڠٖ نَ مَمَنْڠُ  * كِشَ كٖشٗ نِتَكُيَ }} \\* 
 \OLTcl{ nitakuya kesho kisha *  mamangu na nimuwage *  wangu mwalimu na nina *  zangu nenda kamwambiya} \\* 
\SB{70} (\textbf{72}) \OLTst{kamwambiya nenda zangu  * nina na mwalimu wangu\footnote{Ja'far has to go to the \Swa{chuo}, \E{school}, first in order to gain the permission of his teacher.  See 79b.}  * nimuwage na mamangu  * kisha kesho nitakuya } \\ 
\E{and told him: I am going off --   I have my teacher    whom I must take leave of, and my mother.   Then I will come tomorrow.   } \\ 
\\[8mm] 

\textarabic{(٧٣) \textcolor{mygreen}{كٖشٗ كُكِپَمْبَؤُكَ  * نَ مَپِمَ تَئِنُكَ  * سَاءَ مٗيَ إِكِفِكَ  * بَبَ تَكُوَصِلِيَ }} \\* 
 \OLTcl{ takuwaṣiliya baba *  ikifika moya saa *  tainuka mapima na *  kukipambauka kesho} \\* 
\SB{71} (\textbf{73}) \OLTst{kesho kukipambauka  * na mapima tainuka  * saa moya ikifika\footnote{i.e. around 7.00pm.}  * baba takuwasiliya } \\ 
\E{When tomorrow has dawned,  I will get up early,    and when the first hour comes   I will arrive with you, father.  } \\ 
\\[8mm] 

\textarabic{(٧٤) \textcolor{mygreen}{أَكَمْطِبُ قَوْلِ  * أُيَپٗ نْدِيَ يَ مْبَلِ  * يَ مَكَه نِ يَ كُڤُلِ  * أُسِتَكٖ كُپٗتٖيَ }} \\* 
 \OLTcl{ kupoteya usitake *  kuvuli ya ni makah ya *  mbali ya ndiya uyapo *  qawli akamṭibu} \\* 
\SB{72} (\textbf{74}) \OLTst{akamtibu qauli\footnote{\Swa{akampa maneno mazuri}, \E{he gave him words of advice}.}  * uyapo ndiya ya mbali  * ya makah ni ya kuvuli\footnote{\Swa{kuvuli} = \Swa{kulia}.}  * usitake\footnote{\Swa{usitake} = \Swa{usije}.} kupoteya\footnote{Ali gives more directions than the ones here (see 200a/b), but in the event Ja'far forgets them all and almost gets lost in the scrubland (see 202-3).} } \\ 
\E{[Ali] gave [Ja'far] some advice:   When you come to the fork in the road,   the way to Mecca is the one on the right --     just so you don't get lost.  } \\ 
\\[8mm] 

\textarabic{(٧٥) \textcolor{mygreen}{كَمْرُدِشِيَ تَمْكٗ  * بَسِ هَيَ نٖنْدَ زَكٗ  * مْوَلِمُ نَ مَمَكٗ  * نَ نْدُزٗ نِسَلِمِيَ }} \\* 
 \OLTcl{ nisalimiya nduzo na *  mamako na mwalimu *  zako nenda haya basi *  tamko kamrudishiya} \\* 
\SB{73} (\textbf{75}) \OLTst{kamrudishiya\footnote{This seems out of place, since Ja'far has not actually said anything for Ali to reply to.  Perhaps we should emend by reading 122, 124, 123, 125.} tamko  * basi haya nenda zako  * mwalimu na mamako  * na nduzo\footnote{\Swa{nduzo} < \Swa{ndugu zako}.  \Swa{ndugu} can mean \q{cousin} as well as \q{brother}.} nisalimiya\footnote{\Swa{-salimu} is used for a person-to-person greeting, and this is the indirect form: \q{greet them on my behalf}.} } \\ 
\E{[Ali] replied to [Ja'far]:  So, now, off you go,    the teacher and your mother   and your relatives -- give them my best wishes."   } \\ 
\\[8mm] 

\textarabic{(٧٦) \textcolor{mygreen}{نَاءٖ مْوَنَ وَ نَسَبَ  * كَمْوَمْبِيَ مَرْحَبَا  * نَمِ هُكٗ أَقْرَبَ  * أُچٖنْدَ نِسَلِمِيَ }} \\* 
 \OLTcl{ nisalimiya uchenda *  aqraba huko nami *  marḥabā kamwambiya *  nasaba wa mwana nae} \\* 
\SB{74} (\textbf{76}) \OLTst{nae mwana wa nasaba\footnote{Ja'far will therefore do what is right and expected of him.}  * kamwambiya marhaba  * nami huko aqraba  * uchenda nisalimiya } \\ 
\E{And [Ja'far], the noble child,    told him: Thank you.  And from me to your relatives there,   when you go [there], greet them for me.  } \\ 
\\[8mm] 

\textarabic{(٧٧) \textcolor{mygreen}{سَلَامُ أَبُوْ بَكَرِ  * أَزْوَاجِ نَ ذُرِيَ  * سُزَاءٗ نَ إِظْهَارِ  * هَؤٗ نَوَفَهَمِيَ }} \\* 
 \OLTcl{ nawafahamiya hao *  iẓhāri na suzao *  dhuriya na azwāji *  bakari abuu salāmu} \\* 
\SB{75} (\textbf{77}) \OLTst{salamu abuu bakari  * azwaji\footnote{Azwaj and Zubeir are the two people who were with Abu Bakr when Ja'far met them on the road (174, 175).  This part of the story (i.e. how Ja'far came to be at the well) has not been told yet -- it is contained in a flashback a little later in the ballad.  In Y it would seem that the name Azwaj has been confused with the word for \q{wife} (understandable in this context), an that \Swa{dhuriya}, \E{children}, has then been substituted for Zubeir (as making better sense) -- note that \Swa{dhuriya} does not rhyme.} na dhuriya  * suzao na idhhari\footnote{This reading is uncertain -- it seems to mean \Swa{nde na ndani}.  Perhaps we should adopt that of R, and translate \q{I know (have heard of) their general qualities}.}  * hao nawafahamiya\footnote{\Swa{-fahamia}, \E{know of someone, hear about someone, while not knowing them personally}.} } \\ 
\E{Greetings to Abu Bakr,   your wives and children,   both close and extended family --   I have heard of them.  } \\ 
\\[8mm] 

\textarabic{(٧٨) \textcolor{mygreen}{بَسِ هَپٗ جَعْفَرِ  * أَكَمْوَمْبِيَ كْوَ هٖرِ  * نَ عَلِى حَيْدَرِ  * مْنْڠُ أَكَمُؤٗمْبٖيَ }} \\* 
 \OLTcl{ akamuombeya mngu *  ḥaydari ʿalii na *  heri kwa akamwambiya *  jaʿfari hapo basi} \\* 
\SB{76} (\textbf{78}) \OLTst{basi hapo ja'fari  * akamwambiya kwa heri  * na 'alii haydari  * mngu akamuombeya\footnote{lit. \q{interceded for him to God}.  If a parent is punishing a child, and a neighbour is present, the neighbour may plead for the child by saying, \Swa{namuombea, namuombea}, \E{I ask mercy for him, I intercede for him}.  If the child is let off lightly, the neighbour will warn the child not to be naughty again, because he will not plead for him a second time.} } \\ 
\E{So then Ja'far   said goodbye to him,   and Ali the Lion-like   commended him to God's care.  } \\ 
\\[8mm] 

\textarabic{(٧٩) \textcolor{mygreen}{مُئِنِ كُنْڠِيَ كْوَكٖ  * كٖنْدَ كْوَ مْوَلِمُ وَكٖ  * كَمْپَ هَبَرِ زَكٖ  * كْوَءٗ هَيَسِكِلِيَ }} \\* 
 \OLTcl{ hayasikiliya kwao *  zake habari kampa *  wake mwalimu kwa kenda *  kwake kungiya muini} \\* 
\SB{77} (\textbf{79}) \OLTst{muini kungiya kwake  * kenda kwa mwalimu wake\footnote{These stanzas show the great importance of the teacher in Swahili life.  The Islamic teacher is greatly respected and honoured.  Ja'far, as a good-mannered child, tells his teacher of his plans even before telling his mother.  Among the Swahili, to bring someone his shoes is humiliating, making you look like a servant, but to bring a teacher his shoes is a mark of respect, and not something humiliating. Teachers get prestige, but no money, and the more students they have, the more esteemed they are.  It is usual, indeed considered necessary, to stay with the same teacher, and to finish his course of instruction.  It is said of one important sheikh that he was forced to move his abode to the next town because of a quarrel.  Even though the next town was a fair distance away, all his original students from the first town came to see him there.  But after a while, one of them stopped coming.  When he next saw this student, the teacher asked him the reason for this, and the student replied that he was prevented from attending the classes because his mother was sick, and, since caring for your parents is a duty in Islam, he had stayed at home to nurse her.  The teacher said that because the student was so dutiful he would have a long life, but since he had unfortunately missed the classes he would never be successful in teaching.  This prediction turned out to be true.}  * kampa habari zake  * kwao hayasikiliya } \\ 
\E{When [Ja'far] entered his village   he went to his teacher's house,    and gave him his news --   before going home.  } \\ 
\\[8mm] 

\textarabic{(٨٠) \textcolor{mygreen}{أَكِشَ كُيَنُظُمُ  * أَكَلِيَ مُعَلِمُ  * كَمبَ وَنِٹِيَ هَمُ  * هَيٗ أُمٖزٗنَمْبِيَ }} \\* 
 \OLTcl{ umezonambiya hayo *  hamu waniţiya kamba *  muʿalimu akaliya *  kuyanuẓumu akisha} \\* 
\SB{78} (\textbf{80}) \OLTst{akisha kuyanudhumu\footnote{\Swa{-nuẓumu} usually means \q{compose}, (\Swa{-tunga}), but here it means \q{explain}, (\Swa{-eleza}).}  * akaliya mu'alimu\footnote{\Swa{anampenda yule mwanafunzi wake}, \E{he is very fond of that pupil of his}.}  * kamba wanitiya hamu\footnote{Everybody, even a teacher, is apprehensive about the future.}  * hayo umezonambiya } \\ 
\E{When he had finished explaining [everything],  the teacher wept,  and said: You are  making me worried   with these [things] you have told me.  } \\ 
\\[8mm] 

\textarabic{(٨١) \textcolor{mygreen}{وَنِٹِيَ سِكِتِكٗ  * نَ وِنْڠِ وَ مَؤُذِكٗ  * وَلَ نَ هُكٗ وٖنْدَكٗ  * سِوٖزِ كُكُزِوِيَ }} \\* 
 \OLTcl{ kukuziwiya siwezi *  wendako huko na wala *  maudhiko wa wingi na *  sikitiko waniţiya} \\* 
\SB{79} (\textbf{81}) \OLTst{wanitiya sikitiko  * na wingi wa maudhiko\footnote{lit. \q{you are bringing me sadness and many anxieties}.}  * wala na huko wendako  * siwezi \dotuline{kukuzuwiya} } \\ 
\E{You are making me sad  and and very anxious.    Yet that place you are going to --    I cannot keep you back from it.  } \\ 
\\[8mm] 

\textarabic{(٨٢) \textcolor{mygreen}{سِ رَحِمُ كُئٖنْدَنِ  * نَ هُكٗ أُ حَلِ ڠَنِ  * نَاءٖ بَبَكٗ زِٹَنِ  * نْدِيٖ وَ كُٹَنْڠُلِيَ }} \\* 
 \OLTcl{ kuţanguliya wa ndiye *  ziţani babako nae *  gani ḥali u huko na *  kuendani raḥimu si} \\* 
\SB{80} (\textbf{82}) \OLTst{si rahimu kuendani\footnote{i.e. the journey is dangerous.}  * na huko u hali gani  * nae babako zitani  * ndiye wa kutanguliya } \\ 
\E{It is not easy to go there,   and what sort of situation will you be in there,     with your father at war,   always in the front line?"   } \\ 
\\[8mm] 

\textarabic{(٨٣) \textcolor{mygreen}{نَ كُكِكِنْدَ سِتَكِ  * كْوَنِ نَيُوَ نِ هَكِ  * نِ رَضِ أَلْفُ لَكِ  * نَ زَيْدِ كِكْوٖٹٖيَ }} \\* 
 \OLTcl{ kikweţeya zaydi na *  laki alfu raḍi ni *  haki ni nayuwa kwani *  sitaki kukikinda na} \\* 
\SB{81} (\textbf{83}) \OLTst{na \dotuline{kukukinda} sitaki\footnote{\Swa{sitaki kushinda na wewe}.}  * kwani nayuwa ni haki  * ni radhi\footnote{\Swa{radi}, \E{consent, blessing}, is of great importance to a person, whether it be from his mother, his father, or his teacher.  People will become afraid if any of these three persons withhold their \Swa{radi}, since it is held that without \Swa{radi} you cannot prosper -- anything you set your hand to will be blighted and fail.  The teacher here gives Ja'far his \Swa{radi} -- if he had not given it, Ja'far would not have gone -- and says that not only will he give his complete consent, but also (83d) that he will not change his mind once Ja'far has gone.} alfu laki  * na zaydi kikweteya } \\ 
\E{Yet I don't intend to oppose you,   because I know it is proper.    you have my consent a hundred thousand times,    and I give you [even] more [than that].   } \\ 
\\[8mm] 

\textarabic{(٨٤) \textcolor{mygreen}{أَكَمْوَمْبِيَ مْوَلِمُ  * مَمَكٗ أَيَفَهَمُ  * كَمْجِبِشَ كَلِمُ  * كْوَكٖ سِيَسِكِلِيَ }} \\* 
 \OLTcl{ siyasikiliya kwake *  kalimu kamjibisha *  ayafahamu mamako *  mwalimu akamwambiya} \\* 
\SB{82} (\textbf{84}) \OLTst{akamwambiya mwalimu  * mamako ayafahamu  * kamjibisha \dotuline{kalamu}  * kwake siyasikiliya\footnote{Amu \Swa{-sika} = \Swa{fika}.  Compare \Swa{-sita / fita}.} } \\ 
\E{The teacher said to him:  Is your mother aware of these [things]?  And [Ja'far] answered him:  "I have not yet gone home.  } \\ 
\\[8mm] 

\textarabic{(٨٥) \textcolor{mygreen}{كَمْوَمْبِيَ إٖنْدَ زَكٗ  * أُكَمُؤُلِزٖ مَمَكٗ  * أُسِكِزٖ مَتَمْكٗ  * نَاءٖ تَكَلٗكْوَمْبِيَ }} \\* 
 \OLTcl{ takalokwambiya nae *  matamko usikize *  mamako ukamuulize *  zako enda kamwambiya} \\* 
\SB{83} (\textbf{85}) \OLTst{kamwambiya enda zako  * ukamuulize mamako\footnote{The teacher is teaching Ja'far obedience to his mother.}  * usikize matamko  * nae takalokwambiya } \\ 
\E{And [the teacher] told him: Off you go,   and ask your mother.  Pay heed to the things  that she will tell you.  } \\ 
\\[8mm] 

\textarabic{(٨٦) \textcolor{mygreen}{َكِشَ هِيٗ كَلِمَ  * هَپٗ نْدِيَ كَيَنْدَمَ  * مٗيٗ أُنَ هَلِمَمَ  * كْوَ مَمَكٖ أَكِنْڠِيَ }} \\* 
 \OLTcl{ akingiya mamake kwa *  halimama una moyo *  kayandama ndiya hapo *  kalima hiyo akisha} \\* 
\SB{84} (\textbf{86}) \OLTst{akisha hiyo kalima  * hapo ndiya kayandama  * moyo una halimama\footnote{\Swa{halimama} = \Swa{wasiwasi}.}  * kwa mamake akingiya } \\ 
\E{Once [the teacher] had finished these words,   [Ja'far] then continued on his way.   His heart was heavy   as he went in to his mother's [house].   } \\ 
\\[8mm] 

\textarabic{(٨٧) \textcolor{mygreen}{كِنْڠِيَ كَوٖكَ كِبُ  * كْوَ أُپٗلٖ نَ تَرَتِبُ  * مَمَكٖ كَتَعَجَبُ  * جَعْفَرِ أَكَمْوَمْبِيَ }} \\* 
 \OLTcl{ akamwambiya jaʿfari *  kataʿajabu mamake *  taratibu na upole kwa *  kibu kaweka kingiya} \\* 
\SB{85} (\textbf{87}) \OLTst{kingiya kaweka kibu\footnote{\Swa{fimbo ya mbuzi}?}  * kwa upole na taratibu\footnote{Ja'far is trying to sneak back into the house.  He is apprehensive about what he is going to tell his mother, and is also hurt that she did not tell him the full story about his past (100-101).}  * mamake kata'ajabu  * ja'fari akamwambiya } \\ 
\E{When he went in he put his stick away   quietly and carefully.    His mother was surprised,   and spoke to Ja'far.  } \\ 
\\[8mm] 

\textarabic{(٨٨) \textcolor{mygreen}{كَمْبَ سِوٖ جَعْفَرِ  * وَٹُؤٗنٖشَ جَوْرِ  * هُنْڠِيَ كَمَ كْوَ سِرِ  * مْٹُ أَمٖزٗكِمْبِيَ }} \\* 
 \OLTcl{ amezokimbiya mţu *  siri kwa kama hungiya *  jawri waţuonesha *  jaʿfari siwe kamba} \\* 
\SB{86} (\textbf{88}) \OLTst{kamba siwe ja'fari  * watuonesha \dotuline{jeuri}  * hungiya kama kwa siri  * mtu amezokimbiya\footnote{\Swa{anajifita}, \E{he is hiding himself}.} } \\ 
\E{She said: That's not [like] you, Ja'far --   are you being insolent to us,  entering as if secretly,    [like] a person who has run away [and is trying to hide]?  } \\ 
\\[8mm] 

\textarabic{(٨٩) \textcolor{mygreen}{نَاصِرِ نِ نْدُڠُ يَكٖ  * پَپٗ هَپٗ أَتَمْكٖ  * عَيْنِ يَ مَتٗ يَكٖ  * هَتَكِ كُٹْوَنْڠَلِيَ }} \\* 
 \OLTcl{ kuţwangaliya hataki *  yake mato ya ʿayni *  atamke hapo papo *  yake ndugu ni nāṣiri} \\* 
\SB{87} (\textbf{89}) \OLTst{nasiri ni ndugu yake  * papo hapo atamke  * \dotuline{aina} ya mato yake  * hataki kutwangaliya } \\ 
\E{Nasir was [Ja'far's] brother,    and at that moment he spoke up:   To judge by his eyes,    he doesn't want to look at us.  } \\ 
\\[8mm] 

\textarabic{(٩٠) \textcolor{mygreen}{إٖوٖ مَمَ سِؤُذِكٖ  * تَكْوَمْبِيَ حُجَ يَكٖ  * يٖؤٗ أُنَ مْبُزِ وَكٖ  * وَوِلِ وَمٖپٗتٖيَ }} \\* 
 \OLTcl{ wamepoteya wawili *  wake mbuzi una yeo *  yake ḥuja takwambiya *  siudhike mama ewe} \\* 
\SB{88} (\textbf{90}) \OLTst{ewe\footnote{= \Swa{wewe}.} mama siudhike  * takwambiya huja yake\footnote{Nasir teases Ja'far, saying that he knows why Ja'far is quiet: (1) he lost two of the goats he was herding (90d), which would be a shameful thing, and (2) he is not strong enough to put up with the warmth of the day (91d).}  * yeo una mbuzi wake  * wawili wamepoteya } \\ 
\E{Don't worry, mother --   I'll tell you the reason:   today he was with his goats,    [and] two of them went missing.  } \\ 
\\[8mm] 

\textarabic{(٩١) \textcolor{mygreen}{وَمٖتٗكَ صَفُنِ  * مٖوَتَنْڠَ هَوَؤٗنِ  * كِشَ أَتٗكَ مْوِٹُنِ  * يُوَ نِ كَلِ لَ نْدِيَ }} \\* 
 \OLTcl{ ndiya la kali ni yuwa *  mwiţuni atoka kisha *  hawaoni mewatanga *  ṣafuni wametoka} \\* 
\SB{89} (\textbf{91}) \OLTst{wametoka safuni  * mewatanga\footnote{\Swa{-tanga}, \E{scatter, spread out}.} hawaoni  * kisha atoka mwituni  * yuwa ni kali la ndiya } \\ 
\E{They left the herd,  they went off and he couldn't find them.  And of course he is coming back from the forest --   the sun is fierce on the way.     } \\ 
\\[8mm] 

\textarabic{(٩٢) \textcolor{mygreen}{كَئِنُكَ جَعْفَرِ  * أَكَمْپِجَ نَاصِرِ  * زِتَكُتٗكَ جٖؤُرِ  * يٖؤٗ نِكِكْوَنْڠَلِيَ }} \\* 
 \OLTcl{ nikikwangaliya yeo *  jeuri zitakutoka *  nāṣiri akampija *  jaʿfari kainuka} \\* 
\SB{90} (\textbf{92}) \OLTst{kainuka ja'fari  * akampija nasiri  * zitakutoka jeuri  * yeo nikikwangaliya\footnote{If someone is impudent, and you try to remind him that he should behave better by asking him where his manners are, he may say: \Swa{zimeningia kwa huku, zimetoka kwa huku}, \E{they came into me here, and went out there}, that is, they went in one ear and out the other.  If this is too much for the other person, he may say, like Ja'far: \Swa{zitakutoka jeuri, zitakuingia adabu}, \E{your insolence will leave you, and good manners will enter you}, and proceed to teach him a lesson, after which he may say, if successful: \Swa{umekwisha pata adabu}, \E{you have finished getting manners}, that is, I've taught you a lesson.} } \\ 
\E{Ja'far got up  and hit Nasir:  Your impudence will leave you  today, I'll see to it.  } \\ 
\\[8mm] 

\textarabic{(٩٣) \textcolor{mygreen}{هَپٗ مَمَ أَسِكِرِ  * كَمْشِكَ جَعْفَرِ  * إِوَپٗ نِ مِيْ نَاصِرِ  * هَپٗ سِنْڠٖلِكِمْبِيَ }} \\* 
 \OLTcl{ singelikimbiya hapo *  nāṣiri mii ni iwapo *  jaʿfari kamshika *  asikiri mama hapo} \\* 
\SB{91} (\textbf{93}) \OLTst{hapo mama asikiri  * kamshika ja'fari  * iwapo ni mii nasiri  * hapo singelikimbiya\footnote{i.e. don't run away from a fight.} } \\ 
\E{But his mother would have none of that,   and grabbed Ja'far [and said to Nasir:]  If I were you, Nasir,     I would not have run away just now.  } \\ 
\\[8mm] 

\textarabic{(٩٤) \textcolor{mygreen}{نَاصِرِ نِ نْدُڠُ يَكٗ  * مْوَنَنْڠُ مْٹٗٹٗ وَكٗ  * كِشَ نِ عَوْنِ يَكٗ  * وَتَ كُمْٹَنْڠُلِيَ }} \\* 
 \OLTcl{ kumţanguliya wata *  yako ʿawni ni kisha *  wako mţoţo mwanangu *  yako ndugu ni nāṣiri} \\* 
\SB{92} (\textbf{94}) \OLTst{nasiri ni ndugu yako  * mwanangu mtoto\footnote{Mvita \Swa{mdogo}. \E{small} = Amu \Swa{mtoto} = Gunya \Swa{mdodi}.} wako  * kisha ni 'auni yako  * wata\footnote{Perhaps emend to \Swa{kumshanguliya}.  The mother tells Ja'far not to hit Nasir, because he was not serious and he was only teasing.  She also reminds him that blood is thicker than water, and that in the last resort your family is your best friend.} kumtanguliya } \\ 
\E{[To Jaafar she said:] "Nasir is your brother,    my son, your younger brother --   you can depend on him [when you need help],    so do not attack him.  } \\ 
\\[8mm] 

\textarabic{(٩٥) \textcolor{mygreen}{أَلِئِنَمِيَ تِنِ  * أَكِؤُلِزْوَ هَنٖنِ  * يَمٖكُپَٹَ مْوٖنْدَنِ  * هَيَ نِمٖزٗكْوَمْبِيَ }} \\* 
 \OLTcl{ nimezokwambiya haya *  mwendani yamekupaţa *  haneni akiulizwa *  tini aliinamiya} \\* 
\SB{93} (\textbf{95}) \OLTst{aliinamiya tini  * akiulizwa haneni  * yamekupata mwendani  * haya nimezokwambiya\footnote{Nasir says that his words have affected Ja'far, so there must have been some truth in them.  But the mother stops his teasing this time.} } \\ 
\E{[Ja'far] lay down --  he did not answer when spoken to.  [Nasir said:] "It was right on the mark, my friend,  what I said to you."  } \\ 
\\[8mm] 

\textarabic{(٩٦) \textcolor{mygreen}{هَپٗ مَمَكٖ أَجِبُ  * نَاصِرِ هُتَأَدَبُ  * كِوَ وٖوٖ نِ هَرَبُ  * نَمِ نَيُوَ طَبِيَ }} \\* 
 \OLTcl{ ṭabiya nayuwa nami *  harabu ni wewe kiwa *  hutaadabu nāṣiri *  ajibu mamake hapo} \\* 
\SB{94} (\textbf{96}) \OLTst{hapo mamake ajibu  * nasiri hutaadabu  * kiwa wewe ni harabu  * nami nayuwa tabiya } \\ 
\E{Then his mother retorted:   You are ill-mannered, Nasir --  when you are being naughty    I can tell from your behaviour.   } \\ 
\\[8mm] 

\textarabic{(٩٧) \textcolor{mygreen}{نَتَكَ زَكٗ هَبَرِ  * هَيَ نِپَ جَعْفَرِ  * أُسٗ أُسِمٖمٖ هَرِ  * صُوْرَ زِمٖكُپٗتٖيَ }} \\* 
 \OLTcl{ zimekupoteya ṣūra *  hari usimeme uso *  jaʿfari nipa haya *  habari zako nataka} \\* 
\SB{95} (\textbf{97}) \OLTst{nataka zako habari  * haya nipa ja'fari\footnote{She knows something has happened.}  * uso usimeme hari\footnote{\Swa{hari} = \Swa{jasho}.}  * sura zimekupoteya\footnote{lit. \q{[your] features have changed}.  \Swa{-poteya} here = \Swa{-geuka, -badilika}.} } \\ 
\E{[She told Ja'far:] I want [to hear] your news,   tell it to me, Ja'far.   Your face is flushed,   and you are not your ordinary self.  } \\ 
\\[8mm] 

\textarabic{(٩٨) \textcolor{mygreen}{جَعْفَرِ أَكَبَئِنِ  * وَتَكَ هَبَرِ ڠَنِ  * نِكْوَمْبِيٖ لُغَ ڠَنِ  * كْوَكٗ إِوٖ نِ پِيَ }} \\* 
 \OLTcl{ piya ni iwe kwako *  gani lugha nikwambiye *  gani habari wataka *  akabaini jaʿfari} \\* 
\SB{96} (\textbf{98}) \OLTst{ja'fari akabaini\footnote{\Swa{-baini} = \Swa{-sema}.}  * wataka habari gani  * nikwambiye lugha gani  * kwako iwe\footnote{Ja'far is angry that his mother hid the truth about his father from him.} ni piya\footnote{Amu \Swa{piya} = Mvita \Swa{mpya}.} } \\ 
\E{Ja'far said:  What news do you want?   In what language should I tell you,    so that it will be new to you?   } \\ 
\\[8mm] 

\textarabic{(٩٩) \textcolor{mygreen}{كِكْوَمْبِيَ كِعَرَبُ  * نَيُوَ أُتَنِجِبُ  * تٖنَ نَؤٗنَ عَجَبُ  * مِمِ مْوٖنْيٖوٖ كُكْوَمْبِيَ }} \\* 
 \OLTcl{ kukwambiya mwenyewe mimi *  ʿajabu naona tena *  utanijibu nayuwa *  kiʿarabu kikwambiya} \\* 
\SB{97} (\textbf{99}) \OLTst{kikwambiya ki'arabu  * nayuwa utanijibu  * tena naona 'ajabu  * mimi mwenyewe kukwambiya } \\ 
\E{If I tell you in Arabic  I know you will answer:  "I am perplexed again"   [even if] I myself tell you.   } \\ 
\\[8mm] 

\textarabic{(١٠٠) \textcolor{mygreen}{كِكُؤُلِزَ أَلِكٗ  * بَبَ هُنَمْبِيَ هَكٗ  * تَنْڠُ نِنَ مِمْبَ يَكٗ  * أَلِفَرِكِ دُنِيَ }} \\* 
 \OLTcl{ duniya alifariki *  yako mimba nina tangu *  hako hunambiya baba *  aliko kikuuliza} \\* 
\SB{98} (\textbf{100}) \OLTst{kikuuliza aliko  * baba hunambiya hako\footnote{\Swa{hako} is the negative form of \Swa{yuko}, \E{he is there}, just as \Swa{siko} is the negative form of \Swa{niko}, \E{I am there}.}  * tangu nina mimba yako  * alifariki duniya } \\ 
\E{If I ask you [whether] he is alive,  my father, you tell me he is not:   "When I was still pregnant with you    he passed away from this world."  } \\ 
\\[8mm] 

\textarabic{(١٠١) \textcolor{mygreen}{كِكْوَمْبِيَ يُمُئِنِ  * أُتَڠٖؤُزَ مَنْڠِنٖ  * أُتَنَمْبِيَ وَفٖنٖ  * خٖيْرِ كُئِنْيَمَزِيَ }} \\* 
 \OLTcl{ kuinyamaziya khēri *  wafene utanambiya *  mangine utageuza *  yumuini kikwambiya} \\* 
\SB{99} (\textbf{101}) \OLTst{kikwambiya yumuini\footnote{i.e. that he has seen someone who might be his father.}  * utageuza mangine  * utanambiya wafene  * heri kuinyamaziya } \\ 
\E{If I tell you he is in the town,  you will change to other [words] --  you will tell me [I've seen someone who] looks like him,  and it's better to keep quiet about it.  } \\ 
\\[8mm] 

\textarabic{(١٠٢) \textcolor{mygreen}{وٖوٖ هُيَوَ فَرِسِ  * وَلَ مَكَه هُكُئِسِ  * وٖنْدٖلٖپِ مَجْلِسِ  * أُكَمُؤٗنَ عَلِيَ }} \\* 
 \OLTcl{ ʿaliya ukamuona *  majlisi wendelepi *  hukuisi makah wala *  farisi huyawa wewe} \\* 
\SB{100} (\textbf{102}) \OLTst{wewe huyawa\footnote{< \Swa{kuwa}.} farisi\footnote{\Swa{farisi}, \E{clever, skilful}, originally meant \q{horseman, rider}, for which skill is necessary.  Compare \Swa{farasi}, \E{horse}, 162b.}  * wala makah hukuisi  * wendelepi majlisi\footnote{\Swa{majlisi} = \Swa{baraza}: a meeting-place where men gather to chat and pass the time.}  * ukamuona 'aliya\footnote{i.e. your daily life does not take you to the sorts of places where you might meet Ali. } } \\ 
\E{[His mother said:] You are not worldly-wise,   nor do you know Mecca --   where did you go among people,  that you saw Ali?"  } \\ 
\\[8mm] 

\textarabic{(١٠٣) \textcolor{mygreen}{جَعْفَرِ كَبَئِنِ  * ٹُمٖؤٗنَنَ مْوِٹُنِ  * صِفَ زَكٖ مُعَيَنِ  * أُكِتَكَ تَكْوَمْبِيَ }} \\* 
 \OLTcl{ takwambiya ukitaka *  muʿayani zake ṣifa *  mwiţuni ţumeonana *  kabaini jaʿfari} \\* 
\SB{101} (\textbf{103}) \OLTst{ja'fari kabaini  * tumeonana mwituni  * sifa zake mu'ayani  * ukitaka takwambiya } \\ 
\E{Jaafar said:  We met in the forest --  a clear description,   if you want it, I will tell you.  } \\ 
\\[8mm] 

\textarabic{(١٠٤) \textcolor{mygreen}{نِسِكِزَ نِرَدِدِ  * كِوَ سِيٗ أُنِرُدِ  * كِمٗ چَكٖ هَكِزِدِ  * كَمَ چَنْڠُ أَنْڠَلِيَ }} \\* 
 \OLTcl{ angaliya changu kama *  hakizidi chake kimo *  unirudi siyo kiwa *  niradidi nisikiza} \\* 
\SB{102} (\textbf{104}) \OLTst{nisikiza niradidi  * kiwa siyo unirudi  * kimo chake hakizidi\footnote{See note to 67c.  People believe anecdotes about famous people, even if they are not likely or academically proven -- as the many magazines retailing celebrity gossip can attest.}  * kama changu angaliya } \\ 
\E{Listen to me, let me speak --  if it is not him, correct me --   his height is not much taller   than my own, look.   } \\ 
\\[8mm] 

\textarabic{(١٠٥) \textcolor{mygreen}{نَ لَ پِلِ نِبَئِنِ  * نْيٖيْ زَكٖ زَ كِتْوَنِ  * هَكُمٖيَ أُپَآنِ  * نَ كَمَ سِيٗ نَمْبِيَ }} \\* 
 \OLTcl{ nambiya siyo kama na *  upaãni hakumeya *  kitwani za zake nyee *  nibaini pili la na} \\* 
\SB{103} (\textbf{105}) \OLTst{na la pili nibaini  * nyee zake za kitwani  * hakumeya upaani\footnote{lit. \q{does not grow on the bald patch [that he has]}.  \Swa{ana upaa [mkubwa]}, \E{he's bald}.}  * na kama siyo nambiya } \\ 
\E{And let me tell you the second thing:    the hair on his head    does not cover his bald patch,  and if that is not so, tell me.    } \\ 
\\[8mm] 

\textarabic{(١٠٦) \textcolor{mygreen}{نَ يَ ٹَاٹُ أُفَهَمُ  * أُنَ ٹُنْدُ يَ كُزِمُ  * صِفَ زَكٖ زٖمٖتِمُ  * نِ هِزٗ نِمٖكْوَمْبِيَ }} \\* 
 \OLTcl{ nimekwambiya hizo ni *  zemetimu zake ṣifa *  kuzimu ya ţundu una *  ufahamu ţāţu ya na} \\* 
\SB{104} (\textbf{106}) \OLTst{na ya tatu ufahamu  * una tundu ya kuzimu\footnote{The meaning of this line is unclear.  \Swa{tundu} means \q{hole, pit}, and \Swa{kuzimu} means \q{the Underworld} -- (\Swa{kuzimu hakuna nyota}, \E{in the Underworld there are no stars}) -- but the implication here is obscure.}  * sifa zake zemetimu  * ni hizo nimekwambiya } \\ 
\E{And know the third thing:    he has a hole [leading to the] Underworld (?).    His description is complete --   it consists of these things that I have told you.   } \\ 
\\[8mm] 

\textarabic{(١٠٧) \textcolor{mygreen}{نِ رَعُوفُ وَ مَنٖنٗ  * كِشَ نِ جَڠِنَ مْنٗ  * نَ أُكِتَكَ مْفَنٗ  * هُنُ نِمٖكُپِجِيَ }} \\* 
 \OLTcl{ nimekupijiya hunu *  mfano ukitaka na *  mno jagina ni kisha *  maneno wa raʿūfu ni} \\* 
\SB{105} (\textbf{107}) \OLTst{ni ra'ufu\footnote{= \Swa{taratibu}, \E{polite}.} wa maneno  * kisha ni jagina\footnote{See 67c.} mno  * na ukitaka mfano\footnote{\Swa{-piga mfano}, \E{give an example of}.}  * hunu nimekupijiya } \\ 
\E{He is courteous of speech,    and further, he is a great warrior.    If you want a likeness of him,   I have given you this one.  } \\ 
\\[8mm] 

\textarabic{(١٠٨) \textcolor{mygreen}{هَيٗ أُنَمْبِزِيٖؤٗ  * نِ كْوٖلِ نْدِيٗ يَلِيٗ  * نِپَ جِنْسِ يَوٖءٖؤٗ  * هَتَ كُمْفَهَمِيَ }} \\* 
 \OLTcl{ kumfahamiya hata *  yaweeo jinsi nipa *  yaliyo ndiyo kweli ni *  unambiziyeo hayo} \\* 
\SB{106} (\textbf{108}) \OLTst{hayo unambiziyeo  * ni kweli ndiyo yaliyo  * nipa jinsi yaweeo\footnote{Amu \Swa{yaweeo} = Mvita \Swa{yalivyokuwa}.}  * hata kumfahamiya } \\ 
\E{[His mother said:] These things you have told me  are indeed exactly correct.    Tell me how it was   that you came to recognise him."  } \\ 
\\[8mm] 

\textarabic{(١٠٩) \textcolor{mygreen}{تَكُپَ ٹَنْڠُ أَوَلِ  * هِكِ چَكَ نِ ثَقِلِ  * كِتَنْڠَ مَاءِ نِ غَالِ  * نِكَتَكَسَ نَ نْدِيَ }} \\* 
 \OLTcl{ ndiya na nikatakasa *  ghāli ni mai kitanga *  thaqili ni chaka hiki *  awali ţangu takupa} \\* 
\SB{107} (\textbf{109}) \OLTst{takupa tangu awali  * hiki\footnote{\Swa{hiki} implies that the mother knows what drought he is referring to, i.e. she has experienced it too.} chaka\footnote{\Swa{chaka} < \Swa{-waka}, \E{burn}.} ni thaqili\footnote{Because of this, Ja'far had to travel farther than normal with his goats to find water, and this led to his meeting up with Ali.}  * kitanga mai ni ghali  * nikatakasa na ndiya\footnote{This comes very close to the English expression \q{hit the road}. \Swa{-takasa}, \E{stride along making a noise when your feet hit the ground}.} } \\ 
\E{[Jaafar said:] "I will give you [the whole story] from the beginning.   This drought was severe,    and wandering around [looking for] water was difficult,    so I travelled further afield.   } \\ 
\\[8mm] 

\textarabic{(١١٠) \textcolor{mygreen}{صَدِقِ يَنْڠُ قَوْلِ  * نِمٖتَنْڠَ بَرَ هِلِ  * مَاءِ هَپَنَ مَهَلِ  * نِ خٖيْرِ كَئِرُدِيَ }} \\* 
 \OLTcl{ kairudiya khēri ni *  mahali hapana mai *  hili bara nimetanga *  qawli yangu ṣadiqi} \\* 
\SB{108} (\textbf{110}) \OLTst{sadiqi yangu qauli  * nimetanga bara hili  * mai hapana mahali  * ni heri kairudiya } \\ 
\E{Believe my words:   I wandered around in the hinterland,   but there was no water anywhere,   [and I thought] I'd better come back.   } \\ 
\\[8mm] 

\textarabic{(١١١) \textcolor{mygreen}{هٖنْدَ هِوَزَ مٗيٗنِ  * نٖنٖنْدٖ جَنِبُ ڠَنِ  * كُؤٗنَ وَٹُ وٖنْڠِنٖ  * وَٹَٹُ وٖنْدٖمٖ نْدِيَ }} \\* 
 \OLTcl{ ndiya wendeme waţaţu *  wengine waţu kuona *  gani janibu nenende *  moyoni hiwaza henda} \\* 
\SB{109} (\textbf{111}) \OLTst{henda hiwaza moyoni  * \dotuline{ninende} janibu\footnote{\Swa{janibu} = \Swa{upande}.} gani  * \dotuline{kaona} watu wengine\footnote{Or we could emend to \Swa{wageni}, \E{strangers}.}  * watatu wendeme ndiya\footnote{The three of them were walking along \Swa{moja kwa moja} in Indian file - see the note on \Swa{-andama}.} } \\ 
\E{As I went along, pondering in my heart   which direction I should take,   I saw some people,   three of them, coming along the road.   } \\ 
\\[8mm] 

\textarabic{(١١٢) \textcolor{mygreen}{كَوَءٗنَ وَكٗ مْبَلِ  * كَنٖنَ تَئِمُهُلِ  * مَرَ نِكِوَصِلِ  * مَاءِ وَتَنَمْبِيَ }} \\* 
 \OLTcl{ watanambiya mai *  nikiwaṣili mara *  taimuhuli kanena *  mbali wako kawaona} \\* 
\SB{110} (\textbf{112}) \OLTst{kawaona wako mbali  * kanena\footnote{Note that \Swa{-nena}, \E{speak, say}, here means \q{intend}.} taimuhuli\footnote{\Swa{-i-muhuli} = \Swa{-ji-ngojesha}.}  * mara\footnote{\Swa{mara} here = \Swa{pengine}.} \dotuline{wakiwasili}  * mai watanambiya } \\ 
\E{I saw them when they were far off,   and I said to myself that I should wait --  once they get here  they can tell me [where to find] water."  } \\ 
\\[8mm] 

\textarabic{(١١٣) \textcolor{mygreen}{پَنَ كِڤُلِ كِنْيٖسَ  * هَتَ نَاءٗ وَكَپِٹَ  * نِوَوٖنٖ وَكِنُسَ  * سَلَامُ كَوَپِسِيَ }} \\* 
 \OLTcl{ kawapisiya salāmu *  wakinusa niwawene *  wakapiţa nao hata *  kinyesa kivuli pana} \\* 
\SB{111} (\textbf{113}) \OLTst{pana kivuli kinyesa  * hata nao wakapita  * niwawene \dotuline{wakitusa}  * salamu kawapisiya\footnote{Amu \Swa{-pisa salamu} = Mvita \Swa{-toa salamu}.  The greeting \Swa{salaam alekum} is used only to groups of more than one person.} } \\ 
\E{Waiting (?) there in the shade   until they had passed,   when I had seen them go by  I greeted them.  } \\ 
\\[8mm] 

\textarabic{(١١٤) \textcolor{mygreen}{كَمْبَ مْوٖنْدَپِ مَتِتِ  * هِكِ نِ كِپُنْڠُ كَٹِ  * هٖلَ نْدٗنِ مُكٖيْتِ  * لِپَٹٖ كُپِنْدُكِيَ }} \\* 
 \OLTcl{ kupindukiya lipaţe *  mukēti ndoni hela *  kaţi kipungu ni hiki *  matiti mwendapi kamba} \\* 
\SB{112} (\textbf{114}) \OLTst{kamba mwendapi matiti\footnote{\Swa{kama Wazungu}, \E{like Europeans}!}  * hiki ni kipungu kati\footnote{\Swa{kipungu-kati} = \Swa{mti-kati, saa sita, jua kali}.}  * hela\footnote{Amu \Swa{hela} = Mvita \Swa{hebu}.  Ja'far's invitation to the three men is not as polite as it might be, which partly accounts for their response.} ndoni muketi  * lipate kupindukiya\footnote{We are to understand \Swa{jua}.  \Swa{-pindukia} is lit. \q{change direction}, i.e. the sun ascends through the sky until noon, and then begins to decline.} } \\ 
\E{I said: Where are you going in such a hurry?   This is high noon --    why don't you come and sit down   until the sun goes down a bit?  } \\ 
\\[8mm] 

\textarabic{(١١٥) \textcolor{mygreen}{نَ هِلِ يُوَ سِ زُرِ  * كُلَنْدَمَ نِ خَطَرِ  * وَلَ زٖؤٗ سِ أَخِرِ  * كَمَ مُتَلِمَٹِيَ }} \\* 
 \OLTcl{ mutalimaţiya kama *  akhiri si zeo wala *  khaṭari ni kulandama *  zuri si yuwa hili na} \\* 
\SB{113} (\textbf{115}) \OLTst{na hili yuwa si zuri\footnote{Even though he is only a child, Ja'far gives advice to the men.}  * kulandama\footnote{\Swa{-andama} = \Swa{-fuata}.  See 34d.} ni hatari  * wala zeo\footnote{Amu \Swa{zeo}, 9/10 = Mvita \Swa{wakati}.  Compare \Swa{njeo} in Muyaka.} si ahiri\footnote{lit. \q{end}.}  * kama mutalimatiya\footnote{\Swa{-limatiya} = \Swa{-chelewa}.  In other words, if they stop for a bit, they will not arrive at their destination so late that they will sleep in the next morning.} } \\ 
\E{You should know that this [sun] is not good [for you] --     to go about in it [for long] is dangerous,   nor is the time so late    that you will be delayed [if you stop here]."  } \\ 
\\[8mm] 

\textarabic{(١١٦) \textcolor{mygreen}{وَكِسِكِيَ كَلِمَ  * وَكِزُنْڠُكِيَ نْيُمَ  * وَوِلِ وَكَسِمَمَ  * مْمٗيَ كَنِئِلِيَ }} \\* 
 \OLTcl{ kaniiliya mmoya *  wakasimama wawili *  nyuma wakizungukiya *  kalima wakisikiya} \\* 
\SB{114} (\textbf{116}) \OLTst{wakisikiya kalima  * wakizungukiya nyuma  * wawili wakasimama  * mmoya kaniiliya } \\ 
\E{When they heard my words,  they turned round.  Two stood where they were,  and one came up to me.  } \\ 
\\[8mm] 

\textarabic{(١١٧) \textcolor{mygreen}{أَكِجَ أَكَبَئِنِ  * كَنِؤُزَ نْدِوٖ نَنِ  * أَوْ وَٹُتَكِيَنِ  * خَطَرِ كُٹُفِكِيَ }} \\* 
 \OLTcl{ kuţufikiya khaṭari *  waţutakiyani aw *  nani ndiwe kaniuza *  akabaini akija} \\* 
\SB{115} (\textbf{117}) \OLTst{akija akabaini  * kaniuza ndiwe nani\footnote{This is a rude response.  They are suspicious because he is being over-familiar, and yet they do not know him.}  * au watutakiyani  * hatari \dotuline{kitufikiya} } \\ 
\E{When he came over he spoke,  and asked me: Who are you?   And why are you concerned about us,  and about danger coming to us?  } \\ 
\\[8mm] 

\textarabic{(١١٨) \textcolor{mygreen}{سِسِ هَٹُشِكِ يَكٗ  * وَلَ سِ نْدُڠُ زَكٗ  * أُئٖٹَيٗ مَتَمْكٗ  * يَپٖسٖنِ كُٹْوَمْبِيَ }} \\* 
 \OLTcl{ kuţwambiya yapeseni *  matamko ueţayo *  zako ndugu si wala *  yako haţushiki sisi} \\* 
\SB{116} (\textbf{118}) \OLTst{sisi hatushiki yako\footnote{We understand \Swa{maneno}.  This is very rude.  There is a saying: \Swa{usishike maneno ya wanawake}, \E{don't take the word of women}.}  * wala si ndugu zako\footnote{Again, very rude.}  * uetayo matamko  * yapeseni\footnote{\Swa{ilifaa vipi}. \Swa{-pasa}, \E{be obliged to, have to}.} kutwambiya } \\ 
\E{We will not take your [advice] --   we are not relatives of yours.    The words you have spoken,  what good is it to tell us them?"  } \\ 
\\[8mm] 

\textarabic{(١١٩) \textcolor{mygreen}{وٖوٖ نِ مْوَنَ آدَمُ  * نِ مْپٗتٖزَ قَوْمُ  * ٹْوَمْبِيٖ ٹُكُفَهَمُ  * كْوَنْدَ ٹُيُوٖ طَبِيَ }} \\* 
 \OLTcl{ ṭabiya ţuyuwe kwanda *  ţukufahamu ţwambiye *  qawmu mpoteza ni *  ãdamu mwana ni wewe} \\* 
\SB{117} (\textbf{119}) \OLTst{wewe \dotuline{si} mwana adamu  * ni mpoteza qaumu\footnote{i.e. a jinn or the Devil.}  * twambiye tukufahamu  * kwanda tuyuwe tabiya } \\ 
\E{You are not a human being,    you are [a spirit] who makes people lose their way.   Tell us so that we can know you,  let us first know your character.   } \\ 
\\[8mm] 

\textarabic{(١٢٠) \textcolor{mygreen}{كَوَجِبِشَ قَوْلِ  * نِ دِيْنِ يَكٖ رَسُوْلِ  * كِوَ وٖوٖ نِ جَهِلِ  * مْبٖلٖ زَنْڠُ نٗنْدٗكٖيَ }} \\* 
 \OLTcl{ nondokeya zangu mbele *  jahili ni wewe kiwa *  rasūli yake dı̄ni ni *  qawli kawajibisha} \\* 
\SB{118} (\textbf{120}) \OLTst{kawajibisha qauli  * ni dini yake rasuli  * kiwa wewe ni jahili\footnote{\Swa{jahili}, \E{someone ignorant of the truth}, in this case of Islam.}  * mbele zangu nondokeya } \\ 
\E{I answered them with the words:  [My religion] is the religion of the Prophet ---    if you are an unbeliever,    go away from in front of me.   } \\ 
\\[8mm] 

\textarabic{(١٢١) \textcolor{mygreen}{كَسِكِيَ هُفَسِرِ  * هُمْوِٹَ أَبُوْ بَكَرِ  * نْدٗوْ وٖوٖ نَ زُبَيْرِ  * وٗتٖ وَوِلِ وَكَيَ }} \\* 
 \OLTcl{ wakaya wawili wote *  zubayri na wewe ndoo *  bakari abuu humwiţa *  hufasiri kasikiya} \\* 
\SB{119} (\textbf{121}) \OLTst{kasikiya hufasiri  * humwita abuu bakari  * ndoo wewe na zubayri\footnote{Abu Bakr and Zubeir are two of the \Swa{masahaba}, the Companions of the Prophet.}  * wote wawili wakaya } \\ 
\E{And I heard him speak  and call: "Abu Bakr,   come here, and you Zubeir!    And both of them came over.   } \\ 
\\[8mm] 

\textarabic{(١٢٢) \textcolor{mygreen}{وَكَيَ وَكَسِمَمَ  * أُسٗ وَكَنِٹِزَمَ  * كَمْبَ مُكِمْفَهَمَ  * وَجْهِ وَكٖ نَبِيَ }} \\* 
 \OLTcl{ nabiya wake wajhi *  mukimfahama kamba *  wakaniţizama uso *  wakasimama wakaya} \\* 
\SB{120} (\textbf{122}) \OLTst{wakaya wakasimama  * uso wakanitizama  * kamba mukimfahama  * wajhi\footnote{\Swa{wajhi} = \Swa{uso}.} wake nabiya } \\ 
\E{They came over and stood,  and looked at my face.  [The first man] said: When you look at him closely,  his face [resembles] the Prophet's.   } \\ 
\\[8mm] 

\textarabic{(١٢٣) \textcolor{mygreen}{هُسِكِيَ هُنِجِبُ  * عَلِي بِنْ طَالِبُ  * مَمَ وَمٖتَعَجَبُ  * كْوَ وٗتٖ هُنَنْڠَلِيَ }} \\* 
 \OLTcl{ hunangaliya wote kwa *  wametaʿajabu mama *  ṭālibu bin ʿalii *  hunijibu husikiya} \\* 
\SB{121} (\textbf{123}) \OLTst{husikiya\footnote{\Swa{hu-} here = \Swa{waka-}.} hunijibu  * 'alii bin talibu  * mama\footnote{\Swa{mama!} is an expression of disbelief.} wameta'ajabu  * kwa wote hunangaliya } \\ 
\E{When [the others] heard this, [they said:] He reminds me  of Ali ibn Talib.   Impossible!, they [said] in amazement,  as they all stared at me.   } \\ 
\\[8mm] 

\textarabic{(١٢٤) \textcolor{mygreen}{هٗىٗ نِ أَبُوْ بَكَرِ  * نَ هٗيٗ هُئِٹْوَ زُبٖئْرِ  * نَوٖ لَكٗ هُفَسِرِ  * أِنَ ٹُكَلِسِكِيَ }} \\* 
 \OLTcl{ ţukalisikiya ina *  hufasiri lako nawe *  zuberi huiţwa hoyo na *  bakari abuu ni hoyo} \\* 
\SB{122} (\textbf{124}) \OLTst{hoyo ni abuu bakari  * na hoyo huitwa zuberi  * nawe lako hufasiri\footnote{\Swa{mbona husemi?}, \E{why aren't you speaking?}.}  * ina tukalisikiya } \\ 
\E{This is Abu Bakr, [said the first man],    and this is Zubeir.    But you have not spoken your   name for us to hear it.  } \\ 
\\[8mm] 

\textarabic{(١٢٥) \textcolor{mygreen}{أَكَتَمْكَ مْبُجِ  * مِمِ هُئِٹْوَ أَزْوَجِ  * نَمِ سَسَ نَتَرَجِ  * نَ إِنَ لَكٗ نَمْبِيَ }} \\* 
 \OLTcl{ nambiya lako ina na *  nataraji sasa nami *  azwaji huiţwa mimi *  mbuji akatamka} \\* 
\SB{123} (\textbf{125}) \OLTst{akatamka mbuji  * mimi huitwa azwaji\footnote{\Swa{azwaji} is literally \q{a couple}, so this name is strange.}  * nami sasa nataraji  * na ina lako nambiya } \\ 
\E{This gentleman spoke:  I am called Azwaj,   and I now hope   [you] will tell me your name too.    } \\ 
\\[8mm] 

\textarabic{(١٢٦) \textcolor{mygreen}{كَوَمْبِيَ نِمٖكِرِ  * نَمِ إِنَ كُفَسِرِ  * مِمِ نْدِيٖ جَعْفَرِ  * وَ مَوْلَانَا عَلِيَّ }} \\* 
 \OLTcl{ ʿaliyaّ mawlānā wa *  jaʿfari ndiye mimi *  kufasiri ina nami *  nimekiri kawambiya} \\* 
\SB{124} (\textbf{126}) \OLTst{kawambiya nimekiri  * nami ina kufasiri  * mimi ndiye ja'fari  * wa maulana 'aliyaU+0651 } \\ 
\E{And I told them: I have decided  to speak my name too.   I am Ja'far,   [son] of Lord Ali.   } \\ 
\\[8mm] 

\textarabic{(١٢٧) \textcolor{mygreen}{وَنِؤُزٖ تَرَتِبُ  * وٖنْدَءٗ وَپِ غَرِيْبُ  * هَپٗ مَمَ كَوَجِبُ  * نِتَكَلٗ كَوَمْبِيَ }} \\* 
 \OLTcl{ kawambiya nitakalo *  kawajibu mama hapo *  gharı̄bu wapi wendao *  taratibu waniuze} \\* 
\SB{125} (\textbf{127}) \OLTst{waniuze taratibu  * wendao wapi gharibu\footnote{\Swa{gharibu} = \Swa{mgeni}.}  * hapo mama kawajibu  * nitakalo kawambiya } \\ 
\E{And they asked me politely,  Where are you going, stranger?   Then, Mother, I answered them:   telling them what I had [earlier] intended.  } \\ 
\\[8mm] 

\textarabic{(١٢٨) \textcolor{mygreen}{نِمٖتَنْڠَ مَاءِ بَرَنِ  * نِمٖچٗكَ سِيَؤٗنِ  * مْبُزِ وَمٖلِشَ يَنِ  * سَسَ كُيُتَ هَلِيَ }} \\* 
 \OLTcl{ haliya kuyuta sasa *  yani wamelisha mbuzi *  siyaoni nimechoka *  barani mai nimetanga} \\* 
\SB{126} (\textbf{128}) \OLTst{nimetanga mai barani  * nimechoka siyaoni  * mbuzi wamelisha yani  * sasa \dotuline{kwa nyota} haliya } \\ 
\E{I have wandered about [searching for] water in the scrubland --   I am tired and I still haven't seen any.  The goats have eaten all the grass,   and now they are bleating for want of water.   } \\ 
\\[8mm] 

\textarabic{(١٢٩) \textcolor{mygreen}{هُكٗ مْمٖزٗپِٹَ  * مَاءِ هَمْكُيَكُٹَ  * مْبُزِ وَنْڠُ وَنَ نْيٗٹَ  * چَمْبَ مْوَيُوَ نَمْبِيَ }} \\* 
 \OLTcl{ nambiya mwayuwa chamba *  nyoţa wana wangu mbuzi *  hamkuyakuţa mai *  mmezopiţa huko} \\* 
\SB{127} (\textbf{129}) \OLTst{huko mmezopita  * mai hamkuyakuta  * mbuzi wangu wana nyota  * chamba mwayuwa nambiya } \\ 
\E{In that area you've passed through,  have you not come upon any water there?  My goats are thirsty --    if you know of [a well], tell me."   } \\ 
\\[8mm] 

\textarabic{(١٣٠) \textcolor{mygreen}{وَكَنِجِبُ قَوْلِ  * كٔوَمْبَ كِسِمَ سِ مْبَلِ  * لَكِنِ كِنَ ثَقِلِ  * هِيٗ نْدٗوْ كُئِٹِيَ }} \\* 
 \OLTcl{ kuiţiya ndoo hiyo *  thaqili kina lakini *  mbali si kisima kwamba *  qawli wakanijibu} \\* 
\SB{128} (\textbf{130}) \OLTst{wakanijibu qauli  * kwamba kisima si mbali  * lakini kina thaqili  * hiyo ndoo kuitiya\footnote{Because Ali has covered it up.} } \\ 
\E{They answered me with word  that there was a well not far away.    But, [they said,] it is difficult   to put the bucket into it.   } \\ 
\\[8mm] 

\textarabic{(١٣١) \textcolor{mygreen}{هَيٗ مَاءِ نِ مَتَمُ  * مْفَنٗ وَ زَمْزَمُ  * لَكِنِ سِسِ فَهَمُ  * كُكُؤٗنْيَ هُچٖلٖيَ }} \\* 
 \OLTcl{ hucheleya kukuonya *  fahamu sisi lakini *  zamzamu wa mfano *  matamu ni mai hayo} \\* 
\SB{129} (\textbf{131}) \OLTst{hayo mai ni matamu  * mfano wa zamzamu\footnote{Zamzam is a sacred spring in Mecca, situated close to the Ka'aba.}  * lakini sisi fahamu  * kukuonya hucheleya\footnote{\Swa{tunaogopa}.  Because they are not sure how Ali will react to someone else using the well.} } \\ 
\E{The water is sweet,    just like Zamzam's,   but we, you understand,   are afraid to show it to you.  } \\ 
\\[8mm] 

\textarabic{(١٣٢) \textcolor{mygreen}{أُوَپٗ أُمٖخِتَرِ  * كُكُپٖكَ ٹُتَيَرِ  * وَلَ أُسِٹُفَسِرِ  * نِ سِسِ ٹٗلٗكْوَمْبِيَ }} \\* 
 \OLTcl{ ţolokwambiya sisi ni *  usiţufasiri wala *  ţutayari kukupeka *  umekhitari uwapo} \\* 
\SB{130} (\textbf{132}) \OLTst{uwapo umehitari  * kukupeka tutayari  * wala usitufasiri  * ni sisi \dotuline{tulokwambiya} } \\ 
\E{If you want to risk it,  we are ready to take you there,  but do not mention us,  [that] it was us who told you [about it].   } \\ 
\\[8mm] 

\textarabic{(١٣٣) \textcolor{mygreen}{وَكَنِؤٗنْيَ أُسِٹَ  * هَپٗ نْدِيَ كَفُوَٹَ  * كْوَ مْوِٹُنِ وَكَپِٹَ  * مْبِيٗ وَكَنِتٗلٖيَ }} \\* 
 \OLTcl{ wakanitoleya mbiyo *  wakapiţa mwiţuni kwa *  kafuwaţa ndiya hapo *  usiţa wakanionya} \\* 
\SB{131} (\textbf{133}) \OLTst{wakanionya usita\footnote{\Swa{usita} = \Swa{barabara}.}  * hapo ndiya kafuwata  * kwa mwituni wakapita  * \dotuline{mbee} wakanitoleya\footnote{\Swa{-toleya}, \E{give directions by accompanying a person to a good place to give them from}.} } \\ 
\E{They showed me the way,  and then I followed the path.   They went into the forest,   and directed me onwards.  } \\ 
\\[8mm] 

\textarabic{(١٣٤) \textcolor{mygreen}{هَتَ كِدُسَ كِسِمَ  * لِپٗ بَاءٗ كَسُكُمَ  * كْوَ كِوَڤُ كُٹِزَمَ  * وَءٗ هُنِئَنْڠَلِيَ }} \\* 
 \OLTcl{ huniangaliya wao *  kuţizama kiwavu kwa *  kasukuma bao lipo *  kisima kidusa hata} \\* 
\SB{132} (\textbf{134}) \OLTst{hata \dotuline{kidosa}\footnote{\Swa{-dosa} = \Swa{-gota, -gogota}, \E{knock, rap}.  As Ja'far walks over the planks covering the well, he hears the resonating sound of the well beneath them.} kisima  * lipo bao kasukuma  * kwa kiwavu kutizama\footnote{The meaning of this line is unclear. }  * wao huniangaliya } \\ 
\E{Until, when I came to the well,   there was a plank there that I pushed away.   They watched me from one side,   looking at me.  } \\ 
\\[8mm] 

\textarabic{(١٣٥) \textcolor{mygreen}{جَعْفَرِ أَتَمْكٖ  * بُوْ بَكَرِ سِشُٹُكٖ  * كِسِمَ سِكِفُنِكٖ  * نِتَرُدِ كُكْوَمْبِيَ }} \\* 
 \OLTcl{ kukwambiya nitarudi *  sikifunike kisima *  sishuţuke bakari buu *  atamke jaʿfari} \\* 
\SB{133} (\textbf{135}) \OLTst{ja'fari atamke\footnote{Past tense.}  * buu bakari sishutuke  * kisima sikifunike  * nitarudi \dotuline{nakwambiya}\footnote{Presumably Ja'far means that now he has found this well he will come back each day with his goats, so there is no point covering the well. } } \\ 
\E{Ja'far said:  Abu Bakr, do not worry --   do not cover the well.  I will return, I tell you.  } \\ 
\\[8mm] 

\textarabic{(١٣٦) \textcolor{mygreen}{سِنَ خٗوْفُ مٗيٗ وَنْڠُ  * كُنْوَ مَاءِ مْبُزِ وَنْڠُ  * كَنٖنَ نٖنْدَ زَنْڠُ  * نِزِتَكَسٖ نَ نْدِيَ }} \\* 
 \OLTcl{ ndiya na nizitakase *  zangu nenda kanena *  wangu mbuzi mai kunwa *  wangu moyo khōfu sina} \\* 
\SB{134} (\textbf{136}) \OLTst{sina hofu moyo wangu  * kunwa mai mbuzi wangu  * kanena nenda zangu  * nizitakase\footnote{\Swa{-takasa}, \E{shake}, with \Swa{nyayo}, \E{footsteps} understood.  The meaning is to shake the road by travelling a lot.} na ndiya } \\ 
\E{I have no fear [in] my heart    that my goats should drink the water.    I said: I am going now,   so that I can herd them along the road."   } \\ 
\\[8mm] 

\textarabic{(١٣٧) \textcolor{mygreen}{يُوَ كُكِپَمْبَؤُكَ  * نِنَ فُرَهَ هُتٖكَ  * مَلِشٗنِ كِوَپٖكَ  * مُدَ وَ يُوَ كُوَاءَ }} \\* 
 \OLTcl{ kuwaa yuwa wa muda *  kiwapeka malishoni *  huteka furaha nina *  kukipambauka yuwa} \\* 
\SB{135} (\textbf{137}) \OLTst{yuwa kukipambauka  * nina furaha huteka  * malishoni kiwapeka  * muda wa yuwa kuwaa\footnote{\Swa{-waa} = \Swa{-waka}, \E{burn}.} } \\ 
\E{When the sun rose [next day]  I was laughing with joy,   and took [the goats] to the pastures  when the sun was burning hot.    } \\ 
\\[8mm] 

\textarabic{(١٣٨) \textcolor{mygreen}{أَوَلِ يَ سَاءَ سِتَ  * هَپٗ مْبُزِ كَوَسُتَ  * نَؤٗنَ وَمٖنِوَتَ  * هُتُرَ نَ كُكُمْبِيَ }} \\* 
 \OLTcl{ kukumbiya na hutura *  wameniwata naona *  kawasuta mbuzi hapo *  sita saa ya awali} \\* 
\SB{136} (\textbf{138}) \OLTst{awali ya saa sita  * hapo mbuzi kawasuta\footnote{Amu \Swa{-suta} = Mvita \Swa{-shunga, -fukuza}, \E{shoo animals on, drive animals along}.}  * naona\footnote{The tense here gives the nuance of \q{suddenly}.} wameniwata  * hutura\footnote{\Swa{-tura} = \Swa{-ruka}, \E{jump, bound}.} na kukumbiya } \\ 
\E{Just before the seventh hour (noon),    I was then driving the goats along,   and I saw that they had broken away from me,  running and frisking.   } \\ 
\\[8mm] 

\textarabic{(١٣٩) \textcolor{mygreen}{كَوَمْبِيَ إٖنٖنْدَنِ  * مْوَكُيُوَ كِسِمَنِ  * لَِكِنِ هُفَلِيَنِ  * نْدِمِ وَ كُوَٹٖكٖيَ }} \\* 
 \OLTcl{ kuwaţekeya wa ndimi *  hufaliyani laikini *  kisimani mwakuyuwa *  enendani kawambiya} \\* 
\SB{137} (\textbf{139}) \OLTst{kawambiya enendani  * mwakuyuwa kisimani  * laikini hufaliyani  * ndimi wa kuwatekeya\footnote{i.e. there is no point in the goats running ahead of Ja'far and reaching the well before him, because once they are there they will have to stand and wait for him to get the water for them.} } \\ 
\E{I told them: On you go --  you know where the well is.  But what good will it do you,  when I'm the only one who can draw water for you?   } \\ 
\\[8mm] 

\textarabic{(١٤٠) \textcolor{mygreen}{وَكٖنْدَ وَكَسِمَمَ  * كُفُنِشِوٖ كِسِمَ  * بَاءٗ نِكَلِسُكُمَ  * نِمٗنٖ مْٹُ أَكِيَ }} \\* 
 \OLTcl{ akiya mţu nimone *  nikalisukuma bao *  kisima kufunishiwe *  wakasimama wakenda} \\* 
\SB{138} (\textbf{140}) \OLTst{wakenda wakasimama  * kufunishiwe kisima\footnote{i.e. the well has been covered over again, in spite of Ja'far uncovering it the day before (134d) and telling Abu Bakr that there was no point in covering it (135c).  The reason, of course, as we know from the earlier verses is that Ali has come to check on the well, and covered it (53b).}  * bao nikalisukuma  * nimone\footnote{Again, the tense gives the nuance of \Eit{suddenly} -- see 138c.} mtu akiya } \\ 
\E{They went on and stood  where the well had been covered over.  I pushed away the plank,  and I saw someone coming.   } \\ 
\\[8mm] 

\textarabic{(١٤١) \textcolor{mygreen}{أَكِيَ أَكَنِشِكَ  * مَاءٖ نِسِيَٹٖكَ  * مَمَ هَپٗ كَتَمْكَ  * يَ غَضَبُ كَمْوَمْبِيَ }} \\* 
 \OLTcl{ kamwambiya ghaḍabu ya *  katamka hapo mama *  nisiyaţeka mae *  akanishika akiya} \\* 
\SB{139} (\textbf{141}) \OLTst{akiya akanishika\footnote{What happens next has already been described in 58 ff.}  * mae nisiyateka  * mama\footnote{\Swa{mama!} -- see 123c.} hapo katamka  * ya ghadhabu kamwambiya } \\ 
\E{When he arrived he grabbed hold of me  before I had drawn any water.  Gosh!  At that point I spoke,   and addressed him angrily.   } \\ 
\\[8mm] 

\textarabic{(١٤٢) \textcolor{mygreen}{نِكَمُحِمِدِ مْنْڠُ  * كُنٖٹٖيَ بَبَنْڠُ  * كَنِؤُزَ پٖٹٖ يَنْڠُ  * چَنْدَنِ كَمْتٗلٖيَ }} \\* 
 \OLTcl{ kamtoleya chandani *  yangu peţe kaniuza *  babangu kuneţeya *  mngu nikamuḥimidi} \\* 
\SB{140} (\textbf{142}) \OLTst{nikamuhimidi\footnote{cf. 152b. } mngu  * kuneteya babangu  * kaniuza pete yangu  * chandani kamtoleya } \\ 
\E{I pleaded with God  to send me my father.  [The man] asked me about the ring   on my finger, and I gave it to him.  } \\ 
\\[8mm] 

\textarabic{(١٤٣) \textcolor{mygreen}{پٖٹٖ أَكَئِٹِزِمَ  * كَپِجَ نَ هَلِمَمَ  * يَپِسِيٖ يَ نْيُمَ  * يٗتٖ يَكَمْرُدِيَ }} \\* 
 \OLTcl{ yakamrudiya yote *  nyuma ya yapisiye *  halimama na kapija *  akaiţizima peţe} \\* 
\SB{141} (\textbf{143}) \OLTst{pete akaitizima  * kapija na halimama  * yapisiye ya nyuma\footnote{\Swa{yale mambo yaliyopita zamani}.}  * yote yakamrudiya } \\ 
\E{He looked at the ring  and became anxious.   everything that had happened in the past,   all of it came back to him.  } \\ 
\\[8mm] 

\textarabic{(١٤٤) \textcolor{mygreen}{جِنَ أَلِپٗنِؤُلِزَ  * نِسِمْوَمْبِيٖ كَئِزَ  * أَكَنِپَ مِؤُجِزَ  * پِيَ نَ كُنِپِجِيَ }} \\* 
 \OLTcl{ kunipijiya na piya *  miujiza akanipa *  kaiza nisimwambiye *  aliponiuliza jina} \\* 
\SB{142} (\textbf{144}) \OLTst{jina aliponiuliza  * nisimwambiye kaiza  * akanipa miujiza\footnote{i.e. unless he really was Ja'far's father.}  * piya na kunipijiya\footnote{\Swa{-piga mifano}, \E{give examples}.} } \\ 
\E{When he asked me my name  I wouldn't tell him -- I refused.  He told me things he could not have known [unless he was my father]  giving me example after example.   } \\ 
\\[8mm] 

\textarabic{(١٤٥) \textcolor{mygreen}{أُنِپِيٖ صُوْرَ زَكٗ  * نَ صِفَ زَ نْيُمْبَ يَكٗ  * هَپٗ كَئٖٹَ تَمْكٗ  * إِنَ لَنْڠُ كَمْوَمْبِيَ }} \\* 
 \OLTcl{ kamwambiya langu ina *  tamko kaeţa hapo *  yako nyumba za ṣifa na *  zako ṣūra unipiye} \\* 
\SB{143} (\textbf{145}) \OLTst{\dotuline{unipee} sura zako  * na sifa za nyumba yako  * hapo kaeta tamko  * ina langu kamwambiya } \\ 
\E{He described your features to me,   and the characteristics of your house.     Then I spoke   and told him my name.   } \\ 
\\[8mm] 

\textarabic{(١٤٦) \textcolor{mygreen}{نِكَمْوَمْبِيَ نْيَكَ  * نِزٖزٗوٖءٗ هَكِكَ  * نِ تِسِيَ زِسِزٗ شَكَ  * نَ وٖوٖ تَرٖهٖ ٹِيَ }} \\* 
 \OLTcl{ ţiya tarehe wewe na *  shaka zisizo tisiya ni *  hakika nizezoweo *  nyaka nikamwambiya} \\* 
\SB{144} (\textbf{146}) \OLTst{nikamwambiya nyaka  * \dotuline{nizeweo}\footnote{= \Swa{nilizozaliwa}.} hakika  * ni tisiya zisizo shaka  * na wewe tarehe tiya\footnote{= \Swa{kumbuka tarehe}.} } \\ 
\E{I told him [the number of] years  since I was born -- definitely  it is nine, and no mistake;    and you should remember the number.    } \\ 
\\[8mm] 

\textarabic{(١٤٧) \textcolor{mygreen}{خَبَرِ زَكٖ تِمَمُ  * نِمٖكُپَ أُفَهَمُ  * تٖنَ نَ كْوَ مْوَلِمُ  * نِمٖرُدِ كُمْوَمْبِيَ }} \\* 
 \OLTcl{ kumwambiya nimerudi *  mwalimu kwa na tena *  ufahamu nimekupa *  timamu zake khabari} \\* 
\SB{145} (\textbf{147}) \OLTst{habari zake timamu  * nimekupa ufahamu  * tena na kwa mwalimu  * nimerudi kumwambiya } \\ 
\E{The news about him is finished.   I have completed it so that you may understand.  And via my teacher's [house]    I came back to tell him [about it].  } \\ 
\\[8mm] 

\textarabic{(١٤٨) \textcolor{mygreen}{نِمٖمُؤَڠَ كْوَ خٖيْرِ  * أَسُبهِ نِ سَفَرِ  * نِؤٗمْبٖيَ كْوَ جَبَارِ  * نَ رَضِ كُنِوٖيَ }} \\* 
 \OLTcl{ kuniweya raḍi na *  jabāri kwa niombeya *  safari ni asubhi *  khēri kwa nimemuaga} \\* 
\SB{146} (\textbf{148}) \OLTst{nimemuaga kwa heri  * \dotuline{asubuhi} ni safari  * niombeya kwa jabari  * na radhi kuniweya } \\ 
\E{I have said farewell to him.   [tomorrow] morning I will set off [to go to my father].   Intercede for me to the Almighty,   and give me your blessing.   } \\ 
\\[8mm] 

\textarabic{(١٤٩) \textcolor{mygreen}{تٖنَ نِؤٗمْبٖيَ مْنْڠُ  * ٹُپٖنْدَنٖ نَ بَبَنْڠُ  * نَمِ كِشَ مُئِ وَنڠُ  * تَكُيَ كُوَنْڠَلِيَ }} \\* 
 \OLTcl{ kuwangaliya takuya *  wangu mui kisha nami *  babangu na ţupendane *  mngu niombeya tena} \\* 
\SB{147} (\textbf{149}) \OLTst{tena niombeya mngu  * tupendane na babangu  * nami kisha mui wangu  * takuya\footnote{i.e. he will return to his town for periodic visits.} kuwangaliya } \\ 
\E{And intercede for me to God   that my father and I will get along well together.   And then [the people] in my town    I will come and visit them.  } \\ 
\\[8mm] 

\textarabic{(١٥٠) \textcolor{mygreen}{أَكِسِكِيَ قَوْلِ  * مَمَكٖ أَسِحِمِلِ  * أَكَتَرَدَدِ عَقِلِ  * كْوَ مَكٗنْدٖ كَئِٹِيَ }} \\* 
 \OLTcl{ kaiţiya makonde kwa *  ʿaqili akataradadi *  asiḥimili mamake *  qawli akisikiya} \\* 
\SB{148} (\textbf{150}) \OLTst{akisikiya qauli  * mamake asihimili  * akataradadi\footnote{\Swa{taradadi} = \Swa{-badilika}.} 'aqili  * kwa makonde kaitiya } \\ 
\E{When she heard these words  his mother could not bear it.  She went out of her mind  and beat herself with her fists.   } \\ 
\\[8mm] 

\textarabic{(١٥١) \textcolor{mygreen}{كَئِٹُنْدَ كَيِنْڠُشَ  * تِيَتِ أَكَئِرُشَ  * هَتَ نْڠُوٗ كَمْڤِشَ  * إِكَوَ كُمْسٗمٖيَ }} \\* 
 \OLTcl{ kumsomeya ikawa *  kamvisha nguwo hata *  akairusha tiyati *  kayingusha kaiţunda} \\* 
\SB{149} (\textbf{151}) \OLTst{kaitunda\footnote{Amu \Swa{-tunda} = Mvita \Swa{-twaa}.} kayingusha  * tiyati akairusha\footnote{She does not know what she is doing.}  * hata nguwo kamvisha\footnote{The \Swa{kanga} is a wraparound garment, which is knotted, not sewn closed, so if someone is ill, tossing and turning, it can become undone.  Ja'far holds it on and re-knots it.  Similarly, it is considered unwise for a man to go into the kitchen, because while the woman is working there her \Swa{leso}, \E{upper garment}, may become undone.}  * ikawa kumsomeya\footnote{This is somewhat exaggerated in this situation.  The point is that her behaviour makes her look as if she is ill, and in such a case a common practice is to read to the sick person from the Qur'an, especially Chapter 36, \Eit{Ya Sin}.  The main message of this chapter is that human beings are created by God, and wholly dependent upon him.  Reading it comforts the sick person and their relatives, and is a sign of sympathy.  Reciting the Word of God has beneficial effects in general.  For instance, a rich man may pay a \Swa{mwalimu}, \E{Islamic scholar} to read the Qur'an over the man's wife every Friday, to keep her safe.  If someone is going on a long journey, wellwishers may pass verses from the Qur'an around them while saying \Swa{Ngwakuhifadhi}, \E{may God protect you}, and then give them the verses to protect them.} } \\ 
\E{She took and threw herself down,  she hurled herself to the ground,  so that her clothing came undone,   as if she was being read over.  } \\ 
\\[8mm] 

\textarabic{(١٥٢) \textcolor{mygreen}{أَلِپٗپَٹَ فَهَمُ  * كَمُحِمِدِ كَرِيْمُ  * كِشَ أَكَتَكَلَمُ  * مْنْڠُ أَكَمُؤٗمْبٖيَ }} \\* 
 \OLTcl{ akamuombeya mngu *  akatakalamu kisha *  karı̄mu kamuḥimidi *  fahamu alipopaţa} \\* 
\SB{150} (\textbf{152}) \OLTst{alipopata fahamu  * kamuhimidi karimu  * kisha akatakalamu  * mngu akamuombeya } \\ 
\E{When she regained her senses  she thanked the Generous One,  and then she spoke  and prayed to God.  } \\ 
\\[8mm] 

\textarabic{(١٥٣) \textcolor{mygreen}{يَا أَللّٰهُ مٗلَ وَنْڠُ  * نِنُصُرِيَ مْوَنَنْڠُ  * نَ وَٹٗٹٗ وَ وٖنْزَنْڠُ  * حِفَظِنِ نِٹِلِيَ }} \\* 
 \OLTcl{ niţiliya ḥifaẓini *  wenzangu wa waţoţo na *  mwanangu ninuṣuriya *  wangu mola alllähu yā} \\* 
\SB{151} (\textbf{153}) \OLTst{ya alllahu mola wangu  * ninusuriya mwanangu  * na watoto wa wenzangu\footnote{It would be selfish to pray only for yourself or your own children.  The proper thing is to pray for others too, e.g. Muslims, or unbelievers who will become Muslims.}  * hifadhini nitiliya\footnote{In other words, she is giving Ja'far her \Swa{radhi} -- see 83c.  In order to leave, Ja'far must have this.  Hence the verse: \Swa{mwate asumbuke / hana radhi ya mamake}, \E{let him remain troubled / he does not have the blessing of his mother}.  Likewise, an unsuccessful person may be referred to as someone \Swa{asiyekupata radhi ya babake}, \E{sho did not get his father's blessing}.  However, \Swa{watoto wa jeuri} \E{cheeky children}, will say things like \Swa{radhi yako kaiweke mbuyuni}, \E{stick your blessing in a baobab tree}. } } \\ 
\E{Oh God,  my Lord,  protect my child for me,  and the children of my friends,    place them for me in your care.  } \\ 
\\[8mm] 

\textarabic{(١٥٤) \textcolor{mygreen}{إٖنٖنْدَ هُنَ مَضَرَ  * نِرَضِ أَلْفُ مَرَ  * نَاوٖ أُوٖ نَ فِكِرَ  * مٗيٗ نِمٖكُؤُصِيَ }} \\* 
 \OLTcl{ nimekuuṣiya moyo *  fikira na uwe nāwe *  mara alfu niraḍi *  maḍara huna enenda} \\* 
\SB{152} (\textbf{154}) \OLTst{enenda huna madhara  * niradhi alfu mara  * nawe uwe na fikira\footnote{In other words, \Swa{siwe kama ng'ombe} -- don't act stupidly.}  * \dotuline{moya}\footnote{We understand \Swa{jambo}, i.e. this is the one important thing she asks him to do.} nimekuusiya } \\ 
\E{[To Ja'far she said:] Off you go -- no harm will come to you.   I bless you a thousand times.   And that you should be sensible    is the one [thing] I charge you to do.  } \\ 
\\[8mm] 

\textarabic{(١٥٥) \textcolor{mygreen}{مِمِ أُيَپٗنِؤُذِ  * سِتٗكُوَ نَ غَيْظِ  * إِوَپٗ وَتَكَ رَضِ  * نَ كْوَ عَلِيْ زٖنْڠٖيَ }} \\* 
 \OLTcl{ zengeya ʿalii kwa na *  raḍi wataka iwapo *  ghayẓi na sitokuwa *  uyaponiudhi mimi} \\* 
\SB{153} (\textbf{155}) \OLTst{mimi uyaponiudhi  * sitokuwa na ghaydhi\footnote{\Swa{ghaizi} = \Swa{hasira}, \E{crossness, annoyance}.  A mother is always soft-hearted towards her children, unlike a father.}  * iwapo wataka radhi  * na kwa 'alii zengeya\footnote{\Swa{-zengeya} = \Swa{-tafuta}.  She is telling him: \Swa{fanya bidii kupata radhi ya Ali}, \E{make an effort to secure Ali's blessing}.} } \\ 
\E{Even if you were to anger me  I would not hold it against you.   If you want a blessing [from him],   then go and visit Ali.    } \\ 
\\[8mm] 

\textarabic{(١٥٦) \textcolor{mygreen}{إِتُنْدٖ أُوٖ نْيَؤٗنِ  * كْوَ بَبَكٗ أُوٖ تِنِ  * نَ أَتَكَلٗبَئِنِ  * كْوَكٗ لِوٖ مَرْضِيَ }} \\* 
 \OLTcl{ marḍiya liwe kwako *  atakalobaini na *  tini uwe babako kwa *  nyaoni uwe itunde} \\* 
\SB{154} (\textbf{156}) \OLTst{itunde uwe nyaoni\footnote{lit. \q{take care that you are under his feet}.  That is, be humble, and also obedient.}  * kwa babako uwe tini  * na atakalobaini\footnote{\Swa{-baini}, \E{say}.}  * kwako liwe mardhiya\footnote{That is, do not refuse anything -- the opposite of \Swa{-legea}, \E{be remiss}.} } \\ 
\E{Take care that you be humble   and subservient to your father,    and [accept] whatever he says to you   without demur.  } \\ 
\\[8mm] 

\textarabic{(١٥٧) \textcolor{mygreen}{نَ مْٹُمٖ مُحَمَدِ  * هَنَ بُدِ كُكُزِدِ  * أَللّٰهَ اللّٰهَ جِتَهِدِ  * نْڠَاءَ أُپَٹٖ وَصِيَ }} \\* 
 \OLTcl{ waṣiya upaţe ngaa *  jitahidi lläha allläha *  kukuzidi budi hana *  muḥamadi mţume na} \\* 
\SB{155} (\textbf{157}) \OLTst{na mtume muhamadi  * hana budi kukuzidi\footnote{i.e. it goes without saying that \Swa{anakushinda}, \E{he is superior to you}.}  * alllaha\footnote{\Swa{hala} = \Swa{hara}, an exhortation to effort, as in \Swa{hara mbee!}, \E{forward!}.} llaha jitahidi  * ngaa\footnote{\Swa{ngaa}, \E{even without} is similar to \Swa{ingawa}, \E{although, even though}, but distinct from it.  Compare: \Swa{ngaa hungii ndani ukauliza}, \E{even without getting in you can ask, even if you don't get in you can ask} and \Swa{ingawa umengia ndani, lakini ...}, \E{even though you get in, yet ..., even if you've got in, still ...}.  In this line, the meaning is that even if Ja'far picks up little or no wisdom, he should still attempt to do it.} upate wasiya\footnote{\Swa{waṣia} is often translated as \q{last will}, but its wider meaning is \q{wisdom}, or \q{dos and donts}.} } \\ 
\E{And the Prophet Muhammad,   there is no doubt that he is better than you,   so mind you exert yourself   to gain even a little wisdom [from him].   } \\ 
\\[8mm] 

\textarabic{(١٥٨) \textcolor{mygreen}{نَاءٖ بِنْتِ حَبِيْبُ  * كٖتِ نَاءٖ كْوَ ثَوَابُ  * كْوَكٖ أُوٖ نَ أَدَبُ  * أُمْطِيْ نَ كُمْوَنْڠُكِيَ }} \\* 
 \OLTcl{ kumwangukiya na umṭii *  adabu na uwe kwake *  thawābu kwa nae keti *  ḥabı̄bu binti nae} \\* 
\SB{156} (\textbf{158}) \OLTst{nae binti habibu\footnote{i.e. Fatima.  The Prophet is also known as \Swa{habibu'llah}, \E{Beloved of God}.}  * keti nae kwa thawabu\footnote{i.e. \Swa{vizuri, kama mama wa kambo}, \E{nicely, as with a stepmother}.  Because Ja'far will be living in Ali's house, he must be a polite guest.}  * kwake uwe na adabu  * umtii na kumwangukiya\footnote{\Swa{-angukia}, \E{fall down before, prostrate oneself before}, = \Swa{-sujudia}, i.e. submit.  This would normally be humiliating (the only time you prostrate yourself should be before God), but Ja'far's mother is impressing on him the need for humility.} } \\ 
\E{And as for the daughter of the Beloved One,   stay with her politely;    be courteous towards her,    obey her and be humble towards her.   } \\ 
\\[8mm] 

\textarabic{(١٥٩) \textcolor{mygreen}{أَكَتٗكَ جَعْفَرِ  * نَ چَكُلَ كِتَيَرِ  * كَلَ نَ نْدُيٖ نَاصِرِ  * نَ مْوَلِمُ كَتٗكٖيَ }} \\* 
 \OLTcl{ katokeya mwalimu na *  nāṣiri nduye na kala *  kitayari chakula na *  jaʿfari akatoka} \\* 
\SB{157} (\textbf{159}) \OLTst{akatoka ja'fari  * na chakula\footnote{A meal with someone before they go on a journey is traditional, but  \Swa{watu wakenda mbali, chakula hukosa baraka}, \E{if people are going far away, food lacks savour, lit. blessing}.} kitayari  * kala na nduye nasiri  * na mwalimu katokeya } \\ 
\E{Ja'far arose [the next morning]  and a meal was ready.   He ate with his brother Nasir    and then his teacher arrived [while they were eating].   } \\ 
\\[8mm] 

\textarabic{(١٦٠) \textcolor{mygreen}{وَكَتَنْڠَنْيَ مِكٗنٗ  * وٗتٖ وَٹَٹُ مفَنٗ  * أَكِنٖنَ نَ مَنٖنٗ  * نْدُڠُيٖ أَكِمْوَمْبِيَ }} \\* 
 \OLTcl{ akimwambiya nduguye *  maneno na akinena *  mfano waţaţu wote *  mikono wakatanganya} \\* 
\SB{158} (\textbf{160}) \OLTst{wakatanganya mikono\footnote{They all eat from the same bowl, since this is a special day -- Ja'far is leaving.  When you do not know if you will meet again, sharing a meal brings a special feeling of closeness.}  * wote watatu mfano  * \dotuline{akanena} na maneno  * nduguye \dotuline{akamwambiya} } \\ 
\E{They all put their hands [in the communal bowl]  all three of them as equals.   Then [Ja'far] said these words,   speaking to his brother:  } \\ 
\\[8mm] 

\textarabic{(١٦١) \textcolor{mygreen}{أَكَمْوَمْبِيَ نَاصِرِ  * يٖؤٗ نْدُيَنْڠُ كْوَ خٖيْرِ  * تَكَپٗرُدِ سَفَرِ  * تَمَشَ تَكُلٖٹٖيَ }} \\* 
 \OLTcl{ takuleţeya tamasha *  safari takaporudi *  khēri kwa nduyangu yeo *  nāṣiri akamwambiya} \\* 
\SB{159} (\textbf{161}) \OLTst{akamwambiya nasiri  * yeo nduyangu kwa heri  * takaporudi safari  * tamasha\footnote{i.e. \Swa{zawadi}, \E{a present}.} takuleteya } \\ 
\E{He told Nasir:  Goodbye today, my brother --    when I come back from my journey  I will bring you something nice.  } \\ 
\\[8mm] 

\textarabic{(١٦٢) \textcolor{mygreen}{أَكَمْجِبُ أُپٖسِ  * نِئٖٹٖيَ نَ فَرَسِ  * نْيَمَ هُيٗ سِمُئِسِ  * نَتَكَ كُمْوَنْڠَلِيَ }} \\* 
 \OLTcl{ kumwangaliya nataka *  simuisi huyo nyama *  farasi na nieţeya *  upesi akamjibu} \\* 
\SB{160} (\textbf{162}) \OLTst{akamjibu upesi  * nieteya na farasi  * nyama huyo\footnote{Mvita \Swa{huyo} = Amu \Swa{hoyo}.} simuisi\footnote{\Swa{-isa}, \E{not know}, is only used in the negative.}  * nataka\footnote{Nasir considers the horse an exotic animal, which suggests he comes from an isolated village.  People may say: \Swa{yeyeni maskini ameona ngamia -- labda anakaa mji mdogo}, \E{that poor fellow there has just seen a camel [for the first time] -- he must live in a little village}.} kumwangaliya } \\ 
\E{[Nasir] answered him quickly:  Bring me a horse --   I don't know [what] that animal [looks like],   I would like to see one.  } \\ 
\\[8mm] 

\textarabic{(١٦٣) \textcolor{mygreen}{كِشَ هَپٗ كَتَمْكَ  * أَكَمْبَ مَمَ هُتٗكَ  * مَمَكٖ أَكَئِنُكَ  * صَدَكَ كَمْتٗلٖيَ }} \\* 
 \OLTcl{ kamtoleya ṣadaka *  akainuka mamake *  hutoka mama akamba *  katamka hapo kisha} \\* 
\SB{161} (\textbf{163}) \OLTst{kisha hapo katamka  * akamba mama hutoka\footnote{= \Swa{natoka}.}  * mamake akainuka  * sadaka\footnote{Contrast \Swa{sadaka} with \Swa{kafara}, \E{expiatory offering} -- the former is given before doing something, the latter after doing something.  The purpose of the \Swa{sadaka} is to protect Ja'far.  If a person is ill, you might put money under his pillow, or rice under his bed, and then give that away as alms, in the hope that he will get better.  Or to bring blessings to someone, you might circle them three times with the \Swa{sadaka}, and then give it away (compare the note to 151d).} kamtoleya } \\ 
\E{When he had finished, then [Ja'far] spoke,   and said: Mother, I am leaving.   His mother got up  and gave alms for him.  } \\ 
\\[8mm] 

\textarabic{(١٦٤) \textcolor{mygreen}{هَپٗ مَمَكٖ أَتٗكٖ  * إٖنٖنْدٖ أَمْفُوَتٖ  * أَكِمْٹٖمٖيَ مَٹٖ  * مْنْڠُ أَكِمُؤٗمْبٖيَ }} \\* 
 \OLTcl{ akimuombeya mngu *  maţe akimţemeya *  amfuwate enende *  atoke mamake hapo} \\* 
\SB{162} (\textbf{164}) \OLTst{hapo mamake atoke\footnote{Past tense.}  * enende\footnote{= \Swa{alienda}.} amfuwate  * akimtemeya mate\footnote{Making gentle spitting sounds (\Eit{pp-pp-pp}) at him, \Swa{-mtia mate}, signifies that she thinks he is \Swa{sharifu}, \E{noble}, and to be admired.}  * mngu akimuombeya } \\ 
\E{Then his mother went out [after him],   she went and followed him,  spitting at him,  praying to God for him.  } \\ 
\\[8mm] 

\textarabic{(١٦٥) \textcolor{mygreen}{هُيٗ نَاصِرِ مْوَلِمُ  * مْسٗمٖشٖ أَهِتِمُ  * أُمْفُنْدٖ نَ عِلِمُ  * عَادَ يَكٗ تَكْوٖٹٖيَ }} \\* 
 \OLTcl{ takweţeya yako ʿāda *  ʿilimu na umfunde *  ahitimu msomeshe *  mwalimu nāṣiri huyo} \\* 
\SB{163} (\textbf{165}) \OLTst{huyo nasiri mwalimu  * msomeshe\footnote{Ja'far's mother asks the \Swa{mwalimu} to \q{cause Nasir to read}, i.e. teach him how to read the Qur'an.  Being able to read the Qur'an, even without understanding the detailed meaning of the words, is considered a first step in learning.  The student will attend the \Swa{chuo}, \E{school}, for 3-4 years, and while he is there the \Swa{mfunzi} has wide latitude in terms of discipline -- the student may be chastised with a \Swa{kikoto}, \E{whip made of plaited grass}, made by the student himself, if he makes mistakes, and it is said that the only constraint on the \Swa{mfunzi} is that \Swa{asaze mifupa na mato}, \E{he should omit [damaging] the bones and the eyes}.  The books used will all have brown or tan covers, because white is considered harmful.  Taha Hussein's \E{The Stream of Days} includes a passage on his similar schooling in Egypt in the early 1900s.} ahitimu\footnote{\Swa{kumaliza Kurani}.}  * umfunde na\footnote{This knowledge would include detailed exegesis of the Qur'an, intricate knowledge of \Swa{fikhri}, \E{grammar}, awareness of religious ritual, etc.} 'ilimu  * 'ada\footnote{These fees will be paid in stages once certain portions of the Qur'an have been learned, and can be paid in kind (e.g. in food items such as \Swa{bisi}, \E{roasted corn}).} yako takweteya } \\ 
\E{[Then she said:] Teacher, Nasir here,   teach him to read [the Qur'an] so that he may complete it.  Teach him knowledge.   I will pay your fee.   } \\ 
\\[8mm] 

\textarabic{(١٦٦) \textcolor{mygreen}{أَكِتٗكَ جَعْفَرِ  * هَپٗ كَلِيَ نَاصِرِ  * مَمَكٖ أَكَفَسِرِ  * أُسِكُ أَتَرٖجٖيَ }} \\* 
 \OLTcl{ atarejeya usiku *  akafasiri mamake *  nāṣiri kaliya hapo *  jaʿfari akitoka} \\* 
\SB{164} (\textbf{166}) \OLTst{akitoka ja'fari  * hapo kaliya nasiri  * mamake akafasiri  * usiku atarejeya\footnote{She tries to comfort the child by saying things like \Swa{hendi mbali -- atakuja atakuletea peremendi}, \E{he is not going far -- he will come back and bring you sweets}.} } \\ 
\E{As Ja'far was setting off  then Nasir began to cry.   His mother said:  [ja'far] will be back by nightfall.  } \\ 
\\[8mm] 

\textarabic{(١٦٧) \textcolor{mygreen}{نَاصِرِ أَكَتَمْكَ  * نَمُيُوَ إٖنْدَ مَكَه  * كُتُنْڠَ هَنْڠَلِتٗكَ  * أَسِپٗئِيٗنَ نْدِيَ }} \\* 
 \OLTcl{ ndiya asipoiyona *  hangalitoka kutunga *  makah enda namuyuwa *  akatamka nāṣiri} \\* 
\SB{165} (\textbf{167}) \OLTst{nasiri akatamka  * namuyuwa enda makah\footnote{Nasir is not stupid, and sees through her words.}  * kutunga\footnote{\Swa{-tunga}, \E{graze}.} hangalitoka  * asipoiyona ndiya\footnote{\Swa{kama hakuona ndia}, as if he does not know the right road, i.e. he is going in a completely different direction to his normal route.} } \\ 
\E{Nasir spoke:  I know he is going to Mecca.   If he were going [to take the animals] to graze  he would not take that road.  } \\ 
\\[8mm] 

\textarabic{(١٦٨) \textcolor{mygreen}{أَوْ يَنَ سِكُوَكٗ  * أُكِمْوَمْبِيَ تَمْكٗ  * كَوَڠٖ وٖنْدَنِ وَكٗ  * يٗتٖ نَلِيَسِكِيَ }} \\* 
 \OLTcl{ naliyasikiya yote *  wako wendani kawage *  tamko ukimwambiya *  sikuwako yana aw} \\* 
\SB{166} (\textbf{168}) \OLTst{au yana sikuwako  * ukimwambiya tamko  * kawage wendani wako  * yote naliyasikiya } \\ 
\E{For was I not there yesterday   when you said him [those] words to him:  "Go and say goodbye to your friends."   I heard everything.  } \\ 
\\[8mm] 

\textarabic{(١٦٩) \textcolor{mygreen}{أَوْ وٖنْدَ مَتُنْڠَنِ  * نْڠُوٗ هُتُكُلِيَنِ  * سِكُ زٗتِ سِمُؤٗنِ  * هَتَ هَيٗ كُنَمْبِيَ }} \\* 
 \OLTcl{ kunambiya hayo hata *  simuoni zoti siku *  hutukuliyani nguwo *  matungani wenda aw} \\* 
\SB{167} (\textbf{169}) \OLTst{au wenda matungani  * nguwo hutukuliyani  * siku \dotuline{zote} simuoni  * hata hayo kunambiya } \\ 
\E{Or if he is going to the pastures,   what is he carrying clothes for?  I have never seen him [do that before].   So explain these [things] to me.   } \\ 
\\[8mm] 

\textarabic{(١٧٠) \textcolor{mygreen}{جَعْفَرِ كَبَئِنِ  * بَسِ وَلِلِيَنِ  * أَوْ ٹْوَلِأڠَنَنِ  * مَنٖنٗ نَلٗكْوَمْبِيَ }} \\* 
 \OLTcl{ nalokwambiya maneno *  ţwaliganani aw *  waliliyani basi *  kabaini jaʿfari} \\* 
\SB{168} (\textbf{170}) \OLTst{ja'fari kabaini\footnote{\Swa{-baini}, lit. \E{explain}.}  * basi waliliyani  * au twaliganani  * maneno nalokwambiya\footnote{Refers to 174-5.  \Swa{amkumbusha, umesahau ...}, \E{he reminds him, you have forgotten ...}.  He says something like: \q{You said you wanted a horse -- how can I get one if I don't go?}} } \\ 
\E{Ja'far spoke:  So why are you crying?  Did we not agree  on the things I said to you?  } \\ 
\\[8mm] 

\textarabic{(١٧١) \textcolor{mygreen}{نَ كَمَ هُكِرِضِكَ  * نَمْبِيَ نِسِيَتٗكَ  * هُنَ هَتَ كُؤُذِكَ  * سِكِتِكٗ كُنِٹِيَ }} \\* 
 \OLTcl{ kuniţiya sikitiko *  kuudhika hata huna *  nisiyatoka nambiya *  hukiriḍika kama na} \\* 
\SB{169} (\textbf{171}) \OLTst{na kama hukiridhika  * nambiya nisiyatoka  * huna hata kuudhika\footnote{\Swa{haina maana}, \E{there's no sense}.}  * sikitiko kunitiya } \\ 
\E{ And if you are not pleased,  tell me before I go.  You have no cause to be hurt   and make me feel sad.  } \\ 
\\[8mm] 

\textarabic{(١٧٢) \textcolor{mygreen}{أَكَجِبُ تَمْكٗ  * سِكُئِزَ هَيٗ يَكٗ  * سِكُ زٗتٖ نِكٗ  * هَمُنِؤٗنِ كُلِيَ }} \\* 
 \OLTcl{ kuliya hamunioni *  niko zote siku *  yako hayo sikuiza *  tamko akajibu} \\* 
\SB{170} (\textbf{172}) \OLTst{akajibu tamko  * sikuiza hayo yako  * siku zote niko  * hamunioni kuliya } \\ 
\E{[Nasir] answered with the words  I don't disagree with these [plans] of yours --   all the days of my life   you have never seen me cry.  } \\ 
\\[8mm] 

\textarabic{(١٧٣) \textcolor{mygreen}{سَسَ هَيَ نْدَ عَقِلِ  * نِمٖزٗيَتَأَمَلِ  * نَ كُوَ مَتُلِ تُلِ  * زٖءٗ زَكٗ زَ كُئِنُكِيَ }} \\* 
 \OLTcl{ kuinukiya za zako zeo *  tuli matuli kuwa na *  nimezoyataamali *  ʿaqili nda haya sasa} \\* 
\SB{171} (\textbf{173}) \OLTst{sasa haya nda 'aqili  * nimezoyataamali  * na kuwa matuli tuli  * zeo zako za kuinukiya\footnote{\Swa{wakati wako wa kutoka}.  Being sad is natural for Nasir -- his brother has stopped being a playmate and has now become a young man.  } } \\ 
\E{Now, these [things] are [a matter of] commonsense,    [the things] which I observed. I am sad [because]    it is time for you to go.    } \\ 
\\[8mm] 

\textarabic{(١٧٤) \textcolor{mygreen}{أَكَمْوَمْبِيَ نْدُيَكٖ  * هى إِنُكَ أُتٗكٖ  * جَعْفَرِ أَتَمْكٖ  * كْوَنْدَ مْنْڠُ نِؤٗمْبٖيَ }} \\* 
 \OLTcl{ niombeya mngu kwanda *  atamke jaʿfari *  utoke inuka hı̄ *  nduyake akamwambiya} \\* 
\SB{172} (\textbf{174}) \OLTst{akamwambiya nduyake\footnote{See 170d.}  * \dotuline{haya} inuka utoke  * ja'fari atamke  * kwanda mngu niombeya } \\ 
\E{He told his brother:  So, off you go.   Ja'far said:  First intercede to God for me.   } \\ 
\\[8mm] 

\textarabic{(١٧٥) \textcolor{mygreen}{نَاصِرِ أَكَبَئِنِ  * أَتَكُپٖكَ مَنَنِ  * سَلَامَ سَلِمِيْنِ  * كْوَ عَفِيَ نَ عَفُوَ }} \\* 
 \OLTcl{ ʿafuwa na ʿafiya kwa *  salimı̄ni salāma *  manani atakupeka *  akabaini nāṣiri} \\* 
\SB{173} (\textbf{175}) \OLTst{nasiri akabaini  * atakupeka\footnote{\Swa{-peka} = \Swa{-peleka}, \E{send}.  In other words, may God make it possible for you to go.} manani  * salama salimini  * kwa 'afiya na 'afuwa\footnote{The verb \Swa{-afu}, \E{preserve, deliver}, is the opposite of \Swa{-tesa}, \E{suffer, be afflicted by}, e.g. sickness, poverty, love.} } \\ 
\E{Nasir said:  May Providence keep you  safe and sound,  in health and free from affliction.    } \\ 
\\[8mm] 

\textarabic{(١٧٦) \textcolor{mygreen}{هَپٗ نْدِيَ كَيَنْدَمَ  * كِنٖنْدَ كُتٗسِمَمَ  * سَاءَ كُوِ إِكِكٗمَ  * نَاءٖ مَكَه أَمٖنْڠِيَ }} \\* 
 \OLTcl{ amengiya makah nae *  ikikoma kuwi saa *  kutosimama kinenda *  kayandama ndiya hapo} \\* 
\SB{174} (\textbf{176}) \OLTst{hapo ndiya kayandama\footnote{\Swa{fuata ndia}.}  * kinenda kutosimama  * saa kuwi ikikoma  * nae makah amengiya } \\ 
\E{Then [Ja'far] set out,   going on, not stopping,   and at four o'clock   he entered Mecca.   } \\ 
\\[8mm] 

\textarabic{(١٧٧) \textcolor{mygreen}{مَكَه أَلِپٗجِلِسِ  * كْوَ بَبَكٖ هَكُئِسِ  * كَمْبَ نْدِيَ سِتَكَسِ  * كْوَنْدَ تَئِكٖتِلِيَ }} \\* 
 \OLTcl{ taiketiliya kwanda *  sitakasi ndiya kamba *  hakuisi babake kwa *  alipojilisi makah} \\* 
\SB{175} (\textbf{177}) \OLTst{makah alipojilisi\footnote{\Swa{-jilisi} = \Swa{-keti}.}  * kwa babake hakuisi  * kamba ndiya \dotuline{sitakisi}\footnote{= \Swa{siioni}.}  * kwanda taiketiliya } \\ 
\E{When he arrived in Mecca  he did not know his father's home.   He said: I will not [try to] guess the road,   first I will sit myself down.  } \\ 
\\[8mm] 

\textarabic{(١٧٨) \textcolor{mygreen}{أَكَكٖتِ جَعْفَرِ  * أَكَمُؤٗنَ زُبٖيْرِ  * وٖنْدٖمٖنٖ نَ بَشِيْرِ  * مِكٗنٗ كَمْوِنُلِيَ }} \\* 
 \OLTcl{ kamwinuliya mikono *  bashı̄ri na wendemene *  zubēri akamuona *  jaʿfari akaketi} \\* 
\SB{176} (\textbf{178}) \OLTst{akaketi ja'fari  * akamuona zuberi  * wendemene na bashiri  * mikono kamwinuliya\footnote{lit. \q{raised his hands to him}.} } \\ 
\E{Ja'far sat down  and saw Zubayr --  he was walking along with the Bearer of Good News --   and Ja'far waved to him.  } \\ 
\\[8mm] 

\textarabic{(١٧٩) \textcolor{mygreen}{مْكٗنٗ كَؤُٹِزَمَ  * زُبٖيْرِ أَكَسِمَمَ  * أَكَمْوَمْبِيَ هَشِمَ  * مْڠٖنِ وٖٹُ هُنْڠِيَ }} \\* 
 \OLTcl{ hungiya weţu mgeni *  hashima akamwambiya *  akasimama zubēri *  kauţizama mkono} \\* 
\SB{177} (\textbf{179}) \OLTst{mkono kautizama\footnote{= \Swa{}akauona.}  * zuberi akasimama  * akamwambiya hashima  * mgeni wetu hungiya\footnote{\Swa{anaanza kufika sasa}, \E{he is arriving just now}.} } \\ 
\E{Zubayr saw the wave  and stopped.  He told the Hashimite:  Our guest has just arrived.   } \\ 
\\[8mm] 

\textarabic{(١٨٠) \textcolor{mygreen}{أَكَمُؤُزَ أَمِيْنِ  * مْڠٖنِ وٖٹُ نِ نَنِ  * وَسِكِيٖ هُبَئِنِ  * مْكٗنٗ نِپٖ نَبِيَ }} \\* 
 \OLTcl{ nabiya nipe mkono *  hubaini wasikiye *  nani ni weţu mgeni *  amı̄ni akamuuza} \\* 
\SB{178} (\textbf{180}) \OLTst{akamuuza amini  * mgeni wetu ni nani  * wasikiye hubaini  * mkono nipe nabiya\footnote{Ja'far, despite never having met the Prophet, recognises him immediately.} } \\ 
\E{The Trustworthy One asked:  Who is our guest?    And they heard [Ja'far] say:  Give me your hand, Prophet.   } \\ 
\\[8mm] 

\textarabic{(١٨١) \textcolor{mygreen}{زُبٖيْرِ كَتَعَجَبُ  * وَمُيُوَپِ حَبِيْبُ  * جَعْفَرِ كَمْجِبُ  * وَجْهِ وَكٖ نَبِيَ }} \\* 
 \OLTcl{ nabiya wake wajhi *  kamjibu jaʿfari *  ḥabı̄bu wamuyuwapi *  kataʿajabu zubēri} \\* 
\SB{179} (\textbf{181}) \OLTst{zuberi kata'ajabu  * wamuyuwapi habibu  * ja'fari kamjibu  * wajhi wake nabiya\footnote{i.e. \Swa{uso wake unamwonyesha}, \E{his countenance identifies him} -- the Prophet's features show a qualitative difference (\Swa{tofauti}) from everyone elses's.} } \\ 
\E{Zubayr was amazed:  How do you know the Prophet?  Ja'far answered him:  His face is that of the Prophet.   } \\ 
\\[8mm] 

\textarabic{(١٨٢) \textcolor{mygreen}{كِشَ هَپٗ كَبَئِنِ  * كْوَ بَبَنْڠُ نِپٖكٖنِ  * كَمْتُكُوَ أَمِيْنِ  * كٖنْدَ نَاءٖ كْوَ عَلِيَ }} \\* 
 \OLTcl{ ʿaliya kwa nae kenda *  amı̄ni kamtukuwa *  nipekeni babangu kwa *  kabaini hapo kisha} \\* 
\SB{180} (\textbf{182}) \OLTst{kisha hapo kabaini\footnote{\Swa{akasema}.}  * kwa babangu nipekeni  * kamtukuwa\footnote{\Swa{-enda naye}.} amini  * kenda nae kwa 'aliya } \\ 
\E{Then he said:   Could you show me to my father's [house]?   The Trustworthy One accompanied him  and took him to Ali's [house].    } \\ 
\\[8mm] 

\textarabic{(١٨٣) \textcolor{mygreen}{أَلِپٗكْوٖنْدَ سَيِّدِ  * كْوَءٗ أَكَپِجَ هٗدِ  * فَتُمَ أَكَرَدِدِ  * أَكَمْبَ هَكٗ نَبِيَ }} \\* 
 \OLTcl{ nabiya hako akamba *  akaradidi fatuma *  hodi akapija kwao *  sayyidi alipokwenda} \\* 
\SB{181} (\textbf{183}) \OLTst{alipokwenda sayyidi  * kwao akapija hodi  * fatuma akaradidi\footnote{The original Arabic words means \q{repeat}, but in Swahili it is another word for \Swa{-sema}, \E{speak}.}  * akamba hako nabiya\footnote{\Swa{alifikiri mtu anamtaka Mtume} -- Fatima thinks that someone has come to the house looking for the Prophet.} } \\ 
\E{When the Lord got there  he called: Hello!   Fatima answered  and said: The Prophet is not here.   } \\ 
\\[8mm] 

\textarabic{(١٨٤) \textcolor{mygreen}{أَكَمُؤُزَ حُسَيْنِ  * أَمْكُوَءٗ نِ نْيَانِ  * كِجَنَ أَكَبَئِنِ  * نِ جَدِ يَنْڠُ سِكِيَ }} \\* 
 \OLTcl{ sikiya yangu jadi ni *  akabaini kijana *  nyāni ni amkuwao *  ḥusayni akamuuza} \\* 
\SB{182} (\textbf{184}) \OLTst{akamuuza\footnote{\Swa{-uza} = \Swa{-uliza}.} husayni  * amkuwao\footnote{= \Swa{aitaye}.} ni nyani  * kijana akabaini  * ni jadi\footnote{= \Swa{babu}.} yangu sikiya } \\ 
\E{She asked Husayn:  Who is calling?   And the boy answered:  Listen -- it is my grandfather.    } \\ 
\\[8mm] 

\textarabic{(١٨٥) \textcolor{mygreen}{نَ جَدِ يَنْڠُ رَسُوْلِ  * يٖيٖ نَ وَٹُ وَوِلِ  * كُسِكِيَكْوٖ قَوْلِ  * كَتٗكَ كَمْوَنْڠَلِيَ }} \\* 
 \OLTcl{ kamwangaliya katoka *  qawli kusikiyakwe *  wawili waţu na yeye *  rasūli yangu jadi na} \\* 
\SB{183} (\textbf{185}) \OLTst{\dotuline{ni} jadi yangu rasuli  * yeye na watu wawili  * kusikiyakwe qauli  * katoka kamwangaliya } \\ 
\E{It is my grandfather the Prophet,    he and two people.    When she heard these words,  [Fatima] went out to see him.  } \\ 
\\[8mm] 

\textarabic{(١٨٦) \textcolor{mygreen}{أَكِتٗكَ جَعْفَرِ  * كَمُؤُلِزَ خَبَرِ  * أَكِكٗمَ كَفَسِرِ  * فَتُمَ كَمْپٗكٖيَ }} \\* 
 \OLTcl{ kampokeya fatuma *  kafasiri akikoma *  khabari kamuuliza *  jaʿfari akitoka} \\* 
\SB{184} (\textbf{186}) \OLTst{akitoka ja'fari  * kamuuliza habari\footnote{i.e. he asked \Swa{hujambo?}}  * akikoma kafasiri  * fatuma kampokeya\footnote{= \Swa{-itikia}.} } \\ 
\E{When she came out, Ja'far  asked her how she was.  When he had finished speaking  Fatima answered him.  } \\ 
\\[8mm] 

\textarabic{(١٨٧) \textcolor{mygreen}{فَتُمَ كَتَكَلَمُ  * أَكَمُؤُزَ هَشِمُ  * بَبَ سِيَمْفَهَمُ  * كِجَنَ هُيُ نَبِيَ }} \\* 
 \OLTcl{ nabiya huyu kijana *  siyamfahamu baba *  hashimu akamuuza *  katakalamu fatuma} \\* 
\SB{185} (\textbf{187}) \OLTst{fatuma katakalamu  * akamuuza hashimu  * baba siyamfahamu  * kijana huyu nabiya } \\ 
\E{Fatima spoke  And asked the Hashimite:  Father, I still don't recognise  this boy, Prophet.   } \\ 
\\[8mm] 

\textarabic{(١٨٨) \textcolor{mygreen}{سِكُ زٗتٖ سِمُؤٗنِ  * إِنَ لَكٖ نٔدِيٖ نَنِ  * مْٹُمِ أَكَبَئِنِ  * فَتُمَ أَكَمْوَمْبِيَ }} \\* 
 \OLTcl{ akamwambiya fatuma *  akabaini mţumi *  nani ndiye lake ina *  simuoni zote siku} \\* 
\SB{186} (\textbf{188}) \OLTst{siku zote simuoni  * ina lake ndiye nani  * mtumi akabaini  * fatuma akamwambiya } \\ 
\E{I have never seen him before,   What is his name?    The Prophet spoke  and addressed Fatima.  } \\ 
\\[8mm] 

\textarabic{(١٨٩) \textcolor{mygreen}{أَكَمْوَمْبِيَ بَشِيْرِ  * هُيُ نْدِيٖ جَعْفَرِ  * وَ عَلِيْ حَيْدَرِ  * هِزٗ صُوْرَ هُكْوَمْبِيَ }} \\* 
 \OLTcl{ hukwambiya ṣūra hizo *  ḥaydari ʿalii wa *  jaʿfari ndiye huyu *  bashı̄ri akamwambiya} \\* 
\SB{187} (\textbf{189}) \OLTst{akamwambiya bashiri  * huyu ndiye ja'fari  * wa 'alii haydari  * hizo sura hukwambiya\footnote{In the Mombasa expression, \Swa{umejizaa mwenyewe}, \E{he's the spitting image of you}.} } \\ 
\E{The Bringer of Good Tidings told her:  This is Ja'far   [son] of Ali the Lion-like --   his features would tell you that.   } \\ 
\\[8mm] 

\textarabic{(١٩٠) \textcolor{mygreen}{هَپٗ سَيِدِ أَمِيْنِ  * أَكَمْٹُمَ حُسَيْنِ  * إٖنٖنْدَ مْسِكِٹِنِ  * بَبَكٗ نَمْكُلِيَ }} \\* 
 \OLTcl{ namkuliya babako *  msikiţini enenda *  ḥusayni akamţuma *  amı̄ni sayidi hapo} \\* 
\SB{188} (\textbf{190}) \OLTst{hapo sayidi amini  * akamtuma husayni  * enenda msikitini  * babako namkuliya\footnote{= \Swa{nimtia, namwita}.} } \\ 
\E{Then the Lord, the Trustworthy One,   sent Husayn:  Go to the mosque,  and call your father for me.  } \\ 
\\[8mm] 

\textarabic{(١٩١) \textcolor{mygreen}{هَپٗ كَتٗكَ حُسَيْنِ  * كَفِكَ مْسِكِٹِنِ  * بَبَ كُئِيٖ مْڠٖنِ  * ٹُمِوٖ كُكْوَنْدَمِيَ }} \\* 
 \OLTcl{ kukwandamiya ţumiwe *  mgeni kuiye baba *  msikiţini kafika *  ḥusayni katoka hapo} \\* 
\SB{189} (\textbf{191}) \OLTst{hapo katoka husayni  * kafika msikitini  * baba kuiye mgeni  * tumiwe kukwandamiya\footnote{= \Swa{nimetumwa kukufuatia}.} } \\ 
\E{So Husayn went off   and arrived at the mosque.  Father, a visitor has come --    I have been sent to fetch you.  } \\ 
\\[8mm] 

\textarabic{(١٩٢) \textcolor{mygreen}{كُنَ كِجَنَ مْزُرِ  * چٖنْدٖمٖنٖ نَ بَشِيْرِ  * نَ إِنَ نِ جَعْفَرِ  * نِ هِلٗ نِمٖكْوَمْبِيَ }} \\* 
 \OLTcl{ nimekwambiya hilo ni *  jaʿfari ni ina na *  bashı̄ri na chendemene *  mzuri kijana kuna} \\* 
\SB{190} (\textbf{192}) \OLTst{kuna kijana mzuri  * \dotuline{endemene}\footnote{\Swa{kijana} (Class 7) is not a diminutive here.} na bashiri  * na ina ni ja'fari  * ni hilo nimekwambiya\footnote{i.e. I have told you the real cause.  He tells the whole story in case Ali gets worried that there has been a crisis at home.  Compare 13b.} } \\ 
\E{There is a handsome boy there.   He came with the Bringer of Good Tidings,   and his name is Ja'far --     I have told you everything now.   } \\ 
\\[8mm] 

\textarabic{(١٩٣) \textcolor{mygreen}{چَنْبِوَ هِيٗ كَلِمَ  * پَپٗ نْدِيَ كَيَنْدَمَ  * مْلَنْڠٗنِ أَكِكٗمَ  * سَلَامُ كَوَپِسِيَ }} \\* 
 \OLTcl{ kawapisiya salāmu *  akikoma mlangoni *  kayandama ndiya papo *  kalima hiyo cham̱biwa} \\* 
\SB{191} (\textbf{193}) \OLTst{chambiwa hiyo kalima  * papo ndiya kayandama  * mlangoni akikoma  * salamu kawapisiya\footnote{\Swa{-pisiya} = \Swa{-pitisha}.  In other words, he said \Swa{salaam alekum}.} } \\ 
\E{When these words had been said   [Ali] set off immediately.   When he reached the door [of his house]  he greeted [those inside].  } \\ 
\\[8mm] 

\textarabic{(١٩٤) \textcolor{mygreen}{سَلَامُ أَكِفَسِرِ  * هَپٗ عَلِيْ حَيْدَرِ  * كَئِنُكَ جَعْفَرِ  * مكٗنٗ كَمْپٗكٖيَ }} \\* 
 \OLTcl{ kampokeya mkono *  jaʿfari kainuka *  ḥaydari ʿalii hapo *  akifasiri salāmu} \\* 
\SB{192} (\textbf{194}) \OLTst{salamu akifasiri  * hapo 'alii haydari  * kainuka ja'fari  * mkono kampokeya } \\ 
\E{While he was greeting [them],  Ali the Lion-like,   Ja'far got up  and took his hand.  } \\ 
\\[8mm] 

\textarabic{(١٩٥) \textcolor{mygreen}{عَلِيْ كٖٹَ تَمْكٗ  * نِ سَلَامَ أُتٗكَكٗ  * جُمْلَ وٖنْدَنِ وَكٗ  * حَالِ زَءٗ نَمْبِيَ }} \\* 
 \OLTcl{ nambiya zao ḥāli *  wako wendani jumla *  utokako salāma ni *  tamko keţa ʿalii} \\* 
\SB{193} (\textbf{195}) \OLTst{'alii keta tamko  * ni salama utokako  * jumla wendani\footnote{\Swa{wendani} can also cover friends as well as relatives.} wako  * hali zao nambiya } \\ 
\E{Ali spoke:   Is everything well where you came from?   All your relatives,   tell me how they are.   } \\ 
\\[8mm] 

\textarabic{(١٩٦) \textcolor{mygreen}{أَكَمْجِبُ كَلَمُ  * نِتٗكَكٗ نِ سَلَامَ  * سِيُوِ يَ هٗكٗ نْيُمَ  * سِپَٹِ لَكُكْوَمْبِيَ }} \\* 
 \OLTcl{ lakukwambiya sipaţi *  nyuma hoko ya siyuwi *  salāma ni nitokako *  kalamu akamjibu} \\* 
\SB{194} (\textbf{196}) \OLTst{akamjibu kalamu  * nitokako ni salama  * siyuwi ya hoko nyuma  * sipati lakukwambiya } \\ 
\E{And [Ja'far] answered him with the words:  Everything is well where I come from,   [though] I don't know about after [I left].    I have nothing to tell you [since my departure].  } \\ 
\\[8mm] 

\textarabic{(١٩٧) \textcolor{mygreen}{تٖنَ بَبَ مْوَلِمُ  * نَ مَمَ وَكُسَلِمُ  * وَعَلَيْكَ السَّلَامَ  * عَلِيْ كَپٗكٖيَ }} \\* 
 \OLTcl{ kapokeya ʿalii *  ās-salāma waʿalayka *  wakusalimu mama na *  mwalimu baba tena} \\* 
\SB{195} (\textbf{197}) \OLTst{tena baba mwalimu  * na mama wakusalimu  * wa-alaika as-salama  * 'alii kapokeya } \\ 
\E{Also, father, [my] teacher   and [my] mother greet you.   Peace be with you,   Ali answered.  } \\ 
\\[8mm] 

\textarabic{(١٩٨) \textcolor{mygreen}{عَلِيْ أَكَبَئِنِ  * نِمٖكُنْڠٗجَ نْدِيَنِ  * وَلِتٗكَ زٖءٗ ڠَنِ  * مْبٗنَ أُمٖلِمَٹِيَ }} \\* 
 \OLTcl{ umelimaţiya mbona *  gani zeo walitoka *  ndiyani nimekungoja *  akabaini ʿalii} \\* 
\SB{196} (\textbf{198}) \OLTst{'alii akabaini  * nimekungoja ndiyani  * walitoka zeo gani  * mbona umelimatiya\footnote{Amu \Swa{-limatia} = Zanzibar, Mvita \Swa{-chelewa}, Mvita \Swa{-kawia}.} } \\ 
\E{Ali spoke:  I waited for you on the road --   what time did you set out?   Why are you late?  } \\ 
\\[8mm] 

\textarabic{(١٩٩) \textcolor{mygreen}{أَكَمْجِبُ قَوْلِ  * أَصُبُحِ نَلِصَلِ  * لَكِنِ بَبَ نِ مْبَلِ  * إِنَ أُرٖفُ وَ نْدِيَ }} \\* 
 \OLTcl{ ndiya wa urefu ina *  mbali ni baba lakini *  naliṣali aṣubuḥi *  qawli akamjibu} \\* 
\SB{197} (\textbf{199}) \OLTst{akamjibu qauli  * asubuhi nalisali\footnote{i.e. he had got up in time for prayers at 5.00am.}  * lakini baba ni mbali  * ina urefu wa ndiya } \\ 
\E{[ja'far] answered him with the words:  I prayed in the morning,  but, father, it is far --    the road is a long one.    } \\ 
\\[8mm] 

\textarabic{(٢٠٠) \textcolor{mygreen}{نَمِ كِپَٹَ پَنْڠٗنِ  * پَنَ مْٹٖنْدٖ نْدِيَنِ  * يَلِنِتٗكَ مٗيٗنِ  * يَلٖ وَلٗنَمْبِيَ }} \\* 
 \OLTcl{ walonambiya yale *  moyoni yalinitoka *  ndiyani mţende pana *  pangoni kipaţa nami} \\* 
\SB{198} (\textbf{200}) \OLTst{nami kipata\footnote{Contrast \Swa{-pata}, \E{arrive at somewhere en route to a destination}, and \Swa{-fika}, \E{arrive at the destination}.} pangoni  * pana mtende ndiyani\footnote{These points were presumably part of the directions that Ali gave Ja'far in 74.}  * yalinitoka moyoni\footnote{i.e. \Swa{nilisahau}, \E{I forgot}.}  * yale walonambiya } \\ 
\E{And when I reached the cave   there was the date-tree by the road,   but they left my head,  the [directions] you had told me.  } \\ 
\\[8mm] 

\textarabic{(٢٠١) \textcolor{mygreen}{كَئِوَتَ يَ كُڤُلِ  * كَأَنْدَمَ إِلٗ مْبَلِ  * هَتَ كِتَأَمَلِ  * سَاءَ إِمٖنِپِٹِيَ }} \\* 
 \OLTcl{ imenipiţiya saa *  kitaamali hata *  mbali ilo kaandama *  kuvuli ya kaiwata} \\* 
\SB{199} (\textbf{201}) \OLTst{kaiwata ya kuvuli  * kaandama ilo mbali  * hata kitaamali  * saa\footnote{\Swa{saa}, \E{hour}, is used here to signify the passage of time.} imenipitiya\footnote{We are to understand, \q{and I still hadn't found the place I was trying to go to}.} } \\ 
\E{I left behind the [road] to the right   and walked on for a long way   until I realised  a long time had passed.  } \\ 
\\[8mm] 

\textarabic{(٢٠٢) \textcolor{mygreen}{كِشَ أُوِنْڠَ كَئٖٹَ  * إِيُ لَ بَرَ كَپِٹَ  * إِلِ نْدِيَ كُئِوَتَ  * نْيُمَ نِسِپٗرٖجٖيَ }} \\* 
 \OLTcl{ nisiporejeya nyuma *  kuiwata ndiya ili *  kapiţa bara la iyu *  kaeţa uwinga kisha} \\* 
\SB{200} (\textbf{202}) \OLTst{kisha uwinga kaeta\footnote{i.e. \Swa{-fanya jinga la kipumbavu}, \E{do something blockheaded}, lit. \q{do the stupidity of an idiot}.}  * iyu la bara kapita\footnote{Where, of course, there are no paths.}  * ili ndiya kuiwata  * nyuma nisiporejeya\footnote{i.e. instead of trying to retrace his footsteps.} } \\ 
\E{Then I did something stupid --    I walked out into the scrubland    and left the road behind,   instead of going back.  } \\ 
\\[8mm] 

\textarabic{(٢٠٣) \textcolor{mygreen}{سُرَ نٖنْدَءٗ بَرَنِ  * إِلٖ نْدِيَ سِئِيٗنِ  * هُؤٗنَ نِكٗ بَرَنِ  * زٗتٖ زِمٖنِپٗتٖيَ }} \\* 
 \OLTcl{ zimenipoteya zote *  barani niko huona *  siiyoni ndiya ile *  barani nendao sura} \\* 
\SB{201} (\textbf{203}) \OLTst{sura\footnote{= \Swa{namna yeyote}, \E{whatever kind}.} nendao barani  * ile ndiya siiyoni  * huona niko barani  * zote\footnote{We understand \Swa{ndia}, \E{paths}.} zimenipoteya } \\ 
\E{Wherever I went in the scrubland   I couldn't find the road --   I realised I was [lost] in the scrubland,    and I had lost track of all [the roads].  } \\ 
\\[8mm] 

\textarabic{(٢٠٤) \textcolor{mygreen}{كِپِجَ فِكِرَ زَنْڠُ  * كَلَنْدَمَ ڠُوْ لَنْڠُ  * نَرُدِيَ پَلٖ پَنْڠُ  * كِشَ نْيُمَ كَرٖجٖيَ }} \\* 
 \OLTcl{ karejeya nyuma kisha *  pangu pale narudiya *  langu guu kalandama *  zangu fikira kipija} \\* 
\SB{202} (\textbf{204}) \OLTst{kipija fikira zangu  * kalandama guu langu  * narudiya pale pangu  * kisha nyuma karejeya } \\ 
\E{I cudgelled my brains   and then retraced my footsteps   and returned to my [correct] place   and finally I got back.   } \\ 
\\[8mm] 

\textarabic{(٢٠٥) \textcolor{mygreen}{كِشَ كَرُدِيَ نْيُمَ  * هَپٗ نْدِيَ كَيَنْدَمَ  * پٖنْيٖ مْٹٖنْدٖ كَكٗمَ  * صَالَ إِمٖنِسِمَمِيَ }} \\* 
 \OLTcl{ imenisimamiya ṣāla *  kakoma mţende penye *  kayandama ndiya hapo *  nyuma karudiya kisha} \\* 
\SB{203} (\textbf{205}) \OLTst{kisha karudiya nyuma  * hapo ndiya kayandama  * penye mtende kakoma  * sala imenisimamiya } \\ 
\E{At last I turned back   and then I followed the road.   At the place with the date-tree I stopped --   it was time to pray.  } \\ 
\\[8mm] 

\textarabic{(٢٠٦) \textcolor{mygreen}{أَوَلِ يَ أَظُهُرِ  * نْدِپٗ نِْيَ كَعَبِرِ  * حُجَ يَ كُجَ أَخِيْرِ  * مَعَانَ نِمٖكْوَمْبِيَ }} \\* 
 \OLTcl{ nimekwambiya maʿāna *  akhı̄ri kuja ya ḥuja *  kaʿabiri niya ndipo *  aẓuhuri ya awali} \\* 
\SB{204} (\textbf{206}) \OLTst{awali ya adhuhuri\footnote{Midday, when the sun is approaching its zenith, or just afterwards.}  * ndipo niya ka'abiri\footnote{\Swa{-abiri} < \AS{عبر}, \E{traverse, cross} was used in older Swahili to mean \E{travel from continent to continent in a ship}, but now it refers to travel in general.}  * huja\footnote{\Swa{huja}, \E{reason, argument, proof}.} ya kuja ahiri  * ma'ana nimekwambiya } \\ 
\E{Just after noon   was when I set out [again] on the road.   Regarding coming late,    I have told you the reason.  } \\ 
\\[8mm] 

\textarabic{(٢٠٧) \textcolor{mygreen}{كِمَلِزَ كُپُلِكَ  * عَلِيْ أَكَتَمْكَ  * مْوَنَنْڠُ أُمٖسُمْبُكَ  * هَپٗ كَنٖنَ نَبِيَ }} \\* 
 \OLTcl{ nabiya kanena hapo *  umesumbuka mwanangu *  akatamka ʿalii *  kupulika kimaliza} \\* 
\SB{205} (\textbf{207}) \OLTst{kimaliza kupulika\footnote{\Swa{-pulika} = \Swa{-sikiza}, \E{listen carefully}.}  * 'alii akatamka  * mwanangu umesumbuka\footnote{\Swa{umepata taabu}.}  * hapo kanena nabiya } \\ 
\E{When he had finished listening  Ali spoke:  My child, you have been through a lot.  Then the Prophet spoke.   } \\ 
\\[8mm] 

\textarabic{(٢٠٨) \textcolor{mygreen}{هَپٗ كَنٖنَ هَشِمَ  * سِ هَبَ كُيَ سَلَام  * نْدِيَ مٖزٗإِيَنْدَمَ  * خَطَرِ هُمْزٖنْڠٖيَ }} \\* 
 \OLTcl{ humzengeya khaṭari *  mezoiyandama ndiya *  salām kuya haba si *  hashima kanena hapo} \\* 
\SB{206} (\textbf{208}) \OLTst{hapo\footnote{Note the use of \Swa{hapo} to refer to time instead of place: \q{at this point}.} kanena hashima  * si haba kuya \dotuline{salamu}\footnote{\Swa{ingawa umetaabika}, \E{even though you were in distress}.}  * ndiya mezoiyandama  * hatari humzengeya\footnote{He could have been attacked by robbers, lions, etc.} } \\ 
\E{Then the Hashimite spoke:   It is no small thing to arrive safely --    [on] the road he came along  danger stalked him.  } \\ 
\\[8mm] 

\textarabic{(٢٠٩) \textcolor{mygreen}{أَمْكِنْڠَ وَدُوْدِ  * أَسِؤٗوْنٖ مَيَهُوْدِ  * كْوَنِ وَنْڠَلِمْزِدِ  * وَٹُ وَنْڠِ سِ مْمٗيَ }} \\* 
 \OLTcl{ mmoya si wangi waţu *  wangalimzidi kwani *  mayahūdi asiōne *  wadūdi amkinga} \\* 
\SB{207} (\textbf{209}) \OLTst{\dotuline{amemkinga} wadudi  * asione mayahudi\footnote{Although the literal meaning is \q{Jews}, it is important to note that this word now has a much wider meaning of unbelievers in general, \Swa{makafiri}.}  * kwani wangalimzidi\footnote{i.e. \q{they would have been too much for him} -- \Swa{-zidi} here = \Swa{shinda}.}  * watu wangi si mmoya\footnote{\Swa{mtu mmoya si sawasawa na watu wengi}, \E{one person is no match for many}.} } \\ 
\E{The Loving One protected him  so that he met no unbelievers,  for they would have overwhelmed him --   one against many.    } \\ 
\\[8mm] 

\textarabic{(٢١٠) \textcolor{mygreen}{فَتُمَ أُكٗ كِٹِنِ  * أَكَمْوٖپُكَ أَمِيْنِ  * كْوَ مْكٗنٗ كَبَئِنِ  * نَ نْدَنِ كَمُأَمْكُوَ }} \\* 
 \OLTcl{ kamuamkuwa ndani na *  kabaini mkono kwa *  amı̄ni akamwepuka *  kiţini uko fatuma} \\* 
\SB{208} (\textbf{210}) \OLTst{fatuma uko kitini  * akamwepuka amini  * kwa mkono kabaini\footnote{\Swa{kumwambia siri}, \E{to tell him a secret}.}  * na ndani kamuamkuwa\footnote{\Swa{kumwita ndani}, \E{to call him into the private quarters}.} } \\ 
\E{Fatima had been sitting down.   She moved back from the Trustworthy One  and made a sign with her hand   and beckoned [Ali] into the inner [room].   } \\ 
\\[8mm] 

\textarabic{(٢١١) \textcolor{mygreen}{مكٗنٗ كِؤُٹِزَمَ  * عَلِيْ أَكَفَهَمَ  * أَكَمْوَمْبِيَ هَشِمَ  * نٖنْدَ نْدَنِ مَرَ مٗيَ }} \\* 
 \OLTcl{ moya mara ndani nenda *  hashima akamwambiya *  akafahama ʿalii *  kiuţizama mkono} \\* 
\SB{209} (\textbf{211}) \OLTst{mkono kiutizama  * 'alii akafahama  * akamwambiya hashima  * nenda ndani mara moya } \\ 
\E{When he saw her hand [sign]  Ali understood [what it meant].  He told the Hashimite:  I am just now going to the inner [room].    } \\ 
\\[8mm] 

\textarabic{(٢١٢) \textcolor{mygreen}{كْوَ نْدَنِ أَلِپٗفِكَ  * فَتُمَ أَكَتَمْكَ  * چَكُلَ نِمٖكِپِكَ  * هَيَتَسَ كْوَنْدِكِوَ }} \\* 
 \OLTcl{ kwandikiwa hayatasa *  nimekipika chakula *  akatamka fatuma *  alipofika ndani kwa} \\* 
\SB{210} (\textbf{212}) \OLTst{kwa ndani alipofika  * fatuma akatamka  * chakula nimekipika  * hayatasa kwandikiwa } \\ 
\E{When he entered the inner [room]   Fatima spoke:  I have cooked some food  is it not time for it to be served?  } \\ 
\\[8mm] 

\textarabic{(٢١٣) \textcolor{mygreen}{كْوَنْدَ سِكُمَكِنِكَ  * خَبَرِ أَكَتَمْكَ  * مٗيٗ هُمْپَپَٹِكَ  * كْوَ أُثَقِلِ وَ نْدِيَ }} \\* 
 \OLTcl{ ndiya wa uthaqili kwa *  humpapaţika moyo *  akatamka khabari *  sikumakinika kwanda} \\* 
\SB{211} (\textbf{213}) \OLTst{kwanda \dotuline{ni kumakinika}  * habari akatamka\footnote{Ali is trying to put off as long as possible the inevitable point at which Fatima will hear that Ja'far is his son.}  * moyo humpapatika  * kwa uthaqili wa ndiya\footnote{i.e. \Swa{taabu ya ndiani}.} } \\ 
\E{[Ali said:] First he must relax  and tell his news --   his heart is fluttering  because of the hardships of the journey.    } \\ 
\\[8mm] 

\textarabic{(٢١٤) \textcolor{mygreen}{نَ زٖءٗ أَلِزٗتٗكَ  * أُمٖزِيُوَ هَكِكَ  * نَ سِسِ هُمْپُلِكَ  * مَنٖنٗيٖ هُٹْوَمْبِيَ }} \\* 
 \OLTcl{ huţwambiya manenoye *  humpulika sisi na *  hakika umeziyuwa *  alizotoka zeo na} \\* 
\SB{212} (\textbf{214}) \OLTst{na zeo\footnote{Amu \Swa{zeo} = Mvita \Swa{wakati}, Mu \Swa{njeo}.} alizotoka  * umeziyuwa hakika  * na sisi humpulika  * manenoye hutwambiya } \\ 
\E{And the time he took [to get here],   you know it well,  and we are listening to him   as he tells us his story.  } \\ 
\\[8mm] 

\textarabic{(٢١٥) \textcolor{mygreen}{عَلِيْ أَكَبَئِنِ  * هُمْصُبِرِ حُسَيْنِ  * نِمٖمْٹُمَ حَسَنِ  * إٖنْدٖ كُمُؤَمْكُوَ }} \\* 
 \OLTcl{ kumuamkuwa ende *  ḥasani nimemţuma *  ḥusayni humṣubiri *  akabaini ʿalii} \\* 
\SB{213} (\textbf{215}) \OLTst{'alii akabaini  * humsubiri\footnote{\Swa{-subiri} = \Swa{-ngoja}.} husayni  * nimemtuma hasani  * ende kumuamkuwa\footnote{This is another attempt to put off the moment of truth.} } \\ 
\E{Ali said:  we are [still] waiting for Husayn --  I have sent Hasan  to go and fetch him.  } \\ 
\\[8mm] 

\textarabic{(٢١٦) \textcolor{mygreen}{عَلِيْ كُتٗكَ نْدَنِ  * أَلِپٗكٖتِ كِٹِنِ  * أَمْسِكِيٖ حَسَنِ  * سَلَامُ هُوَپِسِيَ }} \\* 
 \OLTcl{ huwapisiya salāmu *  ḥasani amsikiye *  kiţini alipoketi *  ndani kutoka ʿalii} \\* 
\SB{214} (\textbf{216}) \OLTst{'alii kutoka ndani  * alipoketi kitini  * amsikiye hasani  * salamu huwapisiya\footnote{Hasan has obviously not left yet (assuming Ali has even told him to fetch Husayn), because he is still greeting the visitors.} } \\ 
\E{Ali came back from the inner [room]   and sat down on a chair.  He heard Hasan  greeting them.  } \\ 
\\[8mm] 

\textarabic{(٢١٧) \textcolor{mygreen}{حَسَنِ أَكَنُظُمُ  * كُوَپِسِزَ سَلَامُ  * أَمْرُدِشٖ كَلِمُ  * جَعْفَرِ كَمْوَمْبِيَ }} \\* 
 \OLTcl{ kamwambiya jaʿfari *  kalimu amrudishe *  salāmu kuwapisiza *  akanuẓumu ḥasani} \\* 
\SB{215} (\textbf{217}) \OLTst{hasani akanudhumu  * kuwapisiza salamu  * \dotuline{wamrudishe} kalimu  * ja'fari kamwambiya } \\ 
\E{Hasan spoke  and greeted them  so that they could return the greeting.  He spoke to Ja'far.  } \\ 
\\[8mm] 

\textarabic{(٢١٨) \textcolor{mygreen}{كْوَ أُنْدَنِ كْوِمَ  * كَمُؤُلِزَ سَلَامَ  * وَلِؤٗپٗ وَكَسِمَ  * وٗتٖ وَكَمْپٗكٖيَ }} \\* 
 \OLTcl{ wakampokeya wote *  wakasima waliopo *  salāma kamuuliza *  kwima undani kwa} \\* 
\SB{216} (\textbf{218}) \OLTst{kwa undani\footnote{= \Swa{kwa taratibu}. (?)} kwima  * kamuuliza salama  * waliopo \dotuline{wakasema}  * wote wakampokeya\footnote{i.e. answered \Swa{wa alekum as-salaam}.} } \\ 
\E{With politeness (?)   he asked how [Ja'far] was.  Those present spoke  and all returned his greeting.  } \\ 
\\[8mm] 

\textarabic{(٢١٩) \textcolor{mygreen}{كَمْوَمْبِيَ مَتَمْكٗ  * نِ سَلَامَ نِتٗكَكٗ  * سِيُوِ مْبٖيْ نٖنْدَكٗ  * أَيُوَءٖ نِ جَلِيَ }} \\* 
 \OLTcl{ jaliya ni ayuwae *  nendako mbee siyuwi *  nitokako salāma ni *  matamko kamwambiya} \\* 
\SB{217} (\textbf{219}) \OLTst{kamwambiya matamko  * ni salama nitokako\footnote{This is an echo of 196b, but neatly turns it to refer to time instead of space.}  * siyuwi mbee nendako  * ayuwae ni jaliya } \\ 
\E{[ja'far] spoke [these] words to him:  All is well where I come from;   I do not know about where I am going --   the one who knows is the Almighty.   } \\ 
\\[8mm] 

\textarabic{(٢٢٠) \textcolor{mygreen}{أَكَتَمْكَ أَمِيْنِ  * كَوَمْبِيَ كْوَ هٖرِنِ  * عَلِيْ أَكَبَئِنِ  * مْبٗنَ هُتٗكَ نَبِيَ }} \\* 
 \OLTcl{ nabiya hutoka mbona *  akabaini ʿalii *  herini kwa kawambiya *  amı̄ni akatamka} \\* 
\SB{218} (\textbf{220}) \OLTst{akatamka amini  * kawambiya kwa herini  * 'alii akabaini  * mbona hutoka nabiya } \\ 
\E{The Trustworthy One spoke  and bade them goodbye.   Ali spoke:  Surely you are not leaving, Prophet?   } \\ 
\\[8mm] 

\textarabic{(٢٢١) \textcolor{mygreen}{مْٹُمٖ أَكَتَمْكَ  * زٖءٗ زِمٖأَخِرِكَ  * سَاءَ تِسِيَ هَكِكَ  * نْيُمْبَنِ سِيَرٖجٖيَ }} \\* 
 \OLTcl{ siyarejeya nyumbani *  hakika tisiya saa *  zimeakhirika zeo *  akatamka mţume} \\* 
\SB{219} (\textbf{221}) \OLTst{mtume akatamka  * zeo zimeahirika  * saa tisiya hakika  * nyumbani siyarejeya } \\ 
\E{The Prophet spoke:  the time is late --   it is now the ninth hour for certain,   and I have not yet returned home.  } \\ 
\\[8mm] 

\textarabic{(٢٢٢) \textcolor{mygreen}{كَمْبَ صُبِرِ بَشِيْرِ  * ٹُمْلِشٖ جَعْفَرِ  * چَكُلَ كِكٗ تَيَرِ  * هَپٗ كَكٖتِ نَبِيَ }} \\* 
 \OLTcl{ nabiya kaketi hapo *  tayari kiko chakula *  jaʿfari ţumlishe *  bashı̄ri ṣubiri kamba} \\* 
\SB{220} (\textbf{222}) \OLTst{kamba subiri bashiri  * tumlishe ja'fari\footnote{i.e. \Swa{tule naye}, \E{so that we may share a meal with him}.  Sharing a meal with someone shows respect.}  * chakula kiko tayari  * hapo kaketi nabiya } \\ 
\E{[Ali] said: Wait, Bringer of Good Tidings,   until we have given Ja'far something to eat.  The food is ready.   So the Prophet sat down.   } \\ 
\\[8mm] 

\textarabic{(٢٢٣) \textcolor{mygreen}{فَتُمَ أَكَئِنُكَ  * كْوَ أُپٖسِ نَ هَرَكَ  * تَمَشَ أَكَئِوٖكَ  * نَ مَاءِ كَوَپٖكٖيَ }} \\* 
 \OLTcl{ kawapekeya mai na *  akaiweka tamasha *  haraka na upesi kwa *  akainuka fatuma} \\* 
\SB{221} (\textbf{223}) \OLTst{fatuma akainuka  * kwa upesi na haraka\footnote{After being told not to server the food yet, she now has to do it in a hurry.}  * tamasha\footnote{\Swa{vitu vizuri vizuri}.} akaiweka  * na mai\footnote{This could either be \Swa{ya kunawa}, \E{to wash with}, or \Swa{ya kunwa}, \E{to drink}.  The Swahili custom is not to eat food without water.} kawapekeya } \\ 
\E{Fatima got up  quickly, in a hurry,    and placed delicacies [before them]  and brought them water.   } \\ 
\\[8mm] 

\textarabic{(٢٢٤) \textcolor{mygreen}{وَكَكٖتِ كْوَ مْفَنٗ  * وَكَتَنْڠَنْيَ مِكٗنٗ  * جُمْلَ وَٹُ وَ تَنٗ  * وَلَ أَسِلٖ نَبِيَ }} \\* 
 \OLTcl{ nabiya asile wala *  tano wa waţu jumla *  mikono wakatanganya *  mfano kwa wakaketi} \\* 
\SB{222} (\textbf{224}) \OLTst{wakaketi kwa mfano\footnote{i.e. \Swa{kwa sawasawa}.}  * wakatanganya mikono\footnote{In other words, they eat together, \Swa{kula pamoja}, which brings \Swa{baraka}, \E{blessings}.}  * jumla watu wa tano  * wala asile nabiya } \\ 
\E{They sat equally,   sharing the same dish,  all five of them,    though the Prophet did not eat [much].   } \\ 
\\[8mm] 

\textarabic{(٢٢٥) \textcolor{mygreen}{مَرَ ٹَاٹُ كْوَ هَكِكَ  * مْكٗنٗ أَلِؤُپٖكَ  * أَكِشَ أَكَئِنُكَ  * وَءٗ أَكَوَتِيَ }} \\* 
 \OLTcl{ akawatiya wao *  akainuka akisha *  aliupeka mkono *  hakika kwa ţāţu mara} \\* 
\SB{223} (\textbf{225}) \OLTst{mara tatu kwa hakika  * mkono aliupeka  * akisha akainuka  * wao akawatiya\footnote{In other words, the Prophet stays for a little while for politeness' sake, but leaves as soon as he can.} } \\ 
\E{Three times indeed    he put his hand [into the dish]  and then he got up  and left [the food] to them.  } \\ 
\\[8mm] 

\textarabic{(٢٢٦) \textcolor{mygreen}{أَكَئِنُكَ كِٹِنِ  * كَپٗوَ مَاءِ أَمِيْنِ  * نَ يَ كُنْوَ كِكٗمْبٖنِ  * نَاءٖ أَكَسُكُتُوَ }} \\* 
 \OLTcl{ akasukutuwa nae *  kikombeni kunwa ya na *  amı̄ni mai kapowa *  kiţini akainuka} \\* 
\SB{224} (\textbf{226}) \OLTst{akainuka\footnote{This and the previous stanza are very vivid depictions of Swahili customs.} kitini  * kapowa mai amini  * na ya kunwa kikombeni  * nae akasukutuwa\footnote{After eating, you rinse your mouth with water and spit it out.} } \\ 
\E{He rose from his chair  and accepted water, the Trustworthy One,   in a cup to drink,    and rinsed his mouth.  } \\ 
\\[8mm] 

\textarabic{(٢٢٧) \textcolor{mygreen}{تَمْبُوْ يَ كُخِتَرِ  * هَپٗ كَپٗوَ بَشِيْرِ  * أَكَوَآڠَ كْوَ هٖرِ  * ٹُمْوَ أَكَئِتٗكٖيَ }} \\* 
 \OLTcl{ akaitokeya ţumwa *  heri kwa akawaãga *  bashı̄ri kapowa hapo *  kukhitari ya tambuu} \\* 
\SB{225} (\textbf{227}) \OLTst{tambuu\footnote{\Swa{tambuu} is some lime wrapped in a betel leaf, used like chewing tobacco.  Protracted use stains the teeth red.  Offering \Swa{tambuu} is a particular feature of northern Swahili culture.  However, it is very unlikely that Swa{tambuu} would have been offered in the original Arabian setting of the story, and it is even less likely that the Prophet would have accepted it even if it was.} ya kuhitari\footnote{\Swa{ya kuteua nzuri}.}  * hapo kapowa bashiri  * akawaaga kwa heri  * tumwa akaitokeya } \\ 
\E{Choice tambuu   he was then given, the Bearer of Good Tidings,   and he bade them farewell --    the Prophet went off.  } \\ 
\\[8mm] 

\textarabic{(٢٢٨) \textcolor{mygreen}{ٹُمْوَ أَكِشَ كُتٗكَ  * نَاءٗ كُلَ وَمٖكْوِشَ  * فَتُمَ أَكَئِنُكَ  * مَاءِ أَكَوَپٖكٖيَ }} \\* 
 \OLTcl{ akawapekeya mai *  akainuka fatuma *  wamekwisha kula nao *  kutoka akisha ţumwa} \\* 
\SB{226} (\textbf{228}) \OLTst{tumwa akisha kutoka  * nao kula wamekwisha  * fatuma akainuka  * mai akawapekeya } \\ 
\E{When the Prophet had left   and they had finished eating   Fatima got up  and offered them water.  } \\ 
\\[8mm] 

\textarabic{(٢٢٩) \textcolor{mygreen}{أَكَئِيٗنَ فَتُمَ  * پٹٖ أَكَئِٹِزَمَ  * يَپِسِيٗ يَ نْيُمَ  * يٗتٖ يَكَمْرُدِيَ }} \\* 
 \OLTcl{ yakamrudiya yote *  nyuma ya yapisiyo *  akaiţizama pţe *  fatuma akaiyona} \\* 
\SB{227} (\textbf{229}) \OLTst{akaiyona fatuma  * \dotuline{pete} akaitizama  * yapisiyo ya nyuma  * yote yakamrudiya } \\ 
\E{And Fatima saw it --   she caught sight of the ring.  What had happened in the past   all came back to her.  } \\ 
\\[8mm] 

\textarabic{(٢٣٠) \textcolor{mygreen}{عَلِيْ أَكَتَمْكَ  * مْبٗنَ أُمٖبَدِلِكَ  * كَمَ أُمٖزٗؤُذِكَ  * يٖؤٗ نِكِكْوَنْڠَلِيَ }} \\* 
 \OLTcl{ nikikwangaliya yeo *  umezoudhika kama *  umebadilika mbona *  akatamka ʿalii} \\* 
\SB{228} (\textbf{230}) \OLTst{'alii akatamka  * mbona umebadilika  * kama umezoudhika\footnote{Like \Swa{ambaye mekasirika}, \E{someone who is angry}.}  * yeo\footnote{In this case, \Swa{yeo / leo}, \E{today}, means \q{now}.} nikikwangaliya } \\ 
\E{Ali spoke:  Why has [your mood] changed,  as if you are angry,  now as I look at you?  } \\ 
\\[8mm] 

\textarabic{(٢٣١) \textcolor{mygreen}{فَتُمَ َكَرَدِدِ  * وٖوٖ هُنٖنْدِ بَعِيْدِ  * نِلٗنَلٗ سِنَ بُدِ  * إِلَّا نَاوٖ كُكْوَمْبِيَ }} \\* 
 \OLTcl{ kukwambiya nāwe illā *  budi sina nilonalo *  baʿı̄di hunendi wewe *  akaradidi fatuma} \\* 
\SB{229} (\textbf{231}) \OLTst{fatuma akaradidi  * wewe hunendi ba'idi\footnote{\Swa{baidi} = \Swa{mbali}, \E{far}, but here it has the meaning \q{yet}.  i.e. Ali is going to be there for a while, so she will tell him now.}  * nilonalo sina budi  * illa nawe kukwambiya } \\ 
\E{Fatima replied:  You are not leaving yet.    [the thing] I have [in my heart], I have no choice   but to tell you.   } \\ 
\\[8mm] 

\textarabic{(٢٣٢) \textcolor{mygreen}{أَكِسِكِيَ حَسَنِ  * كَتٗكَ كُلٖ نْيُمْبَنِ  * أَكَنٖنْدَ كْوَ أَمِيْنِ  * حبر أَكَمْوَمْبِيَ }} \\* 
 \OLTcl{ akamwambiya ḥbr *  amı̄ni kwa akanenda *  nyumbani kule katoka *  ḥasani akisikiya} \\* 
\SB{230} (\textbf{232}) \OLTst{akisikiya hasani  * katoka kule nyumbani  * akanenda kwa amini  * \dotuline{habari} akamwambiya } \\ 
\E{When Hasan heard this  he left the house   and went to the Trustworthy One   and told him the news.  } \\ 
\\[8mm] 

\textarabic{(٢٣٣) \textcolor{mygreen}{حَسَنِ كِشَ كُفِكَ  * كْوَ مْٹُمٖ كَتَمْكَ  * مِمِ أَمٖكَسِرِكَ  * بِبِ يَنْڠُ نَكْوَمْبِيَ }} \\* 
 \OLTcl{ nakwambiya yangu bibi *  amekasirika mimi *  katamka mţume kwa *  kufika kisha ḥasani} \\* 
\SB{231} (\textbf{233}) \OLTst{hasani kisha kufika  * kwa mtume katamka  * mimi amekasirika  * bibi\footnote{\Swa{bibi} is a more polite way of saying \Swa{mama}.} yangu nakwambiya } \\ 
\E{When Hassan got there   he told the Prophet:   she is angry,  my mother - I'm telling you.   } \\ 
\\[8mm] 

\textarabic{(٢٣٤) \textcolor{mygreen}{أَكَمُؤُزَ أَمِيْنِ  * مٖكَسِرِكِيَ نِنِ  * مْوٖنْيٖوٖ هَكُبَئِنِ  * مٗيَ سِكُفَهَمِيَ }} \\* 
 \OLTcl{ sikufahamiya moya *  hakubaini mwenyewe *  nini mekasirikiya *  amı̄ni akamuuza} \\* 
\SB{232} (\textbf{234}) \OLTst{akamuuza amini  * mekasirikiya nini  * mwenyewe \dotuline{sikubaini}  * moya sikufahamiya } \\ 
\E{The Trustworthy One asked him:  Why has she got angry?  [Hasan said:] Myself I don't know --   I don't understand [a thing about it].  } \\ 
\\[8mm] 

\textarabic{(٢٣٥) \textcolor{mygreen}{كُرُدِ كْوَكٖ نْدِيَنِ  * أَكَمُؤٗنَ حُسَيْنِ  * آٹِ مٖكُيَ مْڠٖنِ  * هُكُ كْوٖٹُ نَسِكِيَ }} \\* 
 \OLTcl{ nasikiya kweţu huku *  mgeni mekuya ãţi *  ḥusayni akamuona *  ndiyani kwake kurudi} \\* 
\SB{233} (\textbf{235}) \OLTst{kurudi kwake ndiyani  * akamuona husayni  * ati mekuya mgeni  * huku kwetu nasikiya } \\ 
\E{As [Hasan] went back along the road   he saw Husayn [who said:]  So, a visitor has come   to our house, I hear.   } \\ 
\\[8mm] 

\textarabic{(٢٣٦) \textcolor{mygreen}{أَكَمْجِبُ حَسَنِ  * مْوٖنْيٖ خَبَرِ مُئِنِ  * إِنَ يٖيٖ تَمْكِنِ  * هٗيٗ أَمٖزٗوَمْبِيَ }} \\* 
 \OLTcl{ amezowambiya hoyo *  tamkini yeye ina *  muini khabari mwenye *  ḥasani akamjibu} \\* 
\SB{234} (\textbf{236}) \OLTst{akamjibu hasani  * mwenye habari muini\footnote{Hasan is annoyed because someone is spreading gossip about the fact that Ali has a hitherto-unknown son.  To avoid confirming the rumours he does a typically Swahili thing -- if someone asks you if such-and-such a rumour is true, you say: \q{The one who told you is the one who knows -- go back and ask him}.}  * ina yeye\footnote{i.e. Ja'far's name.} tamkini\footnote{= \Swa{hakika}, \E{certainly}.}  * hoyo amezowambiya } \\ 
\E{Hasan answered him:  the gossip-monger in the town,   [ja'far's] name, certainly,   [it is] this person who has told [people] that.  } \\ 
\\[8mm] 

\textarabic{(٢٣٧) \textcolor{mygreen}{أَكَپِٹَ مْلَنْڠٗنِ  * أَكَرَدِدِ حُسَيْنِ  * آسَ وَمْتَكِيَنِ  * پٖنُ أَلِوَپٗتٖيَ }} \\* 
 \OLTcl{ aliwapoteya penu *  wamtakiyani ãsa *  ḥusayni akaradidi *  mlangoni akapiţa} \\* 
\SB{235} (\textbf{237}) \OLTst{akapita mlangoni  * akaradidi husayni  * \dotuline{basi} wamtakiyani\footnote{\Swa{-m-} here = \Swa{-ni-}.  i.e. it's no-one else's business.}  * penu\footnote{We understand \Swa{pahali}.} aliwapoteya\footnote{The meaning seems to be that there is no justification for any gossip, because it is not as if Ja'far has gone wandering around the town like a child or a pet, giving people cause to talk about it.} } \\ 
\E{He came to the door [of Ali's house]  and told Husayn:  So, why are you bothering me?  Has he left our house [and gone wandering about]?  } \\ 
\\[8mm] 

\textarabic{(٢٣٨) \textcolor{mygreen}{أَكَمُؤُزَ بَبَكٖ  * أُنَنِ هُنٖنَ پْوٖكٖ  * أَكَمْبَ خٖيْرِ نِتٗكٖ  * خَبَرِ زِمٖئٖنٖيَ }} \\* 
 \OLTcl{ zimeeneya khabari *  nitoke khēri akamba *  pweke hunena unani *  babake akamuuza} \\* 
\SB{236} (\textbf{238}) \OLTst{akamuuza babake  * unani\footnote{= \Swa{una nini?}.  See 244a, 263b.} hunena pweke  * akamba heri nitoke\footnote{We have to assume that Hasan and Husayn have told Ali what they were talking about.}  * habari zimeeneya\footnote{In other words, Ali thinks it would be better to give the word officially, instead of having people gossip about it as a scandal.} } \\ 
\E{His father asked him:  What's the matter?  You are speaking amongst yourselves.   Then [Ali] said: It is best I go out.   The news has spread.  } \\ 
\\[8mm] 

\textarabic{(٢٣٩) \textcolor{mygreen}{كُمٖپَنَنَ خَبَرِ  * جَمِيْعِ يَ أَنْصَارِ  * أَمٖكُيَ جَعْفَرِ  * وَ مَوْلَانَا عَلِيَ }} \\* 
 \OLTcl{ ʿaliya mawlānā wa *  jaʿfari amekuya *  anṣāri ya jamı̄ʿi *  khabari kumepanana} \\* 
\SB{237} (\textbf{239}) \OLTst{kumepanana habari  * jami'i ya ansari\footnote{\Swa{ansari} are the people of Medina who gave sanctuary to the Prophet when he was forced to flee from Mecca in 622 CE.}  * amekuya ja'fari  * wa maulana 'aliya } \\ 
\E{The news is being passed about  among all the Helpers:   Ja'far has arrived,  [the son] of Lord Ali.   } \\ 
\\[8mm] 

\textarabic{(٢٤٠) \textcolor{mygreen}{هَپٗ عَلِيْ حَيْدَرِ  * نٖنٗ أَلِلٗفَسِرِ  * أُوِنْڠَ أُنَ خَطَرِ  * مَمْبٗ يَكِتُمِلِيَ }} \\* 
 \OLTcl{ yakitumiliya mambo *  khaṭari una uwinga *  alilofasiri neno *  ḥaydari ʿalii hapo} \\* 
\SB{238} (\textbf{240}) \OLTst{hapo 'alii haydari  * neno alilofasiri\footnote{\Swa{Ali anamlaumu mtoto wake}, \E{Ali is criticising his son [Hasan]}.  i.e. Ali is telling them they should not be upset by gossip.}  * uwinga\footnote{Cognate with \Swa{jinga} in 202a.} una hatari  * mambo \dotuline{yakitokweleya} } \\ 
\E{Then Ali the Lionlike,   the words that he said [were]:  Foolishness is dangerous,   if someone does not understand how things are.  } \\ 
\\[8mm] 

\textarabic{(٢٤١) \textcolor{mygreen}{كْوٖنُ كُئِيٖ مْڠٖنِ  * هِلٗ هُكُؤُذِيَنِ  * تَمُوٖكَ هُكُ نْيُمْبَنِ  * مُئِنَ هَتٗتٖمْبٖيَ }} \\* 
 \OLTcl{ hatotembeya muina *  nyumbani huku tamuweka *  hukuudhiyani hilo *  mgeni kuiye kwenu} \\* 
\SB{239} (\textbf{241}) \OLTst{kwenu kuiye mgeni\footnote{Ali is asking Hasan: \Swa{kwa nini umekasirika?}, \E{why are you angry?}.  You must know that I have a duty of care to Ja'far -- I cannot disown him and leave him to wander around the town by himself.}  * hilo hukuudhiyani  * tamuweka huku nyumbani  * muina hatotembeya\footnote{This is a rhetorical question: Ali is saying that trying to keep Ja'far's existence secret by locking him in the house would be just as bad as disowning him and leaving him to wander about like a beggar.} } \\ 
\E{A visitor has come to your house --   why does this disturb you?  Should I keep him here in the house   so that he will not wander around the town?  } \\ 
\\[8mm] 

\textarabic{(٢٤٢) \textcolor{mygreen}{فَتُمَ أَكَنُطُمُ  * كْوَنِ سِ مْوَنَ حَرَمُ  * مْوَنَ هَنَ تَبَسَمُ  * عَلِيْ كِمْوَنْڠَلِيَ }} \\* 
 \OLTcl{ kimwangaliya ʿalii *  tabasamu hana mwana *  ḥaramu mwana si kwani *  akanuṭumu fatuma} \\* 
\SB{240} (\textbf{242}) \OLTst{fatuma \dotuline{akanudhumu}\footnote{Fatima supports the point Ali is making to his children. }  * kwani si mwana haramu  * mwana\footnote{\Swa{mwana} in the previous line meant \E{child}, but in this line it is used a respectful title, \E{lady, mistress}. } hana tabasamu\footnote{\Swa{amehuzunika}, \E{he has become sad}, because everyone seems to be against him.}  * 'alii kimwangaliya\footnote{In spite of supporting Ali's comments, Fatima is still upset about her discovery.} } \\ 
\E{Fatima spoke [to the boys]:  Why [do you want to hide him]?  He is not an illegitimate child.    [But] the Lady [Fatima] appeared sad   when Ali looked at her.  } \\ 
\\[8mm] 

\textarabic{(٢٤٣) \textcolor{mygreen}{عَلِيْ هَپٗ كَسٖمَ  * هٖلَ نْدٗوْ فَطُمَ  * أَكَئِنُكَ كْوَ هِمَ  * مْكٖوٖ كَمُئٖنْدٖيَ }} \\* 
 \OLTcl{ kamuendeya mkewe *  hima kwa akainuka *  faṭuma ndoo hela *  kasema hapo ʿalii} \\* 
\SB{241} (\textbf{243}) \OLTst{'alii hapo kasema  * hela\footnote{= \Swa{hebu}.} ndoo fatuma  * akainuka kwa hima\footnote{= \Swa{taratibu}.}  * mkewe kamuendeya } \\ 
\E{So Ali said:   Come now, Fatima.   He got up carefully   and went to his wife.  } \\ 
\\[8mm] 

\textarabic{(٢٤٤) \textcolor{mygreen}{أَكَمُؤُزَ أُنَنِ  * مْبٗنَ أُنَ كِسِرَنِ  * فَتُمَ أَكَمْبَ كُنِ  * يَ مَتُنْڠُ هُكْوَمْبِيَ }} \\* 
 \OLTcl{ hukwambiya matungu ya *  kuni akamba fatuma *  kisirani una mbona *  unani akamuuza} \\* 
\SB{242} (\textbf{244}) \OLTst{akamuuza unani\footnote{= \Swa{una nini?}.  See 238b, 263b.}  * mbona una kisirani\footnote{\Swa{haṯeki}, \E{she is not laughing}.  If someone is in a bad mood, you might say: \Swa{ameamka na kisirani}, \E{he got out of the wrong side of the bed}.  A \Swa{siku wa kisirani} is a \q{bad hair day}, a day on which nothing goes right.}  * fatuma akamba \dotuline{kwani}\footnote{\Swa{kwani}, \E{why?}.}  * ya matungu\footnote{\E{bitterness}.} hukwambiya } \\ 
\E{He asked her: What is the matter?  Why are you frowning?   Fatima said: What is the point   of telling you bitter things.   } \\ 
\\[8mm] 

\textarabic{(٢٤٥) \textcolor{mygreen}{أَكَمْجِبُ تَمْكٗ  * هِيٗ سِ طَبِيَ يَكٗ  * مِمِ سِ كِجَنَ چَكٗ  * أَمْبَ هَيٗ هُنَمْبِيَ }} \\* 
 \OLTcl{ hunambiya hayo amba *  chako kijana si mimi *  yako ṭabiya si hiyo *  tamko akamjibu} \\* 
\SB{243} (\textbf{245}) \OLTst{akamjibu tamko  * hiyo si tabiya yako  * mimi si kijana chako\footnote{It is said: \Swa{mtu mzima, huwezi kumdanganya}, \E{you cannot hoodwink a mature person}.  Ali is telling Fatima: \Swa{usinihadae, mimi si mtoto}, \E{don't try to fool me, I am not a child}.  He knows something is troubling her, and wants her to say what it is.}  * amba hayo hunambiya } \\ 
\E{[Ali] answered her with the words:  This is not like you.    I am not your child,    say what it is, and tell me.   } \\ 
\\[8mm] 

\textarabic{(٢٤٦) \textcolor{mygreen}{فَتُمَ أَكَبَئِنِ  * سِ إِلٖ پٖٹٖ چَنْدَنِ  * يَلٗنْڠِيَ كِسِمَنِ  * آٹِ زِيَپٗ هُٹِيَ }} \\* 
 \OLTcl{ huţiya ziyapo ãţi *  kisimani yalongiya *  chandani peţe ile si *  akabaini fatuma} \\* 
\SB{244} (\textbf{246}) \OLTst{fatuma akabaini  * si ile pete chandani  * yalongiya kisimani  * ati\footnote{\Swa{ati} here implies that what was said is a lie.} ziyapo\footnote{\E{oaths}.} hutiya } \\ 
\E{Fatima spoke:  That ring on his finger, is it not [the one]    which "fell into the well",  as you swore?   } \\ 
\\[8mm] 

\textarabic{(٢٤٧) \textcolor{mygreen}{عَلِيْ أَكَمْبَ هَكِكَ  * نْدِپٗ أُكَكَسِرِكَ  * پَلٖ أُنْڠٖلِؤُذِكَ  * كَمَ كِلٖ نَكْوَمْبِيَ }} \\* 
 \OLTcl{ nakwambiya kile kama *  ungeliudhika pale *  ukakasirika ndipo *  hakika akamba ʿalii} \\* 
\SB{245} (\textbf{247}) \OLTst{'alii akamba hakika  * ndipo\footnote{\E{that is why}.} ukakasirika  * pale ungeliudhika  * kama kile\footnote{Amend translation.} nakwambiya\footnote{In other words, \q{You would have got angry if I had not told you lies}.} } \\ 
\E{Ali said: Indeed,    so that's why you are angry --  you would have got angry at that time [as well],  if I had told you the truth.   } \\ 
\\[8mm] 

\textarabic{(٢٤٨) \textcolor{mygreen}{نَ سَسَ نٖنْڠٖكُؤُذِ  * نْدُڠُ يَنْڠُ وَتَ غَرَضِ  * مٗيٗ وَكٖ أُوٖ رَضِ  * نَ أُتَكَلٗ نَمْبِيَ }} \\* 
 \OLTcl{ nambiya utakalo na *  raḍi uwe wake moyo *  gharaḍi wata yangu ndugu *  nengekuudhi sasa na} \\* 
\SB{246} (\textbf{248}) \OLTst{na sasa nengekuudhi  * ndugu yangu wata gharadhi\footnote{\Swa{gharadhi} = \Swa{hasira}.}  * moyo wake uwe radhi  * na utakalo nambiya\footnote{Ali is trying to mollify his wife.} } \\ 
\E{And now, even if I have hurt you,   stop being angry, my dear.    let your heart be forgiving    and tell me what you want.   } \\ 
\\[8mm] 

\textarabic{(٢٤٩) \textcolor{mygreen}{فَتُمَ كٖٹَ قَوْلِ  * كِٹُ سِ يَ كُلَ دَلِيْلِ  * أُوَپٗ أُمٖكُبَلِ  * رَضِ نِمٖكْوٖلٖيَ }} \\* 
 \OLTcl{ nimekweleya raḍi *  umekubali uwapo *  dalı̄li kula ya si kiţu *  qawli keţa fatuma} \\* 
\SB{247} (\textbf{249}) \OLTst{fatuma keta qauli  * kitu si ya kula dalili\footnote{The meaning of this line is not entirely clear.  \Swa{dalili} is usuallly translated as \q{sign}, but it is also a term for \q{proof}, as used in logic.  So the line might be paraphrased as: \q{between us, the issue (\Swa{kitu}) does not need to be proved on every point, because we love each other}.}  * uwapo\footnote{= \Swa{ukiwa}.} umekubali  * radhi nimekweleya\footnote{= \Swa{nimekusamehe}, \E{I have forgiven you}.    The mollification works -- Fatima forgives him.} } \\ 
\E{Fatima spoke these words:   the matter is of little importance.     Since you have now agreed [you were wrong],  I forgive you.  } \\ 
\\[8mm] 

\textarabic{(٢٥٠) \textcolor{mygreen}{وَكَكٖتِ كْوَ لِسَنِ  * يٖيٖ نَ مْوَنَ نْيُمْبَنِ  * نَوٖ نٖنْدَپٗ زِٹَنِ  * وَچٖنْدَ وٗتٖ پَمٗيَ }} \\* 
 \OLTcl{ pamoya wote wachenda *  ziţani nendapo nawe *  nyumbani mwana na yeye *  lisani kwa wakaketi} \\* 
\SB{248} (\textbf{250}) \OLTst{wakaketi kwa \dotuline{hisani}\footnote{\Swa{hisani}, \E{kindness, goodness}.}  * yeye na mwana nyumbani  * \dotuline{naye} \dotuline{wendapo} zitani  * wachenda wote pamoya\footnote{i.e. Ali took Ja'far with him on his campaigns.} } \\ 
\E{They lived happily,   [Ali] and the boy, in the house.    When [Ali] went to war   they both went together.   } \\ 
\\[8mm] 

\textarabic{(٢٥١) \textcolor{mygreen}{جَعْفَرِ نِ مْڠٖنِ  * هَيَزٗوٖيَ زِٹَنِ  * مَهَلَ پَ مَيْتِنِ  * عَلِيْ أَكِمْوٖنْدٖلٖيَ }} \\* 
 \OLTcl{ akimwendeleya ʿalii *  maytini pa mahala *  ziţani hayazoweya *  mgeni ni jaʿfari} \\* 
\SB{249} (\textbf{251}) \OLTst{ja'fari ni mgeni  * hayazoweya zitani  * mahala pa \dotuline{miyateni}  * 'alii akimwendeleya\footnote{Unlike Ali, Ja'far cannot yet fight 200 opponents alone!} } \\ 
\E{Ja'far was a stranger [to war]   he was not accustomed to battle --  where there were 200 [opponents]   Ali would go to him [to help].  } \\ 
\\[8mm] 

\textarabic{(٢٥٢) \textcolor{mygreen}{هَتَ أَكِتِمُ مْوَكَ  * زِٹَنِ أَمٖصِفِكَ  * سَبَا مِيَ هَكِكَ  * هُتِنْدَ أَسِپٗيُوَ }} \\* 
 \OLTcl{ asipoyuwa hutinda *  hakika miya sabā *  ameṣifika ziţani *  mwaka akitimu hata} \\* 
\SB{250} (\textbf{252}) \OLTst{hata akitimu mwaka  * zitani amesifika  * saba miya hakika  * hutinda asipoyuwa\footnote{i.e. he could do it without realising.} } \\ 
\E{Until at the end of one year   he was renowned in battle.  Indeed, 700 [opponents]   he would cut down with no effort.  } \\ 
\\[8mm] 

\textarabic{(٢٥٣) \textcolor{mygreen}{هَتَ مْوَكَ أُكِزِدِ  * هَپٗ أَكٖنْدَ جِهَدِ  * لَكِ مٗيَ مَيَهُدِ  * هَكُنَ هَتَ مْمٗيَ }} \\* 
 \OLTcl{ mmoya hata hakuna *  mayahudi moya laki *  jihadi akenda hapo *  ukizidi mwaka hata} \\* 
\SB{251} (\textbf{253}) \OLTst{hata mwaka\footnote{\Swa{mwaka}, \E{year}, is used here to mean \q{time} in general.  Compare \Swa{saa} in 201d.} ukizidi  * hapo akenda jihadi  * laki\footnote{\Swa{laki} < \AS{لَكٌّ}, \E{100,000}.} moya mayahudi\footnote{See 209b.  This word can be used for anyone who is bad or evil.}  * hakuna hata mmoya } \\ 
\E{Until, as time went by,   when he went on a crusade,   of 100,000 unbelievers   there was not one [left alive].   } \\ 
\\[8mm] 

\textarabic{(٢٥٤) \textcolor{mygreen}{أَلِپٗكُيَ مُئِنِ  * عَلِيْ أَكَبَئِنِ  * أَكَمْوَمْبِيَ أَمِنِ  * نِنَ يَمْبٗ تَكْوَمْبِيَ }} \\* 
 \OLTcl{ takwambiya yambo nina *  amini akamwambiya *  akabaini ʿalii *  muini alipokuya} \\* 
\SB{252} (\textbf{254}) \OLTst{alipokuya muini  * 'alii akabaini  * akamwambiya amini  * nina yambo takwambiya } \\ 
\E{When he came back to the town [after one campaign]  Ali spoke  and told the Trustworthy One:  I have something to tell you.   } \\ 
\\[8mm] 

\textarabic{(٢٥٥) \textcolor{mygreen}{نَپٖنْدَ سَسَ بَشِرِ  * إٖنْدٖ پْوٖكٖ جَعْفَرِ  * أَكَپِجٖ مَكُفَرِ  * هُتٗشَ كِمْوَنْڠَلِيَ }} \\* 
 \OLTcl{ kimwangaliya hutosha *  makufari akapije *  jaʿfari pweke ende *  bashiri sasa napenda} \\* 
\SB{253} (\textbf{255}) \OLTst{napenda sasa bashiri  * ende pweke ja'fari  * akapije makufari  * hutosha\footnote{\E{he is capable of [doing something]}.} kimwangaliya\footnote{Compare: \Swa{kila kimwangaliya, naona ana mambo yule}, \E{every time I look at him, I see that guy has something}.} } \\ 
\E{I would now like, Bringer of Good Tidings,   for Ja'far to go on his own   to fight the unbelievers.  He is fully able, in my opinion.  } \\ 
\\[8mm] 

\textarabic{(٢٥٦) \textcolor{mygreen}{أَكَشُكَ جِبْرِيْلِ  * أَكَمْوَمْبِيَ رَسُوْلِ  * هَوٖكِ سِمْبَ وَوِلِ  * أَكُسَلِمُ نَبِيَ }} \\* 
 \OLTcl{ nabiya akusalimu *  wawili simba haweki *  rasūli akamwambiya *  jibrı̄li akashuka} \\* 
\SB{254} (\textbf{256}) \OLTst{akashuka jibrili  * akamwambiya rasuli  * haweki\footnote{We understand \Swa{Mungu}, \E{God}.} simba\footnote{Ali is known as \Swa{simba wa Mungu}.} wawili  * akusalimu nabiya } \\ 
\E{Gabriel descended  and told the Prophet:  [God] cannot have two Lions,   and he greets you, Prophet.  } \\ 
\\[8mm] 

\textarabic{(٢٥٧) \textcolor{mygreen}{أَكَتَمْكَ أَمِيْنِ  * عَلِيْ ٹْوٖنْدٖ نْيُمْبَنِ  * مْوَنٗ هُيٗ يَقِيْنِ  * هُئِفَرِكِ دُنِيَ }} \\* 
 \OLTcl{ duniya huifariki *  yaqı̄ni huyo mwano *  nyumbani ţwende ʿalii *  amı̄ni akatamka} \\* 
\SB{255} (\textbf{257}) \OLTst{akatamka amini\footnote{Unlike Ali, the Prophet immediately understands the implications of the angel's message.}  * 'alii twende nyumbani  * mwano\footnote{= \Swa{mwanayo, mwana wako, mtoto wako}.} huyo yaqini  * huifariki duniya } \\ 
\E{The Trustworthy One spoke:  Ali, let us go to your house --   this son of yours, it seems,   is departing this world.  } \\ 
\\[8mm] 

\textarabic{(٢٥٨) \textcolor{mygreen}{چَمْبِوَ هِيٗ قَوْلِ  * هَپٗ أَسِيَمُهَلِ  * كَئِنُكَ نَ رَسُوْلِ  * هَپٗ وَكَنْدَمَ نْدِيَ }} \\* 
 \OLTcl{ ndiya wakandama hapo *  rasūli na kainuka *  asiyamuhali hapo *  qawli hiyo chambiwa} \\* 
\SB{256} (\textbf{258}) \OLTst{chambiwa hiyo qauli  * hapo asiyamuhali\footnote{i.e. he did not delay.}  * kainuka na rasuli  * hapo\footnote{i.e. \Swa{pale pale}, \E{then and there}.} wakandama ndiya } \\ 
\E{When he was told these words   [Ali] did not linger there --  he got up with the Prophet   and then they set out on the road.   } \\ 
\\[8mm] 

\textarabic{(٢٥٩) \textcolor{mygreen}{أَكِپَٹَ مْلَنْڠٗنِ  * فَتُمَ أَكَبَئِنِ  * أَلِهُتٗكَ حَسَنِ  * أُيَاءٗ كُكْوَنْدَمِيَ }} \\* 
 \OLTcl{ kukwandamiya uyao *  ḥasani alihutoka *  akabaini fatuma *  mlangoni akipaţa} \\* 
\SB{257} (\textbf{259}) \OLTst{akipata mlangoni\footnote{i.e. \Swa{hajangia ndani} -- he has not gone into the house yet.}  * fatuma\footnote{In a fairytale we would immediately conclude that \Swa{mama wa kambo anamdhuru}, \E{his stepmother is doing him harm}, but nothing could be further from the truth in this case -- it is God who has determined Ja'far's fate.} akabaini  * alihutoka hasani  * uyao kukwandamiya } \\ 
\E{When he got to the door  Fatima spoke:  Hasan has [just] left   to go and fetch you.  } \\ 
\\[8mm] 

\textarabic{(٢٦٠) \textcolor{mygreen}{مْوَنٗ أَلِپٗ كِٹِنِ  * نَ حَسَنِ نَ حُسَيْنِ  * غَفُلَ أَكَبَئِنِ  * بَبَنْڠُ نَمْكُلِيَ }} \\* 
 \OLTcl{ namkuliya babangu *  akabaini ghafula *  ḥusayni na ḥasani na *  kiţini alipo mwano} \\* 
\SB{258} (\textbf{260}) \OLTst{mwano alipo kitini  * na hasani na husayni  * ghafula akabaini  * babangu namkuliya\footnote{i.e. call my father for me.} } \\ 
\E{Your son was sitting there   with Hasan and Husayn    and all of a sudden he said:  I need to call my father.  } \\ 
\\[8mm] 

\textarabic{(٢٦١) \textcolor{mygreen}{أَمٖئِنُكَ كِٹِنِ  * أَمٖپَنْدَ فِرَشَنِ  * نَاءٖ مْوَنْڠَلِيِنِ  * يَمْبٗ لَلٗمْزِدِيَ }} \\* 
 \OLTcl{ lalomzidiya yambo *  mwangaliyini nae *  firashani amepanda *  kiţini ameinuka} \\* 
\SB{259} (\textbf{261}) \OLTst{ameinuka kitini  * amepanda firashani  * nae \dotuline{mwangaliyeni}  * yambo lalomzidiya\footnote{Or: \q{what misfortune has overwhelmed him?}.  Compare \Swa{kumezidi nini?}, \E{what has happened?} for something disastrous or catastrophic.} } \\ 
\E{He got up from the chair  and climbed onto the bed.  Go and look at him --  what has happened to him?  } \\ 
\\[8mm] 

\textarabic{(٢٦٢) \textcolor{mygreen}{هَپٗ أَكٖنْدَ بَشِيْرِ  * نَ عَلِىْ حَىْدَرِ  * كِمْوٗنَ جَعْفَرِ  * هَپٗ بَبَكٖ كَلِيَ }} \\* 
 \OLTcl{ kaliya babake hapo *  jaʿfari kimwona *  ḥaydari ʿalii na *  bashı̄ri akenda hapo} \\* 
\SB{260} (\textbf{262}) \OLTst{hapo akenda bashiri  * na 'alii haydari  * kimwona ja'fari  * hapo babake kaliya } \\ 
\E{So the Bringer of Good Tidings went in   with Ali the Lion-like.   and when he saw Ja'far  his father wept.   } \\ 
\\[8mm] 

\textarabic{(٢٦٣) \textcolor{mygreen}{أَكَلِيَ كِبَنِ  * إٖوٖ مْوَنَنْڠُ أُنَنِ  * أُپٖٹْوٖ نِ يَمْبٗ ڠَنِ  * كَٹِكَ كْوَنْدَم نْدِيَ }} \\* 
 \OLTcl{ ndiya kwandam kaţika *  gani yambo ni upeţwe *  unani mwanangu ewe *  kibani akaliya} \\* 
\SB{261} (\textbf{263}) \OLTst{akaliya \dotuline{kibaini}  * ewe mwanangu unani\footnote{= \Swa{una nini?}.  See 238b, 244b.}  * upetwe ni yambo gani  * katika \dotuline{kwandama} ndiya } \\ 
\E{He wept, saying:  Oh, my son, what is the matter with you?   What misfortune has stricken you    as you went on your way?   } \\ 
\\[8mm] 

\textarabic{(٢٦٤) \textcolor{mygreen}{أَكِسِكِيَ كَلِمَ  * جَعْفَرِ كَفَهَمَ  * كْوَ ضَرُبُ كَٹِزَمَ  * مَتٗ أَكَمْوَنْڠَلِيَ }} \\* 
 \OLTcl{ akamwangaliya mato *  kaţizama ḍarubu kwa *  kafahama jaʿfari *  kalima akisikiya} \\* 
\SB{262} (\textbf{264}) \OLTst{akisikiya kalima  * ja'fari kafahama  * kwa dharubu\footnote{\Swa{dharubu} = \Swa{taabu, mashaka}} katizama  * mato akamwangaliya } \\ 
\E{When he heard these words  Ja'far regained consciousness  and looked about with difficulty   and focussed his eyes on him.  } \\ 
\\[8mm] 

\textarabic{(٢٦٥) \textcolor{mygreen}{كِمْوَنْڠَلِيَ أَمِيْنِ  * أَكَمْبَ نِپَ يَسِنِ  * أَكِكٗمَ كُبَئِنِ  * أَمٖكْوِشَ كُئِفِيَ }} \\* 
 \OLTcl{ kuifiya amekwisha *  kubaini akikoma *  yasini nipa akamba *  amı̄ni kimwangaliya} \\* 
\SB{263} (\textbf{265}) \OLTst{kimwangaliya amini  * akamba nipa\footnote{lit. \q{give me}, as a favour.  The sick person will also be offered watr.} yasini\footnote{See note to 151d.  Chapter 36, \Eit{Ya Sin}, of the Qur'an is read over the sick or dying.  It is considered unfortunate to die without having it read over you.}  * akikoma kubaini  * amekwisha kuifiya } \\ 
\E{When he saw the Trustworthy One  he said: Read me [the chapter] Ya Sin.   By the time he had finished speaking,  [Ja'far] was already dead.  } \\ 
\\[8mm] 

\textarabic{(٢٦٦) \textcolor{mygreen}{هَپٗ عَلِيْ حَيْدَرِ  * يٗتٖ أَسِيَفِكِرِ  * أَكَسِمَمَ بَشِيْرِ  * مَصَحَبَ كَوَمْبِيَ }} \\* 
 \OLTcl{ kawambiya maṣaḥaba *  bashı̄ri akasimama *  asiyafikiri yote *  ḥaydari ʿalii hapo} \\* 
\SB{264} (\textbf{266}) \OLTst{hapo 'alii haydari  * yote asiyafikiri\footnote{\Swa{hajui mambo}.  Usually only women are in this state after someone has died -- the men try to concentrate on making the funeral arrangements.  In this case, the Prophet steps in to organise the funeral.}  * akasimama\footnote{\Swa{-simama} does not mean just \q{stand up}; it also means \q{do anything that needs to be done}, i.e. in this case, step into the breach as regards the aftermath of Ja'far's death.} bashiri  * masahaba kawambiya } \\ 
\E{Then Ali the Lion-like   became insensible to anything.  The Bringer of Good Tidings had to do the needful,  and spoke to the Companions.  } \\ 
\\[8mm] 

\textarabic{(٢٦٧) \textcolor{mygreen}{كَوَمْبِيَ كِپُلِكَ  * نَ جَمِيْعِ وَكَتٗكَ  * هَيَ وَكِشَ كُزِكَ  * عَلِيْ أَسِپٗيُوَ }} \\* 
 \OLTcl{ asipoyuwa ʿalii *  kuzika wakisha haya *  wakatoka jamı̄ʿi na *  kipulika kawambiya} \\* 
\SB{265} (\textbf{267}) \OLTst{kawambiya kipulika  * na jami'i wakatoka\footnote{With Ja'far's corpse.}  * haya wakisha kuzika  * 'alii asipoyuwa } \\ 
\E{He spoke to them and they listened,  and they all went out   and they completed the burial ceremony,   Ali still insensible.  } \\ 
\\[8mm] 

\textarabic{(٢٦٨) \textcolor{mygreen}{أَلِپٗكْوِشَ كُزِكَ  * مْٹُمِ أَكَمُوٖكَ  * مَنٖنٗ أَكَتَمْكَ  * عَلِيْ أَكَمْوَمْبِيَ }} \\* 
 \OLTcl{ akamwambiya ʿalii *  akatamka maneno *  akamuweka mţumi *  kuzika alipokwisha} \\* 
\SB{266} (\textbf{268}) \OLTst{alipokwisha kuzika  * mtumi akamuweka\footnote{In a chair.}  * maneno akatamka  * 'alii akamwambiya } \\ 
\E{When he had completed the burial  the Prophet sat [Ali] down  and spoke [these] words  and addressed Ali.  } \\ 
\\[8mm] 

\textarabic{(٢٦٩) \textcolor{mygreen}{كَمْوَمْبِيَ كِپُلِكَ  * صُبِرِ كْوَكٖ رَبُك  * مْٹُ هَنْڠَلِكُپٗكَ  * نَوٖ أُكَمْوَنْڠَلِيَ }} \\* 
 \OLTcl{ ukamwangaliya nawe *  hangalikupoka mţu *  rabuk kwake ṣubiri *  kipulika kamwambiya} \\* 
\SB{267} (\textbf{269}) \OLTst{kamwambiya kipulika  * subiri\footnote{You have to endure whatever God sends you.  If someone wanted to take your child, you would not just stand there and look at him, but what else can you do in this case?} kwake \dotuline{rabuka}  * mtu hangalikupoka\footnote{Amu \Swa{-poka} = Mvita \Swa{-pokonya}, \E{seize}.}  * nawe ukamwangaliya } \\ 
\E{He told him as [Ali] listened:  Have trust in Him, your Lord --   a person may be seized [by death]  even if you were to stand watch over him.  } \\ 
\\[8mm] 

\textarabic{(٢٧٠) \textcolor{mygreen}{وَ أَمَّا نِ مْٹُ ڠَنِ  * أَدُمُوٗ دُنِيَنِ  * إِسِپٗكُوَ مَنَّانِ  * نَوٖ وَيَفَهَمِيَ }} \\* 
 \OLTcl{ wayafahamiya nawe *  mannāni isipokuwa *  duniyani adumuwo *  gani mţu ni ammā wa} \\* 
\SB{268} (\textbf{270}) \OLTst{wa amma ni mtu gani  * adumuwo\footnote{\Swa{-dumu} < \AS{دَامَ}, \E{endure}, cognate of \Swa{daima}, \E{always}.} duniyani  * isipokuwa mannani\footnote{\Swa{ela Mannani tu}.  \Swa{Mannani} < \AS{المنّان}, the Benevolent One, < \AS{مَنُّ}, \E{bestow favours}.}  * nawe wayafahamiya } \\ 
\E{And indeed, what kind of person is it     who remains in existence,  unless it is God alone,  and you know that well.  } \\ 
\\[8mm] 

\textarabic{(٢٧١) \textcolor{mygreen}{عَلِيْ سٖنٖنْدٖ مْنٗ  * مَمْبٗ هُپِجْوَ مْفَنٗ  * كَپٖوَ هَيٗ مَنٖنٗ  * عَقِلِ إِكَمْنْڠِيَ }} \\* 
 \OLTcl{ ikamngiya ʿaqili *  maneno hayo kapewa *  mfano hupijwa mambo *  mno senende ʿalii} \\* 
\SB{269} (\textbf{271}) \OLTst{'alii senende mno\footnote{\Swa{usizidi huzuni sana}, \E{do not wallow in sadness}.}  * mambo hupijwa mfano  * kapewa\footnote{The passive of \Swa{-pa}, \E{give} is \Swa{powa} in Amu, \Swa{-pawa} in Mvita, and \Swa{-pewa} in Zanzibar.} hayo maneno  * 'aqili ikamngiya\footnote{He realised the truth -- to be sorrowful is a mistake, as the Prophet has said.  This sort of bereavement has always happened -- it is the same for everyone, and you cannot help it.  The Swahili practice is to console people by saying things like this -- if the bereaved family thought that they were the only ones to whom this was happening, they would become very distraught.} } \\ 
\E{Ali, don't go on about this too much --    things have turned out like this.   And when he was given this advice   [Ali] regained his senses.  } \\ 
\\[8mm] 

\textarabic{(٢٧٢) \textcolor{mygreen}{هَپٗ أَكِشَ كُتٗوَ  * زُبَيْرِ كَمْوَمْكُوَ  * تَكُپَ زَنْڠُ بَرُوَ  * أُپَٹٖ كُنِپٖكٖيَ }} \\* 
 \OLTcl{ kunipekeya upaţe *  baruwa zangu takupa *  kamwamkuwa zubayri *  kutowa akisha hapo} \\* 
\SB{270} (\textbf{272}) \OLTst{hapo akisha kutowa\footnote{Fix kutuwa}  * zubayri kamwamkuwa  * takupa zangu baruwa  * upate kunipekeya } \\ 
\E{So when he had calmed down   He summoned Zubayr [and said:]  I will give you my letters   so that you may deliver them for me.  } \\ 
\\[8mm] 

\textarabic{(٢٧٣) \textcolor{mygreen}{هِيْ مٗيَ نْدَ مَمَكٖ  * مٗيَ نْدَ مْوَلِمُ وَكٖ  * نَ أُچٖنْدَ سِتَمْكٖ  * نٖنٗ مٗيَ كُوَمْبِيَ }} \\* 
 \OLTcl{ kuwambiya moya neno *  sitamke uchenda na *  wake mwalimu nda moya *  mamake nda moya hii} \\* 
\SB{271} (\textbf{273}) \OLTst{hii moya nda mamake  * moya nda mwalimu wake  * na uchenda sitamke  * neno moya kuwambiya } \\ 
\E{This one is for his mother    and this one for his teacher,    and when you go there do not say   one word to tell them [what has happened].   } \\ 
\\[8mm] 

\textarabic{(٢٧٤) \textcolor{mygreen}{زُبَيْرِ أَسِجِلِسِ  * كَپٗكٖيَ كَرَتَسِ  * كَنٖنْدَ نَزٗ أُپٖسِ  * كَمْپٖكٖيَ عَطِيَ }} \\* 
 \OLTcl{ ʿaṭiya kampekeya *  upesi nazo kanenda *  karatasi kapokeya *  asijilisi zubayri} \\* 
\SB{272} (\textbf{274}) \OLTst{zubayri asijilisi\footnote{i.e. Zubayr did not sit and wait.}  * kapokeya karatasi  * kanenda nazo upesi  * kampekeya 'atiya\footnote{\q{Atika} is changed to \q{Atiya} at the end of the line for the sake of the rhyme.  See also 278d, 293d, and 304d} } \\ 
\E{Zubayr did not delay --  he took the papers  and went quickly with them.   He delivered one to Atiya [Ja'far's mother].  } \\ 
\\[8mm] 

\textarabic{(٢٧٥) \textcolor{mygreen}{بَرُوَ كُمْپَ كْوَكٖ  * كِشَ زُبٖىْرِ أَتٗكٖ  * إِلٖ يَ مْوَلِمُ وَكٖ  * كٖنٖنْدَ كُمْپٖكٖيَ }} \\* 
 \OLTcl{ kumpekeya kenenda *  wake mwalimu ya ile *  atoke zubēri kisha *  kwake kumpa baruwa} \\* 
\SB{273} (\textbf{275}) \OLTst{baruwa kumpa kwake  * kisha zuberi atoke  * ile ya mwalimu wake  * kenenda kumpekeya } \\ 
\E{When he had given her the letter   then Zubayr left,   and the one for [Ja'far's] teacher    he went on to deliver [it] to him.  } \\ 
\\[8mm] 

\textarabic{(٢٧٦) \textcolor{mygreen}{زُبَىْرِ كُتٗكَ كْوَكٖ  * يٖيٖ أَسٗمٖ مَمَكٖ  * أَكَتٗكَ پْوٖكٖ يَكٖ  * مْٹُ أَسِپٗزٖنْڠٖيَ }} \\* 
 \OLTcl{ asipozengeya mţu *  yake pweke akatoka *  mamake asome yeye *  kwake kutoka zubayri} \\* 
\SB{274} (\textbf{276}) \OLTst{zubayri kutoka kwake\footnote{i.e. immediately he left.}  * yeye asome mamake  * akatoka pweke yake\footnote{Usually if a mother is going somewhere and she has a small child she will take the child with her, but in this case Atika is  so distraught that she rushes out immediately, forgetting about Nasir.}  * mtu asipozengeya } \\ 
\E{When Zubayr had left   [ja'far's] mother read [the letter],   and she left home on her own   without telling anyone.  } \\ 
\\[8mm] 

\textarabic{(٢٧٧) \textcolor{mygreen}{نَاءٖ أَكِيَنُظُمُ  * يُوَ لِمٗ هُسَلِمُ  * وَلَ أَسِپٗفَهَمُ  * أُسِكُ هُمْنْڠِلِيَ }} \\* 
 \OLTcl{ humngiliya usiku *  asipofahamu wala *  husalimu limo yuwa *  akiyanuẓumu nae} \\* 
\SB{275} (\textbf{277}) \OLTst{nae akiyanudhumu  * yuwa limo husalimu\footnote{It is dangerous for a woman to be out alone at night, but she is grief-stricken.}  * wala asipofahamu  * usiku humngiliya } \\ 
\E{And as she repeated [the contents]  the sun was going down,   but she did not realise  that night was drawing on.  } \\ 
\\[8mm] 

\textarabic{(٢٧٨) \textcolor{mygreen}{نَ هُكٗ نْيُمَ زُبٖيْرِ  * أَسِپَٹٖ تَقْصِيْرِ  * كِلَ نْيُمَ كِعَبِرِ  * كِمْزٖنْڠٖيَ عَطِيَ }} \\* 
 \OLTcl{ ʿaṭiya kimzengeya *  kiʿabiri nyuma kila *  taqṣı̄ri asipaţe *  zubēri nyuma huko na} \\* 
\SB{276} (\textbf{278}) \OLTst{na huko nyuma zuberi  * \dotuline{asifanye} taqsiri\footnote{\Swa{-fanya taksiri}, \E{put in the effort, do the needful}.  Note that the Swahili negative here corresponds to a positive in English.}  * kila \dotuline{nyumba} ki'abiri\footnote{i.e. going to every house and calling \Swa{Hodi!}.  Zubeir is trying to find the \Swa{mwalimu}'s house, to deliver his second letter (273b, 275).  He could not simply ask Atika where it was, because he was told by Ali not to speak to the recipients (273c/d).}  * kimzengeya\footnote{\Swa{anamtafuta}.} 'atiya } \\ 
\E{And meanwhile Zubayr    was doing his best [to find the teacher's house],  calling at every house   while Atiya was looking for him.  } \\ 
\\[8mm] 

\textarabic{(٢٧٩) \textcolor{mygreen}{هَتَ نْدِيَ كِفُوَٹَ  * كِنٖنْدَ كِتٗمْكُٹَ  * نْيُمْبَ يَ كْوَنْدَ كِپَٹَ  * مْلَنْڠٗنِ أَكِنْڠِيَ }} \\* 
 \OLTcl{ akingiya mlangoni *  kipaţa kwanda ya nyumba *  kitomkuţa kinenda *  kifuwaţa ndiya hata} \\* 
\SB{277} (\textbf{279}) \OLTst{hata ndiya kifuwata  * \dotuline{kenenda} kitomkuta  * nyumba ya kwanda kipata  * mlangoni\footnote{In other words, \Swa{alikwenda usiku kucha}, \E{she travelled all night}, and arrived at Mecca, where she immediately makes for the first house in the village -- this just happens to be one where Hamza and Umar are present.  Note that in her distress Atika does not even ask permission to enter (\Swa{Hodi!}).} akingiya } \\ 
\E{So as she followed the road   she went on without finding him.  When she reached the first house    she went in the door.  } \\ 
\\[8mm] 

\textarabic{(٢٨٠) \textcolor{mygreen}{حَمْزَة أَكَفَسِرِ  * نَ مَوْلَانَا عُمَرِ  * هِنِ نِ أَلْفَجِرِ  * صَلَ إِمٖسِمَمِيَ }} \\* 
 \OLTcl{ imesimamiya ṣala *  alfajiri ni hini *  ʿumari mawlānā na *  akafasiri ḥamzaẗ} \\* 
\SB{278} (\textbf{280}) \OLTst{hamzat akafasiri  * na maulana 'umari  * hini ni alfajiri\footnote{\Swa{alfajiri}, \E{dawn}.  In other words, it was time for morning prayers.}  * sala imesimamiya\footnote{\Swa{yali tayari sala}.} } \\ 
\E{Hamza was talking  with Lord Umar [in the house].   It was dawn   and prayers were about to begin.  } \\ 
\\[8mm] 

\textarabic{(٢٨١) \textcolor{mygreen}{مْوَنَمْكٖ كَتَمْكَ  * كَنٖنَ نْدِمِ أَتْوِكَ  * چَمْبَ مْتَنِپِلٖكَ  * كْوَ مَوْلَانَا عَلِيَ }} \\* 
 \OLTcl{ ʿaliya mawlānā kwa *  mtanipileka chamba *  atwika ndimi kanena *  katamka mwanamke} \\* 
\SB{279} (\textbf{281}) \OLTst{mwanamke katamka  * kanena ndimi atwika  * chamba \dotuline{mtanipeleka}  * kwa maulana 'aliya } \\ 
\E{The woman spoke:  and said: I am Atiya.   perhaps you could show me  to Lord Ali's [house]?   } \\ 
\\[8mm] 

\textarabic{(٢٨٢) \textcolor{mygreen}{نْدِيَ نِمٖعَبِرِ  * نْيُمْبَ سِكُئِفَسِرِ  * نْدِمِ أُمِ جَعْفَرِ  * كَمَ هَيَ يَوٖلٖيَ }} \\* 
 \OLTcl{ yaweleya haya kama *  jaʿfari umi ndimi *  sikuifasiri nyumba *  nimeʿabiri ndiya} \\* 
\SB{280} (\textbf{282}) \OLTst{ndiya nime'abiri  * nyumba sikuifasiri\footnote{\Swa{-fasiri} = \Swa{-jua}.}  * ndimi umi ja'fari  * kama haya yaweleya } \\ 
\E{I have come along the road  and I don't know the house.  I am Ja'far's mother   if that clarifies things for you.   } \\ 
\\[8mm] 

\textarabic{(٢٨٣) \textcolor{mygreen}{هَپٗ حَمْزَ كَتٗكَ  * كْوَ أُپٖسِ نَ هَرَكَ  * مْلَنْڠٗ أَكَؤُشِكَ  * إِلِ كُمْفُنْڠُلِيَ }} \\* 
 \OLTcl{ kumfunguliya ili *  akaushika mlango *  haraka na upesi kwa *  katoka ḥamza hapo} \\* 
\SB{281} (\textbf{283}) \OLTst{hapo hamza katoka  * kwa upesi na haraka  * mlango akaushika  * ili kumfunguliya\footnote{When someone asks you directions, the Swahili consider it polite to accompany them to their destination, call the person they are looking for, and hand over the visitor to them: \Swa{nakuletea mgeni wako}, \E{I'm bringing your visitor to you}.} } \\ 
\E{Then Hamza went out   quickly and speedily    and took hold of the door  to open it for her.  } \\ 
\\[8mm] 

\textarabic{(٢٨٤) \textcolor{mygreen}{نْدٖ كِتٗكَ كَٹِكَ هٖمَ  * هَپٗ نْدِيَ هَيَنْدَمَ  * كْوَ عَلِيْ أَكِكٗمَ  * مْلَنْڠٗ كِمْبِشِيَ }} \\* 
 \OLTcl{ kimbishiya mlango *  akikoma ʿalii kwa *  hayandama ndiya hapo *  hema kaţika kitoka nde} \\* 
\SB{282} (\textbf{284}) \OLTst{nde kitoka katika hema  * hapo ndiya \dotuline{kayandama}  * kwa 'alii akikoma\footnote{\Swa{-koma} here means \q{end up at}.  Compare \Swa{ndia hii imekoma wapi?}, \E{where does this road go to?}}  * mlango kimbishiya } \\ 
\E{Then, leaving the tent,    he set out on the way.   When he finally came to Ali's [house]   he knocked on the door.  } \\ 
\\[8mm] 

\textarabic{(٢٨٥) \textcolor{mygreen}{كِنٖنَ أَكِمَلِزَ  * عَلِيْ نْدِمِ حَمْزَ  * مْڠٖنِ هُكُؤُلِزَ  * هَپٗ كَتٗكَ عَلِيَ }} \\* 
 \OLTcl{ ʿaliya katoka hapo *  hukuuliza mgeni *  ḥamza ndimi ʿalii *  akimaliza kinena} \\* 
\SB{283} (\textbf{285}) \OLTst{kinena \dotuline{akimweleza}  * 'alii ndimi hamza  * mgeni\footnote{i.e. \Swa{kuna mgeni wako hapa}.} hukuuliza  * hapo katoka 'aliya } \\ 
\E{And he said, explaining [things] to him:  Ali, it's me,  Hamza.   A visitor is asking for you.  Then Ali went out.   } \\ 
\\[8mm] 

\textarabic{(٢٨٦) \textcolor{mygreen}{يٖيٖ يُپٗ مْلَنْڠٗنِ  * هُلِيَ أَكِبَئِنِ  * قَبُرِنِ نِپٖكَنِ  * نَپٖنْدَ كُيَنْڠَلِيَ }} \\* 
 \OLTcl{ kuyangaliya napenda *  nipekani qaburini *  akibaini huliya *  mlangoni yupo yeye} \\* 
\SB{284} (\textbf{286}) \OLTst{yeye yupo mlangoni  * huliya akibaini  * qaburini nipekani  * napenda kuyangaliya } \\ 
\E{[Atiya] was at the door,   weeping and saying:  Take me to his grave --   I want to see it.  } \\ 
\\[8mm] 

\textarabic{(٢٨٧) \textcolor{mygreen}{هَپٗ عَلِيْ كَتٗكَ  * مَتٗزِ يَكِمْشُكَ  * هَتَ نْدٖ كَتَمْكَ  * مَنٖنٗ أَكَمْوَمْبِيَ }} \\* 
 \OLTcl{ akamwambiya maneno *  katamka nde hata *  yakimshuka matozi *  katoka ʿalii hapo} \\* 
\SB{285} (\textbf{287}) \OLTst{hapo 'alii katoka  * matozi yakimshuka  * hata nde katamka  * maneno akamwambiya } \\ 
\E{Then Ali went out,   his tears flowing,  and outside he spoke,   telling her these words.  } \\ 
\\[8mm] 

\textarabic{(٢٨٨) \textcolor{mygreen}{كَمْوَمْبِيَ كِنُظُمُ  * سٖنْدٖلٖيْ إِسِلَامُ  * صُبِرِ كْوَكٖ كَرِيْمُ  * أَمْبَيٗ أَكُلٖٹٖيَ }} \\* 
 \OLTcl{ akuleţeya ambayo *  karı̄mu kwake ṣubiri *  isilāmu sendelee *  kinuẓumu kamwambiya} \\* 
\SB{286} (\textbf{288}) \OLTst{kamwambiya kinudhumu  * sendelee\footnote{i.e. \Swa{usifanye sana}.  If you get carried away by grief, you may say something that is \Swa{kufru}, i.e. something an unbeliever might say.  So a wife at the death of her husband may say that he was her lion, or her pillar in the world, or tht she depended on him, and she will be told: don't say that, or you will become a \Swa{kafiri}.  On the contrary, you have to be loyal to God even in a time of grief, and endure whatever he sends you.  Debate on the \q{problem of evil} (why does a good God allow bad things to happen) is unknown in Islam -- God knows best, and we cannot begin to fathom His motives.} isilamu  * subiri\footnote{Compare 269b.  } kwake karimu  * ambayo akuleteya } \\ 
\E{He spoke, saying:  don't go on so -- submit to God's will,  trust in Providence   who has brought you here.  } \\ 
\\[8mm] 

\textarabic{(٢٨٩) \textcolor{mygreen}{مِمِ هُىُ نِ مَمَكٖ  * نْدِيٖ مْوٖنْي كِٹِ چَكٖ  * نِؤٗنْيَ قَبُرِ يَكٖ  * نَتَكَ كُيَنْڠَلِيَ }} \\* 
 \OLTcl{ kuyangaliya nataka *  yake qaburi nionya *  chake kiţi mwenı̄ ndiye *  mamake ni huyu mimi} \\* 
\SB{287} (\textbf{289}) \OLTst{mimi huyu ni mamake  * ndiye mweni \dotuline{kite}\footnote{\Swa{kite}, \E{birth pangs}.  These give a mother a special love (\Swa{huruma}) for her child -- she will willingly sacrifice herself for the child.  We see this even in animals.} chake  * nionya qaburi yake  * nataka kuyangaliya } \\ 
\E{[Atiya said:] I am his mother!    I bore his birthpangs!    Show me his grave --   I want to see it.  } \\ 
\\[8mm] 

\textarabic{(٢٩٠) \textcolor{mygreen}{يٗوَ مْوٖنْيٖ مَمْلَكَه  * لَكٖ هٗنْدٗوَ كِوٖكَ  * وَلَ مِمِ سِكُتَكَ  * أَئِفَرِكِ دُنِيَ }} \\* 
 \OLTcl{ duniya aifariki *  sikutaka mimi wala *  kiweka hondowa lake *  mamlakah mwenye yowa} \\* 
\SB{288} (\textbf{290}) \OLTst{\dotuline{yuwa} mwenye mamlakah  * lake hondowa\footnote{We understand \Swa{watu}, \E{human beings}.  } kiweka\footnote{God has the power to do whatever he likes -- he sustains people or brings their life to an end, and we are not in a position to understand his motives.}  * wala mimi sikutaka  * aifariki duniya } \\ 
\E{[Ali said:] Know that the Almighty   [his way] is to take people away and bring [them into existence],   and I did not want    [ja'far] to pass away.  } \\ 
\\[8mm] 

\textarabic{(٢٩١) \textcolor{mygreen}{نَوٖ صُبِرِ نْدُيَنْڠُ  * أُسِپٗتٖيْ كْوَ مْنْڠُ  * وَلَ هُتٗكٗسَ فُنْڠُ  * لَ أَخٖرَ نَ دُنِيَ }} \\* 
 \OLTcl{ duniya na akhera la *  fungu hutokosa wala *  mngu kwa usipotee *  nduyangu ṣubiri nawe} \\* 
\SB{289} (\textbf{291}) \OLTst{nawe subiri nduyangu  * usipotee kwa mngu  * wala \dotuline{hutakosa}\footnote{If you \Swa{subiri}, you will receive a reward from God: \Swa{hutakosa maneno kwa Mungu}, \E{you will not fail [to receive] comfort from God}.} fungu  * la ahera\footnote{It is said: \Swa{ukisema mambo mabaya, utapata madhambi kwa Mungu; ukisubiri, utapata malipo mazuri}, \E{if you say irreligious things, God will judge you as having sinned; on the other hand, if you trust [in him], you will be well-rewarded}.} na duniya } \\ 
\E{And have trust [in God], my dear,   so that you do not go astray from God's [path].   or you will not receive your share   in the next world and this one.    } \\ 
\\[8mm] 

\textarabic{(٢٩٢) \textcolor{mygreen}{هُمُؤٗوَ كِپُلِكَ  * مَتٗزِ يَكِمْشُكَ  * هَپٗ كَشُكَ عَطِكَ  * أَكَتَمْكَ عَلِيَ }} \\* 
 \OLTcl{ ʿaliya akatamka *  ʿaṭika kashuka hapo *  yakimshuka matozi *  kipulika humuowa} \\* 
\SB{290} (\textbf{292}) \OLTst{humuowa\footnote{\Swa{anamtizama}.} kipulika  * matozi yakimshuka  * hapo \dotuline{kachoka} 'atika  * akatamka 'aliya } \\ 
\E{[Atiya] looked at him, listening,  tears falling.  Then Atiya stopped [crying]   and Ali spoke.  } \\ 
\\[8mm] 

\textarabic{(٢٩٣) \textcolor{mygreen}{هَپٗ عَلِيْ حَيْدَرِ  * بَسِ نَاءٖ كَفَسِرِ  * مْبٗنَ هَكُيَ زُبٖيْرِ  * أُيِيٖ پْوٖكٖ عَلِيَ }} \\* 
 \OLTcl{ ʿaliya pweke uyiye *  zubēri hakuya mbona *  kafasiri nae basi *  ḥaydari ʿalii hapo} \\* 
\SB{291} (\textbf{293}) \OLTst{hapo 'alii haydari  * basi nae kafasiri  * mbona hakuya zuberi  * uyiye pweke \dotuline{Atiya} } \\ 
\E{So Ali the Lion-like   spoke to her then:   Why did Zubayr not come [with you]?   Did you come by yourself, Atiya?   } \\ 
\\[8mm] 

\textarabic{(٢٩٤) \textcolor{mygreen}{أُنِئٖٹٖيْ بَرُوَ  * كَٹِكَ كُئِفُنْڠُوَ  * أَلِپٗ سِكُمُيُوَ  * وَلَ سِكُمْزٖنْڠٖيَ }} \\* 
 \OLTcl{ sikumzengeya wala *  sikumuyuwa alipo *  kuifunguwa kaţika *  baruwa unieţee} \\* 
\SB{292} (\textbf{294}) \OLTst{unietee baruwa  * katika kuifunguwa  * alipo sikumuyuwa  * wala sikumzengeya } \\ 
\E{[Atiya said:] He brought me a letter,  and when I opened it  I took no heed of where he was,  and I did not look for him.  } \\ 
\\[8mm] 

\textarabic{(٢٩٥) \textcolor{mygreen}{بَرُوَ كِئِفَسِرِ  * نْدِيَ نَلِئِعَبِرِ  * هَنْدَ سَسَ كُفِكِرِ  * تَنَبُهِ كُنِنْڠِيَ }} \\* 
 \OLTcl{ kuningiya tanabuhi *  kufikiri sasa handa *  naliiʿabiri ndiya *  kiifasiri baruwa} \\* 
\SB{293} (\textbf{295}) \OLTst{baruwa kiifasiri  * ndiya \dotuline{nali'abiri}  * handa\footnote{Amu for \Swa{naanza}.} sasa kufikiri  * tanabuhi\footnote{= \Swa{hatari}.} kuningiya } \\ 
\E{When I realised what was in the letter  I set out on the road --   I am beginning now to realise   I put myself in danger.  } \\ 
\\[8mm] 

\textarabic{(٢٩٦) \textcolor{mygreen}{فَطِمَ كَمْكَلِمُ  * أَكَمْپَ مَجِ تَمُ  * دُعَ سَبَا تِمَمُ  * نْدَنِ أَلِيَسٗمٖيَ }} \\* 
 \OLTcl{ aliyasomeya ndani *  timamu sabā duʿa *  tamu maji akampa *  kamkalimu faṭima} \\* 
\SB{294} (\textbf{296}) \OLTst{fatima kamkalimu  * akampa maji tamu  * du'a saba timamu  * ndani aliyasomeya\footnote{This is anachronistic, in that this would not have been done at the time the story is supposed to take place.  The reference is to the practice of reading the Qur'an and then breathing into the water -- the efficacy of the verses is piously considered to transfer into the water.} } \\ 
\E{Fatima spoke to her,  and gave her sweet water --   seven whole prayers   she had read into it.  } \\ 
\\[8mm] 

\textarabic{(٢٩٧) \textcolor{mygreen}{أَكَمْجِبُ كَلِمَ  * سِيَوٖزِ يَ فَتُمَ  * مٗيٗ نِنَ هَلِمَمَ  * رُوْحُ يَتَكَ كُلِيَ }} \\* 
 \OLTcl{ kuliya yataka rūḥu *  halimama nina moyo *  fatuma ya siyawezi *  kalima akamjibu} \\* 
\SB{295} (\textbf{297}) \OLTst{akamjibu kalima  * siyawezi\footnote{She means \Swa{chakula hakinishuki}, \E{I have no heart for eating}.  If someone dies, he is buried the next day, and the women keen and lament all day and night.  They may not eat at all until after the burial.  People who are not close relatives of the deceased may make food and bring it secretly, encouraging the bereaved to eat, as if they were sick.} ya fatuma  * moyo nina halimama  * ruhu\footnote{\Swa{ruhu} = \Swa{roho}.} yataka kuliya } \\ 
\E{[Atiya] answered her with the words:  I cannot [take it], oh Fatima --   my heart is in confusion,   and my soul wants to cry out.   } \\ 
\\[8mm] 

\textarabic{(٢٩٨) \textcolor{mygreen}{صُبِرِ كْوَ بْوَنَ وٖٹُ  * عَطِكَ مَاءِ سِ كِٹُ  * پِجَ مَٹَمَ مَٹَٹُ  * يَبَكِيٖؤٗ تَٹُوَ }} \\* 
 \OLTcl{ taţuwa yabakiyeo *  maţaţu maţama pija *  kiţu si mai ʿaṭika *  weţu bwana kwa ṣubiri} \\* 
\SB{296} (\textbf{298}) \OLTst{subiri kwa bwana wetu\footnote{i.e. \Swa{Mungu}.}  * 'atika mai si kitu\footnote{i.e. it is not food, so if she is fasting because of the bereavement it is reasonable to take it.}  * pija matama\footnote{\Swa{-piga tama} or \Swa{-shika tama}, \E{take a drink, fill your mouth with liquid}.} matatu  * yabakiyeo \dotuline{tatwaa} } \\ 
\E{[Fatima replied:] Trust in our Lord,    Atiya, the water is not something [to eat] --    take three sips,   and whatever is left I will take.  } \\ 
\\[8mm] 

\textarabic{(٢٩٩) \textcolor{mygreen}{مَاءِ أَسِپٗيَتَكَ  * مِيٗمٗنِ كَيَپٖكَ  * أَكَٹُمْوَ نَ عَطِكَ  * فَتُمَ كَمْپٗكٖيَ }} \\* 
 \OLTcl{ kampokeya fatuma *  ʿaṭika na akaţumwa *  kayapeka miyomoni *  asipoyataka mai} \\* 
\SB{297} (\textbf{299}) \OLTst{mai asipoyataka  * miyomoni kayapeka  * akatumwa na 'atika  * fatuma kampokeya } \\ 
\E{Although [Atiya] did not really want the water,  she took some into her mouth.  [The cup] was given back by Atiya,   and Fatima took it. (?)  } \\ 
\\[8mm] 

\textarabic{(٣٠٠) \textcolor{mygreen}{هَپٗ كَمْكَلِفِشَ  * وَعَظِ كُمُؤٗنٖشَ  * مَاءِ أَلِپٗيَشُشَ  * مُنْڠُ أَكَمُؤٗمْبٖيَ }} \\* 
 \OLTcl{ akamuombeya mungu *  alipoyashusha mai *  kumuonesha waʿaẓi *  kamkalifisha hapo} \\* 
\SB{298} (\textbf{300}) \OLTst{hapo kamkalifisha\footnote{\Swa{-kalifisha} =  \Swa{-lazimisha}.}  * wa'adhi kumuonesha  * mai alipoyashusha  * mungu akamuombeya } \\ 
\E{So Fatima persuaded her  and showed her [what to do] by exhortation.  When [Atiya] had swallowed the water  [Fatima] interceded to God for her.  } \\ 
\\[8mm] 

\textarabic{(٣٠١) \textcolor{mygreen}{بَسِ هَپٗ أَمُؤُزٖ  * زَ مْوَنَوٖ خَبَرِزٖ  * نَ عَلِيْ أَمْوٖلٖزٖ  * كِشَ أَكِلِيَ }} \\* 
 \OLTcl{ akiliya kisha *  amweleze ʿalii na *  khabarize mwanawe za *  amuuze hapo basi} \\* 
\SB{299} (\textbf{301}) \OLTst{basi hapo amuuze  * za mwanawe habarize  * na 'alii amweleze  * kisha akiliya } \\ 
\E{Then [Atiya] asked   for news of her son,   and Ali explained [everything] to her   weeping at the end.  } \\ 
\\[8mm] 

\textarabic{(٣٠٢) \textcolor{mygreen}{هَتَ كُكِپَمْبَؤُكَ  * فَتُمَ أَكَئِنُكَ  * كَمْپِكِيَ عَطِكَ  * وَكَلَ وٗتٖ پَمٗيَ }} \\* 
 \OLTcl{ pamoya wote wakala *  ʿaṭika kampikiya *  akainuka fatuma *  kukipambauka hata} \\* 
\SB{300} (\textbf{302}) \OLTst{hata kukipambauka  * fatuma akainuka  * kampikiya 'atika  * wakala wote pamoya } \\ 
\E{Until, when dawn came,  Fatima got up  and cooked [food] for Atiya  and they all ate together.   } \\ 
\\[8mm] 

\textarabic{(٣٠٣) \textcolor{mygreen}{عَطِكَ أَكَبَئِنِ  * سَسَ نَمِ كْوَ خٖرِنِ  * مْوَنَنْڠُ أُكٗ مُئِنِ  * مْٹُ سِكُمُوَتِيَ }} \\* 
 \OLTcl{ sikumuwatiya mţu *  muini uko mwanangu *  kherini kwa nami sasa *  akabaini ʿaṭika} \\* 
\SB{301} (\textbf{303}) \OLTst{'atika akabaini  * sasa nami kwa herini  * mwanangu uko muini  * mtu sikumuwatiya } \\ 
\E{Atiya said:  Now I [bid you] farewell.    My son is [back] at home   and I left no-one with him.  } \\ 
\\[8mm] 

\textarabic{(٣٠٤) \textcolor{mygreen}{نَ فَتُمَ أَتَمْكٖ  * أَمْوَمْبِيٖ مُمٖ وَكٖ  * نَاوٖ نٖنْدَ كَمْپٖكٖ  * أَسٖنْدٖ پْوٖكٖ عَطِيَ }} \\* 
 \OLTcl{ ʿaṭiya pweke asende *  kampeke nenda nāwe *  wake mume amwambiye *  atamke fatuma na} \\* 
\SB{302} (\textbf{304}) \OLTst{na fatuma atamke  * amwambiye mume wake  * nawe nenda kampeke  * asende pweke 'atiya } \\ 
\E{And Fatima spoke   and said to her husband:   And you go and accompany her   so that Atiya [need] not go on her own.   } \\ 
\\[8mm] 

\textarabic{(٣٠٥) \textcolor{mygreen}{عَلِيْ كَنْڠِيَ نْدَنِ  * كْوَ مْكٗنٗ كَبَئِنِ  * أَكَئِنُكَ كِٹِنِ  * فَتُمَ أَكَمْوٖنْدٖيَ }} \\* 
 \OLTcl{ akamwendeya fatuma *  kiţini akainuka *  kabaini mkono kwa *  ndani kangiya ʿalii} \\* 
\SB{303} (\textbf{305}) \OLTst{'alii kangiya ndani  * kwa mkono kabaini\footnote{He does not want Atika to hear.}  * akainuka kitini  * fatuma akamwendeya } \\ 
\E{Ali went into the inner [room]   and signalled [Fatima] with his hand.   She rose from her chair,  Fatima, and went to him.  } \\ 
\\[8mm] 

\textarabic{(٣٠٦) \textcolor{mygreen}{كَمْبَ چٖنْدَ كِمْپٖكَ  * فَتُمَ هُتٗؤُذِكَ  * كْوَمْبَ وَٹُ هُتَمْكَ  * كَمَ هَيٗ كُنَمْبِيَ }} \\* 
 \OLTcl{ kunambiya hayo kama *  hutamka waţu kwamba *  hutoudhika fatuma *  kimpeka chenda kamba} \\* 
\SB{304} (\textbf{306}) \OLTst{kamba chenda kimpeka  * fatuma hutoudhika  * kwamba watu hutamka  * kama hayo kunambiya } \\ 
\E{He said: If I go and accompany her,   Fatima, will you not be angry  if people talk [about it]   and gossip about me?   } \\ 
\\[8mm] 

\textarabic{(٣٠٧) \textcolor{mygreen}{كَنٖنَ شَهِدِ مْنْڠُ  * هَيَمٗ مٗيٗنِ مْوَنڠُ  * سَسَ نِ كَمَ نْدُيَنْڠُ  * نَ قَاسِمُ نِ مَمٗيَ }} \\* 
 \OLTcl{ mamoya ni qāsimu na *  nduyangu kama ni sasa *  mwangu moyoni hayamo *  mngu shahidi kanena} \\* 
\SB{305} (\textbf{307}) \OLTst{kanena shahidi mngu  * hayamo moyoni mwangu\footnote{Fatima, in contrast to her behaviour at the beginning of the ballad, has learnt to be magnanimous.}  * sasa ni kama nduyangu  * na qasimu\footnote{Qasim was Fatima's brother, and died in infancy.  The Prophet had 7 children (3 boys and 4 girls), but they all pre-deceased him except Fatima.} ni \dotuline{mmoya} } \\ 
\E{She said: I swear to God,   [such things] are not in my heart.   [Atiya] is like a sister to me --     exactly the same as Qasim.    } \\ 
\\[8mm] 

\textarabic{(٣٠٨) \textcolor{mygreen}{هَپٗ عَلِيْ كَتٗكَ  * كَنْدَمَنَ نَ عَطِكَ  * أَكٖنْدَ أَكَمْپٖكَ  * هَتَ كْوَءٗ أَكَنْڠِيَ }} \\* 
 \OLTcl{ akangiya kwao hata *  akampeka akenda *  ʿaṭika na kandamana *  katoka ʿalii hapo} \\* 
\SB{306} (\textbf{308}) \OLTst{hapo 'alii katoka  * kandamana na 'atika\footnote{Compare 281b and 298b.}  * akenda akampeka  * hata kwao akangiya } \\ 
\E{So Ali went out   and went along with Atiya.   He went and accompanied her  until he reached her home.   } \\ 
\\[8mm] 

\textarabic{(٣٠٩) \textcolor{mygreen}{هَپٗ عَلِيْ حَيْدَرِ  * أَكَصَلِ أَظُهُرِ  * كَنْدَمَنَ نَ زُبٖيْرِ  * مُئِنِ أَكَرٖجٖيَ }} \\* 
 \OLTcl{ akarejeya muini *  zubēri na kandamana *  aẓuhuri akaṣali *  ḥaydari ʿalii hapo} \\* 
\SB{307} (\textbf{309}) \OLTst{hapo 'alii haydari  * akasali adhuhuri  * kandamana na zuberi  * muini akarejeya } \\ 
\E{Then Ali the Lion-like   said the midday prayers  and walked along with Zubayr   and returned to the town.  } \\ 
\\[8mm] 

\textarabic{(٣١٠) \textcolor{mygreen}{نَ هَپٗ أَلِپٗرُدِ  * نْدِپٗ أَلِپٗرَدِدِ  * مَنٖنٗ كِجِتَهِدِ  * مصَحَبَ كِوَمْبِيَ }} \\* 
 \OLTcl{ kiwambiya mṣaḥaba *  kijitahidi maneno *  aliporadidi ndipo *  aliporudi hapo na} \\* 
\SB{308} (\textbf{310}) \OLTst{na hapo aliporudi  * ndipo aliporadidi  * maneno kijitahidi  * msahaba kiwambiya } \\ 
\E{And when he returned,   it was then that he told [the story],  striving for [exact] words,  telling the Companions.  } \\ 
\\[8mm] 

\textarabic{(٣١١) \textcolor{mygreen}{هَپَ نِمٖٹِيَ تَمَ  * بَيْتِ زَنْڠُ هُكٗمَ  * نَ أَمْبَؤٗ وَتَسٗمَ  * كْوَ زٗتٖ زِكَوٖلٖيَ }} \\* 
 \OLTcl{ zikaweleya zote kwa *  watasoma ambao na *  hukoma zangu bayti *  tama nimeţiya hapa} \\* 
\SB{309} (\textbf{311}) \OLTst{hapa nimetiya tama  * bayti zangu hukoma  * na ambao watasoma  * kwa zote zikaweleya } \\ 
\E{Here I have finished,   my verses have come to an end,   and whoever reads [them]   will be made aware of everything [that happened].   } \\ 
\\[8mm] 

\textarabic{(٣١٢) \textcolor{mygreen}{بَيْتِزٖ زِيُوٖنِ  * مْٹُ أَكَزِبَئِنِ  * أَصُبُحِ نَ جِيٗنِ  * مٗيٗ أُسٗپُنْڠُلِيَ }} \\* 
 \OLTcl{ usopunguliya moyo *  jiyoni na aṣubuḥi *  akazibaini mţu *  ziyuweni baytize} \\* 
\SB{304} (\textbf{312}) \OLTst{baytize ziyuweni  * mtu akazibaini  * asubuhi na jiyoni  * \dotuline{moya} usopunguliya } \\ 
\E{Learn [the poem's] verses,  so that a person may say them  morning and evening.   not omitting one.  } \\ 
\\[8mm] 

\textarabic{(٣١٣) \textcolor{mygreen}{كْوَنْدَ هَتٗذَلِلِكَ  * مْٹُيٖ هَتٗسُمْبُكَ  * نَ أَتَكَلٗتَمْكَ  * مٗلَ هُمُوَفِقِيَ }} \\* 
 \OLTcl{ humuwafiqiya mola *  atakalotamka na *  hatosumbuka mţuye *  hatodhalilika kwanda} \\* 
\SB{305} (\textbf{313}) \OLTst{kwanda hatodhalilika\footnote{\Swa{-dhalilika}, \E{be humble, be humiliated, no agent specified}.}  * mtuye hatosumbuka  * na atakalotamka  * mola humuwafiqiya\footnote{These claims are somewhat overblown -- this is not a religious text.} } \\ 
\E{First, he will never be brought low,  that person, he will not be troubled,  and whatever he asks for  the Lord will bring to him.  } \\ 
\\[8mm] 

\textarabic{(٣١٤) \textcolor{mygreen}{أَؤٗمْبَلٗ كْوَ وَهَابُ  * أَتَجِبِوَ جَوَبُ  * أَوْ مْٹُ نَجَرِبُ  * أَدَلِلِشٖ وَصِيَ }} \\* 
 \OLTcl{ waṣiya adalilishe *  najaribu mţu aw *  jawabu atajibiwa *  wahābu kwa aombalo} \\* 
\SB{306} (\textbf{314}) \OLTst{aombalo kwa wahabu  * atajibiwa jawabu  * au mtu \dotuline{ajaribu}  * adalilishe wasiya\footnote{In other words, if the reader is doubtful that this is true, let him just try it.  See note to 157d.} } \\ 
\E{Whatever he prays for from the Generous One,   he will be vouchsafed an answer,  and let the person try [it],   that he may demonstrate its wisdom.  } \\ 
\\[8mm] 

\textarabic{(٣١٥) \textcolor{mygreen}{نِمٖپٖنْدَ كُكَرِرِ  * نَنْيِ سٗمَنِ ضَمِيْرِ  * أُتٖنْدِ وَ جَعْفَرِ  * وَ مَوْلَانَا عَلِيَ }} \\* 
 \OLTcl{ ʿaliya mawlānā wa *  jaʿfari wa utendi *  ḍamı̄ri somani nanyi *  kukariri nimependa} \\* 
\SB{307} (\textbf{315}) \OLTst{nimependa kukariri  * nanyi somani dhamiri  * utendi wa ja'fari  * wa maulana 'aliya } \\ 
\E{I have been pleased to recite it,  and you, read it inwardly --   the Ballad of Ja'far   and Lord Ali.   } \\ 
\\[8mm] 

\end{center} 

\renewcommand{\bibname}{References} 
\begingroup 
\printbibliography 
\endgroup 

\end{document}
