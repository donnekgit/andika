\documentclass[a4paper, 10pt]{report}

\usepackage{titlesec}  % Allow the chapter/section heading settings to be fine-tuned.  Needs to come before bidi, in polyglossia.
\usepackage{polyglossia}  % multilingual support
\usepackage{longtable}  % tables that carry across multiple pages
\usepackage[x11names]{xcolor}  % can't use color with polyglossia
% The following lines set up commands for superscript (SP) and subscript (SB), which can also be stacked (SPSB).
% These can be used to give the stanza number as appearing in the MS (which, because of human error, can be misnumbered).
% http://tex.stackexchange.com/questions/8255/superscript-and-subscript-together and http://tex.stackexchange.com/questions/146098/llap-or-rlap-at-the-begining-of-an-indented-paragraph
\def\SP#1{\textsuperscript{\E{#1}}}
\def\SB#1{\textsubscript{\E{#1}}}
\def\SPSB#1#2{\indent\rlap{\textsuperscript{\E{#1}}}\SB{#2}}

\usepackage{ulem}  % Allow dotted underlines.

\usepackage{pdfpages}

%--------------------------------
%%% Font definitions %%%
%--------------------------------

% Note that these definitions malfunction if used in \chapter{}.
\defaultfontfeatures{Mapping=tex-text}
\setmainfont{Charis SIL}  % Set the default font for the document. = \setdefaultfont
% Footnotes will by default also use this font -- http://tex.stackexchange.com/questions/4779/how-to-change-font-family-in-footnote).
\defaultfontfeatures{Scale=MatchLowercase}  % needs to be below main font declaration

\setsansfont{Liberation Sans}
\setmonofont{DejaVu Sans Mono}

\setmainlanguage{english}
\setotherlanguage{arabic}

\newfontfamily\arabicfont[Script=Arabic, Scale=2]{Scheherazade} % Arabic transcription -- coloured black, double size.
% One font needs to be called \arabicfont in order for XeTeX to load Arabic-related hyphenation and other stuff.
%  The default \textarabic will use this \arabicfont.  Use the \begin{Arabic} ..... \end{Arabic} environment for longer stretches (eg paras).
% Use \textarabic{\aemph{با}} to give overline emphasis.
% Omitting Script=Arabic for Amiri or Granada will mean that letters are written in their standalone forms, not connected.  (Omitting Script=Arabic for Scheherazade seems to cause no problem, though.)

\newfontfamily\citationfont[Script=Arabic, Scale=1.5]{Scheherazade}  % Citations, or stand-alone Arabic script in the middle of Roman script -- coloured black, one-and-a-half size.
\newcommand\AS[1]{{\citationfont\RLE{#1}}}
% \RLE (from the bidi package, which polyglossia loads automatically) is to allow multiple words of Arabic to be written right-to-left -- if omitted, each word in the sequence will be written RTL, but the sequence as a whole will be written LTR.

%You can either, as above, define a new \fontfamily, and then use it in a \newcommand, or you can, as below, include the font in the \newcommand by calling \fontspec directly.

\newcommand\Atitle[1]{{\fontspec[Script=Arabic, Scale=2]{GranadaKD}\RLE{#1}}}  % Arabic transcription for titles - uses a version of Granada which has been extended to include glyphs for Swahili.

\newcommand\Am[1]{{\fontspec[Script=Arabic]{Amiri}\RLE{#1}}} % Examples using Amiri --  if using Scheherazade's default scale, set Scale=0.8 here.

%\newfontfamily\translitfont[Scale=1, Color=666666]{Linux Biolinum O}
%\newcommand\Tr[1]{{\translitfont\RLE{#1}}}
\newcommand\Tr[1]{{\fontspec[Scale=1, Color=666666]{Linux Biolinum O}#1}}   %  Transliteration -- Biolinum handles diacritics well.  Coloured grey, slightly less than normal size.
% Scale=1 is required because of Scale=MatchLowercase - otherwise the size is too large.
\newcommand\Trb[1]{{\fontspec[Scale=1, Color=0000BB]{Linux Biolinum O}#1}} 

\newcommand\In[1]{{\fontspec[Scale=1, Color=blue]{Linux Biolinum O}#1}}  % Epenthetic letters in the transliteration -- coloured blue, normal size.

\newcommand\Swa[1]{{\fontspec[Color=00BB33, Scale=1]{Linux Biolinum O}#1}}  % Standard spelling -- coloured green, normal size.

\newcommand\E[1]{{\fontspec[Scale=0.9, Color=333333]{Liberation Serif Italic}#1}}  % English translation layer -- coloured grey, slightly less than normal size.

\newcommand\Eit[1]{{\fontspec{Liberation Serif Italic}#1}}  % English italics.

\newcommand\FN[1]{{\fontspec[Color=00BB33]{Liberation Serif Italic}#1}} % Standout type in footnotes -- coloured green, normal size.

% Older versions:
% \newfontfamily{\Tr}[Scale=0.9, Color=00BB33]{Linux Biolinum O}
% This can be used as \Tr{text}.  But this will change the font outside the argument until the end of that stretch.
% This doesn't show up in the poemlines, because they are self-contained, but it does show up in connected text.
% To avoid this, and have the font only changed within the argument, use \newcommand as above.
% Though you can also enclose \Tr in braces to limit it: {\Tr{}}

%----------------------------------------
%%% End of font definitions %%%
%----------------------------------------

%--------------------------------
%%% Colour definitions %%%
%--------------------------------

\definecolor{mygreen}{RGB}{0, 187, 50}

%----------------------------------------
%%% End of font definitions %%%
%----------------------------------------

  % Bring in the font definitions.

\usepackage{marginnote}
\renewcommand*{\marginfont}{\color{red}\sffamily}

\usepackage{csquotes}
\newcommand{\q}[1]{\enquote{#1}}  % Mark quotes by using \q{text to be quoted}.
% \newcommand{\q}[1]{``#1''}  % Alternative when not using csquotes.

\usepackage[multiple]{footmisc}  % Adds commas between multiple footnotemarkers.

% \usepackage{natbib}
\usepackage[backend=biber, style=authoryear]{biblatex}
\addbibresource{bib/andika.bib}

\interfootnotelinepenalty=10000 % prevents the footnote from breaking across pages
% http://tex.stackexchange.com/questions/32208/footnote-runs-onto-second-page

% Thanks to Manas Tungare (http://manas.tungare.name/software/latex) for these settings.
\setlength{\paperwidth}{210mm}
\setlength{\paperheight}{297mm}

\setlength{\textwidth}{160mm}
\setlength{\textheight}{247mm}

\setlength{\evensidemargin}{1in}
\setlength{\oddsidemargin}{0in}
\setlength{\topmargin}{-0.5in}

\renewcommand\thefootnote{\textcolor{red}{\arabic{footnote}}}  % Alter the colour of the footnote markers - thanks to Gonzalo Medina (http://tex.stackexchange.com/questions/26693/change-the-color-of-footnote-marker-in-latex#26696).

\usepackage{url}  % Use urls in text and captions with sensible linewrap.  Can't use [obeyspaces] - this option clashes with biblatex.
% \urlstyle{rm}  % Set urls in roman.

\titleformat{\chapter}[display]{\normalfont\large}{\bfseries\chaptertitlename\ \thechapter}{10pt}{\large\itshape}[\vspace{2ex}\titlerule\vspace{2ex}]  % 10pt is the space between chapter and chapter name.  [display] sets the chapter and chapter name on separate lines.  The square brackets at the end draw a line under each chapter name, with 2ex gap from the name.
\titlespacing{\chapter}{0pt}{0pt}{10pt}  % First is indent from the side, second is length down from the top, third is gap between heading and text.

% \pagenumbering{gobble}  % Suppress page numbering.  This is best if you want to include the pdf in another LaTeX document -- otherwise you will get two sets of page numbering.

% ===== Endnotes =====
% Uncomment the following two lines to get endnotes instead of footnotes.
% Remember to uncomment the three lines in andika/db/output_pdf.php as well.
% \usepackage{endnotes}
% \let\footnote=\endnote
% ==================

\begin{document}

\begin{center}
\Atitle{أُتٖنْزِ وَ جَعْفَر} \\
\Tr{Utenzi wa Ja'far -- excerpt} \\
% \E{The Ballad of Ja'far} \\
[5mm]
% \textcolor{red}{\AS{بِسْمِ اللّٰهِ الرَحمَنِ الرَّحِيْمِ}} \\
% \Tr{bismillähi ar-rahmani ar-rahīmi} \\
% \E{In the name of God, the Compassionate, the Merciful}
\end{center}




\begin{center} 

\textarabic{(٢٠٠) \textcolor{mygreen}{نَمِ كِپَٹَ پَنْڠٗنِ  * پَنَ مْٹٖنْدٖ نْدِيَنِ  * يَلِنِتٗكَ مٗيٗنِ  * يَلٖ وَلٗنَمْبِيَ }} \\ 
(\textbf{200}) \OLTst{nami kipata pangoni  * pana mtende ndiani  * yalinitoka moyoni  * yale walonambiya } \\ 
[5mm] 

\textarabic{(٢٠١) \textcolor{mygreen}{كَئِوَتَ يَ كُڤُلِ  * كَأَنْدَمَ إِلٗ مْبَلِ  * هَتَ كِتَأَمَلِ  * سَاءَ إِمٖنِپِٹِيَ }} \\ 
(\textbf{201}) \OLTst{kaiwata ya kuvuli  * kaandama ilo mbali  * hata kitaamali  * saa imenipitiya } \\ 
[5mm] 

\textarabic{(٢٠٢) \textcolor{mygreen}{كِشَ أُوِنْڠَ كَئٖٹَ  * إِيُ لَ بَرَ كَپِٹَ  * إِلِ نْدِيَ كُئِوَتَ  * نْيُمَ نِسِپٗرٖجٖيَ }} \\ 
(\textbf{202}) \OLTst{kisha uwinga kaeta  * iyu la bara kapita  * ile ndia kuiwata  * nyuma nisiporejeya } \\ 
[5mm] 

\textarabic{(٢٠٣) \textcolor{mygreen}{سُرَ نٖنْدَءٗ بَرَنِ  * إِلٖ نْدِيَ سِئِيٗنِ  * هُؤٗنَ نِكٗ بَرَنِ  * زٗتٖ زِمٖنِپٗتٖيَ }} \\ 
(\textbf{203}) \OLTst{sura nendao barani  * ile ndia siioni  * huona niko barani  * zote zimenipoteya } \\ 
[5mm] 

\textarabic{(٢٠٤) \textcolor{mygreen}{كِپِجَ فِكِرَ زَنْڠُ  * كَلَنْدَمَ ڠُوْ لَنْڠُ  * نَرُدِيَ پَلٖ پَنْڠُ  * كِشَ نْيُمَ كَرٖجٖيَ }} \\ 
(\textbf{204}) \OLTst{kipija fikira zangu  * kalandama guu langu  * narudia pale pangu  * kisha nyuma karejeya } \\ 
[5mm] 

\textarabic{(٢٠٥) \textcolor{mygreen}{كِشَ كَرُدِيَ نْيُمَ  * هَپٗ نْدِيَ كَيَنْدَمَ  * پٖنْيٖ مْٹٖنْدٖ كَكٗمَ  * صَالَ إِمٖنِسِمَمِيَ }} \\ 
(\textbf{205}) \OLTst{kisha karudia nyuma  * hapo ndia kayandama  * penye mtende kakoma  * sala imenisimamiya } \\ 
[5mm] 

\textarabic{(٢٠٦) \textcolor{mygreen}{أَوَلِ يَ أَظُهُرِ  * نْدِپٗ نْدِيَ كَعَبِرِ  * حُجَ يَ كُجَ أَخِيْرِ  * مَعَانَ نِمٖكْوَمْبِيَ }} \\ 
(\textbf{206}) \OLTst{awali ya adhuhuri  * ndipo ndia kaabiri  * huja ya kuja ahiri  * maana nimekwambiya } \\ 
[5mm] 

\textarabic{(٢٠٧) \textcolor{mygreen}{كِمَلِزَ كُپُلِكَ  * عَلِيْ أَكَتَمْكَ  * مْوَنَنْڠُ أُمٖسُمْبُكَ  * هَپٗ كَنٖنَ نَبِيَ }} \\ 
(\textbf{207}) \OLTst{kimaliza kupulika  * Aliyi akatamka  * mwanangu umesumbuka  * hapo kanena Nabiya } \\ 
[5mm] 

\textarabic{(٢٠٨) \textcolor{mygreen}{هَپٗ كَنٖنَ هَشِمَ  * سِ هَبَ كُيَ سَلَام  * نْدِيَ مٖزٗإِيَنْدَمَ  * خَطَرِ هُمْزٖنْڠٖيَ }} \\ 
(\textbf{208}) \OLTst{hapo kanena Hashima  * si haba kuya \dotuline{salama}  * ndia mezoiandama  * hatari humzengeya } \\ 
[5mm] 

\textarabic{(٢٠٩) \textcolor{mygreen}{أَمْكِنْڠَ وَدُوْدِ  * أَسِؤٗوْنٖ مَيَهُوْدِ  * كْوَنِ وَنْڠَلِمْزِدِ  * وَٹُ وَنْڠِ سِ مْمٗيَ }} \\ 
(\textbf{209}) \OLTst{\dotuline{amemkinga} Wadudi  * asione mayahudi  * kwani wangalimzidi  * watu wangi si mmoya } \\ 
[5mm] 

\textarabic{(٢١٠) \textcolor{mygreen}{فَتُمَ أُكٗ كِٹِنِ  * أَكَمْوٖپُكَ أَمِيْنِ  * كْوَ مْكٗنٗ كَبَئِنِ  * نَ نْدَنِ كَمُأَمْكُوَ }} \\ 
(\textbf{210}) \OLTst{Fatuma uko kitini  * akamwepuka Amini  * kwa mkono kabaini  * na ndani kamuamkuwa } \\ 
[5mm] 

\end{center} 

\renewcommand{\bibname}{References} 
\begingroup 
\printbibliography 
\endgroup 

\end{document}
