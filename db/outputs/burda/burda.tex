\documentclass[a4paper, 10pt]{report}

\usepackage{titlesec}  % Allow the chapter/section heading settings to be fine-tuned.  Needs to come before bidi, in polyglossia.
\usepackage{polyglossia}  % multilingual support
\usepackage{longtable}  % tables that carry across multiple pages
\usepackage[x11names]{xcolor}  % can't use color with polyglossia
% The following lines set up commands for superscript (SP) and subscript (SB), which can also be stacked (SPSB).
% These can be used to give the stanza number as appearing in the MS (which, because of human error, can be misnumbered).
% http://tex.stackexchange.com/questions/8255/superscript-and-subscript-together and http://tex.stackexchange.com/questions/146098/llap-or-rlap-at-the-begining-of-an-indented-paragraph
\def\SP#1{\textsuperscript{\E{#1}}}
\def\SB#1{\textsubscript{\E{#1}}}
\def\SPSB#1#2{\indent\rlap{\textsuperscript{\E{#1}}}\SB{#2}}

\usepackage{ulem}  % Allow dotted underlines.

\usepackage{pdfpages}

%--------------------------------
%%% Font definitions %%%
%--------------------------------
% Note that these definitions malfunction if used in \chapter.
\defaultfontfeatures{Mapping=tex-text}
\setmainfont{Charis SIL}  % Set the default font for the document. = \setdefaultfont
% Footnotes will by default also use this font -- http://tex.stackexchange.com/questions/4779/how-to-change-font-family-in-footnote).
\defaultfontfeatures{Scale=MatchLowercase}  % needs to be below main font declaration

\setsansfont{Liberation Sans}
\setmonofont{Liberation Mono}

\setmainlanguage{english}
\setotherlanguage{arabic}

\newfontfamily\arabicfont[Script=Arabic, Scale=2]{Scheherazade} % Arabic transcription -- coloured black, double size.
% One font needs to be called \arabicfont in order for XeTeX to load Arabic-related hyphenation and other stuff.
%  The default \textarabic will use this \arabicfont.  Use the \begin{Arabic} ..... \end{Arabic} environment for longer stretches (eg paras).
% Use \textarabic{\aemph{با}} to give overline emphasis.
% Omitting Script=Arabic for Amiri or Granada will mean that letters are written in their standalone forms, not connected.  (Omitting Script=Arabic for Scheherazade seems to cause no problem, though.)

\newfontfamily\citationfont[Script=Arabic, Scale=1.5]{Scheherazade}  % Citations, or stand-alone Arabic script in the middle of Roman script -- coloured black, one-and-a-half size.
\newcommand\AS[1]{{\citationfont\RLE{#1}}}
% \RLE (from the bidi package, which polyglossia loads automatically) is to allow multiple words of Arabic to be written right-to-left -- if omitted, each word in the sequence will be written RTL, but the sequence as a whole will be written LTR.

%You can either, as above, define a new \fontfamily, and then use it in a \newcommand, or you can, as below, include the font in the \newcommand by calling \fontspec directly.

\newcommand\Atitle[1]{{\fontspec[Script=Arabic, Scale=2]{GranadaKD}\RLE{#1}}}  % Arabic transcription for titles - uses a version of Granada which has been extended to include glyphs for Swahili.

\newcommand\Am[1]{{\fontspec[Script=Arabic]{Amiri}\RLE{#1}}} % Examples using Amiri --  if using Scheherazade's default scale, set Scale=0.8 here.

%\newfontfamily\translitfont[Scale=1, Color=666666]{Linux Biolinum O}
%\newcommand\Tr[1]{{\translitfont\RLE{#1}}}
\newcommand\Tr[1]{{\fontspec[Scale=1, Color=666666]{Linux Biolinum O}#1}}   %  Transliteration -- Biolinum handles diacritics well.  Coloured grey, slightly less than normal size.
% Scale=1 is required because of Scale=MatchLowercase - otherwise the size is too large.

\newcommand\In[1]{{\fontspec[Scale=1, Color=blue]{Linux Biolinum O}#1}}  % Epenthetic letters in the transliteration -- coloured blue, normal size.

\newcommand\Swa[1]{{\fontspec[Color=00BB33, Scale=1]{Linux Biolinum O}#1}}  % Standard spelling -- coloured green, normal size.

\newcommand\E[1]{{\fontspec[Scale=0.9, Color=333333]{Liberation Serif Italic}#1}}  % English translation layer -- coloured grey, slightly less than normal size.

\newcommand\Eit[1]{{\fontspec{Liberation Serif Italic}#1}}  % English italics.

\newcommand\FN[1]{{\fontspec[Color=00BB33]{Liberation Serif Italic}#1}} % Standout type in footnotes -- coloured green, normal size.

% Older versions:
% \newfontfamily{\Tr}[Scale=0.9, Color=00BB33]{Linux Biolinum O}
% This can be used as \Tr{text}.  But this will change the font outside the argument until the end of that stretch.
% This doesn't show up in the poemlines, because they are self-contained, but it does show up in connected text.
% To avoid this, and have the font only changed within the argument, use \newcommand as above.
% Though you can also enclose \Tr in braces to limit it: {\Tr{}}

%----------------------------------------
%%% End of font definitions %%%
%----------------------------------------
  % Bring in the font definitions.

\usepackage{marginnote}
\renewcommand*{\marginfont}{\color{red}\sffamily}

\usepackage{csquotes}
\newcommand{\q}[1]{\enquote{#1}}  % Mark quotes by using \q{text to be quoted}.
% \newcommand{\q}[1]{``#1''}  % Alternative when not using csquotes.

\usepackage[multiple]{footmisc}  % Adds commas between multiple footnotemarkers.

% \usepackage{natbib}
\usepackage[backend=biber, style=authoryear]{biblatex}
\addbibresource{bib/andika.bib}

\interfootnotelinepenalty=10000 % prevents the footnote from breaking across pages
% http://tex.stackexchange.com/questions/32208/footnote-runs-onto-second-page

% Thanks to Manas Tungare (http://manas.tungare.name/software/latex) for these settings.
\setlength{\paperwidth}{210mm}
\setlength{\paperheight}{297mm}

\setlength{\textwidth}{160mm}
\setlength{\textheight}{247mm}

\setlength{\evensidemargin}{1in}
\setlength{\oddsidemargin}{0in}
\setlength{\topmargin}{-0.5in}

\renewcommand\thefootnote{\textcolor{red}{\arabic{footnote}}}  % Alter the colour of the footnote markers - thanks to Gonzalo Medina (http://tex.stackexchange.com/questions/26693/change-the-color-of-footnote-marker-in-latex#26696).

\usepackage{url}  % Use urls in text and captions with sensible linewrap.  Can't use [obeyspaces] - this option clashes with biblatex.
% \urlstyle{rm}  % Set urls in roman.

\titleformat{\chapter}[display]{\normalfont\large}{\bfseries\chaptertitlename\ \thechapter}{10pt}{\large\itshape}[\vspace{2ex}\titlerule\vspace{2ex}]  % 10pt is the space between chapter and chapter name.  [display] sets the chapter and chapter name on separate lines.  The square brackets at the end draw a line under each chapter name, with 2ex gap from the name.
\titlespacing{\chapter}{0pt}{0pt}{10pt}  % First is indent from the side, second is length down from the top, third is gap between heading and text.

% \pagenumbering{gobble}  % Suppress page numbering.  This is best if you want to include the pdf in another LaTeX document -- otherwise you will get two sets of page numbering.

% ===== Endnotes =====
% Uncomment the following two lines to get endnotes instead of footnotes.
% Remember to uncomment the three lines in andika/db/output_pdf.php as well.
% \usepackage{endnotes}
% \let\footnote=\endnote
% ==================

\begin{document}




\begin{longtable}{cl} 

\textcolor{blue}{\textarabic{أَمِنْ نَذَكُّرِ جِيْرَانٍ بِذِيْ سَلَمٍ * مَزَجْتَ دَمْعًاجَرَى\LTRfootnote{This is a note.}  مِنْ مُقْلَةٍ بِدَمٍ\LTRfootnote{This is a variant.}}} & \textarabic{١} \\* 
\E{ } & \\[2mm] 
\textcolor{mygreen}{\textarabic{نِكَڪُكُمْبُكَ جِرَنِ نْيٖمَ * وَلِيٗكٗ هَپٗ نِذِى سَلَمِ}} &  \\* 
nikakukumbuka jirani nyema * waliyoko hapo nidhii salami & 1c/d \\* 
\E{ } & \\[2mm] 
\textcolor{mygreen}{\textarabic{اُمٖلِتَنْڠَنْيَ تٗزِ كْوَ دَمِ * كْوَمْبَ مَعَنَايٖ نِهَيٗ سٖيْمَا}} &  \\* 
umelitanganya tozi kwa dami * kwamba maanaye nihayo sema & 1e/f \\* 
\E{ } & \\[2mm] 
\\[6mm] 

\textcolor{blue}{\textarabic{أَمْ هُبَّتِ الرِّيحُ مِنْ تِلْقَاءِ كَاظِمَةٍ * وَأَوْ مَضَ الْبَرْقُ فِى الظَّلْمَاءِ مِنْ إِضَمٍ}} & \textarabic{٢} \\* 
\E{ } & \\[2mm] 
\textcolor{mygreen}{\textarabic{اَمَ نِؤُپٖتٗ كُپِتَ كْوَكٖ * كُتٗكَ كَظِمَ جَنِبُ زَكٖ}} &  \\* 
ama niupeto kupita kwake * kutoka kadhima janibu zake & 2c/d \\* 
\E{ } & \\[2mm] 
\textcolor{mygreen}{\textarabic{اَمَ نِوُ مِمِ كْوَ نُوْرُ يَكٖ * كُيْنُكَ كِيْزَ هَپٗ اِظَمَا}} &  \\* 
ama niwu mimi kwa nuru yake * kuynuka kiza hapo idhama & 2e/f \\* 
\E{ } & \\[2mm] 
\\[6mm] 

\textcolor{blue}{\textarabic{نَمَا لِعَيْبَىْكَ أِنْ قُلْتَ اكْفُفَا هَمَتَا * وَمَا لِقَلْبِكَ إِنْ قُلْتُ اسْتَفِقْ يَهِمِ}} & \textarabic{٣} \\* 
\E{ } & \\[2mm] 
\textcolor{mygreen}{\textarabic{مْبٗنَ مَتٗ يَكٗ هُكٗمِ تٗزِ * اُكِيَزِوِ يَا هَيَسِكِزَ}} &  \\* 
mbona mato yako hukomi tozi * ukiyaziwi ya hayasikiza & 3c/d \\* 
\E{ } & \\[2mm] 
\textcolor{mygreen}{\textarabic{كَذَلِكَ مٗويٗ اُنَ سِمَنْزِ * اُكِوُزُ نْدُوَ هُزِدِ هِمَا}} &  \\* 
kadhalika moyo una simanzi * ukiwuzu nduwa huzidi hima & 3e/f \\* 
\E{ } & \\[2mm] 
\\[6mm] 

\textcolor{blue}{\textarabic{أَبَخسَبُ الصَّبُّ أَنَّ الْحُبَّ مُنكَتِمٌ * مَابَيْنَ مُذْسَجِمٍ مِنْهُ وَمُضْطَرِمِ}} & \textarabic{٤} \\* 
\E{ } & \\[2mm] 
\textcolor{mygreen}{\textarabic{مْتُ اَپٖنْدَاوٗ هَسِتَمَنِ * هُوَ نِعَلَامَ مَحَضِرَانِ}} &  \\* 
mtu apendawo hasitamani * huwa nialama mahadhirani & 4c/d \\* 
\E{ } & \\[2mm] 
\textcolor{mygreen}{\textarabic{نَخَصَ اَوَاپٗ تٗزِ مَيٗوْنِ * نَمُئِلِ وَكٖ هَپَمْبِ تَامَا}} &  \\* 
nahasa awapo tozi mayoni * namuili wake hapambi tama & 4e/f \\* 
\E{ } & \\[2mm] 
\\[6mm] 

\textcolor{blue}{\textarabic{لَوْ لَاالْهَوَى لَمْ تُرِقُ دَمْعًا عَلَى طَلَلٍ * وَلَا أَرِقْتَ لِذِكْرِ الْبَانِ وَالْعَلَمِ}} & \textarabic{٥} \\* 
\E{ } & \\[2mm] 
\textcolor{mygreen}{\textarabic{نَلَوُ مَپٖنْدِ يَكُپٖتٖوٗ * نْدِپٗ هُكُلِزَ اَثَرِزَاوٗ}} &  \\* 
nalawu mapendi yakupetewo * ndipo hukuliza atharizawo & 5c/d \\* 
\E{ } & \\[2mm] 
\textcolor{mygreen}{\textarabic{نَوُ كُمْبُكَپٗ مِقُوْجٖيَوٗ * يَمِ نَخَبَالِ زِلِيٗ ثَمَا}} &  \\* 
nawu kumbukapo miqujeyawo * yami nahabali ziliyo thama & 5e/f \\* 
\E{ } & \\[2mm] 
\\[6mm] 

\textcolor{blue}{\textarabic{فَكَيْفَ تُنْكِرُ حَبًّا بَعْدَ مَا شَهِدَتْ * بِهِ عَلَيْكَ عُدُوْلُ الْدَّمْعِ وَالسَّقَمِ}} & \textarabic{٦} \\* 
\E{ } & \\[2mm] 
\textcolor{mygreen}{\textarabic{كْوَنِ كُيَكَنْيَ مَپٖنْدِ هَيٗ * نَسِ هُكُوٗ نَ حَالِ اُلِيٗ}} &  \\* 
kwani kuyakanya mapendi hayo * nasi hukuwo na hali uliyo & 6c/d \\* 
\E{ } & \\[2mm] 
\textcolor{mygreen}{\textarabic{مَتٗزِ نَنْدِيٗ نَوَايٗ وَايٗ * نْدِيٗ اُشَهِدِ اُلِيٗ تَمَا}} &  \\* 
matozi nandiyo nawayo wayo * ndiyo ushahidi uliyo tama & 6e/f \\* 
\E{ } & \\[2mm] 
\\[6mm] 

\textcolor{blue}{\textarabic{وَأَثْبَتَ اْلوَجْدُخَطَّىْ عبْرَةٍ وَضَنًى * مِثْلَ الْمَهَارِ عَلَى خَدَّيْكَ وَالْعَنَمِ}} & \textarabic{٧} \\* 
\E{ } & \\[2mm] 
\textcolor{mygreen}{\textarabic{اِمٖثُبُتِشَ يَكٗ حُزُنِ * مِفُنْدَ مِوِلِ زَتٖفُتٖنِ}} &  \\* 
imethubutisha yako huzuni * mifunda miwili zatefuteni & 7c/d \\* 
\E{ } & \\[2mm] 
\textcolor{mygreen}{\textarabic{رَنْچِ يَبَهَارِ كُوُ كٗنْدٗنِ * يَمَتٗزِ يَكٗ اَوْ عَنَمَا}} &  \\* 
ranchi yabahari kuwu kondoni * yamatozi yako au anama & 7e/f \\* 
\E{ } & \\[2mm] 
\\[6mm] 

\textcolor{blue}{\textarabic{نَعَمْ سَرَى طَيْفُ مَنْ أَهْوَى فَأَرِّ قَنِى * وَالْحُبُّ يَعْتَرِضُ اُلَّلذَّاتِ باْلأَلَمِ}} & \textarabic{٨} \\* 
\E{ } & \\[2mm] 
\textcolor{mygreen}{\textarabic{نِكْوٖلِ مَنٖنٗ يَكٗ نِكْوٖلِ * نِكِڤُلِ ػَكٖ وَنْڠُ خَلِيْلِ}} &  \\* 
nikweli maneno yako nikweli * nikivuli chake wangu halili & 8c/d \\* 
\E{ } & \\[2mm] 
\textcolor{mygreen}{\textarabic{ػَلِنَنْڠَزِشَ بَيَنْڠُ حَلِ * سِنَ لَذَ تٖنَ وَتَ مَلَامَا}} &  \\* 
chalinangazisha bayangu hali * sina ladha tena wata malama & 8e/f \\* 
\E{ } & \\[2mm] 
\\[6mm] 

\textcolor{blue}{\textarabic{يَالَائِمِيْ فِى الْهَوَى الْعُذْرِيِّ مَعْذِرَةً * مِنِّيْ إِلَيْكَ وَلَوْ أَنْصَفْتَ لِمْ يَلُمِ}} & \textarabic{٩} \\* 
\E{ } & \\[2mm] 
\textcolor{mygreen}{\textarabic{وَلَ سِنَ بُدِ سِنَ صُبِرَ * حُبَ يَنْڠُ كَمَ بَنُو عُذٖرَ}} &  \\* 
wala sina budi sina subira * huba yangu kama banuu udhera & 9c/d \\* 
\E{ } & \\[2mm] 
\textcolor{mygreen}{\textarabic{نِمٖكُعَرِفُ يَنْڠُ ضَرُوْرَ * وَتَ كُنِيَايَ فَنْيَ رُحُمَا}} &  \\* 
nimekuarifu yangu dharura * wata kuniyaya fanya ruhuma & 9e/f \\* 
\E{ } & \\[2mm] 
\\[6mm] 

\textcolor{blue}{\textarabic{عَدَتْكَ حَالِيَ لَا سِرِّيْ بِمُسْتَتِرٍ * عَنِ الْوُشَاةِ وَلَا دَائِيْ بِمُنْحَسِمِ}} & \textarabic{١٠} \\* 
\E{ } & \\[2mm] 
\textcolor{mygreen}{\textarabic{سِنَ سِرِ تٖنَ اِمٖفُنُوكَ * هَوٗ تٗنْڠٗلٖزَ وَ مُهُتٗكَ}} &  \\* 
sina siri tena imefunuka * hawo tongoleza wa muhutoka & 10c/d \\* 
\E{ } & \\[2mm] 
\textcolor{mygreen}{\textarabic{وَلَ نْدْوٖى زَنْڠُ سِكُپٗزٖكَ * نِئِپِ فَئِدَ يَنْڠُ كُسٖما}} &  \\* 
wala ndwee zangu sikupozeka * niipi faida yangu kusema & 10e/f \\* 
\E{ } & \\[2mm] 
\\[6mm] 

\textcolor{blue}{\textarabic{مَحَضْتَنِيَ النُّصْحَ لَكِنْ لَسْتُ أَسْمَعُهُ * إنَّ الُحِبُّ عَنِ الْعُذَّ الِ فِيْ صَمَمِ}} & \textarabic{١١} \\* 
\E{ } & \\[2mm] 
\textcolor{mygreen}{\textarabic{اُمٖنِنَصِحِ كْوَطَاقَ يَكٗ * اِلَّا كُسِكِزَ هِلٗ هَللِكٗ}} &  \\* 
umeninasihi kwataqa yako * illa kusikiza hilo halliko & 11c/d \\* 
\E{ } & \\[2mm] 
\textcolor{mygreen}{\textarabic{مْتُ كِنِتَايَ مِمِ نِدُكٗ * كْوَ مَحَبَّا يَنْڠُ سِوَكُكٗمَا}} &  \\* 
mtu kinitaya mimi niduko * kwa mahabba yangu siwakukoma & 11e/f \\* 
\E{ } & \\[2mm] 
\\[6mm] 

\textcolor{blue}{\textarabic{إِنِّى اتَّهَمْتُ نَصِيْحَ الشَّيْبِ فِيْ عَذَلٍ * وَالشَّيْبُ أَبْعَدُ فِيْ نُصْحٍ عَنِ التُّهَمِ}} & \textarabic{١٢} \\* 
\E{ } & \\[2mm] 
\textcolor{mygreen}{\textarabic{مِمِ سِكُكِرِ نَصَحَ بُورَ * يَمْڤِ نْيٖوُپٖ كُنْڠَرَنْڠَرَ}} &  \\* 
mimi sikukiri nasaha bura * yamvi nyewupe kungarangara & 12c/d \\* 
\E{ } & \\[2mm] 
\textcolor{mygreen}{\textarabic{نِمٖووتُهُمُ سِوُپِ سُورَ * نِمٖوُظَنِيَ اُنَ يُهُمَا}} &  \\* 
nimewutuhumu siwupi sura * nimewudhaniya una yuhuma & 12e/f \\* 
\E{ } & \\[2mm] 
\\[6mm] 

\textcolor{blue}{\textarabic{فَإِنَّ أَمَّارَتِيْ بِالسُّوْءِ مَااُتَّعَظَتْ * مِنْ جَهْلِهَا بِنَذِيْرِ الشَّيْبِ وَالْهَرَمِ}} & \textarabic{١٣} \\* 
\E{ } & \\[2mm] 
\textcolor{mygreen}{\textarabic{كْوَ اُوٗڤُ وَكِ نَفُسِ يَنْڠُ * هَئِكُعِظِكَ كْوَ كِلِمْوٖنْڠُ}} &  \\* 
kwa uwovu waki nafusi yangu * haikuidhika kwa kilimwengu & 13c/d \\* 
\E{ } & \\[2mm] 
\textcolor{mygreen}{\textarabic{مْڤِ نَوٗ كٗنْڠْوٖ اَتَ نْدُيَنْڠُ * سِتٗشِكَ يَكٗ كَرُدِ نْيُومَا}} &  \\* 
mvi nawo kongwe ata nduyangu * sitoshika yako karudi nyuma & 13e/f \\* 
\E{ } & \\[2mm] 
\\[6mm] 

\end{longtable} 

\renewcommand{\bibname}{References} 
\begingroup 
\printbibliography 
\endgroup 

\end{document}
